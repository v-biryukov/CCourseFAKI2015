\documentclass{article}
\usepackage[utf8x]{inputenc}
\usepackage{ucs}
\usepackage{amsmath} 
\usepackage{amsfonts}
\usepackage{marvosym}
\usepackage{wasysym}
\usepackage{upgreek}
\usepackage[english,russian]{babel}
\usepackage{graphicx}
\usepackage{float}
\usepackage{textcomp}
\usepackage{hyperref}
\usepackage{geometry}
  \geometry{left=2cm}
  \geometry{right=1.5cm}
  \geometry{top=1cm}
  \geometry{bottom=2cm}
\usepackage{tikz}
\usepackage{ccaption}
\usepackage{multicol}

\hypersetup{
   colorlinks=true,
   citecolor=blue,
   linkcolor=black,
   urlcolor=blue
}

\usepackage{listings}
%\setlength{\columnsep}{1.5cm}
%\setlength{\columnseprule}{0.2pt}

\usepackage[absolute]{textpos}


\usepackage{colortbl,graphicx,tikz}
\definecolor{X}{rgb}{.5,.5,.5}

\renewcommand{\thesubsection}{\arabic{subsection}}

\begin{document}
\pagenumbering{gobble}
\lstset{
  language=C,                % choose the language of the code
  basicstyle=\linespread{1.1}\ttfamily,
  columns=fixed,
  fontadjust=true,
  basewidth=0.5em,
  keywordstyle=\color{blue}\bfseries,
  commentstyle=\color{gray},
  stringstyle=\ttfamily\color{orange!50!black},
  showstringspaces=false,
  numbersep=5pt,
  numberstyle=\tiny\color{black},
  numberfirstline=true,
  stepnumber=1,                   % the step between two line-numbers.        
  numbersep=10pt,                  % how far the line-numbers are from the code
  backgroundcolor=\color{white},  % choose the background color. You must add \usepackage{color}
  showstringspaces=false,         % underline spaces within strings
  captionpos=b,                   % sets the caption-position to bottom
  breaklines=true,                % sets automatic line breaking
  breakatwhitespace=true,         % sets if automatic breaks should only happen at whitespace
  xleftmargin=.2in,
  extendedchars=\true,
  keepspaces = true,
}
\lstset{literate=%
   *{0}{{{\color{red!20!violet}0}}}1
    {1}{{{\color{red!20!violet}1}}}1
    {2}{{{\color{red!20!violet}2}}}1
    {3}{{{\color{red!20!violet}3}}}1
    {4}{{{\color{red!20!violet}4}}}1
    {5}{{{\color{red!20!violet}5}}}1
    {6}{{{\color{red!20!violet}6}}}1
    {7}{{{\color{red!20!violet}7}}}1
    {8}{{{\color{red!20!violet}8}}}1
    {9}{{{\color{red!20!violet}9}}}1
}

\renewcommand{\thesubsection}{\arabic{subsection}}
\makeatletter
\def\@seccntformat#1{\@ifundefined{#1@cntformat}%
   {\csname the#1\endcsname\quad}%    default
   {\csname #1@cntformat\endcsname}}% enable individual control
\newcommand\section@cntformat{}     % section level 
\newcommand\subsection@cntformat{Задача \thesubsection.\space} % subsection level
\newcommand\subsubsection@cntformat{\thesubsubsection.\space} % subsubsection level
\makeatother


\makeatletter
\newcount\my@repeat@count
\newcommand{\myrepeat}[2]{%
  \begingroup
  \my@repeat@count=\z@
  \@whilenum\my@repeat@count<#1\do{#2\advance\my@repeat@count\@ne}%
  \endgroup
}
\makeatother

\title{Семинар \#3: Класс \texttt{std::string}. Инициализация. Оператор \texttt{new}. Домашнее задание.\vspace{-5ex}}\date{}\maketitle

\subsection{Измени регистр первой буквы}
Напишите функцию, которая будет принимать на вход строку типа \texttt{std::string} и возвращать строку с изменённым регистром первой буквы. Например, если на вход пришла строка \texttt{"Cat"}, то функция должна вернуть строку \texttt{"cat"}. Если на вход пришла строка \texttt{"dog"}, то функция должна вернуть строку \texttt{"Dog"}. Если на вход пришла пустая строка, то функция должна вернуть простую строку.


\subsection{Удвоение}
Напишите функции, которые будут принимать на вход строку и возвращать строку, повторённую два раза. То есть, если на вход этой функции приходит строка \texttt{"Cat"}, то функция должна вернуть \texttt{"CatCat"}.
При этом нужно написать несколько функций, которые должны делать одно и то же, но возвращать результат разными методами.
\begin{itemize}
\item \texttt{std::string repeat1(const std::string\& s)}
Должна принимать на вход строку и возвращать результат.
\item \texttt{void repeat2(std::string\& s)}
Должна принимать на вход строку по ссылке и изменять эту строку.
\item \texttt{void repeat3(std::string* ps)}
Должна принимать на вход указатель на строку и изменять строку, чей адрес хранит этот указатель.
\item \texttt{std::string* repeat4(const std::string\& s)}
Эта функция должна создавать удвоенную строку в куче с помощью оператора \texttt{new} и возвращать указатель на неё. После вызова функции \texttt{repeat4} программист, который будет использовать эту функцию, сам должен позаботиться об её удалении.
\end{itemize}

Протестируйте эти функции в \texttt{main}.

\subsection{Умножение строки}
Напишите перегруженный оператор умножения, которая будет принимать на вход строку \texttt{std::string} и некоторое целое число $n$ и возвращать строку, повторённую $n$ раз. Протестируйте эту функцию в функции \texttt{main}.


\subsection{Усечение}
Напишите функцию \texttt{void truncateToDot(std::string\& s)}, которая будет принимать строку по ссылке и усекать её до первого символа точки. Размер и вместимость строки должны стать как можно более маленькими, для этого используйте метод \texttt{shrink\_to\_fit}. Функция не должна ничего возвращать. Протестируйте функцию в \texttt{main}.
\begin{center}
\begin{tabular}{ l | l }
 до & после \\ \hline
 \texttt{"cat.dog.mouse.elephant.tiger.lion"} & \texttt{"cat"} \\
 \texttt{"wikipedia.org"} & \texttt{"wikipedia"}  \\ 
 \texttt{".com"} & \texttt{"{}"} \\
\end{tabular}
\end{center}


\subsection{Сумма из строки}
Напишите функцию, которая принимает на вход строку в следующем формате: \texttt{"[num1, num2, ... numN]"}.
Функция должна возвращать целое число типа \texttt{int} -- сумму всех чисел в квадратных скобках. В случае, если на вход приходит некорректная строка, то функция должна бросать исключение \texttt{std::invalid\_argument}. Протестируйте эту функцию в \texttt{main}.
\begin{center}
\begin{tabular}{ l | l }
 аргумент & возвращаемое значение \\ \hline
 \texttt{"[10, 20, 30, 40, 50]"} & \texttt{150} \\
 \texttt{"[4, 8, 15, 16, 23, 42]"} & \texttt{108}  \\ 
 \texttt{"[20]"} & \texttt{20} \\
 \texttt{"[]"} & \texttt{0} \\
\end{tabular}
\end{center}

\subsection{\texttt{new}}
Используйте операторы \texttt{new} или \texttt{new[]}, чтобы создать в куче и сразу инициализировать следующие объекты:
\begin{itemize}
\item Один объект типа \texttt{int}, равный \texttt{123}.
\item Один объект типа \texttt{std::string}, равный \texttt{"Cats and Dogs"}.
\item Маcсив объектов типа \texttt{int}, равный \texttt{\{10, 20, 30, 40, 50\}}.
\item Маcсив объектов типа \texttt{std::string}, равный \texttt{\{"Cat"{}, "Dog"{}, "Mouse"\}}.
\item Маcсив из 3-х объектов типа \texttt{std::string\_view}, указывающих на строки из предыдущего пункта.
\end{itemize}
Все эти объекты обязательно должны быть созданы в куче, а не на стеке.
Напечатайте все созданные объекты на экран.
Удалите все созданные объекты с помощью операторов \texttt{delete} и \texttt{delete[]}.


\subsection{Кошки-мышки}
Есть некоторое количество групп кошек, которые соревнуются в том, кто поймает больше мышек.
Соревнование командное, побеждает та группа, которая суммарно поймала больше мышек.

Ваша задача - написать класс \texttt{Cat}, который оказался бы полезным для автоматизации подсчёта пойманных мышек. Сумарное количество пойманых мышек для каждой группы будет хранится вне класса \texttt{Cat}, например в локальной переменной (так как кошки из других групп не должны иметь доступ к этой переменной). В самом же классе должна храниться \textit{ссылка} на эту переменную. В классе вам нужно написать следующие методы:

\begin{itemize}
\item Конструктор. Должен конструироваться от переменной, в которой будет храниться суммарное количество мышек.
\item Метод \texttt{catchMice} - поймать мышек. Принимает число и увеличивает суммарно количество пойманых мышек данной группы на это число.
\end{itemize}

Пример использования такого класса (код ниже должен работать и с вашим классом):

\begin{lstlisting}
int miceCaughtA = 0;
int miceCaughtB = 0;

Cat alice(miceCaughtA), alex(miceCaughtA), anna(miceCaughtA);
Cat bob(miceCaughtB), bella(miceCaughtB);

alice.catchMice(2);
alex.catchMice(1);
bella.catchMice(4);
bob.catchMice(2);
anna.catchMice(1);
bella.catchMice(1);
alex.catchMice(4);
bella.catchMice(5);
alice.catchMice(2);

cout << miceCaughtA << endl; // Должно напечатать 10
cout << miceCaughtB << endl; // Должно напечатать 12
\end{lstlisting}


\subsection{Создание \texttt{mipt::String}}
В файле \texttt{miptstring.hpp} содержится класс \texttt{mipt::String}. Создайте объекты этого класса в следующим образом:
\begin{itemize}
\item Создайте объект класса \texttt{mipt::String} на стеке обычным образом и инициализируйте его строкой \texttt{Cat}.
\item Создайте объект класса \texttt{mipt::String} в куче, используя оператор \texttt{new}, и инициализируйте его строкой \texttt{Dog}. Напечатайте этот объект на экран. Удалите этот объект, с помощью оператора \texttt{delete}.
\item Пусть у нас есть массив:
\begin{lstlisting}
char x[sizeof(mipt::String)];
\end{lstlisting}
Создайте объект класса \texttt{mipt::String} в этом массиве с помощью оператора \texttt{placement new} и инициализируйте объект строкой \texttt{Elephant}. Напечатайте этот объект на экран. Удалите этот объект.
\end{itemize}






\newpage
\quad
\newpage
\section{Необязательные задачи}
\subsection{\texttt{StringView}}
Создайте свой класс \texttt{mipt::StringView}, аналог класса \texttt{std::string\_view} для класса \texttt{mipt::String}.
Этот класс должен содержать 2 поля указатель \texttt{mpData} (тип \texttt{const char*}) и размер \texttt{mSize} (тип \texttt{size\_t}). Класс \texttt{mipt::String} можно найти в файле \texttt{miptstring.hpp}.\\

Методы, которые нужно реализовать:
\begin{itemize}
\item Конструктор по умолчанию. Должен устанавливать указатель в \texttt{nullptr}, а размер в 0.
\item Конструктор копирования.
\item Конструктор от \texttt{mipt::String}.
\item Конструктор от \texttt{const char*}
\item Перегруженный \texttt{operator[]}
\item Метод \texttt{at}, аналог \texttt{operator[]}, но если индекс выходит за границы, то данный метод должен бросать исключение \texttt{std::out\_of\_range}
\item Перегруженный \texttt{operator<}
\item Перегруженный \texttt{operator<{}<} с объектом \texttt{std::ostream}.
\item Метод \texttt{size}.
\item Метод \texttt{substr}, должен возвращать объект типа \texttt{std::string\_view}.
\item Методы \texttt{remove\_prefix} и \texttt{remove\_suffix}.
\end{itemize}

Также придётся изменить класс \texttt{mipt::String}. Нужно будет добавить конструктор от \texttt{mipt::StringView}.
Протестируйте этот класс.

\end{document}