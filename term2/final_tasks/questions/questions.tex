\documentclass{article}
\usepackage[utf8x]{inputenc}
\usepackage{ucs}
\usepackage{amsmath} 
\usepackage{mathtext}
\usepackage{amsfonts}
\usepackage{upgreek}
\usepackage[english,russian]{babel}
\usepackage{graphicx}
\usepackage{float}
\usepackage{textcomp}
\usepackage{hyperref}
\usepackage{geometry}
  \geometry{left=2cm}
  \geometry{right=1.5cm}
  \geometry{top=1cm}
  \geometry{bottom=2cm}
\usepackage{tikz}
\usepackage{ccaption}

\usepackage{listings}

\lstdefinelanguage
   [x64]{Assembler}     % add a "x64" dialect of Assembler
   [x86masm]{Assembler} % based on the "x86masm" dialect
   % with these extra keywords:
   {morekeywords={CDQE,CQO,CMPSQ,CMPXCHG16B,JRCXZ,LODSQ,MOVSXD, %
                  POPFQ,PUSHFQ,SCASQ,STOSQ,IRETQ,RDTSCP,SWAPGS, %
                  rax,rdx,rcx,rbx,rsi,rdi,rsp,rbp, %
                  r8,r8d,r8w,r8b,r9,r9d,r9w,r9b}} % etc.

\lstset{language=[x64]Assembler}

\begin{document}
\pagenumbering{gobble}

\section*{Задачи:}
\subsection*{Часть A -- на оценку хор}

\begin{enumerate}

\item Структуры данных\\
\textbf{Что нужно:} Знать что такое O(n) нотация и уметь определять сложность алгоритма. Знать что такое список, дерево и хэш таблица, как они реализуется в языке C. Знать сложности операций добавления, удаления, взятия элемента по индексу и поиска элемента в этих структурах данных.\\
\textbf{Пример задачи:} Объяснить реализацию хэш-таблицы в языке C.

\item Битовые операции.\\
\textbf{Что нужно:} Знать что такое побитовые операторы: побитовое не, побитовое или, побитовое и, исключающее или. Знать что такое битовые сдвиги (логический и арифметический). \\
\textbf{Пример задачи:} Используя битовые операции умножить целое число на 128. Используя битовые операции найти остаток деления целого числа на $2^n$. Используя битовые операции найти значение $k$-го бита целого числа.

\item Представление целых чисел в памяти компьютера.\\
\textbf{Что нужно:} Знать как представляются целые числа в памяти компьютера. Отдельно рассмотреть беззнаковые целые числа и знаковые(дополнительный код). Little endian и big endian.\\
\textbf{Пример задачи:} Как представляется в памяти компьютера число $-5000$?

\item Представление чисел с плавающей точкой в памяти компьютера.\\
\textbf{Что нужно:} Знать как представляются числа с плавающей точкой в памяти компьютера. Знать как представляются особые значения inf, nan, денормализованные числа. Little endian и big endian.\\
\textbf{Пример задачи:} Как представляется в памяти компьютера число $-2.0$?

\item Трансляция C кода в объектный код.\\
\textbf{Что нужно:} Знать все стадии трансляции C кода в объектный код(Препроцессинг, компиляция кода на языке C в код на языке ассемблера, компиляция ассемблерного кода в машинный код, линковка.). Уметь использовать gcc для исполнения этих стадий трансляции кода.

\item Программная модель ассемблера. Основы языка ассемблера.\\
\textbf{Что нужно:} Знать что такое центральный процессор, регистры, счётчик команд, коды условий. Знать чем различаются наборы регистров для 32-битных систем и для 64-битных. Операция mov и lea. Адресация памяти.\\
\textbf{Пример задачи:} Пусть в регистрах eax и ebx лежат значения 0x15 и 0xf соответственно. Что будет лежать в регистре ecx после выполнения следующей операции 
\begin{lstlisting}
leal 0x500(%eax, %ebx, 4), %ecx
\end{lstlisting}

\item Язык ассемблера.\\
\textbf{Что нужно:} Арифметические команды: add, sub, imul, idiv, sal, sar, xor, and, or, inc, dec. Флаги условий(CF, ZF, SF, OF), как эти флаги устанавливаются и используются. Команды cmp, test, set. Условные переходы. Команды j*(jmp, jg, jl и т. д.). Реализация циклов в языке ассемблера с помощью условных переходов. Вызов функции в языке ассемблера.\\
\textbf{Пример задачи:} Написать программу на языке ассемблера, которая считывает 2 целых положительных числа и возводит первое число в степень другого.
\end{enumerate}

\subsection*{Часть B -- на оценку отл}
\begin{enumerate}

\item Машина Тьюринга. \\
\textbf{Что нужно:} Знать что такое машина Тьюринга и зачем она нужна. Что такое детерминированная и недетерминированная машины Тьюринага. Классы сложности задач: P, NP, NP complete. Привести примеры задач, которые:
\begin{enumerate}
\item входят в класс P
\item входят в класс NP, но не входят в P
\item входят в класс NP complete
\item не входят ни в один из перечисленных классов, но не являются невычислимыми
\item невычислимы
\end{enumerate} 


\item Графы. \\
\textbf{Что нужно:} Знать что такое граф, как он может быть представлен (матрица смежности и список смежных вершин) и преемущества/недостатки этих представлений. Различные виды графов: связный граф, взвешенный граф, ациклический граф, дерево. Алгоритмы поиска в глубину и поиска в ширину в графе. Абстрактный тип данных очередь с приоритетом, реализация этого абстрактного типа данных с помощью структуры данных heap(куча). Алгоритм Дейкстры и использование очереди с приоритетом в этом алгоритме. \\


\item Стек. \\
\textbf{Что нужно:} Знать что такое стек и как он используется в языке C и в языке ассемблера. Команды push и pop. Особый регистр rsp(или esp в 32-битных системах). Вызов процедуры в языке ассемблера, адресс возврата, команды call и ret. Передача аргументов в процедуру и возращаемый результат. Использование регистров в x86-64.  \\
\textbf{Пример задачи:} Написать рекурсивную функцию вычисления факториала на языке ассемблера.\\


\item Иерархия памяти. Кэш. \\
\textbf{Что нужно:} Принцип локальности. Иерархия памяти. Что такое регистры, кэш(несколько уровней), основная память, локальные диски, удалённые хранилища. Общие принципы устройства кэша: попадания кэша, промахи кэша(холодный промах, конфликтный промах, промах ёмкости), линия кэша. Характерные значения вероятности промахов и продолжительности доступа в кэш и в основную память в современных архитектурах.    \\
\textbf{Пример задачи:} Вычислить процент промахов в следующем участке кода. Длина кэш-линии -- 64 байта. Размер кэша -- 1 мегабайт.\\
\begin{verbatim}
#define N 10000
...
int A[N], B[2*N], sum=0;
...
for (int i = 0; i < N; ++i) {
    sum += A[i] + B[2*i];
}
\end{verbatim}


\end{enumerate}



\iffalse
\section*{Вопросы:}

\begin{enumerate}

\item Назовите базовые команды командной строки linux
\item Что такое препроцессор?
\item Что делает директива \#include?
\item Что делает директива \#define?
\item Назовите все стандартные целочисленные типы C и их размер.
\item Назовите все стандартные типы с плавающей точкой в C и их размер.
\item Операторы break и continue
\item Что такое прототип функции?
\item Как функция может возвращать значения?
\item Спецификатор типа void
\item Области видимости переменных в функции
\item Передача по ссылке и по значению
\item Аргументы функции main
\item Как объявить константу в C?
\item Объявление массива
\item Как хранятся массивы в памяти
\item Строки в стиле C
\item Функции работы со строками
\item Алгоритмы сортировки вставками, выбором, пузырьком (1 из 3-х)
\item Описание структуры, typedef
\item Что такое указатель? Какой размер переменной указателя? 
\item Как получить адрес переменной? Как получить переменную по адресу?
\item Указатель на 1-й элемент массива, на 5-й.
\item Стэк и куча. Что такое, преемущества и недостатки.
\item Что делает malloc? Выделите память на массив из n элементов типа int.
\item Как освобождать память? Почему плохо не освобождать память?
\item Есть массив A состоящий из 100 элементов. Что будет, если исполнить A[110].
\item Принцип "разделяй и властвуй"
\item Сортировка слиянием или быстрая сортировка
\item Что такое связный список.
\item Сложность добавления и удаления элементов в связный список и в массив.
\item Что такое дерево?
\item Что такое двоичное дерево поиска? 
\item Сложность поиска элемента с помощью двоичного дерева поиска. Сравнение с бинарным поиском.
\end{enumerate}

\fi
\end{document}