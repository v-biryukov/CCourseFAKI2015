\documentclass{article}
\usepackage[utf8x]{inputenc}
\usepackage{ucs}
\usepackage{amsmath} 
\usepackage{amsfonts}
\usepackage{upgreek}
\usepackage[english,russian]{babel}
\usepackage{graphicx}
\usepackage{float}
\usepackage{textcomp}
\usepackage{hyperref}
\usepackage{geometry}
  \geometry{left=2cm}
  \geometry{right=1.5cm}
  \geometry{top=1cm}
  \geometry{bottom=2cm}
\usepackage{tikz}
\usepackage{ccaption}
\usepackage{multicol}

\usepackage{listings}
%\setlength{\columnsep}{1.5cm}
%\setlength{\columnseprule}{0.2pt}


\begin{document}
\pagenumbering{gobble}

\lstset{
  language=C++,                % choose the language of the code
  basicstyle=\linespread{1.1}\ttfamily,
  columns=fixed,
  fontadjust=true,
  basewidth=0.5em,
  keywordstyle=\color{blue}\bfseries,
  commentstyle=\color{gray},
  stringstyle=\ttfamily\color{orange!50!black},
  showstringspaces=false,
  %numbers=false,                   % where to put the line-numbers
  numbersep=5pt,
  numberstyle=\tiny\color{black},
  numberfirstline=true,
  stepnumber=1,                   % the step between two line-numbers.        
  numbersep=10pt,                  % how far the line-numbers are from the code
  backgroundcolor=\color{white},  % choose the background color. You must add \usepackage{color}
  showstringspaces=false,         % underline spaces within strings
  captionpos=b,                   % sets the caption-position to bottom
  breaklines=true,                % sets automatic line breaking
  breakatwhitespace=true,         % sets if automatic breaks should only happen at whitespace
  xleftmargin=.2in,
  extendedchars=\true,
  keepspaces = true,
}
\lstset{literate=%
   *{0}{{{\color{red!20!violet}0}}}1
    {1}{{{\color{red!20!violet}1}}}1
    {2}{{{\color{red!20!violet}2}}}1
    {3}{{{\color{red!20!violet}3}}}1
    {4}{{{\color{red!20!violet}4}}}1
    {5}{{{\color{red!20!violet}5}}}1
    {6}{{{\color{red!20!violet}6}}}1
    {7}{{{\color{red!20!violet}7}}}1
    {8}{{{\color{red!20!violet}8}}}1
    {9}{{{\color{red!20!violet}9}}}1
}

\title{Семинар \#5: Функциональные объекты. \vspace{-5ex}}\date{}\maketitle
\section*{Часть 1: Указатели на функции в алгоритмах \texttt{STL}}
Функциональный объект -- это объект, к которому можно применить оператор вызова функции (\texttt{operator()}). Простейшими функциональными объектами, введёными еще в языке \texttt{C}, являются указатели на функции.
Многие алгоритмы STL могут принимать указатели на функции в качестве одного из параметров
\begin{lstlisting}
#include <iostream>
#include <vector>
#include <algorithm>

int cmp(int a, int b) {
    return a > b;
}

int add_one(int a) {
    return a + 1;
}

int main() {
    std::vector v<int> {18, 51, 2, 25, 14, 97, 73};
    std::sort(v.begin(), v.end(), cmp);
    for (int num : v) {
        std::cout << num << " ";
    }    
    std::cout << std::endl;
    
    std::transform(v.begin(), v.end(), v.begin(), add_one);
    for (int num : v) {
        std::cout << num << " ";
    } 
}
\end{lstlisting}
В данном коде используются 2 алгоритма STL:
\begin{itemize}
\item[--] \texttt{std::sort} третьим объектом принимает функциональный объект - компаратор.
\item[--] \texttt{std::transform} принимает на вход сначала 3 итератора. Первые два итератора задают последовательность объектов. Третий итератор указывает место куда нужно сохранить результат трансформации. В данном случае записывается тот же вектор, откуда берутся числа. Четвёртый аргумент -- это функциональный объект, указывающий какую трансформацию нужно применить.
\end{itemize}

\subsection*{Задачи:}
\begin{itemize}
\item Изменить код программы выше так, чтобы сортировка проходила по последней цифре.
\item Изменить код программы выше так, функция \texttt{std::transform} увеличивала все числа в 2 раза.
\item В файле \texttt{books.cpp} лежит заготовка, содержащая массив структур. Отсортируйте этот массив структур по возрастанию цены и напечатайте.
\item Отсортируйте массив \texttt{books} по названию и напечатайте его.
\item Увеличьте цену каждой книге на 10 процентов, используя функцию \texttt{std::transform}.
\end{itemize}


\newpage
\section*{Часть 2: Функторы}
Другим видом функциональных объектов в \texttt{C++} является функтор. Функторы -- это просто объекты класса, у которого есть перегруженный оператор \texttt{operator()}.
\begin{lstlisting}
#include <iostream>
#include <vector>
#include <algorithm>

struct AddFunctor {
private:
    int x;
public:
    AddFunctor(int x) : x(x) {};
    
    operator()(int a) {
        return a + x;
    }
};


int main() {
    std::vector v<int> {18, 51, 2, 25, 14, 97, 73};
    
    std::transform(v.begin(), v.end(), v.begin(), AddFunctor(5));
    for (int num : v) {
        std::cout << num << " ";
    } 
}
\end{lstlisting}

\subsection*{Задачи:}
\begin{itemize}
\item Напишите \texttt{ModuloCmpFunctor}, который будет принимать в конструкторе некоторое число и в последствии этот функтор должен использоваться для передачи в функцию \texttt{std::sort}. 
\item Используйте этот функтор, чтобы отсортировать все числа 
\begin{itemize}
\item по модулю 2 (то есть сначала должны идти все чётные числа, а потом нечётные)
\item по модулю 3 (то есть сначала должны идти числа, делящиеся на 3, потом числа, которые дают остаток 1, при делении на 3, а потом числа, которые дают остаток 2)
\item по последней цифре
\end{itemize}
\item Пусть есть вектор строк. Напишите функтор \texttt{SortByNthLetterFunctor}, который должен будет использоваться для сортировки строк по \texttt{n}-му символу. Например, следующий код должен будет отсортировать вектор строк по символу с индексом 2. 
\begin{lstlisting}
vector<string> vs {"Cat", "Dog", "Axolotl", "Bear"};
sort(vs.begin(), vs.end(), SortByNthLetterFunctor(2));
\end{lstlisting}
\item В файле \texttt{3functor.cpp} есть пример, в котором есть функтор, который используется для хранения чисел, которые делятся на какое-либо число. Напишите аналогичный функтор, который принимает вектор строк и сохраняет в себе все строки, которые начинаются на определённую букву.
\item В файле \texttt{movies.cpp} содержится заготовка кода. Отсортируйте массив \texttt{movies}, используя функторы:
\begin{itemize}
\item по рейтингу
\item по названию
\item по дате
\end{itemize}
\end{itemize}

\section*{Часть 3: Стандартные функторы}

\newpage
\section*{Часть 4: Лямбда-функции}
Ещё одним функциональным объектом является лямбда-функция:
\begin{lstlisting}
#include <iostream>
#include <vector>
#include <algorithm>

int main() {
    std::vector v {18, 51, 2, 25, 14, 97, 73};
    std::sort(v.begin(), v.end(), [](int a, int b) {return a > b;});
    std::for_each(v.begin(), v.end(), [](int a) {std::cout << a << " ";});
}

\end{lstlisting}

\begin{itemize}
\item Используйте лямбда функцию и функцию \texttt{std::transform}, чтобы по вектору \texttt{v} создать вектор, содержащий последнии цифры чисел.
\item В файле \texttt{movies.cpp} содержится заготовка кода. Отсортируйте массив \texttt{movies}, используя лямбда-функции:
\begin{itemize}
\item по рейтингу
\item по названию
\item по дате
\end{itemize}
\item Измените массив \texttt{movies}, с помощью \texttt{std::transform} и лямбда функций, так, чтобы
\begin{itemize}
\item рейтинг каждого фильма уменьшился на 1
\item название каждого фильма было переведено в верхний регистр
\end{itemize}

\item Создайте новый массив, который будет сорежать только фильмы с рейтингом \texttt{8} и выше. Используйте функцию \texttt{copy\_if} и лямбда выражение.

\item Удалите все фильмы из массива, который вышли в 90-е годы. Используйте функцию \texttt{remove\_if} и \texttt{erase} и лямбда выражение.
\end{itemize}


\section*{Часть 5: Стандартные алгоритмы \texttt{STL}, принимающие функции}

\section*{Часть 6: Лямбда-захваты}

\section*{Часть 7: Тип-обёртка \texttt{std::function}}

\section*{Часть 8: \texttt{std::bind}}

\iffalse
\section*{Часть 1: \texttt{std::stack} и \texttt{std::queue}}

\texttt{std::stack} -- это реализация стека в STL (по умолчанию на основе контейнера \texttt{std::deque}).\\

Основные методы для работы со стеком:

\begin{tabular}{ l | l } 
 метод & описание \\ \hline
 \texttt{push(x)}  & Добавляет элемент в стек\\ \\\hline
 \texttt{top()}    & Возвращает последний добавленный элемент.  \\ 
                   & Но не изменяет стек. \\\\ \hline
 \texttt{pop()}    & Удаляет последний добавленный элемент. \\
                   & Но ничего не возвращает\\ \\ \hline
 \texttt{empty()}  & Возвращает \texttt{true} если стек пуст \\ \\\hline
 \texttt{size()}   & Возвращает количество элементов в стеке \\ \hline
\end{tabular}\\
\\

\texttt{std::queue} -- это реализация очереди в STL (по умолчанию на основе контейнера \texttt{std::deque}).\\

Основные методы для работы с очередью:

\begin{tabular}{ l | l }
 метод & описание \\ \hline
 \texttt{push(x)}  & Добавляет элемент в конец очереди\\ \\\hline
 \texttt{front()}  & Возвращает элемент из начала очереди.  \\  \hline
 \texttt{pop()}    & Удаляет элемент из начала очереди. \\
                   & Но ничего не возвращает\\ \\ \hline
 \texttt{empty()}  & Возвращает \texttt{true} если очередь пуста \\ \\\hline
 \texttt{size()}   & Возвращает количество элементов в очереди \\ \hline
\end{tabular}

\subsection*{Задачи}
\begin{itemize}
\item \textbf{Правильная скобочная последовательность} \\
На вход поступает скобочная последовательность, которая может состоять из круглых, фигурных или квадратных скобок. Необходимо выяснить является ли эта скобочная последовательность правильной.
\begin{center}
\begin{tabular}{ l | l }
 вход & выход \\ \hline
 \texttt{[(\{\})]} & \texttt{Yes}  \\ 
 \texttt{({)}} & \texttt{No}  \\ 
 \texttt{((()])} & \texttt{No} \\
\end{tabular}
\end{center}
\end{itemize}
\fi
\end{document}