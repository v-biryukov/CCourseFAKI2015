\documentclass{article}
\usepackage[utf8x]{inputenc}
\usepackage{ucs}
\usepackage{amsmath} 
\usepackage{amsfonts}
\usepackage{upgreek}
\usepackage[english,russian]{babel}
\usepackage{graphicx}
\usepackage{float}
\usepackage{textcomp}
\usepackage{hyperref}
\usepackage{geometry}
  \geometry{left=2cm}
  \geometry{right=1.5cm}
  \geometry{top=1cm}
  \geometry{bottom=2cm}
\usepackage{tikz}
\usepackage{ccaption}



\begin{document}
\pagenumbering{gobble}

\section*{Теория:}
\begin{enumerate}

\item \textbf{Небольшие различия между C и C++}\\
Подключение библиотек языка C в языке C++. Библиотеки \texttt{cmath}, \texttt{cstdlib} и другие. Встроенный тип \texttt{bool}. Указатель \texttt{nullptr}. Разница между \texttt{nullptr} и \texttt{NULL}. Недостатки \texttt{NULL} из языка C. Присваивание указателя одного типа указателю другого типа в языках C и C++. Разница между функцией \texttt{abs} из библиотеки \texttt{math.h} языка C и функцией \texttt{abs} из библиотеки \texttt{cmath} языка C++. Аналогичная разница для других математических функций. Функции с аргументами по умолчанию.

\item \textbf{Пространство имён и ссылки}\\
Пространство имён: что такое и зачем нужно. \texttt{using}-объявление. Анонимное пространство имён. Что такое ссылки. Различие ссылок и указателей. Ссылки на константу. Три типа передачи аргументов в функцию: передача по значению, передача по ссылке и передача по ссылке на константу. Преимущества/недостатки каждого метода. Возвращение ссылки из функции.   



\item  \textbf{Перегрузка функций}\\
Сигнатуры функций в языках C и C++. Перегрузка функций. Манглирование имён. Ключевое слово \texttt{extern "C"}. Правила разрешения перегрузки функций. 


\item \textbf{Перегрузка операторов}\\
Перегрузка операторов в языке C++. Перегрузка арифметических операторов. Перегрузка унарных операторов. Перегрузка операторов как методов класса. Перегрузка оператора присваивания. Перегрузка оператора присваивания сложения. Реализация оператора сложения с помощью оператора присваивания сложения (\texttt{+=}) и других подобных операторов. Перегрузка операторов ввода вывода \verb|<<| и \verb|>>| с \texttt{cin} и \texttt{cout}. Перегрузка оператора взятия индекса. Перегрузка операторов инкремента и декремента. Перегрузка оператора стрелочка (\texttt{->}). Перегрузка операторов \texttt{new} и \texttt{delete}. Перегрузка оператора вызова функции.

\item  \textbf{Классы. Инкапсуляция}\\
Что такое объектно-ориентированное программирование. Основные принципы ООП: инкапсуляция, композиция, наследование и полиморфизм. Классы. Поля и методы класса. Константные методы класса. Модификаторы доступа \texttt{private} и \texttt{public}.  Указатель \texttt{this}.  Различие ключевых слов \texttt{struct} и \texttt{class} в языке C++. Конструкторы и деструкторы. Список инициализации членов класса. Какие поля можно инициализировать с помощью списка инициализации, но нельзя инициализировать обычным образом. Перегрузка конструкторов. Конструктор по умолчанию. Конструктор копирования. Делегирующий конструктор. Ключевое слово \texttt{explicit}. Перегрузка оператора присваивания. Конструкторы и перегруженные операторы, создаваемые по умолчанию.
Друзья. Ключевое слово \texttt{friend}. 

\item  \textbf{Динамическое создание объектов в Куче}\\
Создание экземпляров класса в стеке и куче в языке C++. Использование операторов \texttt{new} и \texttt{delete}. Основные отличия \texttt{new} и \texttt{delete} от \texttt{malloc} и \texttt{free}. Операторы \texttt{new[]} и \texttt{delete[]}. Создание массива объектов в Куче с вызовом конструкторов по умолчанию у каждого объекта. Оператор \texttt{placement new} и ручной вызов деструктора.

\item \textbf{Реализация строки. Строки std::string}\\
Реализация своей строки с выделением памяти в Куче. Методы такой строки: конструктор, принимающий строку в стиле C, конструктор копирования, конструктор по умолчанию, деструктор, оператор присваивания, оператор сложения, оператор присваивания сложения(\texttt{+=}).  Стандартная строка \texttt{std::string}. Преимущества строки \texttt{std::string} по сравнению со строкой в стиле \texttt{C}.


\item \textbf{Шаблоны.}\\
Шаблонные функции. Использование шаблонных функции в языке C++. Шаблоны классов. Инстанцированием шаблона. Вывод шаблонных аргументов функций и классов. Специализация шаблона. Частичная специализация шаблона.

\item \textbf{Реализация динамического массива}\\
Реализация своего шаблонного класса динамического массива. Конструкторы, деструктор, оператор присваивания, итераторы. Использование \texttt{std::initializer\_list} для реализации одного из конструкторов вектора.

\item \textbf{STL. Контейнер \texttt{std::vector}}\\
С помощью какой структуры данных реализован. Как устроен вектор, где и как хранятся данные в векторе. Размер и вместимость вектора, методы \texttt{resize} и \texttt{reserve}. Методы \texttt{push\_back}, \texttt{pop\_back}, \texttt{insert}, \texttt{erase} и их вычислительная сложность. Когда происходит инвалидация итераторов вектора?

\item \textbf{STL. Итераторы}\\
Идея итераторов. В чём преимущество итераторов по сравнению с обычным обходом структур данных. Операции, которые можно производить с итератором. Обход стандартных контейнеров с помощью итераторов. Константные и обратные итераторы. Методы \texttt{begin}, \texttt{end}, \texttt{cbegin}, \texttt{cend} и другие. Итератор \texttt{std::back\_insert\_iterator}. Использование функции \texttt{std::copy} для вставки элементов в контейнер. Итератор \texttt{std::ostream\_iterator}. Функции \texttt{std::advance}, \texttt{std::next} и \texttt{std::distance}. Категории итераторов (Random access, Biderectional, Forward, Output, Input).

\item \textbf{STL. Контейнер \texttt{std::list}}\\
С помощью какой структуры данных реализован. Как устроен список, где и как хранятся данные в списке. Методы списка: \texttt{insert}, \texttt{erase}, \texttt{push\_back}, \texttt{push\_front}, \texttt{pop\_back}, \texttt{pop\_front}. Вычислительная сложность этих операций. Когда происходит инвалидация итераторов списка? Как удаляются элементы списка во время прохода по нему.

\item \textbf{STL. Другие последовательные контейнеры.}\\
Контейнер \texttt{std::array}, как реализован, операций, которые можно с ним провести и их вычислительная сложность. Контейнер \texttt{std::deque}, как реализован, операций, которые можно с ним провести и их вычислительная сложность. Когда происходит инвалидация итераторов deque? Контейнер \texttt{std::valarray}, как реализован, операций, которые можно с ним провести и их вычислительная сложность. Контейнеры адаптеры \texttt{std::stack}, \texttt{std::queue} и \texttt{std::priority\_queue}.

\item \textbf{STL. Упорядоченные ассоциативные контейнеры}\\
Контейнер \texttt{std::set} -- множество. Его основные свойства. С помощью какой структуры данных он реализован. Методы \texttt{insert}, \texttt{erase}, \texttt{find}, \texttt{count}, \texttt{lower\_bound}, \texttt{upper\_bound} и их вычислительная сложность. Можно ли изменить элемент множества? Контейнер \texttt{std::map} -- словарь. Его основные свойства. Методы \texttt{insert}, \texttt{operator[]}, \texttt{erase}, \texttt{find}, \texttt{count}, \texttt{lower\_bound}, \texttt{upper\_bound} и их вычислительная сложность. Как изменить ключ элемента словаря? Контейнеры \texttt{multuset} и \texttt{multimap}. Как удалить из \texttt{multimap} все элементы с данным ключом. Как удалить из \texttt{multimap} только один элемент с данным ключом? Когда происходит инвалидация итераторов множества и словаря? Пользовательский компаратор для упорядоченных ассоциативных контейнеров. 

\item \textbf{STL. Неупорядоченные ассоциативные контейнеры}\\
Контейнер \texttt{std::unordered\_set} -- неупорядоченное множество. Его основные свойства. С помощью какой структуры данных он реализован. Основные методы этого контейнера и их вычислительная сложность. Контейнер \texttt{std::unordered\_map} -- словарь. Его основные свойства и методы и их вычислительная сложность. Как изменить ключ элемента словаря? Контейнеры \texttt{std::unordered\_multiset} и \texttt{std::unordered\_multimap}. Как удалить из \texttt{unordered\_multimap} только один элемент с данным ключом? Когда происходит инвалидация итераторов неупорядоченных множества и словаря.  Пользовательский компаратор и пользовательская хеш-функция для неупорядоченнных ассоциативных контейнеров.

\item \textbf{Инициализация, ключевое слово auto и другое}\\
Инициализация. Default initialization. Value initialization. Direct initialization. Direct list initialization. Copy initialization. Copy list initialization. Ключевое слово \texttt{auto}. Range-based циклы. Пользовательские литералы. \texttt{std::initializer\_list}. Structure bindings. Copy elision. Return value optimization.


\item \textbf{Функциональные объекты}\\
Указатели на функции в алгоритмах STL. Функторы. Стандартные функторы: \texttt{std::less}, \texttt{std::greater}, \texttt{std::equal\_to}, \texttt{std::plus}, \texttt{std::minus}, \texttt{std::multiplies}. Основы лямбда-функций. Стандартные алгоритмы STL, принимающие функциональные объекты. Тип обёртка \texttt{std::function}. Шаблонная функция \texttt{std::bind}.

\item \textbf{Лямбда-функций}\\
Лямбда-функций. Объявление лямбда-функций. Передача их в другие функции. Преимущества лямбда-функций перед указателями на функции и функторами. Использование лямбда функций со стандартными алгоритмами \texttt{std::sort}, \texttt{std::transform}, \texttt{str::copy\_if}. Лямбда-захват. Захват по значению и по ссылке. Захват всех переменных области видимости по значению и по ссылке. Объявление новых переменных внутри захвата.

\newpage
\item \textbf{Раздельная компиляция}\\
Что такое файл исходного кода и исполняемый файл. Этап сборки программы: препроцессинг, ассемблирование, компиляция и линковка. Директивы препроцессора \texttt{\#include} и \texttt{\#define}. Компиляция программы с помощью \texttt{g++}. Header-файлы. Раздельная компиляция. Преимущества раздельной компиляции. Статические библиотеки и их подключение с помощью компилятора \texttt{gcc}. Динамические библиотеки и их подключение. Скрипты \texttt{bash}. \texttt{Make}-файлы.

\item \textbf{Событийно-ориентированное программирование и библиотека SFML}\\
Библиотека SFML. Класс \texttt{sf::RenderWindow}. Системы координат SFML (координаты пикселей, глобальная система координат, локальные системы координат). Методы \texttt{mapPixelToCoords} и \texttt{mapCoordsToPixel}. Основной цикл программы. Двойная буферизация. Понятие событий. Событийно-ориентированное программирование. События SFML: \texttt{Closed}, \texttt{Resized}, \texttt{KeyPressed}, \texttt{KeyReleased}, \texttt{MouseButtonPressed},  \texttt{MouseButtonReleased}, \texttt{MouseMoved}. Очередь событий. Цикл обработки событий.


\item \textbf{Наследование.}\\
Наследование в языке \texttt{C++}. Добавление новых полей и методов в наследуемый класс. Вызов конструкторов наследуемого класса. Модификатор доступа \texttt{protected}. Переопределение методов. Чем отличается переопределение от перегрузки. Вызов переопределённого метода класса родителя. Object slicing. Множественное наследование. Ромбовидное наследование.\\

\item \textbf{Полиморфизм.}\\
Полиморфизм в \texttt{C++}. Указатели на базовый класс, хранящие адрес объекта наследуемого класса.  Виртуальные функции. Таблица виртуальных функций. Ключевые слова \texttt{override} и \texttt{final}. Виртуальный деструктор. Чистые виртуальные функции. Pure virtual call. Абстрактные классы и интерфейсы.

\item \textbf{Умные указатели.}\\
Недостатки обычных указателей. Умный указатель \texttt{std::unique\_ptr}. Шаблонная функция \texttt{std::make\_unique}. Основы move-семантики. Функция \texttt{std::move}. Перемещение объектов типа \texttt{unique\_ptr}. Умный указатель \texttt{std::shared\_ptr}. Работа с таким указателем. Шаблонная функция \texttt{std::make\_shared}. Базовая реализация \texttt{std::shared\_ptr}. Умный указатель \texttt{std::weak\_ptr}.

\item \textbf{Приведение типов}\\
В чём недостатки приведения в стиле \texttt{C}? Оператор \texttt{static\_cast} и в каких случая он используется. Оператор \texttt{reinterpret\_cast} и в каких случая он используется. Оператор \texttt{const\_cast} и в каких случая он используется. Перегрузка оператора приведения. Использование \texttt{static\_cast} для приведения типов и указателей на типы в иерархии наследования. Оператор \texttt{dynamic\_cast} и в каких случая он используется. Что происходит если \texttt{dynamic\_cast} не может привести тип, рассмотрите случай приведения указателей и случай приведения ссылок.

\item \textbf{Классы std::span и std::string\_view}\\
Класс \texttt{std::span}. Строение объектов этого класса, его размер. Конструкторы этого класса. Методы \texttt{first}, \texttt{last}, \texttt{subspan}. В чём преимущество передачи вектора в функцию, принимающую \texttt{std::span} по сравнению с функцией, принимающей \texttt{std::vector<T>\&}. Класс \texttt{std::string\_view}. Строение объектов этого класса, его размер. Конструкторы этого класса. Методы \texttt{remove\_prefix} и \texttt{remove\_suffix}. В чём преимущество передачи \texttt{string\_view} в функцию. Опасность возврата \texttt{span} и \texttt{string\_view} из функции.

\end{enumerate}

\end{document}