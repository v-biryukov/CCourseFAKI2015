\documentclass{article}
\usepackage[utf8x]{inputenc}
\usepackage{ucs}
\usepackage{amsmath} 
\usepackage{amsfonts}
\usepackage{upgreek}
\usepackage[english,russian]{babel}
\usepackage{graphicx}
\usepackage{float}
\usepackage{textcomp}
\usepackage{hyperref}
\usepackage{geometry}
  \geometry{left=2cm}
  \geometry{right=1.5cm}
  \geometry{top=1cm}
  \geometry{bottom=2cm}
\usepackage{tikz}
\usepackage{ccaption}



\begin{document}
\pagenumbering{gobble}

\section*{Теория к КР \#1 (2-й семестр):}
\begin{enumerate}

\item  \textbf{Деревья}\\
Определение графа. Определение структуры данных -- дерево. Описание дерева в языке C. Обход дерева в ширину и глубину. Бинарное дерево. Бинарное дерево поиска. Добавление, удаление и поиск элементов в бинарном дереве поиска. Алгоритмическая сложность этих операций. Сбалансированное дерево. AVL-деревья. Добавление и удаление элемента в AVL-деревья.
\\
\item  \textbf{Хеш-таблица}\\
Хеш-функция. Свойства хорошей хеш-функции. Хеш-таблица. Описание хеш-таблицы в языке C (нарисовать организацию хеш-таблицы в пвмяти). Коллизии. Разрешение коллизий методом цепочек. Расширение хеш-таблицы. Добавление, удаление и поиск элементов в хеш-таблице. Алгоритмическая сложность (в среднем) этих операций.

\item  \textbf{ООП}\\
Что такое объектно-ориентированное программирование. Основные принципы ООП: инкапсуляция, наследование и полиморфизм. Классы. Поля и методы класса. Указатель this. Модификаторы доступа private и public.   Различие ключевых слов struct и class в языке C++. Друзья. Ключевое слово friend. Конструкторы и деструкторы. Список инициализации членов класса. Перегрузка конструкторов. Создание экземпляров класса в стеке и куче в языке C++. Использование операторов new и delete. Основные отличия new и delete от malloc и free.
\\

\item \textbf{Наследование (это было на лекции).}\\
Наследование в языке C++. Добавление новых полей и методов в наследуемый класс. Вызов конструкторов наследуемого класса. Модификатор доступа protected. Указатели на базовый класс, хранящие адрес объекта наследуемого класса. Виртуальные функции. Таблица виртуальных функций. Абстрактные классы и интерфейсы. \\

\item \textbf{Ссылки. Передача аргументов в функцию. Аргументы по умолчанию.}\\
Ссылки. Определение ссылки в языке C++. 3 типа передачи аргументов в функцию. Передача по значению, передача по ссылке и передача по константной ссылке. Преимущества каждого метода. Возвращение ссылки из функции. Перегрузка функций. Функции с аргументами по умолчанию.\\

\item \textbf{Перегрузка операторов}\\
Перегрузка операторов в языке C++. Перегрузка арифметических операторов с использованием дружественных функций. Перегрузка операторов с использованием методов класса. Перегрузка унарных операторов. Перегрузка операторов ввода вывода \verb|<<| и \verb|>>| с cin и cout. Преимущества перегрузки операторов.\\

\item \textbf{Шаблоны. Стандартная библиотека шаблонов(STL).}\\
Шаблонные функции. Использование шаблонных функции в языке C++. Шаблоны классов. Стандартная библиотека шаблонов. STL. Строка std::string. Присваивание, конкатенация и сложение строк. Преимущества и недостатки std::string по сравнению со строками в стиле C(char*). Динамический массив std::vector и его реализация. Размер и вместимость вектора. Использование std::vector. Методы push\_back, reserve и resize. Связный список std::list и его реализация. Множества std::set и std::unordered\_set и их реализация(с помощью какой структуры данных они реализованы). Методы insert, erase и find. Обход стандартных контейнеров. Итераторы. Применение функции std::sort из библиотеки <algorithm>.\\

\item \textbf{Раздельная компиляция}\\
Что такое файл исходного кода и исполняемый файл. Этап компиляции: препроцессинг, компиляция и линковка. Директивы препроцессора \#include и \#define. Компиляция программы с помощью gcc. Опции gcc: -E, -c. Header-файлы. Раздельная компиляция. Преимущества раздельной компиляции. Скрипты bash. Make-файлы. Cmake.

\end{enumerate}

\end{document}