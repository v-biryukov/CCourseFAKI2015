\documentclass{article}
\usepackage[utf8x]{inputenc}
\usepackage{ucs}
\usepackage{amsmath} 
\usepackage{amsfonts}
\usepackage{upgreek}
\usepackage[english,russian]{babel}
\usepackage{graphicx}
\usepackage{float}
\usepackage{textcomp}
\usepackage{hyperref}
\usepackage{geometry}
  \geometry{left=2cm}
  \geometry{right=1.5cm}
  \geometry{top=1cm}
  \geometry{bottom=2cm}
\usepackage{tikz}
\usepackage{ccaption}



\begin{document}
\pagenumbering{gobble}

\section*{Теория к КР \#1 (2-й семестр):}
\begin{enumerate}

\item  \textbf{Ссылки, пространство имён и другое.}\\
Что такое ссылки.  Различие ссылок и указателей. Ссылки на константу. 3 типа передачи аргументов в функцию. Передача по значению, передача по ссылке и передача по ссылке на константу. Преимущества каждого метода. Возвращение ссылки из функции. Перегрузка функций. Функции с аргументами по умолчанию. Пространство имён: что такое и зачем нужно. Строка \texttt{std::string}. Различие строки в стиле \texttt{C} и строки \texttt{std::string}.\\

\item \textbf{Перегрузка операторов}\\
Перегрузка операторов в языке \texttt{C++}. Перегрузка арифметических операторов. Перегрузка операторов с использованием методов класса. Перегрузка унарных операторов. Перегрузка операторов ввода вывода \verb|<<| и \verb|>>| с \texttt{cin} и \texttt{cout}. Преимущества перегрузки операторов.\\

\item  \textbf{ООП: Инкапсуляция}\\
Что такое объектно-ориентированное программирование. Основные принципы ООП: инкапсуляция, наследование и полиморфизм. Классы. Поля и методы класса. Указатель \texttt{this}. Модификаторы доступа \texttt{private} и \texttt{public}.   Различие ключевых слов \texttt{struct} и \texttt{class} в языке \texttt{C++}. Друзья. Ключевое слово \texttt{friend}. Конструкторы и деструкторы. Список инициализации членов класса. Перегрузка конструкторов. Создание экземпляров класса в стеке и куче в языке \texttt{C++}. Использование операторов \texttt{new} и \texttt{delete}. Основные отличия \texttt{new} и \texttt{delete} от \texttt{malloc} и \texttt{free}.
\\

\item \textbf{ООП: Наследование и полиморфизм.}\\
Наследование в языке \texttt{C++}. Добавление новых полей и методов в наследуемый класс. Вызов конструкторов наследуемого класса. Модификатор доступа \texttt{protected}. Указатели на базовый класс, хранящие адрес объекта наследуемого класса. Полиморфизм в \texttt{C++}. Виртуальные функции. Таблица виртуальных функций. Абстрактные классы и интерфейсы. \\


\item \textbf{Шаблоны.}\\
Шаблонные функции. Использование шаблонных функции в языке C++. Шаблоны классов. Стандартная библиотека шаблонов.

\item \textbf{Стандартная библиотека шаблонов(STL).}\\
Динамический массив \texttt{std::vector} и его реализация. Размер и вместимость вектора. Использование \texttt{std::vector}. Методы \texttt{push\_back}, \texttt{reserve} и \texttt{resize}. Связный список \texttt{std::list} и его реализация. Множества \texttt{std::set} и \texttt{std::unordered\_set} и их реализация(с помощью какой структуры данных они реализованы). Методы \texttt{insert}, \texttt{erase} и \texttt{find}. Обход стандартных контейнеров. Итераторы.\\
 
\item \textbf{Раздельная компиляция}\\
Что такое файл исходного кода и исполняемый файл. Этап сборки программы: препроцессинг, ассемблирование, компиляция и линковка. Директивы препроцессора \texttt{\#include} и \texttt{\#define}. Компиляция программы с помощью \texttt{gcc}. Header-файлы. Раздельная компиляция. Преимущества раздельной компиляции. Статические библиотеки и их подключение с помощью компилятора \texttt{gcc}. Динамические библиотеки и их подключение. Скрипты \texttt{bash}. \texttt{Make}-файлы.

\end{enumerate}

\end{document}