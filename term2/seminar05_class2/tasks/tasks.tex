\documentclass{article}
\usepackage[utf8x]{inputenc}
\usepackage{ucs}
\usepackage{amsmath} 
\usepackage{mathtext}
\usepackage{amsfonts}
\usepackage{upgreek}
\usepackage[english,russian]{babel}
\usepackage{graphicx}
\usepackage{float}
\usepackage{textcomp}
\usepackage{hyperref}
\usepackage{geometry}
  \geometry{left=2cm}
  \geometry{right=1.5cm}
  \geometry{top=1cm}
  \geometry{bottom=2cm}
\usepackage{tikz}
\usepackage{ccaption}
\usepackage{multicol}

\usepackage{listings}
%\setlength{\columnsep}{1.5cm}
%\setlength{\columnseprule}{0.2pt}


\begin{document}
\pagenumbering{gobble}

\lstset{
  language=C,                % choose the language of the code
  basicstyle=\linespread{1.1}\ttfamily,
  columns=fixed,
  fontadjust=true,
  basewidth=0.5em,
  keywordstyle=\color{blue}\bfseries,
  commentstyle=\color{gray},
  stringstyle=\ttfamily\color{orange!50!black},
  showstringspaces=false,
  %numbers=false,                   % where to put the line-numbers
  numbersep=5pt,
  numberstyle=\tiny\color{black},
  numberfirstline=true,
  stepnumber=1,                   % the step between two line-numbers.        
  numbersep=10pt,                  % how far the line-numbers are from the code
  backgroundcolor=\color{white},  % choose the background color. You must add \usepackage{color}
  showstringspaces=false,         % underline spaces within strings
  captionpos=b,                   % sets the caption-position to bottom
  breaklines=true,                % sets automatic line breaking
  breakatwhitespace=true,         % sets if automatic breaks should only happen at whitespace
  xleftmargin=.2in,
  extendedchars=\true,
  keepspaces = true,
}
\lstset{literate=%
   *{0}{{{\color{red!20!violet}0}}}1
    {1}{{{\color{red!20!violet}1}}}1
    {2}{{{\color{red!20!violet}2}}}1
    {3}{{{\color{red!20!violet}3}}}1
    {4}{{{\color{red!20!violet}4}}}1
    {5}{{{\color{red!20!violet}5}}}1
    {6}{{{\color{red!20!violet}6}}}1
    {7}{{{\color{red!20!violet}7}}}1
    {8}{{{\color{red!20!violet}8}}}1
    {9}{{{\color{red!20!violet}9}}}1
}


\section*{Задачи:}
\begin{enumerate}
\item \textbf{Ссылки 1:} Написать функцию \texttt{void swap(int\& x, int\& y)}, которая меняет значение двух переменных, используя ссылки. Вызовите эту функцию в функции main().
\item \textbf{header файлы:} Оформите весь код, описывающий класс Complex в виде одного заголовочного(header) файла. Подключите его к вашему .cpp файлу. Используйте стражи включений, чтобы избежать ошибки множественного включения.
\item \textbf{Ссылки 2:} Написать функцию \texttt{void normalize\_complex(Complex\& z)}, которая нормализует комплексное число z.
\item \textbf{Метод класса:} Написать метод класса Complex \texttt{void normalize()}.
\item \textbf{Конструкторы и деструкторы:} добавьте в соответствующие конструктор и деструктор класса Complex вывод на экран сообщений ``Constructor'' и ``Destructor'' соответственно. Создайте экземпляр класса Complex на стеке вызовов в функции main() и запустите программу. Создайте массив из 10-ти элементов класса Complex и запустите программу.
\item \textbf{new/delete:}  Используёте malloc() и free() чтобы выделить память под экземпляр класса Complex в куче. Используйте операторы new и delete, чтобы создать экземпляр класса Complex в куче. Запустите программу. В чём различие между malloc() и new?
\end{enumerate}


\section*{Задачи. Класс Image:}
\begin{enumerate}
\item \textbf{Класс Image:} В папке \texttt{seminar05\_class/code/} лежит код с описанием класса Image для работы с изображениями в формате .ppm. Изучите этот код и скомпилируйте его. Запустите получившуюся программу.
\item \textbf{Flip vertically:} Перепишите программу так, чтобы она загружала изображение и отзрекаливала его по вертикали. Используйте уже написанную функцию \texttt{filp\_vertically()}.
\item \textbf{Sepia:} Добавьте функцию \texttt{void sepia()}, которая добавляет эффект сепии, изменяя цвета следующим образом:
\begin{equation}
\begin{aligned}
  r' = 0.393*r + 0.769*g + 0.189*b \\ 
  g' = 0.349*r + 0.686*g + 0.168*b \\ 
  b' = 0.272*r + 0.534*g + 0.131*b
\end{aligned}
\end{equation}
Примените этот эффект на различные изображения из папки images.
\item \textbf{Новый конструктор:} Напишите новый конструктор \texttt{Image(int n, int m, Pixel color = Pixel(0, 0, 0))}, который будет создавать изображение цвета color и размера n на m.
\item \textbf{Set pixel:} Напишите методы класса \texttt{void set\_pixel(int i, int j, unsigned char r0, unsigned char g0, unsigned char b0)} и \texttt{void set\_pixel(int i, int j, Pixel color)}, которые устанавливают пиксель под координатами i и j в соответствующий цвет.
\item \textbf{Get pixel:} Напишите метод класса \texttt{Pixel get\_pixel(int i, int j)}, которая возвращает цвет пикселя под координатами i и j.
\item \textbf{Белый шум:} Используйте новый конструктор \texttt{Image(int n, int m)} и функцию \texttt{set\_pixel}, чтобы создавать изображение, каждый пиксель которого будет иметь случайный цвет. Случайное число от 0 до 255 можно получить, используя \texttt{rand() \% 256} из библиотеки stdlib.h. Сохраните это изображение.
\item \textbf{Флаг Японии:} Используйте новый конструктор \texttt{Image(int n, int m)} и функцию \texttt{set\_pixel}, чтобы создавать изображение флага Японии. Размеры 600x400 пикселей. Радиус круга -- 100 пикселей.
\newpage
\item \textbf{Задача об убегающей точке:} Предположим, что у нас есть комплексная функция $f(z) = z^2$. Выберем некоторое комплексное число $z_0$ и будем проводить следующие итерации: $z_1 = f(z_0), z_2 = f(z_1), ..., z_{n+1} = f(z_n)$. В зависимости от выбора точки $z_0$ эта последовательность либо разойдётся, либо останется в некоторой ограниченной области. Нужно найти все точки комплексной плоскости, которые не являются убегающими. \\

Для функции $f(z) = z^2$ эта область тривиальна, но всё становится сложней для функции вида $f(z) = z^2 + c$, где $c$ -- некоторое комплексное число. Численно найдите область убегания для функций такого вида. Для этого создайте изображение размера 1000x1000, покрывающую область [-2:2]x[-2:2] на комплексной плоскости. Для каждой точки этой плоскости проведите $N$ итераций и, в зависимости от результата, окрасьте пиксель в соответствующий цвет (цвет можно подобрать самим). Используйте классы Complex и Image\\

Добавьте параметры командной строки: 2 вещественных числа, соответствующие комплексному числу c, и целое число итераций $N$. Программа должна создавать файл \texttt{julia.ppm}.

\item \textbf{Множество:} Зафиксируем теперь $z_0 = 0$ и будем менять $c$. Численно найдите все параметры $c$, для которых точка не является убегающей.

\end{enumerate}

\end{document}