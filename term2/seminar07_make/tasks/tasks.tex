\documentclass{article}
\usepackage[utf8x]{inputenc}
\usepackage{ucs}
\usepackage{amsmath} 
\usepackage{mathtext}
\usepackage{amsfonts}
\usepackage{upgreek}
\usepackage[english,russian]{babel}
\usepackage{graphicx}
\usepackage{float}
\usepackage{textcomp}
\usepackage{hyperref}
\usepackage{geometry}
  \geometry{left=2cm}
  \geometry{right=1.5cm}
  \geometry{top=1cm}
  \geometry{bottom=2cm}
\usepackage{tikz}
\usepackage{ccaption}
\usepackage{multicol}

\usepackage{listings}
%\setlength{\columnsep}{1.5cm}
%\setlength{\columnseprule}{0.2pt}


\begin{document}
\pagenumbering{gobble}

\lstset{
  language=C,                % choose the language of the code
  basicstyle=\linespread{1.1}\ttfamily,
  columns=fixed,
  fontadjust=true,
  basewidth=0.5em,
  keywordstyle=\color{blue}\bfseries,
  commentstyle=\color{gray},
  stringstyle=\ttfamily\color{orange!50!black},
  showstringspaces=false,
  %numbers=false,                   % where to put the line-numbers
  numbersep=5pt,
  numberstyle=\tiny\color{black},
  numberfirstline=true,
  stepnumber=1,                   % the step between two line-numbers.        
  numbersep=10pt,                  % how far the line-numbers are from the code
  backgroundcolor=\color{white},  % choose the background color. You must add \usepackage{color}
  showstringspaces=false,         % underline spaces within strings
  captionpos=b,                   % sets the caption-position to bottom
  breaklines=true,                % sets automatic line breaking
  breakatwhitespace=true,         % sets if automatic breaks should only happen at whitespace
  xleftmargin=.2in,
  extendedchars=\true,
  keepspaces = true,
}
\lstset{literate=%
   *{0}{{{\color{red!20!violet}0}}}1
    {1}{{{\color{red!20!violet}1}}}1
    {2}{{{\color{red!20!violet}2}}}1
    {3}{{{\color{red!20!violet}3}}}1
    {4}{{{\color{red!20!violet}4}}}1
    {5}{{{\color{red!20!violet}5}}}1
    {6}{{{\color{red!20!violet}6}}}1
    {7}{{{\color{red!20!violet}7}}}1
    {8}{{{\color{red!20!violet}8}}}1
    {9}{{{\color{red!20!violet}9}}}1
}


\section*{Задачи:}
\begin{enumerate}
\item \textbf{bash 1:} Написать bash-скрипт, который печатает на экран Hello world. Используйте команду echo.
\item \textbf{bash 2:} Написать bash-скрипт, который создаёт папку test и файл test.txt в котором будет написано ``Testing''. Используйте команды mkdir, touch и cat.
\item \textbf{Раздельная компиляция:} Создайте файлы hello.h и hello.cpp. Эта мини-библиотека должна содержать 1 функцию hello(), которая печатает на экран Hello world. Создайте файл main.cpp и вызовите из этого файла эту функцию. Скомпилируйте оба исполняемых файла отдельными вызовами gcc.  Создайте bash-скрипт, который будет компилировать оба файла, скомпилируйте оба исполняемых файла отдельными вызовами gcc.
\item \textbf{Раздельная компиляция. Make-файл:} Сделайте то же самое, только с помощью make-файла.
\end{enumerate}


\subsection*{Работа с библиотекой SFML:}
Простейший пример:
\begin{lstlisting}
#include <SFML/Graphics.hpp>

int main()
{
    // создаём окно
    sf::RenderWindow window(sf::VideoMode(800, 600), "My window");
    // цикл, который будет работать пока открыто окно
    while (window.isOpen())
    {
        // проверяем все события, связанные с окном, произошедшие со времени прошедшей итерации
        sf::Event event;
        while (window.pollEvent(event))
        {
            // проверяем на закрытие окна
            if (event.type == sf::Event::Closed)
                window.close();
        }
        // очистить окно черным цветом
        window.clear(sf::Color::Black);

        // рисуем тут...
        // window.draw(...);

        // конец текущего кадра
        window.display();
    }
}
\end{lstlisting}

\begin{enumerate}
\item \textbf{Компиляция:} Скомпилируйте программу, привиденную выше. Помните, что на этапе линковки нужно указать опции \texttt{-lsfml-graphics -lsfml-window -lsfml-system}.
\item \textbf{Рисуем круг:} Нарисовать синий круг радиуса 100 в центре экрана.
\begin{lstlisting}
sf::CircleShape shape(50); // создаём экземпляр круга с помощью конструктора
shape.setFillColor(sf::Color(100, 250, 50)); // задаём цвет круга
shape.setPosition(200, 400); // задаём положение круга
window.draw(shape)
\end{lstlisting}
\newpage
\item \textbf{Анимация 1:} Нарисуйте круг радиуса 10, двигающийся с постоянной скоростью. Используёте метод move(dx, dy):
\begin{lstlisting}
shape.move(5, 10); // задаём положение круга
\end{lstlisting}
\item \textbf{Анимация 2:} Нарисуйте круг радиуса 10, двигающийся по окружности.
\item \textbf{События 1:} \\
Пример обработки события нажатия клавиши:
\begin{lstlisting}
if (event.type == sf::Event::KeyPressed)
    if (event.key.code == sf::Keyboard::Escape)
    {
        std::cout << "the escape key was pressed" << std::endl;
    }
\end{lstlisting}
Дописать программу так, чтобы анимация работала только тогда, когда зажата клавиша M.

\item \textbf{События 2:} \\
Пример обработки события нажатия кнопки мыши:
\begin{lstlisting}
if (event.type == sf::Event::MouseButtonPressed)
{
    if (event.mouseButton.button == sf::Mouse::Right)
    {
        std::cout << "the right button was pressed" << std::endl;
        std::cout << "mouse x: " << event.mouseButton.x << std::endl;
        std::cout << "mouse y: " << event.mouseButton.y << std::endl;
    }
}
\end{lstlisting}
Дописать программу так, чтобы при нажатии клавиши мыши кружок перемещался в на место курсора.

\item \textbf{Класс Ball:} \\
Создать класс Ball с полями x, y, vx, vy, radius. В функции main() создать вектор из экземпляров класса Ball. При нажатии на клавишу мыши должен создаваться новый экземпляр класса в соответствующем месте. Скорость задаётся случайным образом(но не делайте её очень большой). Радиус равен 5. Все экземпляры должны правильно отрисововаться.
\item \textbf{Граничные условия} \\
Добавьте стенки, так чтобы шарики не улетали за пределы экрана.
\item \textbf{Задача N тел} \\
Добавьте гравитационное взаимодействие между шариками. Считайте что масса всех шариков равна 1.
\item \textbf{Задача N тел с массой} \\
Добавьте разную массу шарикам. При создании шарика масса должна задаваться случайным образом(но не делайте массу слишком большой либо слишком маленькой!).
\item \textbf{Солнечная система} \\
Смоделируйте солнечную систему в 2D (масштабы можно не соблюдать).
\item \textbf{Создайте игру pong} \\
Смоделируйте солнечную систему в 2D (масштабы можно не соблюдать). Для рисования прямоугольника:
\begin{lstlisting}
sf::RectangleShape rectangle(sf::Vector2f(120, 50));
\end{lstlisting}
\end{enumerate}
\end{document}