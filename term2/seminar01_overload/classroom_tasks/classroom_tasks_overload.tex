\documentclass{article}
\usepackage[utf8x]{inputenc}
\usepackage{ucs}
\usepackage{amsmath} 
\usepackage{amsfonts}
\usepackage{marvosym}
\usepackage{wasysym}
\usepackage{upgreek}
\usepackage[english,russian]{babel}
\usepackage{graphicx}
\usepackage{float}
\usepackage{textcomp}
\usepackage{hyperref}
\usepackage{geometry}
  \geometry{left=2cm}
  \geometry{right=1.5cm}
  \geometry{top=1cm}
  \geometry{bottom=2cm}
\usepackage{tikz}
\usepackage{ccaption}
\usepackage{multicol}
\usepackage{fancyvrb}

\usepackage{listings}
%\setlength{\columnsep}{1.5cm}
%\setlength{\columnseprule}{0.2pt}

\usepackage{colortbl,graphicx,tikz}
\definecolor{X}{rgb}{.5,.5,.5}

\title{ДЗ. Работа с изображениями в формате \texttt{.ppm}}
\date{}
\begin{document}
\pagenumbering{gobble}

\lstset{
  language=C,                % choose the language of the code
  basicstyle=\linespread{1.1}\ttfamily,
  columns=fixed,
  fontadjust=true,
  basewidth=0.5em,
  keywordstyle=\color{blue}\bfseries,
  commentstyle=\color{gray},
  stringstyle=\ttfamily\color{orange!50!black},
  showstringspaces=false,
  %numbers=false,                   % where to put the line-numbers
  numbersep=5pt,
  numberstyle=\tiny\color{black},
  numberfirstline=true,
  stepnumber=1,                   % the step between two line-numbers.        
  numbersep=10pt,                  % how far the line-numbers are from the code
  backgroundcolor=\color{white},  % choose the background color. You must add \usepackage{color}
  showstringspaces=false,         % underline spaces within strings
  captionpos=b,                   % sets the caption-position to bottom
  breaklines=true,                % sets automatic line breaking
  breakatwhitespace=true,         % sets if automatic breaks should only happen at whitespace
  xleftmargin=.2in,
  extendedchars=\true,
  keepspaces = true,
}
\lstset{literate=%
   *{0}{{{\color{red!20!violet}0}}}1
    {1}{{{\color{red!20!violet}1}}}1
    {2}{{{\color{red!20!violet}2}}}1
    {3}{{{\color{red!20!violet}3}}}1
    {4}{{{\color{red!20!violet}4}}}1
    {5}{{{\color{red!20!violet}5}}}1
    {6}{{{\color{red!20!violet}6}}}1
    {7}{{{\color{red!20!violet}7}}}1
    {8}{{{\color{red!20!violet}8}}}1
    {9}{{{\color{red!20!violet}9}}}1
}

\title{Семинар \#1: Введение в язык C++ (для тех, кто знает C). Классные задачи.\vspace{-5ex}}\date{}\maketitle

\section*{Hello world++}
Пишем первую программу на \texttt{C++}
\begin{lstlisting}
#include <cstdio>
int main()
{
	printf("Hello World++\n");
}
\end{lstlisting}
Все библиотеки из языка \texttt{C} можно использовать и в языке \texttt{C++}. Только название библиотеки без \texttt{.h} на конце и с символом \texttt{c} в начале:
\begin{verbatim}
<stdio.h>  ------->  <cstdio>
\end{verbatim}

Для компиляции используйте компилятор \texttt{g++}. Вот так:
\begin{verbatim}
g++ helloworld.cpp
./a.out
\end{verbatim}
Или, если вы хотите установить у исполняемого файла своё имя за место \texttt{a.out}:
\begin{verbatim}
g++ -o hello helloworld.cpp
./hello
\end{verbatim}

\begin{itemize}
\item \textbf{Задача 1:} Скомпилируйте и запустите простейшую программу \texttt{00helloworld.cpp}.
\item \textbf{Задача 2:} Разберитесь в программе \texttt{01structnaming.cpp}. Скомпилируйте и запустите.
\end{itemize}

\section*{Тип bool}
В прошлом семестре, для хранения результатов логических операций, мы использовали целочисленные типы.\\
В языке \texttt{C++} есть встроенный тип \texttt{bool}, который может принимать 2 значения (\texttt{true} и \texttt{false}).
\begin{lstlisting}
#include <cstdio>
int main()
{
	bool a = true;
	bool b = false;
	bool c = a || b;

	if (c)
		printf("Yes\n");
	else
		printf("No\n");
}
\end{lstlisting}
\begin{itemize}
\item \textbf{Задача 3:} Что напечатает эта программа? Скомпилируйте её и запустите (\texttt{02bool.cpp})
\item \textbf{Задача 4:} Напишите функцию, которая будет принимать на вход целое число и возвращать \texttt{true}, если число оканчивается на \texttt{0} и \texttt{false} иначе. Вызовите эту функцию из \texttt{main}.
\end{itemize}

\newpage
\section*{Пространство имён}
\begin{lstlisting}
#include <cstdio>
// Определяем переменные, структуры, функции внутри пространства имён foo
namespace foo
{
	int a = 5;
	int square(int x)
	{
		return x * x;
	}
}
// Чтобы получить доступ к ним вне пространства имён, нужно добавить к именам foo::
int main()
{
	printf("%d\n", foo::a);
}
\end{lstlisting}
\begin{itemize}
\item \textbf{Задача 5:} Возведите \texttt{foo::a} в квадрат с помощью функции \texttt{foo::square}. 
\item \textbf{Задача 6:} Создайте своё пространство имён по имени \texttt{bar} и определите в нём переменную \texttt{a = 7}. Напечатайте значение этой переменной в \texttt{main}.
\item \textbf{Задача 7:} Сделайте задание в программе \texttt{03namespace.cpp}.
\end{itemize}

\section*{Ссылки}
Ссылка -- это переменная, которая является новым именем для существующего участка памяти. 
\begin{lstlisting}
#include <cstdio>

int main()
{
	int a = 10;
	// Создадим ссылку r на переменную a
	int& r = a;
	// Теперь, если изменить r, то поменяется и a
	r += 5;
	printf("%d\n", a);
}
\end{lstlisting}
Ссылки часто используются для тех же целей, что и указатели (только со ссылкам работать удобнее).
В отличии от указателей, ссылки:
\begin{enumerate}
\item Должны всегда инициализироваться при создании
\item Не могут никуда не ссылаться (т.е. не могут равняться \texttt{NULL})
\item Их нельзя переприсвоить. При использовании оператора \texttt{=} со ссылками изменяется та переменная, на которую ссылка ссылается, а не сама ссылка.
\end{enumerate}
\begin{itemize}
\item \textbf{Задача 8:} Сделайте задание в файлах \texttt{04ref.cpp}, \texttt{05ref.cpp} и \texttt{06ref.cpp}.
\end{itemize}
\section*{Перегрузка функций}

\section*{Перегрузка операторов}

\end{document}