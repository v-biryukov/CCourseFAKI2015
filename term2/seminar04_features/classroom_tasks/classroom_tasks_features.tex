\documentclass{article}
\usepackage[utf8x]{inputenc}
\usepackage{ucs}
\usepackage{amsmath} 
\usepackage{amsfonts}
\usepackage{upgreek}
\usepackage[english,russian]{babel}
\usepackage{graphicx}
\usepackage{float}
\usepackage{textcomp}
\usepackage{hyperref}
\usepackage{geometry}
  \geometry{left=2cm}
  \geometry{right=1.5cm}
  \geometry{top=1cm}
  \geometry{bottom=2cm}
\usepackage{tikz}
\usepackage{ccaption}
\usepackage{multicol}

\usepackage{listings}
%\setlength{\columnsep}{1.5cm}
%\setlength{\columnseprule}{0.2pt}


\begin{document}
\pagenumbering{gobble}

\lstset{
  language=C,                % choose the language of the code
  basicstyle=\linespread{1.1}\ttfamily,
  columns=fixed,
  fontadjust=true,
  basewidth=0.5em,
  keywordstyle=\color{blue}\bfseries,
  commentstyle=\color{gray},
  stringstyle=\ttfamily\color{orange!50!black},
  showstringspaces=false,
  %numbers=false,                   % where to put the line-numbers
  numbersep=5pt,
  numberstyle=\tiny\color{black},
  numberfirstline=true,
  stepnumber=1,                   % the step between two line-numbers.        
  numbersep=10pt,                  % how far the line-numbers are from the code
  backgroundcolor=\color{white},  % choose the background color. You must add \usepackage{color}
  showstringspaces=false,         % underline spaces within strings
  captionpos=b,                   % sets the caption-position to bottom
  breaklines=true,                % sets automatic line breaking
  breakatwhitespace=true,         % sets if automatic breaks should only happen at whitespace
  xleftmargin=.2in,
  extendedchars=\true,
  keepspaces = true,
}
\lstset{literate=%
   *{0}{{{\color{red!20!violet}0}}}1
    {1}{{{\color{red!20!violet}1}}}1
    {2}{{{\color{red!20!violet}2}}}1
    {3}{{{\color{red!20!violet}3}}}1
    {4}{{{\color{red!20!violet}4}}}1
    {5}{{{\color{red!20!violet}5}}}1
    {6}{{{\color{red!20!violet}6}}}1
    {7}{{{\color{red!20!violet}7}}}1
    {8}{{{\color{red!20!violet}8}}}1
    {9}{{{\color{red!20!violet}9}}}1
}


\section*{Задачи по теме ''Строки, Шаблоны и STL``}
\subsection*{Часть 1: Строки в C++ (3 балла)}
\begin{lstlisting}
#include <iostream>
#include <string> 
using std::cout;
using std::endl;

int main () 
{
    /* В языке C++ есть библиотека string, которая предоставляет класс строк
     * std::string.
     * В отличии от строк языка C (char*) строки std::string в языке C++
     * гораздо проще и удобней.
     */
    std::string a = "Deus";
    
    // Можно присваивать ( В языке C пришлось бы использовать strcpy )
    std::string b;
    b = "machina";
	
    // Можно складывать ( В языке C пришлось бы использовать strcat )
    std::string c = a + " ex " + b;
    cout << c << endl;
    
    // Можно сравнивать ( В языке C пришлось бы использовать strcmp )
    if (b > a)
        cout << "String b is greater than string a" << endl;
}

\end{lstlisting}

\begin{enumerate}
\item \textbf{String 1:} Создайте строку ''Hello world`` и напечатайте её на экран.
\item \textbf{String 2:} Создайте программу, которая будет считывать слова (используёте cin) в бесконечном цикле и каждый раз печатать сумму всех слов. Например, если пользователь ввёл Hello, то программа должна напечатать Hello и запросить следуещее слово. Если затем пользователь введёт World, то программа должна будет напечатать HelloWorld и запросить следуещее слово.
\end{enumerate}

\newpage
\subsection*{Часть 2: Шаблонные функции и классы (7 баллов)}

\begin{lstlisting}
#include <iostream>
#include <string>
using std::cout;
using std::endl;

template <class T>
T GetMax (T a, T b) 
{
    if (a > b)
    	return a;
    else 
    	return b;
}
int main () 
{
    int a = 5, b = 6;
    cout << GetMax<int>(a, b) << endl;
    
    long long n = 9645634567, m = 7356735634;
    cout << GetMax<long long>(n, m) << endl;
    
    float x = 4.634, y = 534.346;
    cout << GetMax<float>(x, y) << endl;
    
    std::string s1 = "deus", s2 = "machina";
    cout << GetMax<std::string>(s1, s2) << endl;
}
\end{lstlisting}
\begin{enumerate}
\item \textbf{Шаблоны 1:} Написать шаблонную функцию \texttt{T triple(T x)}, которая увеличивает переменную в 3 раза. Проверить её на переменных типа int, float, Complex, std::string.
\begin{lstlisting}
cout << triple<int>(5) << endl;
// Должно напечатать 15
    
std::string s = "Hello"
cout << triple<std::string>(s) << endl;
// Должно напечатать HelloHelloHello
\end{lstlisting}
\item \textbf{Шаблоны 2:} Написать шаблонную функцию \texttt{T sum(T arr[], int count)}, которая возвращает сумму массива переменных. Проверить её на переменных типа int, float, Complex, std::string.
\begin{lstlisting}
int numbers[] = {4, 8, 15, 16, 23, 42};
cout << sum<int>(numbers, 6) << endl;
// Должно напечатать 108
    
std::string words[] = {"Deus", "Ex", "Machina"};
cout << sum<std::string>(words, 3) << endl;
// Должно напечатать DeusExMachina
\end{lstlisting}
\item \textbf{Шаблон комплексного числа:} Измените реализацию комплексного числа по адресу\\ \href{https://github.com/v-biryukov/cs_mipt_faki/tree/master/term2/classes/Complex}{github.com/v-biryukov/cs\_mipt\_faki/tree/master/term2/classes/Complex}, чтобы создать шаблонный класс комплесного числа.
\end{enumerate}

\newpage
\subsection*{Часть 3: std::vector (10 баллов)}
\begin{lstlisting}
#include <iostream>
#include <string>
#include <vector>
#include <algorithm>
using std::cout;
using std::endl;

int main () 
{
    // std::vector - это удобный динамический массив C++
    // Он хранит все элементы в куче и автоматически увеличивается в размерах, если нужно
    std::vector<int> v = {54, 62, 12, 97, 41, 6, 73};
    cout << v[1] << endl;
    // Напечатает 62
    
    // Для добавления новых элементов в вектор используйте push_back   
    v.push_back(44);
    
    // Можно узнать размер вектора и его вместимость(capacity)
    cout << "Size = " << v.size() << "    Capacity = " << v.capacity() << endl;
    
    
    // Для того, чтобы пройтись по всему вектору:
    for (int i = 0; i < v.size(); i++)
    	cout << v[i] << ' ';
    cout << endl;

    // Другой способ - использование итераторов:
    // Итератор это специальный обьект созданный для удобства обхода структуры данных
    // v.begin() - это итератор, который указывает на первый элемент
    // v.end() - это итератор, который указывает на элемент следующий за последним
    // *it - получить сам элемент по итератору
    // В случаи вектора итераторы во многом похожи на обычные указатели
    // Различия проявляются в случаи более сложных структур данных ( например деревьев )
    for (std::vector<int>::iterator it = v.begin(); it != v.end(); ++it)
        std::cout << *it << ' ';
    cout << endl;
    
    // Для сортировки вектора можно использовать стандартную функцию std::sort из
    // библиотеки algorithm
    std::sort(v.begin(), v.end());
    
    // Для поиска элемента в векторе нужно использовать стандартную функцию std::find из
    // библиотеки algorithm. Эта функция вернёт итератор на элемент. Если элемента
    // в векторе нет, то функция вернёт v.end()
    
    if (std::find(v.begin(), v.end(), 55) != v.end())
    	cout << "Element found" << endl;
    else
    	cout << "Element not found" << endl;
}
\end{lstlisting}

\begin{enumerate}
\item \textbf{Размер и вместимость:} Проверьте как работает автоматическое расширение вектора. Для этого создайте пустой вектор и заполните его числами от 0 до 300 (используйте push\_back). При этом на каждом шаге печатайте размер вектора и его вместимость.
\item \textbf{Reserve:} Постоянные расширения вектора могут быть очень трудозатратны. Используйте метод reserve, чтобы расширить вектор до значения 300 перед добавлением элементов. Проверьте как будет меняться размер и вместимость вектора в этом случае.
\item \textbf{Обратить вектор:} Используйте функцию reverse из библиотеки algorithm, чтобы обратить вектор v.
\item \textbf{Вектор в функции:} Написать функцию square\_vec, которая принимает на вход вектор чисел типа int и возводит все числа вектора в квадрат.

\item \textbf{Вектор строк:} Создадим следующий вектор строк:
\begin{lstlisting}
std::vector<std::string> animals = {"Cat", "Dog", "Bison", "Dolphin", "Eagle", "Pony", "Ape", "Lobster", "Monkey", "Cow", "Deer", "Duck", "Rabbit", "Spider", "Wolf", "Turkey", "Lion", "Pig", "Snake", "Shark", "Bird", "Fish", "Chicken", "Horse"};
\end{lstlisting}
\begin{itemize}
\item Напечатайте этот вектор на экран.
\item Отсортируйте этот вектор и напечатайте его.
\item Обратите этот вектор и напечатайте его.
\item Напечатайте только тех животных, которые начинаются на букву S.
\item Написать функцию get\_first\_letter, которая принимает на вход вектор строк и один символ. Эта функция должна должна возвращать вектор строки слов, которые начинаются на соответствующий символ. Для этого внутри функции вы должны создать новый вектор, заполнить его нужными строками и вернуть. Проверить правильность работы функции, вызвав её в функции main и напечатав результат.
\end{itemize}
\item \textbf{Шаблоны + векторы:} Написать шаблонную функцию \texttt{T sum(const std::vector<T>\& vec)}, которая возвращает сумму вектора переменных. Проверить её на переменных типа int, float, Complex, std::string.
\item \textbf{Is exist:} Напишите функцию is\_exists, которая будет проверять есть ли в векторе элемент x.
\end{enumerate}

\newpage
\subsection*{Часть 4: std::set (10 баллов)}
\begin{lstlisting}
#include <iostream>
#include <string>
#include <set>
#include <algorithm>
using std::cout;
using std::endl;

int main () 
{
    // Множество это контейнер для быстрого добавления, удаления и поиска
    // Под капотом это бинарное дерево поиска с балансировкой
    // Соответственно, все операции выполняются за O(log(n))
    std::set<int> s = {54, 62, 12, 97, 41, 6, 73};
    // Доступ по индексу не работает
    // cout << s[1] << endl;
    
    //  Для добавления новых элементов в множество используйте insert
    s.insert(44);
    
    // Для поиска элементов используйте find
    // Функция find возвращает итератор
    // Если элемента в множестве нет, то функция вернёт s.end()
    std::set<int>::iterator it = s.find(20);
    if (it != s.end())
        cout << "Element is found" << endl;
    else
        cout << "Element isn't found" << endl;
		
    // Для удаления используйте функцию erase
    s.erase (s.find(41));

    // Для прохода по множеству используйте итератор
    for (std::set<int>::iterator it = s.begin(); it != s.end(); ++it)
        cout << *it << " ";
    cout << endl;
    // Обратите внимание, что множество set всегда отсортировано
    // Независимо от того как вы ложили туда элементы
    // Это работает так как set является бинарным деревом поиска
    
    // В STL есть другой контейнер std::unordered_set
    // Он работает похоже на std::set, но реализован не с помощью
    // бинарного дерева поиска, а с помощью хеш - таблицы
    // Элементы в нём не отсортированы, но зато все операции выполняются быстрее
    // O(log(N)) для std::set  и  O(1) в среднем для std::unordered_set
}
\end{lstlisting}

\begin{enumerate}
\item \textbf{Множество строк:} Дано следующее множество строк
\begin{lstlisting}
std::set<std::string> animals = {"Cat", "Dog", "Bison", "Dolphin", "Eagle", "Pony", "Ape", "Lobster", "Monkey", "Cow", "Deer", "Duck", "Rabbit", "Spider", "Wolf", "Turkey", "Lion", "Pig", "Snake", "Shark", "Bird", "Fish", "Chicken", "Horse"};
\end{lstlisting}
\begin{itemize}
\item Добавьте в множество новый элемент ``Hippo''.
\item Напечатайте все элементы множества.
\item Удалите строку ``Monkey''
\item Напечатайте только тех животных, которые начинаются на букву S.
\item Измените вашу структуру данных с set на unordered\_set.
\end{itemize}


\item \textbf{Особые числа:} В папке \texttt{generate\_special\_numbers} лежит файл \texttt{special\_numbers.cpp} с особыми числами. Этот набор чисел сгенерирован особым образом. В нем есть одно уникальное число, которое встречается всего 1 раз. Все остальные числа встречаются по 2 раза. Ваша задача - найти это число. Как считывать числа можно посмотреть в файле \texttt{read\_numbers.cpp}. Алгоритм решения этой задачи.
\begin{itemize}
\item Создайте множество.
\item Считывайте числа и проверяйте есть ли такое число в множестве.
\item Если такого числа в множестве нет, то его нужно добавить.
\item Если такое число в множестве есть, то нужно его из множества удалить.
\item В конце в множестве должно остаться только одно уникальное число.
\end{itemize}
Почему для решения этой задачи нужно использовать множество, а не вектор?

\end{enumerate}

\end{document}