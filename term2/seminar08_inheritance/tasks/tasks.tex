\documentclass{article}
\usepackage[utf8x]{inputenc}
\usepackage{ucs}
\usepackage{amsmath} 
\usepackage{mathtext}
\usepackage{amsfonts}
\usepackage{upgreek}
\usepackage[english,russian]{babel}
\usepackage{graphicx}
\usepackage{float}
\usepackage{textcomp}
\usepackage{hyperref}
\usepackage{geometry}
  \geometry{left=2cm}
  \geometry{right=1.5cm}
  \geometry{top=1cm}
  \geometry{bottom=2cm}
\usepackage{tikz}
\usepackage{ccaption}
\usepackage{multicol}

\usepackage{listings}
%\setlength{\columnsep}{1.5cm}
%\setlength{\columnseprule}{0.2pt}


\begin{document}
\pagenumbering{gobble}

\lstset{
  language=C,                % choose the language of the code
  basicstyle=\linespread{1.1}\ttfamily,
  columns=fixed,
  fontadjust=true,
  basewidth=0.5em,
  keywordstyle=\color{blue}\bfseries,
  commentstyle=\color{gray},
  stringstyle=\ttfamily\color{orange!50!black},
  showstringspaces=false,
  %numbers=false,                   % where to put the line-numbers
  numbersep=5pt,
  numberstyle=\tiny\color{black},
  numberfirstline=true,
  stepnumber=1,                   % the step between two line-numbers.        
  numbersep=10pt,                  % how far the line-numbers are from the code
  backgroundcolor=\color{white},  % choose the background color. You must add \usepackage{color}
  showstringspaces=false,         % underline spaces within strings
  captionpos=b,                   % sets the caption-position to bottom
  breaklines=true,                % sets automatic line breaking
  breakatwhitespace=true,         % sets if automatic breaks should only happen at whitespace
  xleftmargin=.2in,
  extendedchars=\true,
  keepspaces = true,
}
\lstset{literate=%
   *{0}{{{\color{red!20!violet}0}}}1
    {1}{{{\color{red!20!violet}1}}}1
    {2}{{{\color{red!20!violet}2}}}1
    {3}{{{\color{red!20!violet}3}}}1
    {4}{{{\color{red!20!violet}4}}}1
    {5}{{{\color{red!20!violet}5}}}1
    {6}{{{\color{red!20!violet}6}}}1
    {7}{{{\color{red!20!violet}7}}}1
    {8}{{{\color{red!20!violet}8}}}1
    {9}{{{\color{red!20!violet}9}}}1
}


\section*{Задачи:}


\subsection*{Работа с библиотекой SFML:}
\begin{enumerate}
\item \textbf{Класс Ball:} \\
Создать класс Ball с полями shape, vx, vy. В функции main() создать вектор из экземпляров класса Ball. При нажатии на клавишу мыши должен создаваться новый экземпляр класса в соответствующем месте. Скорость задаётся случайным образом(но не делайте её очень большой). Радиус равен 5. Все экземпляры должны правильно отрисововаться.
\item \textbf{Граничные условия} \\
Добавьте стенки, так чтобы шарики не улетали за пределы экрана.
\item \textbf{Задача N тел} \\
Добавьте гравитационное взаимодействие между шариками. Считайте что масса всех шариков равна 1.
\item \textbf{Задача N тел с массой} \\
Добавьте разную массу шарикам. При создании шарика масса должна задаваться случайным образом(но не делайте массу слишком большой либо слишком маленькой!).
\end{enumerate}

\subsection*{Arkanoid:}
\begin{enumerate}
\item \textbf{Arkanoid:} В папке arkanoid лежит код простейшей игры. Разберитесь в исходном коде этой игры, скомпилируйте её и запустите.\\
\item \textbf{} Измените размер шарика и количество блоков.\\
\item \textbf{} Измените цвет заднего фона, шарика, лопатки и блоков.\\
\item \textbf{} Сделайте так, чтобы в игре было 3 шарика.\\
\item \textbf{Добавьте параметер:} Измените класс блока так, чтобы некоторые блоки уничтожались с нескольких ударов. Для этого введите новое поле health. В начале игры это поле задаётся случайным образом от 1 до 5 для каждого блока. При каждом попадании шарика по блоку, блок должен терять 1 единицу health. В зависимости от этого параметра, цвет блока должен меняться.\\
\item \textbf{Бонусы:} Создайте абстрактный класс бонуса Bonus с абстрактными методами draw() и apply\_changes(). Создайте различные классы бонусов (бонус который увеличивает размер лопатки, уменьшает размер лопатки, увеличивает/уменьшает скорость шарика, добавляет ``пол'' на короткое время, удваивает количество шариков и другие). В функции main создайте вектор из указателей на абстрактный класс бонус. При уничтожении блока с некоторой вероятностью должен создаваться случайный бонус и падать вниз. Бонус должен применяться когда он касается лопатки. Используйте текст для рисования значка бонуса.
\item \textbf{Уровни:} Добавьте класс Уровень. Каждый экземпляр этого класса будет определять положение всех блоков.
\end{enumerate}
\end{document}



