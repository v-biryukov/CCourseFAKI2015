\documentclass{article}
\usepackage[utf8x]{inputenc}
\usepackage{ucs}
\usepackage{amsmath} 
\usepackage{mathtext}
\usepackage{amsfonts}
\usepackage{upgreek}
\usepackage[english,russian]{babel}
\usepackage{graphicx}
\usepackage{float}
\usepackage{textcomp}
\usepackage{hyperref}
\usepackage{geometry}
  \geometry{left=2cm}
  \geometry{right=1.5cm}
  \geometry{top=1cm}
  \geometry{bottom=2cm}
\usepackage{tikz}
\usepackage{ccaption}
\usepackage{multicol}

\usepackage{listings}
%\setlength{\columnsep}{1.5cm}
%\setlength{\columnseprule}{0.2pt}


\begin{document}
\pagenumbering{gobble}

\lstset{
  language=C,                % choose the language of the code
  basicstyle=\linespread{1.1}\ttfamily,
  columns=fixed,
  fontadjust=true,
  basewidth=0.5em,
  keywordstyle=\color{blue}\bfseries,
  commentstyle=\color{gray},
  stringstyle=\ttfamily\color{orange!50!black},
  showstringspaces=false,
  %numbers=false,                   % where to put the line-numbers
  numbersep=5pt,
  numberstyle=\tiny\color{black},
  numberfirstline=true,
  stepnumber=1,                   % the step between two line-numbers.        
  numbersep=10pt,                  % how far the line-numbers are from the code
  backgroundcolor=\color{white},  % choose the background color. You must add \usepackage{color}
  showstringspaces=false,         % underline spaces within strings
  captionpos=b,                   % sets the caption-position to bottom
  breaklines=true,                % sets automatic line breaking
  breakatwhitespace=true,         % sets if automatic breaks should only happen at whitespace
  xleftmargin=.2in,
  extendedchars=\true,
  keepspaces = true,
}
\lstset{literate=%
   *{0}{{{\color{red!20!violet}0}}}1
    {1}{{{\color{red!20!violet}1}}}1
    {2}{{{\color{red!20!violet}2}}}1
    {3}{{{\color{red!20!violet}3}}}1
    {4}{{{\color{red!20!violet}4}}}1
    {5}{{{\color{red!20!violet}5}}}1
    {6}{{{\color{red!20!violet}6}}}1
    {7}{{{\color{red!20!violet}7}}}1
    {8}{{{\color{red!20!violet}8}}}1
    {9}{{{\color{red!20!violet}9}}}1
}

\begin{multicols}{2}
\begin{lstlisting}[language=C++,basicstyle=\ttfamily,keywordstyle=\color{blue}]
class Monster
{
private:
    float x, y, z;
    int health, is_alive, power;
public:
    void hurt(int damage);
    void heal(int heal_power);
};
void Monster::hurt(int damage)
{
    m->health -= damage;
    if (m->health < 0)
        m->is_alive = 0;
}
void Monster::heal(int heal_power)
{
    m->health += heal_power;
}
\end{lstlisting}
\begin{lstlisting}[language=C++,basicstyle=\ttfamily,keywordstyle=\color{blue}]
namespace Foo
{
   int a;
   int f(int x)
   {
       std::cout << "Did something";
   }
   namespace Bar
   {
       float b;
       int g(int x)
       {
           std::cout << "Did other thing";
       }
   }
}
Foo::a = 12345;
Foo::Bar::g();
\end{lstlisting}
\end{multicols}
\section*{Задачи:}
\begin{enumerate}
\item \textbf{Hello world C++} Написать программу HelloWorld используя библиотеку iostream. Скомпилировать с помощью g++.
\item \textbf{Namespaces} Создать пространство имён.

\end{enumerate}


\end{document}