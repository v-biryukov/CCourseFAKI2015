\documentclass{article}
\usepackage[utf8x]{inputenc}
\usepackage{ucs}
\usepackage{amsmath} 
\usepackage{amsfonts}
\usepackage{upgreek}
\usepackage[english,russian]{babel}
\usepackage{graphicx}
\usepackage{float}
\usepackage{textcomp}
\usepackage{hyperref}
\usepackage{geometry}
  \geometry{left=2cm}
  \geometry{right=1.5cm}
  \geometry{top=1cm}
  \geometry{bottom=2cm}
\usepackage{tikz}
\usepackage{ccaption}
\usepackage{multicol}

\usepackage{listings}
%\setlength{\columnsep}{1.5cm}
%\setlength{\columnseprule}{0.2pt}


\begin{document}
\pagenumbering{gobble}

\lstset{
  language=C++,                % choose the language of the code
  basicstyle=\linespread{1.1}\ttfamily,
  columns=fixed,
  fontadjust=true,
  basewidth=0.5em,
  keywordstyle=\color{blue}\bfseries,
  commentstyle=\color{gray},
  stringstyle=\ttfamily\color{orange!50!black},
  showstringspaces=false,
  %numbers=false,                   % where to put the line-numbers
  numbersep=5pt,
  numberstyle=\tiny\color{black},
  numberfirstline=true,
  stepnumber=1,                   % the step between two line-numbers.        
  numbersep=10pt,                  % how far the line-numbers are from the code
  backgroundcolor=\color{white},  % choose the background color. You must add \usepackage{color}
  showstringspaces=false,         % underline spaces within strings
  captionpos=b,                   % sets the caption-position to bottom
  breaklines=true,                % sets automatic line breaking
  breakatwhitespace=true,         % sets if automatic breaks should only happen at whitespace
  xleftmargin=.2in,
  extendedchars=\true,
  keepspaces = true,
}
\lstset{literate=%
   *{0}{{{\color{red!20!violet}0}}}1
    {1}{{{\color{red!20!violet}1}}}1
    {2}{{{\color{red!20!violet}2}}}1
    {3}{{{\color{red!20!violet}3}}}1
    {4}{{{\color{red!20!violet}4}}}1
    {5}{{{\color{red!20!violet}5}}}1
    {6}{{{\color{red!20!violet}6}}}1
    {7}{{{\color{red!20!violet}7}}}1
    {8}{{{\color{red!20!violet}8}}}1
    {9}{{{\color{red!20!violet}9}}}1
}

\title{Задача: Дерево навыков. \vspace{-5ex}}\date{}\maketitle

\section*{Описание реализации дерева навыков}
В папке \texttt{code} находится реализация простого дерева навыков.
\begin{itemize}
\item \texttt{Node} -- абстрактный класс, который описывает узел дерева навыков. Узел может быть в трёх состояниях
\begin{itemize}
\item \texttt{Blocked} -- заблокированный узел. При нажатии на такой узел ничего не должно происходить.
\item \texttt{Unblocked} -- разблокированный узел. При нажатии на такой узел он активируется. При активации узла, все его дети разблокируются.
\item \texttt{Activated} -- активированный узел.
\end{itemize}
\item \texttt{HitNode} -- абстрактный класс, который описывает круглый переключаемый узел дерева навыков. При нажатии на этот узел, если он разблокирован, узел активируется. При повторном нажатии на этот узел, он деактивируется и все его потомки становятся заблокированными.

\item \texttt{ShieldSkillNode}, \texttt{HandSkillNode} и другие -- конкретные классы узлов.
\end{itemize}



\section*{Подзадачи}

\begin{enumerate}
\item \textbf{Новый навык} -- добавьте новый навык -- огненный шар (файл изображения \texttt{icon\_fireball.png}). Добавьте этот навык в дерево навыков.

\item \textbf{Накопительный навык} -- создайте новый узел \texttt{AccumulativeNode}, наследник класса \texttt{Node}. В отличии от узла \texttt{HitNode} в этот узел можно вкладывать несколько очков навыка.  При нажатии на этот навык, если он разблокирован, он становится активированным, а все его дети разблокируются. При повторном нажатии на этот левой кнопкой мыши у этого узла увеличивается уровень активации до некоторого максимального значения. При нажатии правой кнопкой мыши уровень ативации должен уменьшаться на 1. Если уровень достигнет нуля, то узел перестаёт быть активированным, а все его потомки становятся заблокированными.

\begin{center}
\includegraphics[scale=0.3]{images/freeze.png}
\end{center}

Визуально этот узел должен быть квадратным (учтите это при определении столкновения при нажатии кнопки мыши). Внизу этого узла должен быть написан уровень активации в виде \texttt{a / b}, где \texttt{a} -- текущий уровень активации, а \texttt{b} -- максимальный уровень активации.

\item \textbf{Наследники накопительного навыка} Создайте 3 класса наследника \texttt{AccumulativeNode}. Изображения для этих навыков можно найти в папке \texttt{icons} (\texttt{icon\_rect\_sword.png}, \texttt{icon\_rect\_freeze.png} и\\ \texttt{icon\_rect\_chain.png}). Добавьте эти узлы в дерево навыков.


\item \textbf{Абстрактное дерево навыков} -- создайте абстрактный класс дерева навыков \texttt{SkillTree}. Этот класс должен описывать всё дерево навыков, а не один узел. Соответственно, объекты этого класса должны содержать указатель на корень дерева. Также они должны хранить количество свободных очков навыка. При активации какого-либо навыка количество свободных очков навыка должно уменьшаться на 1. При деактивации навыка, все свободные очки должны возвращаться. Если свободных очков нет, то получить новый навык нельзя. Количество свободных очков навыка должно также отображаться на экране.

\item \textbf{Деревья навыков} Создайте классы \texttt{MageSkillTree}, \texttt{WarriorSkillTree} и\texttt{RogueSkillTree}, наследники абстрактного класса \texttt{SkillTree}. Эти классы должны создавать деревья навыков для персонажей войнов, магов и разбойников соответственно. Все деревья должны содержать как узлы типа \texttt{HitNode} так и узлы типа \texttt{AccumulativeNode}.

\end{enumerate}

\end{document}