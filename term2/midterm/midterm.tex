\documentclass{article}
\usepackage[utf8x]{inputenc}
\usepackage{ucs}
\usepackage{amsmath} 
\usepackage{amsfonts}
\usepackage{upgreek}
\usepackage[english,russian]{babel}
\usepackage{graphicx}
\usepackage{float}
\usepackage{textcomp}
\usepackage{hyperref}
\usepackage{geometry}
  \geometry{left=2cm}
  \geometry{right=1.5cm}
  \geometry{top=1cm}
  \geometry{bottom=2cm}
\usepackage{tikz}
\usepackage{ccaption}
\usepackage{multicol}

\usepackage{listings}
%\setlength{\columnsep}{1.5cm}
%\setlength{\columnseprule}{0.2pt}


\begin{document}
\pagenumbering{gobble}

\lstset{
  language=C,                % choose the language of the code
  basicstyle=\linespread{1.1}\ttfamily,
  columns=fixed,
  fontadjust=true,
  basewidth=0.5em,
  keywordstyle=\color{blue}\bfseries,
  commentstyle=\color{gray},
  stringstyle=\ttfamily\color{orange!50!black},
  showstringspaces=false,
  %numbers=false,                   % where to put the line-numbers
  numbersep=5pt,
  numberstyle=\tiny\color{black},
  numberfirstline=true,
  stepnumber=1,                   % the step between two line-numbers.        
  numbersep=10pt,                  % how far the line-numbers are from the code
  backgroundcolor=\color{white},  % choose the background color. You must add \usepackage{color}
  showstringspaces=false,         % underline spaces within strings
  captionpos=b,                   % sets the caption-position to bottom
  breaklines=true,                % sets automatic line breaking
  breakatwhitespace=true,         % sets if automatic breaks should only happen at whitespace
  xleftmargin=.2in,
  extendedchars=\true,
  keepspaces = true,
}
\lstset{literate=%
   *{0}{{{\color{red!20!violet}0}}}1
    {1}{{{\color{red!20!violet}1}}}1
    {2}{{{\color{red!20!violet}2}}}1
    {3}{{{\color{red!20!violet}3}}}1
    {4}{{{\color{red!20!violet}4}}}1
    {5}{{{\color{red!20!violet}5}}}1
    {6}{{{\color{red!20!violet}6}}}1
    {7}{{{\color{red!20!violet}7}}}1
    {8}{{{\color{red!20!violet}8}}}1
    {9}{{{\color{red!20!violet}9}}}1
}

\begin{itemize}
\item \textbf{Figures:} В папке \texttt{figures} лежит простая программа, демонстрирующий работу с простыми классами SFML.
\end{itemize}


\subsection*{Arkanoid:}
\begin{enumerate}
\item В папке \texttt{arkanoid} лежит код простейшей игры. Разберитесь в исходном коде этой игры, скомпилируйте её и запустите.\\
Для компиляции используйте следующую команду:
\begin{verbatim}
g++ main.cpp -lsfml-graphics -lsfml-window -lsfml-system
\end{verbatim}
Исходный код игры состоит из 2-х файлов. В файле \texttt{arcanoid.h} содержатся константы, описание классов и функции для работы с этими классами. В файле \texttt{main.cpp} эти классы используются. Скомпилируйте эту программу и запустите.
\item \textbf{} Измените цвет заднего фона, шарика, лопатки и блоков.\\
\item \textbf{} Измените размер шарика и количество блоков.\\
\item \textbf{} Сделайте так, чтобы цвет шарика менялся со временем случайным образом. Случайное число можно получить так:\\
\begin{lstlisting}
#include <cstdlib> 
#include <ctime> 
...
srand(time(0)); 
rand() % 256; 
\end{lstlisting}
\item \textbf{} Сделайте так, чтобы цвет блоков был разным.\\
\item \textbf{} Сделайте так, чтобы в игре было 3 шарика.\\
\item \textbf{Добавьте параметер:} Измените класс блока так, чтобы некоторые блоки уничтожались с нескольких ударов. Для этого введите новое поле health. В начале игры это поле задаётся случайным образом от 1 до 5 для каждого блока. При каждом попадании шарика по блоку, блок должен терять 1 единицу health. В зависимости от этого параметра, цвет блока должен меняться (он должен тускнеть).\\
\item \textbf{Бонусы:} Сделайте так, чтобы при разрушении блока с некоторой вероятностью появлялся бы бонус, который бы начинал падать вниз. При столкновении с ракеткой этот бонус должен сделать ракетку в 2 раза шире.
\item \textbf{Типы бонусов:} 
Создайте различные типу бонусов (бонус который увеличивает размер лопатки, уменьшает размер лопатки, увеличивает/уменьшает скорость шарика, добавляет ``пол'' на короткое время, утраивает количество шариков, делает шарик ``огненым'' и способным пробивать блоки насквозь и другие). Значки бонусов должны быть различны и красивыми.
\item \textbf{Непробиваемый блок:} Создайте новый вид блока - непробиваемый блок. Этот блок не должен уничтожаться от столкновений шарика.
\item \textbf{Блок-динамит:} Создайте новый вид блока - динамит. При столкновении с шариком этот блок должен взрываться и уничтожать все блоки в округе.
\item \textbf{Уровни:} Добавьте класс Уровень. Каждый экземпляр этого класса будет определять положение всех блоков. Создайте несколько уровней в игре. При прохождении всех уровней должен появляться победный экран.
\end{enumerate}

\newpage
\subsection*{Своя игра:}
Создайте свою копию классической игры. Варианты:
\begin{itemize}
\item Тетрис
\item Сапер
\item Пакмен
\end{itemize}
\end{document}



