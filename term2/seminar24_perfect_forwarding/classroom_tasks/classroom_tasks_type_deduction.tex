\documentclass{article}
\usepackage[english,russian]{babel}
\usepackage{textcomp}
\usepackage{geometry}
  \geometry{left=2cm}
  \geometry{right=1.5cm}
  \geometry{top=1.5cm}
  \geometry{bottom=2cm}
\usepackage{tikz}
\usepackage{multicol}
\usepackage{listings}
\pagenumbering{gobble}

\lstdefinestyle{csMiptCppStyle}{
  language=C++,
  basicstyle=\linespread{1.1}\ttfamily,
  columns=fixed,
  fontadjust=true,
  basewidth=0.5em,
  keywordstyle=\color{blue}\bfseries,
  commentstyle=\color{gray},
  texcl=true,
  stringstyle=\ttfamily\color{orange!50!black},
  showstringspaces=false,
  %numbers=false,
  numbersep=5pt,
  numberstyle=\tiny\color{black},
  numberfirstline=true,
  stepnumber=1,      
  numbersep=10pt,
  backgroundcolor=\color{white},
  showstringspaces=false,
  captionpos=b,
  breaklines=true
  breakatwhitespace=true,
  xleftmargin=.2in,
  extendedchars=\true,
  keepspaces = true,
  tabsize=4,
  upquote=true,
}
\lstdefinestyle{csMiptCppBorderStyle}{
  style=csMiptCppStyle,
  framexleftmargin=5mm, 
  frame=shadowbox, 
  rulesepcolor=\color{gray}
}

\lstset{style=csMiptCppStyle}
\lstset{literate={~}{{\raisebox{0.5ex}{\texttildelow}}}{1}}


\begin{document}
\title{Семинар \#1: Вывод типов и идеальная передача \vspace{-5ex}}\date{}\maketitle

\subsection*{Функция для печати типа}

\section*{Вывод типов в шаблонах и при использовании \texttt{auto}}


\subsection*{Вывод типов в шаблонах при передаче по значению}

\subsection*{Вывод типов в шаблонах при передаче по ссылке/константной ссылке}

\subsection*{Вывод типов при использовании \texttt{auto}}

\subsection*{\texttt{std::initializer\_list}}


\section*{Вывод типов при использовании \texttt{decltype}}
\subsection*{Правила decltype}
\subsection*{Вывод возвращаемого значения}
\subsection*{\texttt{decltype(auto)}}
\subsection*{Пример с шаблонной функцией, принимающей контейнер}



\section*{Идеальная передача и универсальные ссылки}
\subsection*{Напоминание про lvalue и rvalue ссылки}
\subsection*{Универсальные ссылки. Свёртка ссылок.}
\subsection*{Идеальная передача. Функция \texttt{std::forward}.}
\subsection*{Идеальная передача и вариативные шаблоны}
\subsection*{Примеры использования идеальной передачи \texttt{emplace\_back} и \texttt{make\_unique}}


\section*{Вывод аргументов шаблонного класса (CTAD)}
\subsection*{Руководства вывода (deduction guides)}

\end{document}
