\documentclass{article}
\usepackage[english,russian]{babel}
\usepackage{textcomp}
\usepackage{geometry}
  \geometry{left=2cm}
  \geometry{right=1.5cm}
  \geometry{top=1.5cm}
  \geometry{bottom=2cm}
\usepackage{tikz}
\usepackage{listings}

\begin{document}
\pagenumbering{gobble}

\lstset{
  language=C++,
  basicstyle=\linespread{1.1}\ttfamily,
  columns=fixed,
  fontadjust=true,
  basewidth=0.5em,
  keywordstyle=\color{blue}\bfseries,
  commentstyle=\color{gray},
  texcl=true,
  stringstyle=\ttfamily\color{orange!50!black},
  showstringspaces=false,
  %numbers=false,
  numbersep=5pt,
  numberstyle=\tiny\color{black},
  numberfirstline=true,
  stepnumber=1,      
  numbersep=10pt,
  backgroundcolor=\color{white},
  showstringspaces=false,
  captionpos=b,
  breaklines=true
  breakatwhitespace=true,
  xleftmargin=.2in,
  extendedchars=\true,
  keepspaces = true,
  tabsize=4,
  upquote=true,
}
\lstset{ literate={~}{{\raisebox{0.5ex}{\texttildelow}}}{1}}

\renewcommand{\thesubsection}{\arabic{subsection}}
\makeatletter
\def\@seccntformat#1{\@ifundefined{#1@cntformat}%
   {\csname the#1\endcsname\quad}%    default
   {\csname #1@cntformat\endcsname}}% enable individual control
\newcommand\section@cntformat{}     % section level 
\newcommand\subsection@cntformat{Задача \thesubsection.\space} % subsection level
\newcommand\subsubsection@cntformat{\thesubsubsection.\space} % subsubsection level
\makeatother

\title{Семинар \#1: Основы C++. Домашнее задание. \\[1ex] \large Пространства имён, ссылки, перегрузка функций, \texttt{std::string} и \texttt{std::vector}. \vspace{-5ex}}\date{}\maketitle

\subsection{Пространства имён}
Пусть есть такой участок кода:
\begin{lstlisting}
namespace mipt
{
	namespace fefm
	{
		struct Point
		{
			int x, y;
		};
	}
	
	namespace frtk
	{
		void print(fefm::Point p)
		{
			std::cout << p.x << " " << p.y << std::endl;
		}
	}
}
\end{lstlisting}
Вам нужно сделать следующее:
\begin{itemize}
\item В функции \texttt{main} cоздать переменную типа \texttt{Point} (из пространства имён \texttt{mipt::fefm}) и инициализировать её поля значениями \texttt{x = 10} и \texttt{y = 20}.
\item Вызвать функцию \texttt{print} из пространства имён \texttt{mipt::frtk}, передав её созданную структуру.
\end{itemize}
Решите эту задачу тремя способами:
\begin{enumerate}
\item Без использования ключевого слова \texttt{using}.
\item С использованием директив \texttt{using namespace}.
\item С использованием \texttt{using}-объявлений.
\end{enumerate}

\subsection{Куб}
Напишите функцию \texttt{cube}, которая будет принимать одно число типа \texttt{int} по ссылке и возводить это число в куб. Вызовите эту функцию из функции \texttt{main}, чтобы возвести переменную типа \texttt{int} в куб..
\begin{lstlisting}
#include <iostream>
// Тут нужно написать функцию cube

int main()
{
	int a = 5;
	cube(a);
	std::cout << a << std::endl;  // Должно напечатать 125
}
\end{lstlisting}


\subsection{Обмен}
Напишите функцию \texttt{swap}, которая будет обменивать значения двух переменных типа \texttt{int}.
\begin{lstlisting}
#include <iostream>
// Тут нужно написать функцию swap

int main()
{
	int a = 10;
	int b = 20;
	std::cout << a << " " << b << std::endl;  // Должно напечатать 10 20
	
	swap(a, b);
	std::cout << a << " " << b << std::endl;  // Должно напечатать 20 10
}
\end{lstlisting}


\subsection{Изменение структуры}
Пусть у нас есть следующая структура:
\begin{lstlisting}
struct Book
{
    std::string title;
    int pages;
    float price;
};
\end{lstlisting}
Напишите функцию \texttt{addPrice}, которая будет принимать на вход структуру \texttt{Book} по ссылке и некоторое число \texttt{x} типа \texttt{float}. Эта функция должна увеличивать цену переданной книги на \texttt{x}.


\subsection{Передача структуры по константной ссылке}
Пусть у нас есть следующая структура:
\begin{lstlisting}
struct Book
{
    std::string title;
    int pages;
    float price;
};
\end{lstlisting}

Напишите функцию \texttt{isExpensive}, которая будет принимать на вход структуру \texttt{Book} по константной ссылке. Эта функция должна возвращать значение типа \texttt{bool}. Если цена книги больше чем 1000, то функция должна вернуть \texttt{true}, иначе функция должна вернуть \texttt{false}. Протестируйте эту функцию в функции \texttt{main}.



\subsection{Подсчёт символов}
Напишите функцию \texttt{void countLetters(const std::string\& str, int\& numLetters, int\& numDigits, int\& numOther)}, которая будет принимать на вход строку \texttt{str} и подсчитывать число букв и цифр в этой строке. Количество букв нужно записать в переменную \texttt{numLetters}, количество цифр -- в переменную \texttt{numDigits}, а количество остальных символов -- в переменную \texttt{numOther}. Вызвать эту функцию из функции \texttt{main}.


\newpage
\section*{Необязательные задачи (не входят в ДЗ, никак не учитываются)}
\setcounter{subsection}{0}

\end{document}