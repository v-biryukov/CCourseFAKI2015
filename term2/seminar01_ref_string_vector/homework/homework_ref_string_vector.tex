\documentclass{article}
\usepackage[utf8x]{inputenc}
\usepackage{ucs}
\usepackage{amsmath} 
\usepackage{amsfonts}
\usepackage{upgreek}
\usepackage[english,russian]{babel}
\usepackage{graphicx}
\usepackage{float}
\usepackage{textcomp}
\usepackage{hyperref}
\usepackage{geometry}
  \geometry{left=2cm}
  \geometry{right=1.5cm}
  \geometry{top=1.5cm}
  \geometry{bottom=2cm}
\usepackage{tikz}
\usepackage{ccaption}
\usepackage{multicol}

\usepackage{listings}
%\setlength{\columnsep}{1.5cm}
%\setlength{\columnseprule}{0.2pt}


\begin{document}
\pagenumbering{gobble}

\lstset{
  language=C++,
  basicstyle=\linespread{1.1}\ttfamily,
  columns=fixed,
  fontadjust=true,
  basewidth=0.5em,
  keywordstyle=\color{blue}\bfseries,
  commentstyle=\color{gray},
  texcl=true,
  stringstyle=\ttfamily\color{orange!50!black},
  showstringspaces=false,
  %numbers=false,
  numbersep=5pt,
  numberstyle=\tiny\color{black},
  numberfirstline=true,
  stepnumber=1,      
  numbersep=10pt,
  backgroundcolor=\color{white},
  showstringspaces=false,
  captionpos=b,
  breaklines=true
  breakatwhitespace=true,
  xleftmargin=.2in,
  extendedchars=\true,
  keepspaces = true,
  tabsize=4,
  upquote=true,
}
\lstset{ literate={~}{{\raisebox{0.5ex}{\texttildelow}}}{1} }

\title{Семинар \#1: Основы C++. Домашнее задание. \\[1ex] \large Пространства имён, ссылки, перегрузка функций, \texttt{std::string} и \texttt{std::vector}. \vspace{-5ex}}\date{}\maketitle
\subsection*{Задача 1: Пространство имён}
Создайте пространство имён по имени \texttt{myspace}. В этом пространстве имён создайте функцию \\
 \texttt{void printNTimes(const char* str, int n = 10)}, которая будет печатать строку \texttt{str} \texttt{n} раз. Для печати используйте \texttt{std::cout} из библиотеки \texttt{iostream}. Вызовите эту функцию из \texttt{main}.



\subsection*{Задача 2: Куб}
Напишите функцию \texttt{cubeV}, которая будет принимать одно число типа \texttt{int} и возвращать куб этого числа. Вызовите эту функцию из функции \texttt{main}, чтобы возвести переменную типа \texttt{int} в куб.

\subsection*{Задача 3: Куб по ссылке}
Напишите функцию \texttt{cubeR}, которая будет принимать одно число типа \texttt{int} по ссылке и возводить это число в куб. Вызовите эту функцию из функции \texttt{main}, чтобы возвести переменную типа \texttt{int} в куб..


\subsection*{Задача 4: Подсчёт символов}
Напишите функцию \texttt{void countLetters(const char* str, int\& numLetters, int\& numDigits, int\& numOther)}, которая будет принимать на вход строку \texttt{str} и подсчитывать число букв и цифр в этой строке. Количество букв нужно записать в переменную \texttt{numLetters}, количество цифр -- в переменную \texttt{numDigits}, а количество остальных символов -- в переменную \texttt{numOther}. Вызвать эту функцию из функции \texttt{main}.


\subsection*{Задача 5: Изменение структуры}
Пусть у нас есть следующая структура:
\begin{lstlisting}
struct Book
{
    char title[100];
    int pages;
    float price;
};
\end{lstlisting}

Напишите функцию \texttt{addPrice}, которая будет принимать на вход структуру \texttt{Book} по ссылке и некоторое число \texttt{x} типа \texttt{float}. Эта функция должна увеличивать цену переданной книги на \texttt{x}.


\subsection*{Задача 6: Передача структуры по константной ссылке}
Пусть у нас есть следующая структура:
\begin{lstlisting}
struct Book
{
    char title[100];
    int pages;
    float price;
};
\end{lstlisting}

Напишите функцию \texttt{isExpensive}, которая будет принимать на вход структуру \texttt{Book} по константной ссылке. Эта функция должна возвращать значение типа \texttt{bool}. Если цена книги больше чем 1000, то функция должна вернуть \texttt{true}, иначе функция должна вернуть \texttt{false}. Протестируйте эту функцию в функции \texttt{main}.



\end{document}