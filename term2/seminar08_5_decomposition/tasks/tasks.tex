\documentclass{article}
\usepackage[utf8x]{inputenc}
\usepackage{ucs}
\usepackage{amsmath} 
\usepackage{amsfonts}
\usepackage{upgreek}
\usepackage[english,russian]{babel}
\usepackage{graphicx}
\usepackage{float}
\usepackage{textcomp}
\usepackage{hyperref}
\usepackage{geometry}
  \geometry{left=2cm}
  \geometry{right=1.5cm}
  \geometry{top=1cm}
  \geometry{bottom=2cm}
\usepackage{tikz}
\usepackage{ccaption}
\usepackage{multicol}

\usepackage{listings}
%\setlength{\columnsep}{1.5cm}
%\setlength{\columnseprule}{0.2pt}


\begin{document}
\pagenumbering{gobble}

\lstset{
  language=C,                % choose the language of the code
  basicstyle=\linespread{1.1}\ttfamily,
  columns=fixed,
  fontadjust=true,
  basewidth=0.5em,
  keywordstyle=\color{blue}\bfseries,
  commentstyle=\color{gray},
  stringstyle=\ttfamily\color{orange!50!black},
  showstringspaces=false,
  %numbers=false,                   % where to put the line-numbers
  numbersep=5pt,
  numberstyle=\tiny\color{black},
  numberfirstline=true,
  stepnumber=1,                   % the step between two line-numbers.        
  numbersep=10pt,                  % how far the line-numbers are from the code
  backgroundcolor=\color{white},  % choose the background color. You must add \usepackage{color}
  showstringspaces=false,         % underline spaces within strings
  captionpos=b,                   % sets the caption-position to bottom
  breaklines=true,                % sets automatic line breaking
  breakatwhitespace=true,         % sets if automatic breaks should only happen at whitespace
  xleftmargin=.2in,
  extendedchars=\true,
  keepspaces = true,
}
\lstset{literate=%
   *{0}{{{\color{red!20!violet}0}}}1
    {1}{{{\color{red!20!violet}1}}}1
    {2}{{{\color{red!20!violet}2}}}1
    {3}{{{\color{red!20!violet}3}}}1
    {4}{{{\color{red!20!violet}4}}}1
    {5}{{{\color{red!20!violet}5}}}1
    {6}{{{\color{red!20!violet}6}}}1
    {7}{{{\color{red!20!violet}7}}}1
    {8}{{{\color{red!20!violet}8}}}1
    {9}{{{\color{red!20!violet}9}}}1
}


\section*{Задачи:}

\subsection*{Текстуры SFML:}
\begin{enumerate}
\item \textbf{Отображение текстуры:} Перейдите в папку \texttt{code/textures} измените файл \texttt{textures.cpp} так, чтобы он отображал текстуру из файла \texttt{wood.png}. Для этого вам нужно создать спрайт и инициализировать его текстурой. Попробуйте сделать изображение очень большим, что произойдёт? Что произойдёт, если задать один из размеров текстурного окна отрицательным?
\item \textbf{Движение текстуры:} Меняйте текстурные координаты с течением времени. Спрайт должен оставаться на месте. Что изменится при использовании опции \texttt{texture.setRepeated(true)}?
\item \textbf{Анимация:} Отобразите анимацию из файла \texttt{hero.png}. Для этого вам нужно изменять текстурные координаты таким образом, таким образом, чтобы каждый кадр спрайт показывал соответствующий фрейм.\\
\end{enumerate}

\subsection*{Minesweeper}
Напишите игру Сапёр. В папке \texttt{code/minesweeper} лежит код \texttt{minesweeper.cpp}, который может вам помочь.
\begin{itemize}
\item При нажатии на левую кнопку мыши: если под текущей ячейкой - бомба, то игра заканчивается и показываются все бомбы на уровне. Если же под кнопкой не бомба, то в этой ячейки показывается число всех бомб по соседству. Если по соседству есть ячейка, которая не граничит ни с какими бомбами, то она открывается автоматически.
\item Спрячьте всю логику игры в один класс Minesweeper. При этом в функции main должен остаться только создание экземпляра этого класса(вызов конструктора) и вызов одной функции (run()).
\end{itemize}

\subsection*{Platformer:}
\begin{enumerate}
\item \textbf{Компиляция:} В папке \texttt{code/platformer} лежит код игры-платформера. Разберитесь в исходном коде этой игры, скомпилируйте её и запустите.\\
\item \textbf{Гравитация:} Измените следующие параметры игры: гравитация и скорость главного героя.\\
\item \textbf{Карта:} Массив TileMap задаёт карту мира. Добавьте в игровой мир новую платформу.\\
\item \textbf{Постройте дом:} В папке \texttt{graphics} лежит файл \texttt{props.png}. Создайте спрайт дома и нарисуйте его в некотором месте игрового мира.\\
\item \textbf{Скелет} Создайте класс скелета и поместите его в игру. Используйте анимацию из файла\\ \texttt{graphics/Creatures/skeleton}. Скелет должен бегать по карте и взаимодействовать с героем.
\item \textbf{Призрак} Создайте класс призрака и поместите его в игру. Используйте анимацию из файла\\ \texttt{graphics/Creatures/ghost}. Призрак должен летать по карте и взаимодействовать с героем.
\item \textbf{Класс Creature:} Создайте абстрактный класс \texttt{Creature}. От него должны наследоваться классы скелета и призрака. Абстрактные функции класса: \texttt{update()}, \texttt{draw}, \texttt{hit\_hero()} и другие.
\item \textbf{Монстры:} добавьте классы Frog и Bat.
\item \textbf{Уровни:} Добавьте класс Уровень. Каждый экземпляр этого класса будет определять положение всех блоков, героя и монстров.
\end{enumerate}
\end{document}



