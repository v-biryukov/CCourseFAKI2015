\documentclass{article}
\usepackage[utf8x]{inputenc}
\usepackage{ucs}
\usepackage{amsmath} 
\usepackage{amsfonts}
\usepackage{upgreek}
\usepackage[english,russian]{babel}
\usepackage{graphicx}
\usepackage{float}
\usepackage{textcomp}
\usepackage{hyperref}
\usepackage{geometry}
  \geometry{left=2cm}
  \geometry{right=1.5cm}
  \geometry{top=1cm}
  \geometry{bottom=2cm}
\usepackage{tikz}
\usepackage{ccaption}
\usepackage{multicol}

\usepackage{listings}
%\setlength{\columnsep}{1.5cm}
%\setlength{\columnseprule}{0.2pt}


\begin{document}
\pagenumbering{gobble}

\lstset{
  language=C++,                % choose the language of the code
  basicstyle=\linespread{1.1}\ttfamily,
  columns=fixed,
  fontadjust=true,
  basewidth=0.5em,
  keywordstyle=\color{blue}\bfseries,
  commentstyle=\color{gray},
  stringstyle=\ttfamily\color{orange!50!black},
  showstringspaces=false,
  %numbers=false,                   % where to put the line-numbers
  numbersep=5pt,
  numberstyle=\tiny\color{black},
  numberfirstline=true,
  stepnumber=1,                   % the step between two line-numbers.        
  numbersep=10pt,                  % how far the line-numbers are from the code
  backgroundcolor=\color{white},  % choose the background color. You must add \usepackage{color}
  showstringspaces=false,         % underline spaces within strings
  captionpos=b,                   % sets the caption-position to bottom
  breaklines=true,                % sets automatic line breaking
  breakatwhitespace=true,         % sets if automatic breaks should only happen at whitespace
  xleftmargin=.2in,
  extendedchars=\true,
  keepspaces = true,
}
\lstset{literate=%
   *{0}{{{\color{red!20!violet}0}}}1
    {1}{{{\color{red!20!violet}1}}}1
    {2}{{{\color{red!20!violet}2}}}1
    {3}{{{\color{red!20!violet}3}}}1
    {4}{{{\color{red!20!violet}4}}}1
    {5}{{{\color{red!20!violet}5}}}1
    {6}{{{\color{red!20!violet}6}}}1
    {7}{{{\color{red!20!violet}7}}}1
    {8}{{{\color{red!20!violet}8}}}1
    {9}{{{\color{red!20!violet}9}}}1
}


\section*{Домашнее задание: Библиотека SFML}
\subsection*{Подключение библиотеки}
Библиотека SFML (Simple and Fast Multimedia Library) - простая и быстрая библиотека для работы с мультимедиа. Кроссплатформенная (т. е. одна программа будет работать на операционных системах Linux, Windows и MacOS). Позволяет создавать окно, рисовать в 2D и 3D, проигрывать музыку и передавать информацию по сети. Для подключения библиотеки вам нужно скачать нужную версию с сайта: \href{https://www.sfml-dev.org/}{sfml-dev.org}. Если вы работаете на Windows с компилятором MinGW, то рекомендуется скачать версию \textbf{GCC 5.1.0 TDM}.

\subsubsection*{Подключение вручную:}
Для подключения библиотеки вручную через опции \texttt{g++} нужно задать путь до папок \texttt{include/} и \texttt{lib/} и названия файлов библиотеки, используя опции \texttt{-I}, \texttt{-L} или \texttt{-l}. 
\begin{verbatim}
g++ .\main.cpp -I<путь до include> -L<путь до lib> -lsfml-graphics -lsfml-window -lsfml-system
\end{verbatim}
Например так:
\begin{verbatim}
g++ .\main.cpp -I./SFLL-2.5.1/include -L./SFLL-2.5.1/lib -lsfml-graphics -lsfml-window -lsfml-system
\end{verbatim}
\subsubsection*{Подключение с помощью make (файл Makefile):}
Если вы программируйте на Linux/MacOS или на Windows с компилятором MinGW, то можно создать файл под названием \texttt{Makefile} в текущей директории. и написать в нём:
\begin{verbatim}
main.exe:
g++ .\main.cpp -o main.exe -o -I<путь до include> -L<путь до lib> -lsfml-graphics -lsfml-window 
-lsfml-system
\end{verbatim}
После этого можно будет скомпилировать программу вызвав \texttt{make} (на Linux) или \texttt{mingw32-make} (на Windows - MinGW)\\
Пример make-файла можно посмотреть в \texttt{classroom\_tasks/2sfml/3makefile}
\subsubsection*{Подключение с помощью cmake (файл CMakeLists.txt):}
Система автоматической сборки cmake позволяет собирать большие проекты. Чтобы работать с ней вам нужно её скачать по адресу \href{https://cmake.org/}{cmake.org} и установить переменную среды \texttt{PATH}. Затем нужно создать файл \texttt{CMakeLists.txt} в директории вашего проекта и написать в нём:
\begin{verbatim}
cmake_minimum_required(VERSION 2.8.0)
project(simple_sfml)
 
# Создадим исполняемый файл по имени simple_sfml из исходного файла main.cpp
add_executable(simple_sfml main.cpp)

# Найдём библиотеку SFML автоматически с компонентами graphics, system и window
find_package(SFML 2.5 COMPONENTS graphics system window)
# Подключим эту библиотеку
target_link_libraries(simple_sfml sfml-graphics sfml-system sfml-window)
\end{verbatim}

После этого, проект можно собрать так:
\begin{verbatim}
cmake -G<генератор> <путь до CMakeLists.txt>
\end{verbatim}

Пример make-файла можно посмотреть в папке \texttt{classroom\_tasks/2sfml/4cmake} и\\ \texttt{classroom\_tasks/2sfml/5cmake\_find\_package}\\
\begin{itemize}
\item Соберите проект в папке \texttt{0basics}, используя один из приведённых выше способов (предпочтительно - make или cmake). 
\end{itemize}
\newpage
\subsection*{Работа с библиотекой:}
Документация и туториалы по библиотеке SFML можно найти на оффициальном сайте:\\ \texttt{https://www.sfml-dev.org/}{sfml-dev.org}. Пример простой программы, для работы с SFML в папке \texttt{1sfml\_basics}. \\
Основные классы SFML и их методы:
\begin{itemize}
\item[--] \texttt{sf::Vector3f, sf::Vector2f, sf::Vector2i} и т. д. Классы для математического вектора с перегруженными операциями. (аналогичные тем, что мы писали на предыдущих занятиях). \\
\href{https://www.sfml-dev.org/documentation/2.5.1/classsf_1_1Vector2.php}{sfml-dev.org/documentation/2.5.1/classsf\_1\_1Vector2.php}
\item[--] \texttt{sf::RenderWindow} - класс для окна.
\href{https://www.sfml-dev.org/documentation/2.5.1/classsf_1_1RenderWindow.php}{sfml-dev.org/documentation/2.5.1/classsf\_1\_1RenderWindow.php}
\item[--] \texttt{sf::CircleShape} - класс для фигуры - круг.
\href{https://www.sfml-dev.org/documentation/2.5.1/classsf_1_1CircleShape.php}{sfml-dev.org/documentation/2.5.1/classsf\_1\_1CircleShape.php}

\end{itemize}


\subsection*{Задачи:}
\begin{itemize}
\item \textbf{Шарики:} \\
В папке \texttt{2sfml\_balls} лежит пример простой программы, которая рисует множество движующихся шаров. Скомпилируйте и запустите эту программу.
\item \textbf{Граничные условия - тор:} \\
Видоизмените программу так, чтобы шарики при выходе за границы экрана телепортировались к противоположной границе (сохраняя скорость).
\item \textbf{Граничные условия - стенки:} \\
Видоизмените программу так, чтобы шарики отражались от границ.
\item \textbf{Задача N тел} \\
Добавьте гравитационное взаимодействие между шариками. Считайте что масса всех шариков равна 1. Один тонкий момент - если шарики подойдут очень близко к друг другу, то они могут нереалистично разлететься. Чтобы это избежать, ограничьте снизу дистанцию гравитационного взаимодействия.
\item \textbf{Задача N тел с массой} \\
Добавьте разную массу шарикам. При создании шарика масса должна задаваться случайным образом. Масса шарика должна быть пропорциональна площади (квадрату радиуса).
\item \textbf{Электрические заряды} \\
Смоделируйте взаимодействие заряженных частиц. Для этого нужно добавить поле в структуру \texttt{Ball}, которое будет определять величину заряда. Эта величина может быть как положительной, так и отрицательной. В начале работы программы заряд должен задаваться случайно. Заряды должны взаимодействовать по закону Кулона. Гравитацией можно пренебречь. Цвета зарядов должны быть различными (красный для положительного зарада и синий для отрицательного, интенсивность цвета - пропорциональна величине заряда).
\item \textbf{Нажатие мыши} \\
События нажатия мыши можно обработать с помощью следующего синтаксиса:
\begin{lstlisting}
if (event.type == sf::Event::MouseButtonPressed) 
{
    if (event.mouseButton.button == sf::Mouse::Right) 
    {
        std::cout << "the right button was pressed" << std::endl;
        std::cout << "mouse x: " << event.mouseButton.x << std::endl;
        std::cout << "mouse y: " << event.mouseButton.y << std::endl;
    }
}
\end{lstlisting}
Внутри цикла \texttt{while (window.pollEvent(event))}.\\
Видоизмените вашу программу так, чтобы при нажатии левой кнопки мыши в том месте, где находится мышь, создавался бы шарик со средними массой и средним положительным зарядом зарядом. При нажатии правой кнопки мыши должен создаваться шарик с очень большой  массой и очень большим положительным зарядом. При аналогичных нажатиях, но с зажатой клавишой Shift, должны создаваться отрицательные заряды.
\end{itemize}
В качестве решения нужно прислать финальную версию программы (1 файл).
\end{document}