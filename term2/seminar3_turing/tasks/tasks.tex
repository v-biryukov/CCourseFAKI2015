\documentclass[12pt]{article}
\usepackage[T2A]{fontenc}
\usepackage[utf8]{inputenc}
\usepackage{ucs}
\usepackage{amsmath} 
\usepackage{mathtext}
\usepackage{amsfonts}
\usepackage{upgreek}
\usepackage[russian]{babel}
\usepackage{graphicx}
\usepackage{float}
\usepackage{textcomp}
\usepackage{multicol}
\usepackage{hyperref}
\usepackage{geometry}
  \geometry{left=2cm}
  \geometry{right=1.5cm}
  \geometry{top=1cm}
  \geometry{bottom=2cm}
\usepackage{tikz}
\usepackage{ccaption}
\usepackage{float}
\usepackage{verbatim}
\restylefloat{table}


\begin{document}
\pagenumbering{gobble}

\section*{Алгоритмы}
\subsection*{Часть А}

\begin{enumerate}
\item \textbf{} Чему равен порядок роста продолжительности работы следующего участка кода?
\begin{verbatim}
int sum = 0;
for (int i = 0; i < N; i++)
    for (int j = i+1; j < N; j++)
        for (int k = j+1; k < N; k++)
                sum++;
\end{verbatim}
\item Расположите следующие функции в порядке увеличения скорости роста:
\begin{multicols}{3}
\begin{enumerate}
\item $\log{n}$
\item $1$
\item $e^n$
\item $n$
\item $\log{\log{n}}$
\item $n\sqrt{n}$
\item $n!$
\item $n^n$
\item $n\log{n}$
\item $n^2$
\item $2^n$
\end{enumerate}
\end{multicols}

Отметьте все функции, чьё O-большое равно $O(n^2)$
\begin{multicols}{2}
\begin{itemize}
\item $1000 n^2$
\item $e^n$
\item $4 n^2 + 10 n + 50$
\item $n^3 + 100 n^2$
\item $n^3 - 100 n^2$
\item $n\log{n}$
\item $n^3 / (1 + n)$
\end{itemize}
\end{multicols}

\item Заполните таблицу:
\begin{center}
  \begin{tabular}{ l || c }
    \hline
    Алгоритм & Сложность(в среднем) \\ \hline \hline
    Сортировка пузырьком & $O(n^2)$  \\ \hline
    Сортировка вставками &   \\ \hline
    Сортировка слиянием &   \\ \hline
    Быстрая сортировка &   \\ \hline
    Сортировка выбором & \\ \hline
    Простейший алгоритм перемножения матриц nxn & \\ \hline
    Простейший алгоритм сложения матриц nxn & \\ \hline
  \end{tabular}
\end{center}

\item Заполните таблицу: Вычислительные сложности(в среднем) операций над структурами данных
\begin{center}
  \begin{tabular}{ l || c | c | c | c }
    \hline
     & Доступ к элементу & Вставка & Удаление & Поиск \\ \hline \hline
    Массив &  &  &  &  \\ \hline
    Односвязный список(list) &  &  &  & \\ \hline
    Двоичное дерево поиска &  &  &  & \\ \hline
    Хеш-таблица &  &  &  & \\ \hline
  \end{tabular}
\end{center}

\newpage

\item Класс NP это:
\begin{itemize}
\item Класс задач, которые нельзя решить за полиномиальное время на детерминированной машине Тьюринга.
\item Класс задач, которые можно решить за полиномиальное время  на недетерминированной машине Тьюринга.
\item Класс алгоритмов, которые нельзя выполнить за полиномиальное время на детерминированной машине Тьюринга.
\item Класс алгоритмов, которые можно выполнить за полиномиальное время на недетерминированной машине Тьюринга.
\end{itemize}

\end{enumerate}

\subsection*{Часть B}
\begin{enumerate}
\item Алиса и Боб любят игры и соревнования. И сейчас они готовы приступить к новой игре. Всего у них есть n плиток шоколада. По правилам игры они могут есть этот шоколад по очереди(первой начинает Алиса). Известно, что Алиса съедает $a$ плиток шоколада за ход, а Боб -- $b$ плиток шоколада. Выйгрывает тот, кто съест последнюю плитку. \\
\null\hspace{1cm}\textit{Входные данные:} \\
\null\hspace{2cm} Целые положительные n, a и b. \\
\null\hspace{1cm} \textit{Выходные данные:} \\
\null\hspace{2cm} "Alice", если победит Алиса, или "Bob", если победит Боб \\
\null\hspace{1cm} \textit{Пример:}
\begin{table}[h]
\centering
\begin{tabular}{|l|l|}
\hline
Вход & Выход                                                      \\ \hline
20 5 3  & Alice  \\ \hline
2 1 1  & Bob                                                        \\ \hline
\end{tabular}
\end{table}

\item \textbf{[Set1]} Дано n чисел, причём известно, что среди этих чисел есть только одно уникальное. Все остальные числа повторяются парами. Найти это уникальное число. \\
\null\hspace{1cm}\textit{Входные данные:} \\
\null\hspace{2cm} Целое положительное нечётное $n \le 2 * 10^9$\\
\null\hspace{2cm} n целых чисел.  \\
\null\hspace{1cm} \textit{Выходные данные:} \\
\null\hspace{2cm} Уникальное число \\
\null\hspace{1cm} \textit{Пример:}
\begin{table}[h]
\centering
\begin{tabular}{|l|l|}
\hline
Вход & Выход                                                      \\ \hline
3 1 8 1  & 8  \\ \hline
11 5 4 7 4 8 78 15 5 78 8 15  & 7                                                      \\ \hline
\end{tabular}
\end{table}

\newpage
\item \textbf{[Set2]} Дано n чисел и некоторое число $A$. Среди этих n чисел найти два таких, что их сумма равна A. Вернуть -1, если таких 2-х чисел нет. \\
\null\hspace{1cm}\textit{Входные данные:} \\
\null\hspace{2cm} Целое положительное $n \le 2 * 10^9$ и целое положительное $A$\\
\null\hspace{2cm} n целых чисел.  \\
\null\hspace{1cm} \textit{Выходные данные:} \\
\null\hspace{2cm} Индексы и значения искомой пары чисел или -1, если такой пары нет. \\
\null\hspace{1cm} \textit{Пример:}
\begin{table}[h]
\centering
\begin{tabular}{|l|l|}
\hline
Вход & Выход                                                      \\ \hline
3 1 8 1  & 8  \\ \hline
11 5 4 7 4 8 78 15 5 78 8 15  & 7                                                      \\ \hline
\end{tabular}
\end{table}



\item Найти последнюю цифру n-го элемента последовательности Фибоначчи. \\
\null\hspace{1cm}\textit{Входные данные:} \\
\null\hspace{2cm} Целое положительное $n \le 2 * 10^9$. \\
\null\hspace{1cm} \textit{Выходные данные:} \\
\null\hspace{2cm} Последняя цифра n-го элемента последовательности Фибоначчи \\
\null\hspace{1cm} \textit{Пример:}
\begin{table}[h]
\centering
\begin{tabular}{|l|l|}
\hline
Вход & Выход                                                      \\ \hline
20 5 3  & Alice  \\ \hline
2 1 1  & Bob                                                        \\ \hline
\end{tabular}
\end{table}


\item Задача hexen в контесте NUMBER5.

\end{enumerate}
\end{document}