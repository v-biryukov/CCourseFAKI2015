\documentclass{article}
\usepackage[utf8x]{inputenc}
\usepackage{ucs}
\usepackage{amsmath} 
\usepackage{amsfonts}
\usepackage{marvosym}
\usepackage{wasysym}
\usepackage{upgreek}
\usepackage[english,russian]{babel}
\usepackage{graphicx}
\usepackage{float}
\usepackage{textcomp}
\usepackage{hyperref}
\usepackage{geometry}
  \geometry{left=2cm}
  \geometry{right=1.5cm}
  \geometry{top=1cm}
  \geometry{bottom=2cm}
\usepackage{tikz}
\usepackage{ccaption}
\usepackage{multicol}

\hypersetup{
   colorlinks=true,
   citecolor=blue,
   linkcolor=black,
   urlcolor=blue
}

\usepackage{listings}
%\setlength{\columnsep}{1.5cm}
%\setlength{\columnseprule}{0.2pt}

\usepackage[absolute]{textpos}


\usepackage{colortbl,graphicx,tikz}
\definecolor{X}{rgb}{.5,.5,.5}

\renewcommand{\thesubsection}{\arabic{subsection}}

\begin{document}
\pagenumbering{gobble}
\lstset{
  language=C++,                % choose the language of the code
  basicstyle=\linespread{1.1}\ttfamily,
  columns=fixed,
  fontadjust=true,
  basewidth=0.5em,
  keywordstyle=\color{blue}\bfseries,
  commentstyle=\color{gray},
  stringstyle=\ttfamily\color{orange!50!black},
  showstringspaces=false,
  numbersep=5pt,
  numberstyle=\tiny\color{black},
  numberfirstline=true,
  stepnumber=1,                   % the step between two line-numbers.        
  numbersep=10pt,                  % how far the line-numbers are from the code
  backgroundcolor=\color{white},  % choose the background color. You must add \usepackage{color}
  showstringspaces=false,         % underline spaces within strings
  captionpos=b,                   % sets the caption-position to bottom
  breaklines=true,                % sets automatic line breaking
  breakatwhitespace=true,         % sets if automatic breaks should only happen at whitespace
  xleftmargin=.2in,
  extendedchars=\true,
  keepspaces = true,
}
\lstset{literate=%
   *{0}{{{\color{red!20!violet}0}}}1
    {1}{{{\color{red!20!violet}1}}}1
    {2}{{{\color{red!20!violet}2}}}1
    {3}{{{\color{red!20!violet}3}}}1
    {4}{{{\color{red!20!violet}4}}}1
    {5}{{{\color{red!20!violet}5}}}1
    {6}{{{\color{red!20!violet}6}}}1
    {7}{{{\color{red!20!violet}7}}}1
    {8}{{{\color{red!20!violet}8}}}1
    {9}{{{\color{red!20!violet}9}}}1
}

\renewcommand{\thesubsection}{\arabic{subsection}}
\makeatletter
\def\@seccntformat#1{\@ifundefined{#1@cntformat}%
   {\csname the#1\endcsname\quad}%    default
   {\csname #1@cntformat\endcsname}}% enable individual control
\newcommand\section@cntformat{}     % section level 
\newcommand\subsection@cntformat{Задача \thesubsection.\space} % subsection level
\newcommand\subsubsection@cntformat{\thesubsubsection.\space} % subsubsection level
\makeatother


\makeatletter
\newcount\my@repeat@count
\newcommand{\myrepeat}[2]{%
  \begingroup
  \my@repeat@count=\z@
  \@whilenum\my@repeat@count<#1\do{#2\advance\my@repeat@count\@ne}%
  \endgroup
}
\makeatother

\title{Семинар \#4: Шаблоны. Класс \texttt{std::vector}. Домашнее задание.\vspace{-5ex}}\date{}\maketitle

\subsection{Сумма чётных}
Напишите функцию \texttt{int sumEven(const std::vector<int>\& v)}, которая будет принимать вектор по константной ссылке и возвращать сумму всех чётных чисел этого вектора.
\begin{center}
\begin{tabular}{ l | l }
 аргумент & возвращаемое значение \\ \hline
 \texttt{\{4, 8, 15, 16, 23, 42\}} & \texttt{70}
\end{tabular}
\end{center}

\subsection{Последние цифры}
Напишите функциии которые должны будут принимать на вход вектор и заменять все его элементы на последние цифры этих элементов. Все функции должны делать одно и то же, но должны будут возвращать результат разными способами.

\begin{itemize}
\item Функция \texttt{std::vector<int> lastDigits1(const std::vector<int>\& v)} должна возвращать вектор.
\item Функция \texttt{void lastDigits2(std::vector<int>\& v)} должна принимать вектор по ссылке и изменять его.
\item Функция \texttt{void lastDigits3(std::vector<int>* pv)} должна принимать указатель, хранящий адрес некоторого вектора. Функция должна изменять вектор, на который указывает указатель.
\item Функция \texttt{void lastDigits4(std::span<int> sp)} должна принимать объект типа \texttt{std::span} и менять вектор, на который указывает этот span.
\end{itemize}

\subsection{Простые делители}
Напишите функцию \texttt{std::vector<std::pair<int, int>{}> factorization(int n)}, которая будет находить все простые делители числа \texttt{n} и их количества.
\begin{center}
\begin{tabular}{ l | l }
 аргумент & возвращаемое значение \\ \hline
 \texttt{60} & \texttt{\{\{2, 2\}, \{3, 1\}, \{5, 1\}\}} \\
 \texttt{626215995} & \texttt{\{\{3, 3\}, \{5, 1\}, \{17, 1\}, \{29, 1\}, \{97, 2\}\}} \\
 \texttt{107} & \texttt{\{\{107, 1\}\}} \\
 \texttt{1} & \texttt{\{\{1, 1\}\}} 
\end{tabular}
\end{center}

\subsection{Времена из строки}
\begin{itemize}
\item Напишите простой класс для работы со временем:
\begin{lstlisting}
class Time {
private:
    int mHours, mMinutes, mSeconds;
public:
    Time(int hours, int minutes, int seconds);
    Time(std::string_view s); // строка в формате "hh:mm:ss"
    Time operator+(Time b) const;
    int hours() const; int minutes() const; int seconds() const;
    friend std::operator<<(std::ostream& out, Time t);
};
\end{lstlisting}
\item Напишите функцию \texttt{std::vector<Time> getTimesFromString(std::string\_view s)}, которая будет принимать строку в формате \texttt{"hh:mm:ss ... hh:mm:ss"}, где за место букв должны стоять некоторые числа. Например, строка может иметь вид \texttt{"12:20:05 05:45:30 22:10:45"}. Функция должна возвращать вектор времен, соответствующий временам в строке.\\
\item Напишите функцию \texttt{Time sumTimes(const std::vector<Time>\& v)}, которая будет суммировать все времена и возвращать эту сумму.
\end{itemize}


\subsection{Максимум}
Напишите шаблонную функцию, которая будет принимать на вход вектор и возвращать максимальный элемент вектора.


\subsection{Разбиение на пары}
Напишите шаблонную функцию, которая будет принимать на вход контейнер (\texttt{vector}, \texttt{array}, \texttt{string} или \texttt{string\_view}) и возвращать вектор пар нечётных и чётных элементов. Протестировать функцию можно в файле \texttt{code/test\_pairing.cpp}.
\end{document}
