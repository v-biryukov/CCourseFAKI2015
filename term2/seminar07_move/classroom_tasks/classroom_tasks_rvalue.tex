\documentclass{article}
\usepackage[utf8x]{inputenc}
\usepackage{ucs}
\usepackage{amsmath} 
\usepackage{amsfonts}
\usepackage{upgreek}
\usepackage[english,russian]{babel}
\usepackage{graphicx}
\usepackage{float}
\usepackage{textcomp}
\usepackage{hyperref}
\usepackage{geometry}
  \geometry{left=2cm}
  \geometry{right=1.5cm}
  \geometry{top=1cm}
  \geometry{bottom=2cm}
\usepackage{tikz}
\usepackage{ccaption}
\usepackage{multicol}

\usepackage{listings}
%\setlength{\columnsep}{1.5cm}
%\setlength{\columnseprule}{0.2pt}


\begin{document}
\pagenumbering{gobble}

\lstset{
  language=C++,                % choose the language of the code
  basicstyle=\linespread{1.1}\ttfamily,
  columns=fixed,
  fontadjust=true,
  basewidth=0.5em,
  keywordstyle=\color{blue}\bfseries,
  commentstyle=\color{gray},
  stringstyle=\ttfamily\color{orange!50!black},
  showstringspaces=false,
  %numbers=false,                   % where to put the line-numbers
  numbersep=5pt,
  numberstyle=\tiny\color{black},
  numberfirstline=true,
  stepnumber=1,                   % the step between two line-numbers.        
  numbersep=10pt,                  % how far the line-numbers are from the code
  backgroundcolor=\color{white},  % choose the background color. You must add \usepackage{color}
  showstringspaces=false,         % underline spaces within strings
  captionpos=b,                   % sets the caption-position to bottom
  breaklines=true,                % sets automatic line breaking
  breakatwhitespace=true,         % sets if automatic breaks should only happen at whitespace
  xleftmargin=.2in,
  extendedchars=\true,
  keepspaces = true,
}
\lstset{literate=%
   *{0}{{{\color{red!20!violet}0}}}1
    {1}{{{\color{red!20!violet}1}}}1
    {2}{{{\color{red!20!violet}2}}}1
    {3}{{{\color{red!20!violet}3}}}1
    {4}{{{\color{red!20!violet}4}}}1
    {5}{{{\color{red!20!violet}5}}}1
    {6}{{{\color{red!20!violet}6}}}1
    {7}{{{\color{red!20!violet}7}}}1
    {8}{{{\color{red!20!violet}8}}}1
    {9}{{{\color{red!20!violet}9}}}1
}

\title{Семинар \#8: Move-семантика и rvalue-ссылки. \vspace{-5ex}}\date{}\maketitle


\section*{Часть 1: \texttt{lvalue} и \texttt{rvalue}}
Пусть есть функция \texttt{func}, принимающая объект некоторого типа по значению. В качестве типа объекта в этом примере возьмём \texttt{std::string}, но вообще это может быть любой тип. 
\begin{lstlisting}
void func(std::string s) {...}
\end{lstlisting}
Предположим, что мы передаём на вход этой функции некоторое выражение. Что лучше использовать при передаче в этом случае: копирование или перемещение? На самом деле это зависит от выражения, которое приходит на вход функции.

\begin{lstlisting}
std::string s1 = get1();
std::string s2 = get2();

func(s1);      // В этом случае лучше использовать копирование, так как s1 нам ещё нужна
func(s1 + s2); // В этом случае лучше использовать перемещение, так как s1 + s2 нам не нужен
\end{lstlisting}
В общем случае нам бы хотелось, чтобы те выражения, у которых есть имя или известный адрес, передавались копированием. Так как они могут быть использованы после вызова функции. Такие выражения называются \texttt{lvalue}-выражениями. Остальные выражения, то есть те, которые мы хотим перемещать, называются \texttt{rvalue}-выражениями.\\



В коде ниже представлены примеры \texttt{lvalue} и \texttt{rvalue} выражений.

\begin{lstlisting}
#include <string>
#include <math>

int main() 
{
    int a = 123;
    int array[5] = {1, 2, 3, 4, 5};
    std::string s = "Cat";
    
    // Примеры lvalue выражений:
    a   array    s   array[i]  
    
    // Примеры rvalue выражений:
    a+1    -a    2*array[i]    s+"!"    sqrt(5)
}
\end{lstlisting}

На самом деле, приведённое выше определение \texttt{lvalue} не полностью верно и не учитывает некоторые случаи. Более точное определение понятий \texttt{lvalue} и \texttt{rvalue} приведено в стандарте, оно слишком громоздко, чтобы описать его здесь.\\

При передаче в функцию по значению компилятор автоматически определяет является выражение \texttt{lvalue} или \texttt{rvalue} и, либо копирует его, либо перемещает. Но иногда бывают ситуации, когда нам хочется переметить \texttt{lvalue} выражение. В этом случае нужно просто использовать стандартную функцию \texttt{std::move}. Эта функция сама по себе ничего не передвигает, а просто превращает \texttt{lvalue} в \texttt{rvalue}.
\begin{lstlisting}
func(s1);             // Произойдёт копирование
func(s1 + s2);        // Произойдёт перемещение временного объекта s1 + s2
func(std::move(s1));  // Произойдёт перемещение s1
// Тут s1 стал пустой строкой, так как мы его переместили
\end{lstlisting}


\newpage
\section*{Часть 2: \texttt{rvalue}-ссылки}
В предыдущей часте мы рассмотрели передачу в функцию по значению. Но чаще используется передача по ссылке. Для того, чтобы можно было различать категорию выражения при передаче по ссылке в язык были введены \texttt{rvalue}-ссылки. Такие ссылки очень похожи на обычные ссылки (которые теперь называются \texttt{lvalue}-ссылками). Основное отличие таких ссылок в том, что они инициализируются только \texttt{rvalue}-выражениями.

\begin{lstlisting}
int a = 123;
// rvalue ссылку можно инициализировать так:
int&& r1 = 10;
int&& r2 = a + 1;
// Следующее не будет работать так как a это lvalue:
int&& r3 = a; 
\end{lstlisting}


С помощью \texttt{rvalue} ссылок и перегрузки функций можно различить категорию выражения, приходящую на вход функции. В примере ниже вызовется соответствующий вариант перегрузки в зависимости от категории выражения.
\begin{lstlisting}
#include <iostream>
#include <string>
using std::cout, std::endl;

void func(std::string& s)
{
    cout << "Pass by lvalue reference" << endl;
}
void func(std::string&& s)
{
    cout << "Pass by rvalue reference" << endl;
}

int main() 
{
    std::string s1 = "Cat";
    std::string s2 = "Dog";

    func(s1);               // Передадим по lvalue ссылке
    func(s1 + s2);          // Передадим по rvalue ссылке
    func(s1.substr(0, 2));  // Передадим по rvalue ссылке
}
\end{lstlisting}

\newpage
\section*{Часть 3: Конструктор перемещения и оператор присваивания перемещения}
Для объекта можно написать конструктор копирования и оператор присваивания, которые должны производить глубокое копирование объекта. По аналогии с копированием, для объекта можно создать конструктор перемещения и оператор присваивания перемещением.

\newpage


\end{document}