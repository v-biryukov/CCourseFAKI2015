\documentclass{article}
\usepackage[utf8x]{inputenc}
\usepackage{ucs}
\usepackage{amsmath} 
\usepackage{amsfonts}
\usepackage{marvosym}
\usepackage{wasysym}
\usepackage{upgreek}
\usepackage[english,russian]{babel}
\usepackage{graphicx}
\usepackage{float}
\usepackage{textcomp}
\usepackage{hyperref}
\usepackage{geometry}
  \geometry{left=2cm}
  \geometry{right=1.5cm}
  \geometry{top=1cm}
  \geometry{bottom=2cm}
\usepackage{tikz}
\usepackage{ccaption}
\usepackage{multicol}

\hypersetup{
   colorlinks=true,
   citecolor=blue,
   linkcolor=black,
   urlcolor=blue
}

\usepackage{listings}
%\setlength{\columnsep}{1.5cm}
%\setlength{\columnseprule}{0.2pt}

\usepackage[absolute]{textpos}


\usepackage{colortbl,graphicx,tikz}
\definecolor{X}{rgb}{.5,.5,.5}

\renewcommand{\thesubsection}{\arabic{subsection}}

\begin{document}
\pagenumbering{gobble}
\lstset{
  language=C++,                % choose the language of the code
  basicstyle=\linespread{1.1}\ttfamily,
  columns=fixed,
  fontadjust=true,
  basewidth=0.5em,
  keywordstyle=\color{blue}\bfseries,
  commentstyle=\color{gray},
  stringstyle=\ttfamily\color{orange!50!black},
  showstringspaces=false,
  numbersep=5pt,
  numberstyle=\tiny\color{black},
  numberfirstline=true,
  stepnumber=1,                   % the step between two line-numbers.        
  numbersep=10pt,                  % how far the line-numbers are from the code
  backgroundcolor=\color{white},  % choose the background color. You must add \usepackage{color}
  showstringspaces=false,         % underline spaces within strings
  captionpos=b,                   % sets the caption-position to bottom
  breaklines=true,                % sets automatic line breaking
  breakatwhitespace=true,         % sets if automatic breaks should only happen at whitespace
  xleftmargin=.2in,
  extendedchars=\true,
  keepspaces = true,
}
\lstset{literate=%
   *{0}{{{\color{red!20!violet}0}}}1
    {1}{{{\color{red!20!violet}1}}}1
    {2}{{{\color{red!20!violet}2}}}1
    {3}{{{\color{red!20!violet}3}}}1
    {4}{{{\color{red!20!violet}4}}}1
    {5}{{{\color{red!20!violet}5}}}1
    {6}{{{\color{red!20!violet}6}}}1
    {7}{{{\color{red!20!violet}7}}}1
    {8}{{{\color{red!20!violet}8}}}1
    {9}{{{\color{red!20!violet}9}}}1
}
\newcommand\upquote[1]{\textquotesingle#1\textquotesingle}

\renewcommand{\thesubsection}{\arabic{subsection}}
\makeatletter
\def\@seccntformat#1{\@ifundefined{#1@cntformat}%
   {\csname the#1\endcsname\quad}%    default
   {\csname #1@cntformat\endcsname}}% enable individual control
\newcommand\section@cntformat{}     % section level 
\newcommand\subsection@cntformat{Задача \thesubsection.\space} % subsection level
\newcommand\subsubsection@cntformat{\thesubsubsection.\space} % subsubsection level
\makeatother


\makeatletter
\newcount\my@repeat@count
\newcommand{\myrepeat}[2]{%
  \begingroup
  \my@repeat@count=\z@
  \@whilenum\my@repeat@count<#1\do{#2\advance\my@repeat@count\@ne}%
  \endgroup
}
\makeatother

\title{Семинар \#7: Move-семантика и умные указатели. Домашнее задание.\vspace{-5ex}}\date{}\maketitle

\subsection{Стек строк}
Напишите класс \texttt{StringStack}, объекты которого будут хранить некоторое количество строк.
Нужно написать следующие методы класса \texttt{StringStack}:

\begin{itemize}
\item \texttt{push}: В этот метод передаётся одна строка и эта строка добавляется в стек. Причём, если строка передаёся по lvalue, то она должна скопироваться, а если по rvalue, то переместиться.
\item \texttt{print}: Этот метод должен печатать на экран все элементы стека в скобках, через запятую.
\item \texttt{pop}: Этот метод должен удалять из стека последний элемент и возвращать его. В этом случае не должно происходить копирования строк, а только перемещения.
\end{itemize}

\begin{lstlisting}
...
StringStack ss;
std::string a {"Cat"};

ss.push(a);                    // должен скопировать строку a внутрь объекта ss
ss.push(std::string{"Mouse"}); // должен переместить временную строку внутрь объекта ss
ss.print();                    // печатает  (Cat, Mouse)
cout << a << endl;             // печатает  Cat

ss.push(std::move(a));         // должен переместить строку a внутрь объекта ss
ss.print();                    // печатает  (Cat, Mouse, Cat)
cout << a << endl;             // печатает  пустую строку

cout << a.pop() << endl;       // печатает  Cat
ss.print();                    // печатает  (Cat, Mouse)
cout << a.pop() << endl;       // печатает  Mouse
cout << a.pop() << endl;       // печатает  Cat
ss.print();                    // печатает  ()
...
\end{lstlisting}
Напишите этот класс БЕЗ использования rvalue ссылок и универсальных ссылок.


\end{document}
