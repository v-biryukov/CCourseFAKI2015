\documentclass{article}
\usepackage[utf8x]{inputenc}
\usepackage{ucs}
\usepackage{amsmath} 
\usepackage{amsfonts}
\usepackage{marvosym}
\usepackage{wasysym}
\usepackage{upgreek}
\usepackage[english,russian]{babel}
\usepackage{graphicx}
\usepackage{float}
\usepackage{textcomp}
\usepackage{hyperref}
\usepackage{geometry}
  \geometry{left=2cm}
  \geometry{right=1.5cm}
  \geometry{top=1cm}
  \geometry{bottom=2cm}
\usepackage{tikz}
\usepackage{ccaption}
\usepackage{multicol}

\hypersetup{
   colorlinks=true,
   citecolor=blue,
   linkcolor=black,
   urlcolor=blue
}

\usepackage{listings}
%\setlength{\columnsep}{1.5cm}
%\setlength{\columnseprule}{0.2pt}

\usepackage[absolute]{textpos}


\usepackage{colortbl,graphicx,tikz}
\definecolor{X}{rgb}{.5,.5,.5}

\renewcommand{\thesubsection}{\arabic{subsection}}

\begin{document}
\pagenumbering{gobble}
\lstset{
  language=C++,                % choose the language of the code
  basicstyle=\linespread{1.1}\ttfamily,
  columns=fixed,
  fontadjust=true,
  basewidth=0.5em,
  keywordstyle=\color{blue}\bfseries,
  commentstyle=\color{gray},
  stringstyle=\ttfamily\color{orange!50!black},
  showstringspaces=false,
  numbersep=5pt,
  numberstyle=\tiny\color{black},
  numberfirstline=true,
  stepnumber=1,                   % the step between two line-numbers.        
  numbersep=10pt,                  % how far the line-numbers are from the code
  backgroundcolor=\color{white},  % choose the background color. You must add \usepackage{color}
  showstringspaces=false,         % underline spaces within strings
  captionpos=b,                   % sets the caption-position to bottom
  breaklines=true,                % sets automatic line breaking
  breakatwhitespace=true,         % sets if automatic breaks should only happen at whitespace
  xleftmargin=.2in,
  extendedchars=\true,
  keepspaces = true,
}
\lstset{literate=%
   *{0}{{{\color{red!20!violet}0}}}1
    {1}{{{\color{red!20!violet}1}}}1
    {2}{{{\color{red!20!violet}2}}}1
    {3}{{{\color{red!20!violet}3}}}1
    {4}{{{\color{red!20!violet}4}}}1
    {5}{{{\color{red!20!violet}5}}}1
    {6}{{{\color{red!20!violet}6}}}1
    {7}{{{\color{red!20!violet}7}}}1
    {8}{{{\color{red!20!violet}8}}}1
    {9}{{{\color{red!20!violet}9}}}1
}
\newcommand\upquote[1]{\textquotesingle#1\textquotesingle}

\renewcommand{\thesubsection}{\arabic{subsection}}
\makeatletter
\def\@seccntformat#1{\@ifundefined{#1@cntformat}%
   {\csname the#1\endcsname\quad}%    default
   {\csname #1@cntformat\endcsname}}% enable individual control
\newcommand\section@cntformat{}     % section level 
\newcommand\subsection@cntformat{Задача \thesubsection.\space} % subsection level
\newcommand\subsubsection@cntformat{\thesubsubsection.\space} % subsubsection level
\makeatother


\makeatletter
\newcount\my@repeat@count
\newcommand{\myrepeat}[2]{%
  \begingroup
  \my@repeat@count=\z@
  \@whilenum\my@repeat@count<#1\do{#2\advance\my@repeat@count\@ne}%
  \endgroup
}
\makeatother


\title{Семинар \#9: Библиотеки. Домашнее задание. \vspace{-5ex}}\date{}\maketitle


\subsection{Стадии компиляции}
В файле \texttt{simple\_image} лежит исходный код простой программы, которая создаёт простое изображение. Пройдите поэтапно все стадии сборки с этой программой. В результате вы должны получить следующие файлы:
\begin{itemize}
\item Файл исходного кода на языке C++, который получается после этапа препроцессинга.
\item Файл кода на языке ассемблера.
\item Объектный файл.
\end{itemize}



\subsection{Класс изображения}

В файле \texttt{simple\_image} лежит исходный код программы, которая использует класс \texttt{Image}. Это простой класс для работы с изображениями в формате \texttt{.ppm}. Скомпилируйте и запустите эту программу.
 
\begin{itemize}
\item \textbf{Шум:}\\
Добавьте в этот класс метод \texttt{void addNoise(float probability)}, который будет добавлять шум на картинку: каждый пиксель с вероятностью \texttt{probability} должен поменять цвет на случайный. Протестируйте этот метод на картинках из папки \texttt{$\sim$/data/ppm\_images}.

\item \textbf{Заголовочный файл} \\
Создайте файл \texttt{image.hpp} и перенесите класс \texttt{Image} в этот файл. Используйте директиву \texttt{\#pragma once}, чтобы избежать в будущем возможного множественного включения. Таким образом получится мини-библиотека. Подключите этот заголовочный файл к вашей основной программу и скомпилируйте программу.

\item \textbf{Раздельная компиляция:} \\
Создайте ещё один файл \texttt{image.cpp}. Перенесите из файла \texttt{image.hpp} в файл \texttt{image.cpp} все определения методов. Таким образом в файле \texttt{image.hpp} должны остаться определение класса и объявление всех его методов, а в файле \texttt{image.cpp} должны быть определения всех методов. Скомпилируйте эту программу.

\item \textbf{Статическая библиотека} \\
Чтобы создать свою статическую библиотеку вам нужно:
\begin{enumerate}
\item Создать объектный файл необходимого исходного файла.
\item Превратить объектный файл (или файлы) в библиотеку, используя утилиту \texttt{ar}:
\begin{verbatim}
ar rvs libimage.a image.o
\end{verbatim}
\item После этого файл \texttt{libimage.a} можно будет подключить к любому другому проекту примерно так:
\begin{verbatim}
g++ main.cpp -I<путь до header-файлов> -L<путь до libimage.a> -limage
\end{verbatim}
\end{enumerate}
Создайте статическую библиотеку из файла \texttt{image.cpp}. Создайте папку \texttt{image/} в которой будет храниться наша библиотека. В этой папке создайте ещё 2 папки: \texttt{include} и \texttt{lib/}. Поместите в папку \texttt{image/include} заголовочный файл \texttt{image.hpp}. Поместите в папку \texttt{image/lib} файл статической библиотеки \texttt{libimage.a}. Затем вам нужно удалить файл \texttt{image.cpp} и собрать программу используя только статическую библиотеку (не забывайте про опции \texttt{-I}, \texttt{-L} и \texttt{-l}).


\item \textbf{Динамическая библиотека} \\
Чтобы создать динамическую библиотеку из файла исходного кода (\texttt{image.cpp}):
\begin{verbatim}
g++ -c -fPic image.cpp -o image.o
g++ -shared -o libimage.so image.o
\end{verbatim}
Чтобы скомпилировать код с подключением динамической библиотеки:
\begin{verbatim}
g++ -o main.exe main.cpp libimage.so
\end{verbatim}
или
\begin{verbatim}
g++ -o main.exe main.cpp -limage
\end{verbatim}
Но для этого понадобится добавить в переменную среды \texttt{LD\_LIBRARY\_PATH} (на Windows нужно добавить в переменную среды \texttt{PATH}) путь до папки, содержащий библиотеку.
\begin{enumerate}
\item Создайте динамическую библиотеку и скомпилируйте саму программу с подключением динамической библиотеки
\item Проверьте чему равен размеры исполняемых файлов в случае подключения статической и динамической библиотеки.
\item Что будет происходить, если перенести файл динамической библиотеки в другую папку. Запустится ли исполняемый файл?
\end{enumerate}
\end{itemize}



\subsection{Движение по окружности}
Напишите программу, которая будет рисовать на экране кружок, двигающийся по окружности. Используйте библиотеку SFML.


\subsection{Задача $n$ тел}
В двумерном пространстве находятся $n$ шариков с различными массами и различными электрическими зарядами. Напишите программу, которая будет моделировать движение таких шариков. Для отрисовки используйте библиотеку \texttt{SFML}.\\

\begin{itemize}
\item Шарики действуют друг на друга силой Кулона:
$$
F  = \frac{q_1 \cdot q_2}{R}
$$
Где $q_{1,2}$ -- заряды шариков, $R$ -- расстояние между шариками. Обратите внимание, что сила взаимодействия обратно пропорциональна первой степени расстояния, а не второй. Это правильная формула для силы Кулона в двух измерениях.

\item Считайте, что силы гравитации между шариками пренебрежимо малы по сравнения с элекрическими силами. Их можно не учитывать.

\item Столкновения шариков друг с другом можно тоже не учитывать. Шарики взаимодействуют только посредством силы Кулона.

\item На границах окна поставьте стенки. Шарики должны упруго отскакивать от стенок.

\item Расчёт ускорений, скоростей и положений всех шариков проводите с шагом $\Delta t$. В этой задаче можно считать, что $\Delta t$ постоянна и равна $1 / fps$. Где значение количества кадров в секунду ($fps$) задаётся с помощью метода \texttt{setFramerateLimit}. 

\item Если два шарика подойдут слишком близко друг к другу, то сила взаимодействия может стать очень большой. Это приведёт к большой погрешности в вычислениях из-за того, что $\Delta t$ больше характерного времени изменения силы взаимодействия между шариками. В результате этого шарики начнут чрезмерно быстро двигаться. Чтобы исправить этот баг просто сделайте так, чтобы сила взаимодействия шариков была равна нулю, если расстояние между шариками меньше некоторой величины.

\item Начальные значения масс, зарядов, положений шариков задайте случайным образом из некоторых диапазонов. Начальные значения скоростей равны нулю.

\item Цвет шарика должен зависит от его заряда. Положительно заряженные шарики рисуйте красным цветом, а отрицательно зарженные -- синим.

\item (\textbf{*}) Добавьте возможно добавлять шарики, используя мышь. При нажатии левой кнопки мыши, в том месте, где находится курсор, должен создаваться шарик с маленькой массой и с отрицательным зарядом. При нажатии правой кнопки мыши должен создаваться шарик с очень большой  массой и положительным зарядом.
\end{itemize}

\end{document}