\documentclass{article}
\usepackage[utf8x]{inputenc}
\usepackage{ucs}
\usepackage{amsmath} 
\usepackage{amsfonts}
\usepackage{upgreek}
\usepackage[english,russian]{babel}
\usepackage{graphicx}
\usepackage{float}
\usepackage{textcomp}
\usepackage{hyperref}
\usepackage{geometry}
  \geometry{left=2cm}
  \geometry{right=1.5cm}
  \geometry{top=1cm}
  \geometry{bottom=2cm}
\usepackage{tikz}
\usepackage{ccaption}
\usepackage{multicol}

\usepackage{listings}
%\setlength{\columnsep}{1.5cm}
%\setlength{\columnseprule}{0.2pt}


\begin{document}
\pagenumbering{gobble}

\lstset{
  language=C++,                % choose the language of the code
  basicstyle=\linespread{1.1}\ttfamily,
  columns=fixed,
  fontadjust=true,
  basewidth=0.5em,
  keywordstyle=\color{blue}\bfseries,
  commentstyle=\color{gray},
  stringstyle=\ttfamily\color{orange!50!black},
  showstringspaces=false,
  %numbers=false,                   % where to put the line-numbers
  numbersep=5pt,
  numberstyle=\tiny\color{black},
  numberfirstline=true,
  stepnumber=1,                   % the step between two line-numbers.        
  numbersep=10pt,                  % how far the line-numbers are from the code
  backgroundcolor=\color{white},  % choose the background color. You must add \usepackage{color}
  showstringspaces=false,         % underline spaces within strings
  captionpos=b,                   % sets the caption-position to bottom
  breaklines=true,                % sets automatic line breaking
  breakatwhitespace=true,         % sets if automatic breaks should only happen at whitespace
  xleftmargin=.2in,
  extendedchars=\true,
  keepspaces = true,
}
\lstset{literate=%
   *{0}{{{\color{red!20!violet}0}}}1
    {1}{{{\color{red!20!violet}1}}}1
    {2}{{{\color{red!20!violet}2}}}1
    {3}{{{\color{red!20!violet}3}}}1
    {4}{{{\color{red!20!violet}4}}}1
    {5}{{{\color{red!20!violet}5}}}1
    {6}{{{\color{red!20!violet}6}}}1
    {7}{{{\color{red!20!violet}7}}}1
    {8}{{{\color{red!20!violet}8}}}1
    {9}{{{\color{red!20!violet}9}}}1
}

\title{Семинар \#11: События. Классные задачи.\vspace{-5ex}}\date{}\maketitle

\section*{Задачи}

\begin{itemize}
\item \textbf{Перетаскивание:} В папке \texttt{1draggable/} содержится заготовка исходного кода для этого задания. Эта программа просто рисует прямоугольник на экране. Сделайте его перетаскиваемым мышью. При нажатии на него и последующим движении мыши он должен начать двигаться вместе с курсором. При отпускании мыши должен остаться на месте.

\item \textbf{Класс \texttt{Draggable}:} Создайте класс \texttt{Draggable}, который будет описывать прямоугольник, который можно перетаскивать мышкой.

\item \textbf{Карточки:} Создайте 20 прямоугольников случайного цвета, но одинакового размера, так чтобы все они были перетаскивыемыми. Если карточки перекрываются, то перетаскиваться должна верхняя карточка. Используйте класс \texttt{Draggable}.


\item \textbf{Кнопка:} Создайте кнопку. Логика работы должна этой кнопки аналогичной логике работы обычной кнопки в ОС Windows:
\begin{itemize}
\item Кнопка представляет собой прямоугольник некоторого цвета и с текстом внутри.
\item Изначально кнопка имеет некоторый заданный цвет.
\item При наведении курсора мыши на кнопку, её цвет меняется.
\item При нажатии и зажатии левой кнопки мыши(ЛКМ) над кнопкой, её цвет меняется.
\item При отпускании ЛКМ, если курсор всё ещё находится на прямоугольнике, происходит некоторое действие (например, печать в консоль).
\item В иных случая действие не происходит (например, если мы зажали ЛКМ вне кнопки и отпустили над кнопкой, или если мы зажали ЛКМ над кнопкой и отпустили вне кнопки).
\end{itemize}

\item \textbf{Класс кнопки:}  Напишите класс \texttt{Button}, который будет описывать кнопку. 
\begin{itemize}
\item Создайте 1 круг. Сделайте так, чтобы при нажатии на кнопку цвет круга менялся бы на случайный.
\item Создайте 4 кнопки. Сделайте так, чтобы при нажатии на эти кнопки положение круга смещалось на 10 пикселей в одном из 3-х направлений (влево, вправо, вверх, вниз).
\end{itemize}
	
\item \textbf{Флажки:} Напишите класс \texttt{Checkbox}, который будет описывать флажок. Флажок должен включать в себя квадратик, на который можно нажимать и менять состояние флажка (вкл/выкл), а также текст рядом с этим квадратиком. Создайте несколько флажков и кнопку. При нажатии на кнопку в консоль должно печататься тексты всех включенных флажков.

\item \textbf{Контекстное меню:} Напишите класс \texttt{ContextMenu}, описывающий контекстное меню.
Создайте круг. И добавьте опции в контекстное меню так, чтобы можно было менять цвет и размер круга.
\end{itemize}




\end{document}