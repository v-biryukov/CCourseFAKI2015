\documentclass{article}
\usepackage[utf8x]{inputenc}
\usepackage{ucs}
\usepackage{amsmath} 
\usepackage{amsfonts}
\usepackage{upgreek}
\usepackage[english,russian]{babel}
\usepackage{graphicx}
\usepackage{float}
\usepackage{textcomp}
\usepackage{hyperref}
\usepackage{geometry}
  \geometry{left=2cm}
  \geometry{right=1.5cm}
  \geometry{top=1cm}
  \geometry{bottom=2cm}
\usepackage{tikz}
\usepackage{ccaption}
\usepackage{multicol}

\usepackage{listings}


\begin{document}
\pagenumbering{gobble}

\lstset{
  language=C++,                % choose the language of the code
  basicstyle=\linespread{1.1}\ttfamily,
  columns=fixed,
  fontadjust=true,
  basewidth=0.5em,
  keywordstyle=\color{blue}\bfseries,
  commentstyle=\color{gray},
  stringstyle=\ttfamily\color{orange!50!black},
  showstringspaces=false,
  %numbers=false,                   % where to put the line-numbers
  numbersep=5pt,
  numberstyle=\tiny\color{black},
  numberfirstline=true,
  stepnumber=1,                   % the step between two line-numbers.        
  numbersep=10pt,                  % how far the line-numbers are from the code
  backgroundcolor=\color{white},  % choose the background color. You must add \usepackage{color}
  showstringspaces=false,         % underline spaces within strings
  captionpos=b,                   % sets the caption-position to bottom
  breaklines=true,                % sets automatic line breaking
  breakatwhitespace=true,         % sets if automatic breaks should only happen at whitespace
  xleftmargin=.2in,
  extendedchars=\true,
  keepspaces = true,
}
\lstset{literate=%
   *{0}{{{\color{red!20!violet}0}}}1
    {1}{{{\color{red!20!violet}1}}}1
    {2}{{{\color{red!20!violet}2}}}1
    {3}{{{\color{red!20!violet}3}}}1
    {4}{{{\color{red!20!violet}4}}}1
    {5}{{{\color{red!20!violet}5}}}1
    {6}{{{\color{red!20!violet}6}}}1
    {7}{{{\color{red!20!violet}7}}}1
    {8}{{{\color{red!20!violet}8}}}1
    {9}{{{\color{red!20!violet}9}}}1
}


\title{Проект \textit{Арканоид}.\vspace{-5ex}}\date{}\maketitle

В файле \texttt{arkanoid.cpp} лежит заготовка игры Arkanoid, написанная на \texttt{C++} с использованием библиотеки SFML. Ваша задача заключается в том, чтобы доработать эту игру, добавив различные виды бонусов, блоков, шариков и ракеток.

\section*{Описание классов игры}
Классы программы:
\begin{itemize}
\item Класс \texttt{Ball} -- описывает поведение одного шарика
\item Класс \texttt{Paddle} -- описывает поведение ракетки
\item Класс \texttt{Brick} -- описывает поведение блока
\item Класс \texttt{BrickGrid} -- описывает поведение сетки из блоков
\item Класс \texttt{Bonus} -- описывает поведение бонуса
\item Класс \texttt{Arkanoid} -- описывает поведение всей игры.
\end{itemize}


\subsection*{Класс \texttt{Paddle}}
Класс \texttt{Paddle} описывает свойства и поведение ракетки. Основные методы этого класса:
\begin{itemize}
\item Конструктор \texttt{Paddle(sf::Vector2f position, sf::Vector2f size)}. Аргумент  \texttt{position} задаёт центр ракетки, а \texttt{size} -- её размер.
\item Метод \texttt{sf::FloatRect getBorder() const} возвращает прямоугольник границ ракетки.
\item Метод \texttt{void draw(sf::RenderWindow\& window)} -- рисует ракетку на окно \texttt{window}
\end{itemize}

\subsection*{Класс \texttt{Ball}}
Класс \texttt{Ball} описывает свойства и поведение шарика. Основные методы этого класса:
\begin{itemize}
\item Конструктор \texttt{Ball(float radius, sf::Vector2f position, sf::Vector2f velocity)}
\item Метод \texttt{void update(float dt)} передвигает шарик в соответствеии его скорости
\item Метод \texttt{void draw(sf::RenderWindow\& window)} -- рисует шарик в окне \texttt{window}
\item Метод \texttt{std::pair<sf::Vector2f, bool> findClosestPoint(const sf::FloatRect\& rect) const} -- находит ближайшую точку прямоугольника, задаваемому объектом типа \texttt{sf::FloatRect}, от центра шарика. Возвращает эту точку и булевое значение. Булевое значение показывает, пересекается ли шарик с этим прямоугольников.
\item Метод \texttt{bool handleRectCollision(const sf::FloatRect\& rect)} -- обрабатывает упругое столкновение шарика с прямоугольником. Задаёт новые положения и скорости шарика. Возвращает \texttt{true}, если столкновение произошло и \texttt{false} иначе.

\item Метод \texttt{void handleWallsCollision(sf::FloatRect boundary)} -- обрабатывает упругое столкновение шарика со стенками.
\item Метод \texttt{std::pair<int, int> handleBrickGridCollision(const BrickGrid\& brickGrid)} -- обрабатывает упругое столкновение шарика с сеткой блоков. Если произошло столкновение с блоком, то возвращает координаты блока в сетке блоков. Если столкновение не произошло, то возвращаем пару \texttt{{-1, -1}}.
\item Метод \texttt{void handlePaddleCollision(const Paddle\& paddle)} -- обрабатывает столкновение с ракеткой. Столкновение не упругое. Угол отражения зависит от места на ракетке, куда стукнулся шарик. Если шарик стукнулся в левую часть ракетки, то он должен полететь влево. Если же шарик стукнулся в правую часть ракетки, то вправо.
\end{itemize}

\subsection*{Класс \texttt{BrickGrid}}
Класс \texttt{BrickGrid} описывает свойства и поведение прямоугольной сетки блоков. Находить пересечение шарика с такой сеткой можно гораздо быстрее, чем если бы все блоки хранились поотдельности и их положение задавалось бы произвольными координатами. Благодаря регулярной сетке можно сразу определить с какими блоками может пересечься шарик, а не проверять пересечение всех шариков со всеми блоками.\\

Поля класса \texttt{BrickGrid}:
\begin{itemize}
\item \texttt{sf::FloatRect m\_border} -- прямоугольник, задающий область в которой находится сетка
\item \texttt{int m\_numBrickColumns} -- размер сетки по горизонтали
\item \texttt{int m\_numBrickRows} -- размер сетки по вертикали
\item \texttt{std::vector<Brick> m\_bricks} -- вектор всех блоков, размер вектора \texttt{m\_numBrickColumns * m\_numBrickRows}.
\item \texttt{sf::RectangleShape m\_brickShape} -- фигура SFML для отрисовки блоков.
\item \texttt{int m\_numActiveBricks} -- текущее количество активных блоков
\end{itemize}

Методы класса \texttt{BrickGrid}:
\begin{itemize}
\item Конструктор \texttt{BrickGrid(sf::FloatRect borders, int numBrickColumns, int numBrickRows)}
\item Метод \texttt{sf::Vector2f getBrickSizes() const} -- возвращает размеры одного блока
\item Метод \texttt{void deactivateBrick(std::pair<int, int> indexes)} -- выключает блок с соответствующими индексами.
\item Метод \texttt{void draw(sf::RenderWindow\& window)} -- рисует сетку блоков на окне \texttt{window}.
\end{itemize}

\subsection*{Класс \texttt{Bonus}}
Класс \texttt{Bonus} описывает свойства бонуса. В базовой версии программы тип бонуса будет только один -- бонус, который утраивает шарики. Данный класс будет являться дружественным основному классу \texttt{Arkanoid}, чтобы бонус мог менять состояние игры.\\

Поля класса \texttt{Bonus}:
\begin{itemize}
\item \texttt{sf::Vector2f m\_position} -- положение центра бонуса
\item \texttt{float m\_time} -- время, прошедшее с создания бонуса в секундах
\end{itemize}

Методы класса \texttt{Bonus}:
\begin{itemize}
\item Конструктор \texttt{Bonus(sf::Vector2f position))}
\item Метод \texttt{void update(float dt)} -- перемещает бонус.
\item Метод \texttt{void draw(sf::RenderWindow\& window) const} -- рисует бонус на окне \texttt{window}
\item Метод \texttt{void activate(Arkanoid\& game)} -- активирует бонус, изменяет состояние игры.
\item Метод \texttt{bool isColiding(const Paddle\& paddle) const} -- проверяет произошло ли столкновение бонуса с ракеткой.
\end{itemize}

\newpage
\subsection*{Класс \texttt{Arkanoid}}
Класс \texttt{Arkanoid} -- основной класс игры, описывает поведение игры и взаимодействие всех объектов игры.\\

Поля класса \texttt{Arkanoid}:
\begin{itemize}
\item \texttt{double m\_time} -- время, прошедшее с начала игры в секундах
\item \texttt{sf::FloatRect m\_border} -- прямоугольник, задающий границы игрового поля
\item \texttt{std::list<Ball> m\_balls} -- связный список всех шариков (объектов типа \texttt{Ball}). Связный список используется так, как в процессе игры будет происходить частое добавление и удаление объектов из этого списка. В качестве альтернативы, можно было использовать \texttt{std::vector}, но удалять/добавлять шарики в конце каждого кадра.
\item \texttt{BrickGrid m\_brickGrid} -- объект, задающий состояние сетки блоков
\item \texttt{Paddle m\_paddle} -- объект, задающий состояние ракетки.
\item \texttt{GameState m\_gameState} -- состояние игры. Может принимать следующие значения:
\begin{itemize}
\item \texttt{GameState::stuck} -- начало игры, когда шарик прикреплён к ракетке
\item \texttt{GameState::running} -- процесс игры
\item \texttt{GameState::endLose} -- окончание игры (поражение)
\item \texttt{GameState::endWin} -- окончание игры (победа)
\end{itemize}
\item \texttt{int m\_numLives} -- текущее количество жизней.
\item \texttt{std::list<Bonus*> m\_bonuses} -- связный список указателей на бонусы. Указатели используются для реализации полиформизма. Так как в будущем мы хотим сделать несколько вариантов бонусов.
\item \texttt{float m\_bonusProbability} -- вероятность выпадения бонуса после разрушения одного блока.
\item \texttt{Ball m\_initialBall} -- объект шарика, используемый для рисования шарика на ракетке в состоянии \texttt{GameState::stuck}. Также этот объект используется для рисования шариков в полоске количества жизней.
\item \texttt{sf::Text m\_endText} -- текст, используемый для отображения сообщения о победе/поражении.
\end{itemize}

Методы класса \texttt{Arkanoid}:
\begin{itemize}
\item Конструктор \texttt{Arkanoid(sf::FloatRect border, sf::Font\& font)}
\item \texttt{void addBall(const Ball\& ball)} -- добавляет новый шарик в игру.
\item \texttt{void update(const sf::RenderWindow\& window, float dt)} -- функция, которая вызывается каждый кадр. Она перемещает все объекты, обрабатывает столкновение всех объектов и меняет состояние игры.
\item \texttt{void draw(sf::RenderWindow\& window)} -- рисует все объекты игры на окно \texttt{window}.
\item \texttt{void onMousePressed(sf::Event\& event)} -- обрабатывает состояние нажатия мыши.
\end{itemize}


\newpage

\section*{Задача 1: Бонусы}
Используйте наследование и полиморфизм и добавьте в игру следующие бонусы:
\begin{itemize}
\item Бонус, который увеличивает количество шариков в 3 раза (этот бонус уже реализован)
\item Бонус, который увеличивает размер ракетки
\item Бонус, который уменьшает размер ракетки
\item Бонус, который временно замедляет движение шариков
\item Бонус, который временно добавляет ''пол'' под ракеткой, от которого могут отскакивать шарики. В этом случае шарики временно вообще не могут упасть вниз.
\item Бонус, который временно делает все шарики красными. Красные шарики пробивают блоки насквозь, то есть не отскакивают от них, а просто уничтожают их и проходят дальше без изменения скорости.
\item Бонус, который временно делает ракетку зелёной. Шарики прилепают к зелёной ракетке и остаются на ней пока пользователь не нажмёт левую кнопку мыши. 
\end{itemize}

Для этого сделайте класс \texttt{Bonus} абстрактным и отнаследуйте от него классы конкретных бонусов. Например класс \texttt{EnlargePaddleBonus} будет описывать класс бонуса, увеличивающего размер ракетки. Этот класс должен наследоваться от абстрактного класса \texttt{Bonus}. Указатели из списка \texttt{m\_bonuses} класса \texttt{Arkanoid} будут указывать на разные типы бонусов. Вам придётся немного изменить и другие классы, чтобы некоторые бонусы работали.

\section*{Задача 2: Блоки}
Используйте наследование и полиморфизм и добавьте в игру разные виды блоков:
\begin{itemize}
\item Обычный блок, разрушается с одного раза (этот блок уже реализован)
\item Блок, который разрушается только с трёх ударов. За исключением когда по нему ударил красный шарик, тогда блок разрушается сразу.
\item Неразрушаемый блок. Не разрушается от шариков. Для победы в игре уничтожать эти блоки не нужно.
\item Взрывающийся блок. Блок, который при столкновении с шариком активируется и через некоторое короткое время взрывается и уничтожает соседние блоки. Если один из соседний блоков тоже взрывающийся, то он должен также взорваться. Создать также простую анимацию взрыва (например, можно нарисовать круг на короткое время в месте взрыва).
\end{itemize}

\section*{Задача 3: Снаряд (\texttt{Bullet})}
Добавьте в игру новый бонус, который будет временно давать возможность ракетке стрелять. При нажатии левой кнопки мыши ракетка должна выстреливать вверх по 2 снаряда с левого и правого конца ракетки. При столкновении с блоками эти снаряды должны уничтожать блоки. Для этого создайте новый класс \texttt{Bullet} который будет описывать поведение снаряда.

\section*{Задача 4: Уровень}
Создайте новый класс под названием Уровень (\texttt{Level}). В этом классе должно храниться расположение и типы всех блоков. Класс должен уметь загружать уровень из файла. Используйте этот класс, чтобы создать в игре несколько разных уровней.

\end{document}