\documentclass{article}
\usepackage[english,russian]{babel}
\usepackage{textcomp}
\usepackage{geometry}
  \geometry{left=2cm}
  \geometry{right=1.5cm}
  \geometry{top=1.5cm}
  \geometry{bottom=2cm}
\usepackage{tikz}
\usepackage{listings}
\usepackage{fancyvrb}
\usepackage{xcolor}
\usepackage{multicol}

\begin{document}
\pagenumbering{gobble}

\lstset{
  language=C++,
  basicstyle=\linespread{1.1}\ttfamily,
  columns=fixed,
  fontadjust=true,
  basewidth=0.5em,
  keywordstyle=\color{blue}\bfseries,
  commentstyle=\color{gray},
  texcl=true,
  stringstyle=\ttfamily\color{orange!50!black},
  showstringspaces=false,
  %numbers=false,
  numbersep=5pt,
  numberstyle=\tiny\color{black},
  numberfirstline=true,
  stepnumber=1,      
  numbersep=10pt,
  backgroundcolor=\color{white},
  showstringspaces=false,
  captionpos=b,
  breaklines=true
  breakatwhitespace=true,
  xleftmargin=.2in,
  extendedchars=\true,
  keepspaces = true,
  tabsize=4,
  upquote=true,
}
\lstset{ literate={~}{{\raisebox{0.5ex}{\texttildelow}}}{1}}

\renewcommand{\thesubsection}{\arabic{subsection}}
\makeatletter
\def\@seccntformat#1{\@ifundefined{#1@cntformat}%
   {\csname the#1\endcsname\quad}%    default
   {\csname #1@cntformat\endcsname}}% enable individual control
\newcommand\section@cntformat{}     % section level 
\newcommand\subsection@cntformat{Задача \thesubsection.\space} % subsection level
\newcommand\subsubsection@cntformat{\thesubsubsection.\space} % subsubsection level
\makeatother

\title{Семинар \#2: Наследование. Домашнее задание.\vspace{-5ex}}\date{}\maketitle


\subsection{Фигуры}
В файле \texttt{shape.cpp} написаны простейшие классы \texttt{Circle}, \texttt{Rectangle} и \texttt{Triangle}, описывающие геометрические фигуры: круг, прямоугольник и треугольник. При написании этих классов наследование не было использовано, в следствии чего некоторые поля и методы одинаковы у всех классов (поле \texttt{mPosition} и методы \texttt{getPosition} и \texttt{setPosition}).

\begin{itemize}
\item Создайте класс \texttt{Shape}, который бы описывал абстрактную фигуру и содержал бы поля и методы, общие для всех фигур. Измените код классов \texttt{Circle}, \texttt{Rectangle} и \texttt{Triangle} так, чтобы они наследовались от класса \texttt{Shape}. Общие для всех фигур поля и методы должны содержаться только в классе \texttt{Shape}, но не в классах-наследниках.

\item Добавьте метод \texttt{void move(Vector2f change)} в класс \texttt{Shape}. Этот метод должен изменять поле \texttt{mPosition} на значение \texttt{change}. Так как остальные классы наследуются от \texttt{Shape}, то этот метод можно будет вызвать у всех объектов дочерних классов. Протестируйте этот метод, изменив положения объектов дочерних классов.
\end{itemize}


\subsection{Приведение типов при наследовании}
Пусть есть два следующих класса:
\begin{lstlisting}
struct Alice
{
	int x;
	void func() const {std::cout << "Alice " << x << std::endl;}
};

struct Bob : public Alice
{
	int y;
	void func() const {std::cout << "Bob " << x << " " << y << std::endl;}
};
\end{lstlisting}

Определите, скомпилируется ли следующий код, использующий эти классы, и, если скомпилируется, то что будет напечатано в следующих программах:
\begin{enumerate}
\item \begin{Verbatim}[commandchars=\\\{\}]
Alice a \{10\};
Bob b \{20, 30\};
a = b;
a.func();
\end{Verbatim}

\item \begin{Verbatim}[commandchars=\\\{\}]
Alice a \{10\};
Bob b \{20, 30\};
b = a;
b.func();
\end{Verbatim}

\item \begin{Verbatim}[commandchars=\\\{\}]
Bob b \{20, 30\};
Alice* p = &b;
p->func();
\end{Verbatim}

\item \begin{Verbatim}[commandchars=\\\{\}]
Alice a \{10\};
Bob* p = &a;
p->func();
\end{Verbatim}
\end{enumerate}


Пусть есть следующие классы:
\begin{lstlisting}
struct Alice
{
	int x;
	void func() const {std::cout << "Alice " << x << std::endl;}
};

struct Bob : public Alice
{
	int y;
	void func() const {std::cout << "Bob " << x << " " << y << std::endl;}
};

struct Casper : public Bob
{
	int z;
	void func() const {std::cout << "Casper " << x << " " << y << " " << z << std::endl;}
};
\end{lstlisting}

Определите, скомпилируется ли следующий код, использующий эти классы, и, если скомпилируется, то что будет напечатано в следующих программах:
\begin{enumerate}
\setcounter{enumi}{4}
\item \begin{Verbatim}[commandchars=\\\{\}]
Alice a \{10\};
Bob b \{20, 30\};
Casper c \{40, 50, 60\};
a = c;
a.func();
\end{Verbatim}

\item \begin{Verbatim}[commandchars=\\\{\}]
Alice a \{10\};
Bob b \{20, 30\};
Casper c \{40, 50, 60\};
Alice* p = &c;
p->func();
\end{Verbatim}

\item \begin{Verbatim}[commandchars=\\\{\}]
Alice a \{10\};
Bob b \{20, 30\};
Casper c \{40, 50, 60\};
Bob* p = &a;
p->func();
\end{Verbatim}

\item \begin{Verbatim}[commandchars=\\\{\}]
Alice a \{10\};
Bob b \{20, 30\};
Casper c \{40, 50, 60\};
Bob* p = &c;
p->func();
\end{Verbatim}

\end{enumerate}


Для того, чтобы сдать эту задачу нужно создать файл в формате \texttt{.txt} и, используя любой текстовый редактор, записать в него ответы в следующем формате (ответы ниже неверны):
\begin{verbatim}
1) Error
2) Alice 10
3) Bob 20 30
...
\end{verbatim} 
После этого, файл нужно поместить в ваш репозиторий на github.



\subsection{Изменение цвета}
В файле \texttt{draggable.cpp} написан класс \texttt{Draggable}, который описывает передвигаемый курсором мыши прямоугольник. Ваша задача -- написать класс \texttt{DraggableWithColorChange} -- наследник класса \texttt{Draggable}. Новый класс должен также описывать передвигаемый прямоугольник, но, во время передвижения прямоугольника, его цвет должен меняться на другой. Конструктор нового класса будет иметь вид:
\begin{lstlisting}
DraggableWithColorChange(sf::RenderWindow& window, Vector2f position, Vector2f size, Color baseColor, Color dragColor) 
\end{lstlisting}
Где \texttt{baseColor} -- это основной цвет прямоугольника, а \texttt{dragColor} -- цвет прямоугольника при перетаскивании.
Протестируйте этот класс в функции \texttt{main}.




\subsection{Окна}
В файле \texttt{window.cpp} написан класс \texttt{BaseWindow}, описывающий простейшее окно. Это окно состоит из двух прямоугольников. Первый прямоугольник опредяляет границы отрисовки окна, а второй опредяляет границы области, за которую прямоугольник можно перетаскивать.

\begin{itemize}

\item Создай класс \texttt{MessageWindow}, наследник \texttt{BaseWindow}.
У объектов этого класса, помимо функционала \texttt{BaseWindow} должен быть текст, в котором отображается некоторая строка. 

\item Создайте классы \texttt{ErrorWindow} и \texttt{DoneWindow}, наследники класса \texttt{MessageWindow}. Эти два окна должны отличаться от окна \texttt{MessageWindow} только тем, что окно типа Error должно всегда рисоваться оттенком красного цвета, а окно типа Done - оттенком зелёного цвета.

\item Создай класс \texttt{QuestionWindow}, наследник \texttt{BaseWindow}.
У объектов этого класса, помимо функционала \texttt{BaseWindow} должен быть текст, в котором отображается некоторая строка. А также две кнопки внизу: Ok и Cancel. При нажатии на кнопку Ok в консоль должно выводиться строка \texttt{Ok}, а при нажатии на кнопку Cancel -- строка \texttt{Cancel}. Используйте класс \texttt{Button} из файла \texttt{button.hpp} для создания кнопок.

\end{itemize}


        

\end{document}