\documentclass{article}
\usepackage[utf8x]{inputenc}
\usepackage{ucs}
\usepackage{amsmath} 
\usepackage{amsfonts}
\usepackage{marvosym}
\usepackage{wasysym}
\usepackage{upgreek}
\usepackage[english,russian]{babel}
\usepackage{graphicx}
\usepackage{float}
\usepackage{textcomp}
\usepackage{hyperref}
\usepackage{geometry}
  \geometry{left=2cm}
  \geometry{right=1.5cm}
  \geometry{top=1cm}
  \geometry{bottom=2cm}
\usepackage{tikz}
\usepackage{ccaption}
\usepackage{multicol}

\hypersetup{
   colorlinks=true,
   citecolor=blue,
   linkcolor=black,
   urlcolor=blue
}

\usepackage{listings}
%\setlength{\columnsep}{1.5cm}
%\setlength{\columnseprule}{0.2pt}

\usepackage[absolute]{textpos}


\usepackage{colortbl,graphicx,tikz}
\definecolor{X}{rgb}{.5,.5,.5}

\renewcommand{\thesubsection}{\arabic{subsection}}

\begin{document}
\pagenumbering{gobble}
\lstset{
  language=C++,                % choose the language of the code
  basicstyle=\linespread{1.1}\ttfamily,
  columns=fixed,
  fontadjust=true,
  basewidth=0.5em,
  keywordstyle=\color{blue}\bfseries,
  commentstyle=\color{gray},
  stringstyle=\ttfamily\color{orange!50!black},
  showstringspaces=false,
  numbersep=5pt,
  numberstyle=\tiny\color{black},
  numberfirstline=true,
  stepnumber=1,                   % the step between two line-numbers.        
  numbersep=10pt,                  % how far the line-numbers are from the code
  backgroundcolor=\color{white},  % choose the background color. You must add \usepackage{color}
  showstringspaces=false,         % underline spaces within strings
  captionpos=b,                   % sets the caption-position to bottom
  breaklines=true,                % sets automatic line breaking
  breakatwhitespace=true,         % sets if automatic breaks should only happen at whitespace
  xleftmargin=.2in,
  extendedchars=\true,
  keepspaces = true,
}
\newcommand\upquote[1]{\textquotesingle#1\textquotesingle}

\renewcommand{\thesubsection}{\arabic{subsection}}
\makeatletter
\def\@seccntformat#1{\@ifundefined{#1@cntformat}%
   {\csname the#1\endcsname\quad}%    default
   {\csname #1@cntformat\endcsname}}% enable individual control
\newcommand\section@cntformat{}     % section level 
\newcommand\subsection@cntformat{Задача \thesubsection.\space} % subsection level
\newcommand\subsubsection@cntformat{\thesubsubsection.\space} % subsubsection level
\makeatother


\makeatletter
\newcount\my@repeat@count
\newcommand{\myrepeat}[2]{%
  \begingroup
  \my@repeat@count=\z@
  \@whilenum\my@repeat@count<#1\do{#2\advance\my@repeat@count\@ne}%
  \endgroup
}
\makeatother

\title{Семинар \#2: Наследование. Домашнее задание.\vspace{-5ex}}\date{}\maketitle


\subsection{Фигуры}
В файле \texttt{shape.cpp} написаны простейшие классы \texttt{Circle}, \texttt{Rectangle} и \texttt{Triangle}, описывающие геометрические фигуры: круг, прямоугольник и треугольник. При написании этих классов наследование не было использовано, в следствии чего некоторые поля и методы одинаковы у всех классов (поле \texttt{mPosition} и методы \texttt{getPosition} и \texttt{setPosition}).

\begin{itemize}
\item Создайте класс \texttt{Shape}, который бы описывал абстрактную фигуру и содержал бы поля и методы, общие для всех фигур. Измените код классов \texttt{Circle}, \texttt{Rectangle} и \texttt{Triangle} так, чтобы они наследовались от класса \texttt{Shape}. Общие для всех фигур поля и методы должны содержаться только в классе \texttt{Shape}, но не в классах-наследниках.

\item Добавьте метод \texttt{void move(Vector2f change)} в класс \texttt{Shape}. Этот метод должен изменять поле \texttt{mPosition} на значение \texttt{change}. Так как остальные классы наследуются от \texttt{Shape}, то этот метод можно будет вызвать у всех объектов дочерних классов. Протестируйте этот метод, изменив положения объектов дочерних классов.
\end{itemize}




\subsection{Изменение цвета}
В файле \texttt{draggable.cpp} написан класс \texttt{Draggable}, который описывает передвигаемый курсором мыши прямоугольник. Ваша задача -- написать класс \texttt{DraggableWithColorChange} -- наследник класса \texttt{Draggable}. Новый класс должен также описывать передвигаемый прямоугольник, но, во время передвижения прямоугольника, его цвет должен меняться на другой. Конструктор нового класса будет иметь вид:
\begin{lstlisting}
DraggableWithColorChange(Vector2f position, Vector2f size, Color baseColor, Color dragColor) 
\end{lstlisting}
Где \texttt{baseColor} -- это основной цвет прямоугольника, а \texttt{dragColor} -- цвет прямоугольника при перетаскивании.
Протестируйте этот класс в функции \texttt{main}.




\subsection{Окна}
В файле \texttt{window.cpp} написан класс \texttt{BaseWindow}, описывающий простейшее окно. Это окно состоит из двух прямоугольников. Первый прямоугольник опредяляет границы отрисовки окна, а второй опредяляет границы области, за которую прямоугольник можно перетаскивать.

\begin{itemize}
\item Создай класс \texttt{QuestionWindow}, наследник \texttt{BaseWindow}.
У объектов этого класса, помимо функционала BaseWindow должен быть текст, в котором отображается некоторая строка. А также две кнопки внизу: Ok и Cancel. При нажатии на кнопку Ok в консоль должно выводиться строка \texttt{Ok}, а при нажатии на кнопку Cancel -- строка \texttt{Cancel}. Используйте класс \texttt{Button} из файла \texttt{button.hpp} для создания кнопок.

\item Создай класс \texttt{MessageWindow}, наследник \texttt{BaseWindow}.
У объектов этого класса, помимо функционала \texttt{BaseWindow} должен быть текст, в котором отображается некоторая строка. Также такое окно должно быть одним из 2-х типов: Error, Done. Окно типа Error должно всегда рисоваться оттенком красного цвета, а окно типа Done - оттенком зелёного цвета.

\item В данной реализации класс \texttt{BaseWindow} использует класс \texttt{Draggable} через композицию (то есть объект класса \texttt{Draggable} является полем класса \texttt{BaseWindow}). Измените класс \texttt{BaseWindow} так, чтобы он получал все возможности класса \texttt{Draggable} через наследование. То есть нужно переписать класс \texttt{BaseWindow}, сделав его наследником класса \texttt{Draggable}. 

\end{itemize}


        

\end{document}