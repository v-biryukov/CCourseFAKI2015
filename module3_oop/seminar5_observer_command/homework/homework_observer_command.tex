\documentclass{article}
\usepackage[english,russian]{babel}
\usepackage{textcomp}
\usepackage{geometry}
  \geometry{left=2cm}
  \geometry{right=1.5cm}
  \geometry{top=1.5cm}
  \geometry{bottom=2cm}
\usepackage{tikz}
\usepackage{listings}
\usepackage{fancyvrb}
\usepackage{xcolor}
\usepackage{multicol}

\begin{document}
\pagenumbering{gobble}

\lstset{
  language=C++,
  basicstyle=\linespread{1.1}\ttfamily,
  columns=fixed,
  fontadjust=true,
  basewidth=0.5em,
  keywordstyle=\color{blue}\bfseries,
  commentstyle=\color{gray},
  texcl=true,
  stringstyle=\ttfamily\color{orange!50!black},
  showstringspaces=false,
  %numbers=false,
  numbersep=5pt,
  numberstyle=\tiny\color{black},
  numberfirstline=true,
  stepnumber=1,      
  numbersep=10pt,
  backgroundcolor=\color{white},
  showstringspaces=false,
  captionpos=b,
  breaklines=true
  breakatwhitespace=true,
  xleftmargin=.2in,
  extendedchars=\true,
  keepspaces = true,
  tabsize=4,
  upquote=true,
}
\lstset{ literate={~}{{\raisebox{0.5ex}{\texttildelow}}}{1}}

\renewcommand{\thesubsection}{\arabic{subsection}}
\makeatletter
\def\@seccntformat#1{\@ifundefined{#1@cntformat}%
   {\csname the#1\endcsname\quad}%    default
   {\csname #1@cntformat\endcsname}}% enable individual control
\newcommand\section@cntformat{}     % section level 
\newcommand\subsection@cntformat{Задача \thesubsection.\space} % subsection level
\newcommand\subsubsection@cntformat{\thesubsubsection.\space} % subsubsection level
\makeatother

\title{Семинар \#5: Наблюдатель и Команда. Домашнее задание.\vspace{-5ex}}\date{}\maketitle


\subsection{Ещё один наблюдатель}
Начальный код для задачи в файле \texttt{observer.cpp}. Добавьте ещё один объект, который будет наблюдателем слайдера. При изменении положения слайдера, должно меняться вертикальное положение объекта-наблюдателя.


\subsection{Два субъекта}
Начальный код для задачи в файле \texttt{observer.cpp}.
Измените код таким образом, чтобы в программе было 2 субъекта-слайдера. При изменении положения одного слайдера все объекты должны вращаться. При изменении положения второго слайдера, все объекты должны менять цвет.


\subsection{Случайные цвета}
Начальный код для этой и последующих задач в файле \texttt{command.cpp}. Создайте новую команду, которая бы меняла цвета всех объектов на случайные. Присоедините эту команду к одной из кнопок.

        
\subsection{Мультикоманда}
Создайте новую команду \texttt{MultiCommand}, которая должна объединять в себе несколько команд. Она должна конструироваться из вектора команд. При исполнении этой команды должны исполняться подряд все входящие в неё команды.


\subsection{Задержка}
Добавьте возможность создания команд с задержкой. Исполнение команды должно происходить с некоторой задаваемой задержкой. При этом, во время этой задержки, можно исполнять другие команды. Используйте классы \texttt{sf::Time} и \texttt{sf::Clock}.


\subsection{Отмена команды}
Добавьте новую кнопку, которая будет отменять последнюю исполненную команду.
При нескольких последовательный нажатиях на эту кнопку, должны отмениться несколько последних команд.
Для решения этой задачи измените интерфейс команды на следующий:
\begin{lstlisting}
class Command
{
public:
    virtual void execute() = 0;
    virtual void unexecute() = 0;
    virtual ~Command() {}
};
\end{lstlisting}
Где виртуальный метод \texttt{unexecute} должен отменять действия, которые прошли в методе \texttt{execute}.



\end{document}