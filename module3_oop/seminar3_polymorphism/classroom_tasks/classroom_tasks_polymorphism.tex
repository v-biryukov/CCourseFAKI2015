\documentclass{article}
\usepackage[english,russian]{babel}
\usepackage{textcomp}
\usepackage{geometry}
  \geometry{left=2cm}
  \geometry{right=1.5cm}
  \geometry{top=1.5cm}
  \geometry{bottom=2cm}
\usepackage{tikz}
\usepackage{multicol}
\usepackage{listings}
\pagenumbering{gobble}

\lstdefinestyle{csMiptCppStyle}{
  language=C++,
  basicstyle=\linespread{1.1}\ttfamily,
  columns=fixed,
  fontadjust=true,
  basewidth=0.5em,
  keywordstyle=\color{blue}\bfseries,
  commentstyle=\color{gray},
  texcl=true,
  stringstyle=\ttfamily\color{orange!50!black},
  showstringspaces=false,
  %numbers=false,
  numbersep=5pt,
  numberstyle=\tiny\color{black},
  numberfirstline=true,
  stepnumber=1,      
  numbersep=10pt,
  backgroundcolor=\color{white},
  showstringspaces=false,
  captionpos=b,
  breaklines=true
  breakatwhitespace=true,
  xleftmargin=.2in,
  extendedchars=\true,
  keepspaces = true,
  tabsize=4,
  upquote=true,
  emph = {override, final},
  emphstyle={\color{blue}\bfseries},
}
\lstdefinestyle{csMiptCppBorderStyle}{
  style=csMiptCppStyle,
  framexleftmargin=5mm, 
  frame=shadowbox, 
  rulesepcolor=\color{gray}
}

\lstset{style=csMiptCppStyle}
\lstset{literate={~}{{\raisebox{0.5ex}{\texttildelow}}}{1}}


\begin{document}
\title{Семинар \#2: Полиморфизм \vspace{-5ex}}\date{}\maketitle

\section*{Полиморфизм}
\textbf{Полиморфизм} -- это способность функций обрабатывать данные разных типов.
\subsection*{Статический полиморфизм}
\subsection*{Динамический полиморфизм}


\section*{Виртуальные функции}

\subsection*{Указатели на базовый класс, хранящие адрес объекта производного класса}


\subsection*{Виртуальный деструктор}

\subsection*{\texttt{override} и \texttt{final}}


\subsection*{Полиморфизм и умные указатели}

\newpage
\subsection*{Виртуальные функции в конструкторах и деструкторах}
Если производный класс вызывает конструктор базового класса (это происходит автоматически в перед вызовом конструктора производного класса) и в конструкторе базового класса вызывается виртуальный метод, то вызовется метод базового класса.
Это происходит просто потому что объект производного класса не может вызывать свои методы, так как он ещё не готов к этому. 
\begin{lstlisting}
#include <iostream>
struct Alice 
{
    Alice()     {say();}
    void func() {say();}
    virtual void say() 
    {
    	std::cout << "Alice" << std::endl;
    }
};

struct Bob : public Alice 
{
    void say() override 
    {
    	std::cout << "Bob" << std::endl;
    }
};

int main() 
{
    Bob b;     // Напечатает Alice
    b.func();  // Напечатает Bob
}
\end{lstlisting}
Похожая ситуация будет и при вызове виртуального метода в деструкторе:
\begin{lstlisting}
#include <iostream>
struct Alice 
{
    virtual void say() {std::cout << "Alice" << std::endl;}
    void func() {say();}
    virtual ~Alice()    {say();}
};

struct Bob : public Alice 
{
    void say() override {std::cout << "Bob" << std::endl;};
    ~Bob() {}
};

int main() 
{
    Bob b;
    b.func();  // Напечатает Bob
               // Напечатает Alice при уничтожении объекта
}
\end{lstlisting}

\subsection*{Как программа понимает, какой виртуальный метод вызывать}
Размер объектов полиморфных типов. Скрытый указатель на таблицу виртуальных функций.


\section*{Абстрактные функции}
\subsection*{Чистые виртуальные функции}
\subsection*{Абстрактные классы и интерфейсы}
\subsection*{Контейнер указателей на базовый класс, хранящих адрес объектов производных классов}
\subsection*{pure virtual call и определение чистых виртуальных методов}




\section*{RTTI и \texttt{dynamic\_cast}}
\subsection*{Полиморфный тип}
\subsection*{\texttt{dynamic\_cast}}

\subsubsection*{\texttt{dynamic\_cast} от родителя к ребёнку}
\subsubsection*{\texttt{dynamic\_cast} в бок}

\subsection*{Оператор \texttt{typeid} и класс \texttt{std::type\_info}}





\section*{Реализация механизма виртуальных функций}


\end{document}
