\documentclass[12pt]{article}
\usepackage[utf8]{inputenc}
\usepackage{ucs}
\usepackage{amsmath} 
\usepackage{mathtext}
\usepackage{amsfonts}
\usepackage{upgreek}
\usepackage[english,russian]{babel}
\usepackage{graphicx}
\usepackage{float}
\usepackage{textcomp}
\usepackage{hyperref}
\usepackage{geometry}
  \geometry{left=2cm}
  \geometry{right=1.5cm}
  \geometry{top=1cm}
  \geometry{bottom=2cm}
\usepackage{tikz}
\usepackage{ccaption}
\usepackage{float}
\usepackage{verbatim}
\restylefloat{table}


\begin{document}
\pagenumbering{gobble}

\section*{Задачи:}
\subsection*{Часть А}

\begin{enumerate}
\item \textbf{vimtutor} Пройти туториал по текстовому редактору vim (только Lesson 1 и Lesson 2).
\item \textbf{FizzBuzz} Напишите программу, которая выводит на экран числа от 1 до 100. При этом вместо чисел, кратных трем, программа должна выводить слово «Fizz», а вместо чисел, кратных пяти — слово «Buzz». Если число кратно и 3, и 5, то программа должна выводить слово «FizzBuzz»
\item \textbf{Функции} Написать функции 
\begin{verbatim}
int sum(int a, int b);      // Вычисляет сумму двух чисел
int mult(int a, int b);     // Вычисляет произведение двух чисел
void add42(int * p);        // Увеличивает число, на которое указывает p, на 42
\end{verbatim}.
Использовать все эти функции, чтобы найти значение выражения $(x + 42) * (y + z)$.\\ $x$, $y$ и $z$ считываются из стандартного входа с помощью функции scanf.

\item \textbf{Прямоугольник} Напишите \textbf{функцию}, которая печатает символами * прямоугольную рамку m x n. \\
\null\hspace{1cm}\textit{Входные данные:} \\
\null\hspace{2cm} Целые положительные m и n меньшие 20. \\
\null\hspace{1cm} \textit{Выходные данные:} \\
\null\hspace{2cm} Прямоугольник из символов * \\
\null\hspace{1cm} \textit{Пример:}
\begin{table}[h]
\centering
\begin{tabular}{|l|l|}
\hline
Вход & Выход                                                      \\ \hline
3 4  & \begin{tabular}[c]{@{}l@{}}****\\ * \ \ *\\ ****\end{tabular} \\ \hline
1 1  & *                                                          \\ \hline
\end{tabular}
\end{table}

\item \textbf{Makefile(Прямоугольник)} Вынести описание и реализацию функции из предыдущего задания в отдельные файлы исходного кода .c и .h. Скомпилировать программу с помощью gcc. Написать makefile для этой программы.

\item \textbf{Makefile(Список)} Вынести описание и реализацию структуры данных список из контеста "Как лист увядший падает на душу..." в отдельные файлы исходного кода .c и .h. Написать makefile для одной из задач этого контеста.

\end{enumerate}

\subsection*{Часть B (Домашние задачи)}
\begin{enumerate}
\item \textbf{Binary tree} Решить задачи tree\_1 и tree\_4 в контесте Древо жизни на ejudge.\\
\href{http://93.175.29.68/cgi-bin/new-register?action=211&contest_id=500111}{Ссылка}
\end{enumerate}


\iffalse
\section*{Вопросы:}

\begin{enumerate}

\item Назовите базовые команды командной строки linux
\item Что такое препроцессор?
\item Что делает директива \#include?
\item Что делает директива \#define?
\item Назовите все стандартные целочисленные типы C и их размер.
\item Назовите все стандартные типы с плавающей точкой в C и их размер.
\item Операторы break и continue
\item Что такое прототип функции?
\item Как функция может возвращать значения?
\item Спецификатор типа void
\item Области видимости переменных в функции
\item Передача по ссылке и по значению
\item Аргументы функции main
\item Как объявить константу в C?
\item Объявление массива
\item Как хранятся массивы в памяти
\item Строки в стиле C
\item Функции работы со строками
\item Алгоритмы сортировки вставками, выбором, пузырьком (1 из 3-х)
\item Описание структуры, typedef
\item Что такое указатель? Какой размер переменной указателя? 
\item Как получить адрес переменной? Как получить переменную по адресу?
\item Указатель на 1-й элемент массива, на 5-й.
\item Стэк и куча. Что такое, преемущества и недостатки.
\item Что делает malloc? Выделите память на массив из n элементов типа int.
\item Как освобождать память? Почему плохо не освобождать память?
\item Есть массив A состоящий из 100 элементов. Что будет, если исполнить A[110].
\item Принцип "разделяй и властвуй"
\item Сортировка слиянием или быстрая сортировка
\item Что такое связный список.
\item Сложность добавления и удаления элементов в связный список и в массив.
\item Что такое дерево?
\item Что такое двоичное дерево поиска? 
\item Сложность поиска элемента с помощью двоичного дерева поиска. Сравнение с бинарным поиском.
\end{enumerate}

\fi
\end{document}