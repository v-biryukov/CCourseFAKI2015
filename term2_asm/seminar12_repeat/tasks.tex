\documentclass[12pt]{article}
\usepackage[utf8x]{inputenc} 
\usepackage{ucs}
\usepackage{amsmath} 
\usepackage{mathtext}
\usepackage{amsfonts}
\usepackage{upgreek}
\usepackage[english,russian]{babel}
\usepackage{graphicx}
\usepackage{float}
\usepackage{textcomp}
\usepackage{hyperref}
\usepackage{geometry}
  \geometry{left=2cm}
  \geometry{right=1.5cm}
  \geometry{top=1cm}
  \geometry{bottom=2cm}
\usepackage{tikz}
\usepackage{ccaption}
\usepackage{float}
\usepackage{verbatim}
\restylefloat{table}


\begin{document}
\pagenumbering{gobble}

\section*{Низкоуровневое программирование}

\begin{enumerate}
\item \textbf{Целое число в памяти компьютера} Даны числа \{10, 4156, -1\}. Как эти числа представляются в памяти компьютера? Рассмотрите несколько случаев: разная разрядность числа (16 и 32 бита) и разный порядок байт (little и big endian).
\item \textbf{Вещественное число в памяти компьютера} Дано число 13.6875. Как оно представляется в памяти компьютера согласно стандарту IEEE 754? (число 32-х битное).\\
Напоминание: вещественное число в памяти представляется в виде: $(-1)^s \cdot M \cdot 2^{E}$, где $s$ - знак, $M$ -- мантисса (двоичное дробное чило от 1 до 2), $E$ - порядок.
\item \textbf{Язык ассемблера} напишите программу на языке ассемблера, реализующую следующий код:
\begin{verbatim}
int func(int x, int n)
{
    int result = 0;
    for ( i = n-1; i > 0; i--)
    {
        result = result + x*i;
    }
    return result;
}
\end{verbatim}
\item \textbf{Язык ассемблера 2} Рассмотрим следующий код на языке ассемблера:
\begin{verbatim}
f:
    cmpq	$-12, %rsi
    jne	.L2
    movq	%rsi, %rax
    movq	%rdi, %rsi
    movq	%rax, %rdi
    call	f
    ret
.L2:
    addq    $1, %rdi
    movq	%rsi, %rax
    imulq	%rdi, %rax
    ret

g:
    leaq	-8(%rdi), %rsi
    leaq	(%rdi,%rdi,2), %rdi
    call	f
    ret
\end{verbatim}
 a) Как могли были выглядеть функции f и g написанные на языке C. \newline
 b) Предположим, что функция f была вызвана и при этом в регистре rdi хранилось число -2, а в регистре rsi хранилось число 3. Проследите как будут меняться содержания регистров rdi, rsi и rax на каждом шаге(инструкции) выполнения функции.

\end{enumerate}



\end{document}