\documentclass{article}
\usepackage[english,russian]{babel}
\usepackage{textcomp}
\usepackage{geometry}
  \geometry{left=2cm}
  \geometry{right=1.5cm}
  \geometry{top=1.5cm}
  \geometry{bottom=2cm}
\usepackage{tikz}
\usepackage{listings}
\usepackage{fancyvrb}
\usepackage{xcolor}
\pagenumbering{gobble}

\lstset{
  language=C++,
  basicstyle=\linespread{1.1}\ttfamily,
  columns=fixed,
  fontadjust=true,
  basewidth=0.5em,
  keywordstyle=\color{blue}\bfseries,
  commentstyle=\color{gray},
  texcl=true,
  stringstyle=\ttfamily\color{orange!50!black},
  showstringspaces=false,
  %numbers=false,
  numbersep=5pt,
  numberstyle=\tiny\color{black},
  numberfirstline=true,
  stepnumber=1,      
  numbersep=10pt,
  backgroundcolor=\color{white},
  showstringspaces=false,
  captionpos=b,
  breaklines=true
  breakatwhitespace=true,
  xleftmargin=.2in,
  extendedchars=\true,
  keepspaces = true,
  tabsize=4,
  upquote=true,
}
\lstset{ literate={~}{{\raisebox{0.5ex}{\texttildelow}}}{1}}

\renewcommand{\thesubsection}{\arabic{subsection}}
\makeatletter
\def\@seccntformat#1{\@ifundefined{#1@cntformat}%
   {\csname the#1\endcsname\quad}%    default
   {\csname #1@cntformat\endcsname}}% enable individual control
\newcommand\section@cntformat{}     % section level 
\newcommand\subsection@cntformat{Задача \thesubsection.\space} % subsection level
\newcommand\subsubsection@cntformat{\thesubsubsection.\space} % subsubsection level
\makeatother

\begin{document}
\title{Семинар \#1: Основы C++. Домашнее задание. \\[1ex] \large Пространства имён, ссылки, перегрузка функций, \texttt{std::string} и \texttt{std::vector}. \vspace{-5ex}}\date{}\maketitle

\subsection{Пространства имён}
Пусть есть такой участок кода:
\begin{lstlisting}
namespace mipt
{
	namespace fefm
	{
		struct Point
		{
			int x, y;
		};
	}
	
	namespace frtk
	{
		void print(fefm::Point p)
		{
			std::cout << p.x << " " << p.y << std::endl;
		}
	}
}
\end{lstlisting}
Вам нужно сделать следующее:
\begin{itemize}
\item В функции \texttt{main} cоздать переменную типа \texttt{Point} (из пространства имён \texttt{mipt::fefm}) и инициализировать её поля значениями \texttt{x = 10} и \texttt{y = 20}.
\item Вызвать функцию \texttt{print} из пространства имён \texttt{mipt::frtk}, передав ей созданную структуру.
\end{itemize}
Решите эту задачу тремя способами:
\begin{enumerate}
\item Без использования ключевого слова \texttt{using}.
\item С использованием директив \texttt{using namespace}.
\item С использованием \texttt{using}-объявлений.
\end{enumerate}

\subsection{Куб}
Напишите функцию \texttt{cube}, которая будет принимать одно число типа \texttt{int} по ссылке и возводить это число в куб. Вызовите эту функцию из функции \texttt{main}, чтобы возвести переменную типа \texttt{int} в куб.
\begin{lstlisting}
#include <iostream>
// Тут нужно написать функцию cube

int main()
{
	int a = 5;
	cube(a);
	std::cout << a << std::endl;  // Должно напечатать 125
}
\end{lstlisting}


\subsection{Обмен}
Напишите функцию \texttt{swap}, которая будет обменивать значения двух переменных типа \texttt{int}.
\begin{lstlisting}
#include <iostream>
// Тут нужно написать функцию swap

int main()
{
	int a = 10;
	int b = 20;
	std::cout << a << " " << b << std::endl;  // Должно напечатать 10 20
	
	swap(a, b);
	std::cout << a << " " << b << std::endl;  // Должно напечатать 20 10
}
\end{lstlisting}


\subsection{Возврат ссылки на максимум}
Написан следующий код, в котором используются указатели:
\begin{lstlisting}
#include <iostream>
int* getPointerToMax(int* pa, int* pb)
{
	if (*pa > *pb)
		return pa;
	else
		return pb;
}

int main()
{
	int a = 10;
	int b = 20;
	
	*getPointerToMax(&a, &b) += 1;
	
	std::cout << a << " " << b << std::endl;
}
\end{lstlisting}
Перепишите этот код, но вместо указателей используйте ссылки. В коде не должно остаться ни одного указателя. Название функции \texttt{getPointerToMax} измените на \texttt{getRefToMax}.

\iffalse
\subsection{Передача структуры по ссылке}
Пусть у нас есть следующая структура:
\begin{lstlisting}
struct Book
{
    std::string title;
    int pages;
    float price;
};
\end{lstlisting}
Напишите функцию \texttt{addPrice}, которая будет принимать на вход структуру \texttt{Book} по ссылке и некоторое число \texttt{x} типа \texttt{float}. Эта функция должна увеличивать цену переданной книги на \texttt{x}. Протестируйте эту функцию в функции \texttt{main}.
\fi

\subsection{Передача структуры по константной ссылке}
Пусть у нас есть следующая структура:
\begin{lstlisting}
struct Book
{
    std::string title;
    int pages;
    float price;
};
\end{lstlisting}
Напишите функцию \texttt{isExpensive}, которая будет принимать на вход структуру \texttt{Book} по константной ссылке. Эта функция должна возвращать значение типа \texttt{bool}. Если цена книги больше чем 1000, то функция должна вернуть \texttt{true}, иначе функция должна вернуть \texttt{false}. Протестируйте эту функцию в функции \texttt{main}.

\subsection{Выбор перегрузки}
Пусть есть следующие перегруженные функции:
\begin{lstlisting}
void cat(char x)      {std::cout << "Char" << std::endl;}
void cat(int x)       {std::cout << "Int" << std::endl;}
void cat(long long x) {std::cout << "Long Long" << std::endl;}
\end{lstlisting}
Какая из перегрузок будет выбрана (или будет ошибка), если вызвать \texttt{cat} следующим образом:
\begin{enumerate}
\item \texttt{cat(10)}
\item \texttt{cat(10LL)}
\item \texttt{cat(\textcolor{blue}{static\_cast}<\textcolor{blue}{short}>(10))} 

\item \begin{Verbatim}[commandchars=\\\{\}]
\textcolor{blue}{char} a = 10;
\textcolor{blue}{char} b = 20;
cat(a + b);
\end{Verbatim}

\item \texttt{cat(1.0f)}
\end{enumerate}
Пусть также есть следующие перегруженные функции:
\begin{lstlisting}
void dog(int& x)        {std::cout << "Ref" << std::endl;}
void dog(const int& x)  {std::cout << "CRef" << std::endl;}
\end{lstlisting}
Какая из перегрузок будет выбрана (или будет ошибка), если вызвать \texttt{dog} следующим образом:
\begin{enumerate}
\setcounter{enumi}{5}
\item \begin{Verbatim}[commandchars=\\\{\}]
\textcolor{blue}{int} a = 10;
dog(a);
\end{Verbatim}

\item \begin{Verbatim}[commandchars=\\\{\}]
\textcolor{blue}{const int} a = 10;
dog(a);
\end{Verbatim}

\item \begin{Verbatim}[commandchars=\\\{\}]
dog(10);
\end{Verbatim}

\item \begin{Verbatim}[commandchars=\\\{\}]
\textcolor{blue}{int} a = 10;
dog(static_cast<int>(a));
\end{Verbatim}

\item \begin{Verbatim}[commandchars=\\\{\}]
\textcolor{blue}{int} a = 10;
dog(static_cast<int&>(a));
\end{Verbatim}

\item \begin{Verbatim}[commandchars=\\\{\}]
\textcolor{blue}{int} a = 10;
dog(+a);
\end{Verbatim}

\item \begin{Verbatim}[commandchars=\\\{\}]
\textcolor{blue}{int} func() \{\textcolor{blue}{return} 10;\}
\textcolor{blue}{int} main()
\{
    dog(func());
\}
\end{Verbatim}
\end{enumerate}
Пусть также есть следующие перегруженные функции:
\begin{lstlisting}
void cow(const char* str)        {std::cout << "C-string" << std::endl;}
void cow(const std::string& str) {std::cout << "std::string" << std::endl;}
\end{lstlisting}
Какая из перегрузок будет выбрана (или будет ошибка), если вызвать \texttt{cow} следующим образом:


\begin{enumerate}
\setcounter{enumi}{12}
\item \begin{Verbatim}[commandchars=\\\{\}]
\textcolor{blue}{char} a[10] = \textcolor{orange!50!black}{"Hello"};
cow(a);
\end{Verbatim}

\item \begin{Verbatim}[commandchars=\\\{\}]
std::string a = \textcolor{orange!50!black}{"Hello"};
cow(a);
\end{Verbatim}

\item \begin{Verbatim}[commandchars=\\\{\}]
cow(\textcolor{orange!50!black}{"Hello"});
\end{Verbatim}

\item \begin{Verbatim}[commandchars=\\\{\}]
cow(std::string(\textcolor{orange!50!black}{"Hello"}));
\end{Verbatim}

\item \begin{Verbatim}[commandchars=\\\{\}]
\textcolor{blue}{using namespace} std::string_literals;
\textcolor{blue}{int} main()
\{
    cow(\textcolor{orange!50!black}{"Hello"}s);
\}
\end{Verbatim}

\item \begin{Verbatim}[commandchars=\\\{\}]
std::string a = \textcolor{orange!50!black}{"Hello"};
cow(a.c_str());
\end{Verbatim}
\end{enumerate}
Для того, чтобы сдать эту задачу нужно создать файл в формате \texttt{.txt} и, используя любой текстовый редактор, записать в него ответы в следующем формате (ответы ниже неверны):
\begin{verbatim}
1) Char
2) C-string
3) Error
\end{verbatim} 
После этого, файл нужно поместить в ваш репозиторий на github.

\subsection{Подсчёт символов}
Напишите функцию:
\begin{lstlisting}
void countLetters(const std::string& str, int& numLetters, int& numDigits)
\end{lstlisting}
которая будет принимать на вход строку \texttt{str} и подсчитывать число букв и цифр в этой строке. Количество букв нужно записать по ссылке \texttt{numLetters}, а количество цифр --  по ссылке \texttt{numDigits}. Используйте библиотеку \texttt{cctype}. Вызвать эту функцию из функции \texttt{main}.


\subsection{Добавление скобок}
Напишите функцию \texttt{addBrackets}, которая будет добавлять к строке квадратные скобки. Протестируйте эту функцию на следующем коде:
\begin{lstlisting}
#include <iostream>
#include <string>
int main()
{
	std::string a = "Cat";
	addBrackets(a);
	std::cout << a << std::endl;  // Должно напечатать [Cat]
	
	addBrackets(a);
	std::cout << a << std::endl;  // Должно напечатать [[Cat]]
}
\end{lstlisting}

\subsection{Повтор числа в строке}
Напишите функцию \texttt{repeat}, которое будет принимать на вход некоторое целое число \texttt{n} и возвращать строку, содержащую это число в десятичной записи, повторённое \texttt{n} раз. Протестируйте функцию на следующем коде:
\begin{lstlisting}
#include <iostream>
int main()
{
	std::cout << repeat(5)  << std::endl;  // Должно напечатать 55555
	std::cout << repeat(10) << std::endl;  // Должно напечатать 10101010101010101010
	std::cout << repeat(-1) << std::endl;  // Не должно ничего печатать
}
\end{lstlisting}

\subsection{Доменное имя}
Напишите функцию \texttt{isDomainName}, которая принимает на вход строку и если строка начинается на \texttt{www.} и заканчивается на \texttt{.com}, то эта функция должна вернуть \texttt{true}, иначе \texttt{false}. Используйте методы класса \texttt{std::string}.
\begin{lstlisting}
#include <iostream>
#include <string>
// Тут нужно написать функцию isDomainName

int main()
{
	std::cout << isDomainName("www.google.com") << std::endl;         // Напечатает 1
	std::cout << isDomainName("abc") << std::endl;                    // Напечатает 0
	std::cout << isDomainName("hello.com") << std::endl;              // Напечатает 0
}
\end{lstlisting}



\subsection{Удвоение вектора}
Напишите функцию \texttt{doubling}, которая будет принимать на вход вектор целых чисел и увеличивать вектор в два раза, путём повтора всех элементов.
\begin{lstlisting}
#include <iostream>
#include <string>
#include <vector>
void print(const std::vector<int>& v)
{
    for (std::size_t i = 0; i < v.size(); ++i)
        std::cout << v[i] << " ";
    std::cout << std::endl;
}
// Тут нужно написать функцию doubling

int main()
{
    std::vector<int> v {10, 20, 30};
    doubling(v);
    print(v);  // Должно напечатать 10 20 30 10 20 30
}
\end{lstlisting}


\subsection{Склейка строк}
Напишите функцию \texttt{concatenate}, которая будет принимать вектор из строк и возращать строку -- результат конкатенации всех строк вектора.
\begin{lstlisting}
#include <iostream>
#include <string>
#include <vector>
// Тут нужно написать функцию concatenate

int main()
{
	std::vector<std::string> v {"Cat", "Dog", "Mouse", "Tiger", "Elk"};
	std::cout << concatenate(v) << std::cout // Должно напечатать CatDogMouseTigerElk
}
\end{lstlisting}

\subsection{Вектор префиксов}
Напишите функцию \texttt{prefixes}, которая должна будет принимать на вход строку, а возращать вектор, содержащий все подстроки, начинающиеся с начала строки.
\begin{lstlisting}
#include <iostream>
#include <string>
#include <vector>
void print(const std::vector<std::string>& v)
{
    for (std::size_t i = 0; i < v.size(); ++i)
        std::cout << v[i] << " ";
    std::cout << std::endl;
}
// Тут нужно написать функцию prefixes

int main()
{
    std::vector<std::string> v = prefixes("Mouse");
    print(v);  // Должно напечатать M Mo Mou Mous Mouse
}
\end{lstlisting}

\subsection{Вектор индексов подстрок}
Напишите функцию \texttt{substringIndexes}, которая будет принимать на вход две строки и будет искать индексы всех вхождений второй строки в первую. Функция должна возвращать вектор этих индексов.
\begin{lstlisting}
#include <iostream>
#include <string>
#include <vector>
void print(const std::vector<std::size_t>& v)
{
    for (std::size_t i = 0; i < v.size(); ++i)
        std::cout << v[i] << " ";
    std::cout << std::endl;
}
// Тут нужно написать функцию substringIndexes


int main()
{
    std::vector<std::size_t> v1 = substringIndexes("cat and dog and cat", "cat");
    print(v1);  // Должно напечатать 0 16
    
    std::vector<std::size_t> v2 = substringIndexes("look, cats were here", "cat");
    print(v2);  // Должно напечатать 6
  
    std::vector<std::size_t> v3 = substringIndexes("catcatcatcatcatcat", "cat");
    print(v3);  // Должно напечатать 0 3 6 9 12 15
    
    std::vector<std::size_t> v4 = substringIndexes("dog mouse elephant", "cat");
    print(v4);  // Не должно ничего печатать
}
\end{lstlisting}



\subsection{Получение элемента}
Напишите три перегрузки функции под названием \texttt{get}, которые могли бы принимать некоторый контейнер и индекс. Функции должны возвращать элемент по индексу. Если значение индекса находится вне допустимых значений, функция должна печатать сообщение об ошибке и выходить из программы (используйте для этого функцию \texttt{std::exit} из \texttt{cstdlib}). Нужно написать три перегрузки функций \texttt{get}:
\begin{enumerate}
\item Перегрузка, которая принимает строку \texttt{std::string} и индекс.
\item Перегрузка, которая принимает вектор целых чисел \texttt{std::vector<int>} и индекс.
\item Перегрузка, которая принимает обычный массив целых чисел типа \texttt{int}, его размер и индекс.
\end{enumerate}


\begin{lstlisting}
#include <iostream>
#include <string>
#include <vector>
void print(const std::vector<int>& v)
{
    for (std::size_t i = 0; i < v.size(); ++i)
        std::cout << v[i] << " ";
    std::cout << std::endl;
}

void print(const int* a, std::size_t n)
{
    for (std::size_t i = 0; i < n; ++i)
        std::cout << a[i] << " ";
    std::cout << std::endl;
}
// Тут нужно написать перегрузки функции get


int main()
{
    std::vector<int> v {10, 20, 30, 40, 50};
    get(v, 2) += 1;
    print(v);                     // Напечатает 10 20 31 40 50
    
    std::string s = "Cat";
    get(s, 0) = 'B';
    std::cout << s << std::endl;  // Напечатает Bat
    
    int a[5] = {10, 20, 30, 40, 50};
    get(a, 5, 2) += 1;
    print(a, 5);                  // Напечатает 10 20 31 40 50
    
    get(v, 10) = 0;               // Должен напечатать сообщение об ошибке и выйти из программы
}
\end{lstlisting} 


\newpage
\section*{Необязательные задачи (не входят в ДЗ, никак не учитываются)}
\setcounter{subsection}{0}
\subsection{Измени регистр первой буквы}
Напишите функцию, которая будет принимать на вход строку типа \texttt{std::string} и возвращать строку с изменённым регистром первой буквы. Например, если на вход пришла строка \texttt{"Cat"}, то функция должна вернуть строку \texttt{"cat"}. Если на вход пришла строка \texttt{"dog"}, то функция должна вернуть строку \texttt{"Dog"}. Если на вход пришла пустая строка, то функция должна вернуть простую строку.


\subsection{Удвоение}
Напишите функции, которые будут принимать на вход строку и возвращать строку, повторённую два раза. То есть, если на вход этой функции приходит строка \texttt{"Cat"}, то функция должна вернуть \texttt{"CatCat"}.
При этом нужно написать несколько функций, которые должны делать одно и то же, но возвращать результат разными методами.
\begin{itemize}
\item \texttt{std::string repeat1(const std::string\& s)}
Должна принимать на вход строку и возвращать результат.
\item \texttt{void repeat2(std::string\& s)}
Должна принимать на вход строку по ссылке и изменять эту строку.
\item \texttt{void repeat3(std::string* ps)}
Должна принимать на вход указатель на строку и изменять строку, чей адрес хранит этот указатель.
\item \texttt{std::string* repeat4(const std::string\& s)}
Эта функция должна создавать удвоенную строку в куче с помощью оператора \texttt{new} и возвращать указатель на неё. После вызова функции \texttt{repeat4} программист, который будет использовать эту функцию, сам должен позаботиться об её удалении.
\end{itemize}

Протестируйте эти функции в \texttt{main}.

\subsection{Умножение строки}
Напишите перегруженный оператор умножения, которая будет принимать на вход строку \texttt{std::string} и некоторое целое число $n$ и возвращать строку, повторённую $n$ раз. Протестируйте эту функцию в функции \texttt{main}.


\subsection{Усечение}
Напишите функцию \texttt{void truncateToDot(std::string\& s)}, которая будет принимать строку по ссылке и усекать её до первого символа точки. Размер и вместимость строки должны стать как можно более маленькими, для этого используйте метод \texttt{shrink\_to\_fit}. Функция не должна ничего возвращать. Протестируйте функцию в \texttt{main}.
\begin{center}
\begin{tabular}{ l | l }
 до & после \\ \hline
 \texttt{"cat.dog.mouse.elephant.tiger.lion"} & \texttt{"cat"} \\
 \texttt{"wikipedia.org"} & \texttt{"wikipedia"}  \\ 
 \texttt{".com"} & \texttt{"{}"} \\
\end{tabular}
\end{center}

\subsection{Сумма чётных}
Напишите функцию \texttt{sumEven}, которая будет принимать вектор чисел типа \texttt{int} по константной ссылке и возвращать сумму всех чётных чисел этого вектора.
\begin{center}
\begin{tabular}{ l | l }
 аргумент & возвращаемое значение \\ \hline
 \texttt{\{4, 8, 15, 16, 23, 42\}} & \texttt{70}
\end{tabular}
\end{center}

\subsection{Последние цифры}
Напишите функциию \texttt{lastDigits} которая должны будут принимать на вход вектор чисел типа \texttt{int} по ссылке и заменять все его элементы на последние цифры этих элементов.


\end{document}


