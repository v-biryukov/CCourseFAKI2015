\documentclass{article}
\usepackage[utf8x]{inputenc}
\usepackage{ucs}
\usepackage{amsmath} 
\usepackage{amsfonts}
\usepackage{upgreek}
\usepackage[english,russian]{babel}
\usepackage{graphicx}
\usepackage{float}
\usepackage{textcomp}
\usepackage{hyperref}
\usepackage{geometry}
  \geometry{left=2cm}
  \geometry{right=1.5cm}
  \geometry{top=1cm}
  \geometry{bottom=2cm}
\usepackage{tikz}
\usepackage{ccaption}
\usepackage{mathrsfs}
\usepackage[shortlabels]{enumitem}
\usepackage{listings}
\lstset{
  language=C,                % choose the language of the code
  basicstyle=\linespread{1.1}\ttfamily,
  columns=fixed,
  fontadjust=true,
  basewidth=0.5em,
  keywordstyle=\color{blue}\bfseries,
  commentstyle=\color{gray},
  stringstyle=\ttfamily\color{orange!50!black},
  showstringspaces=false,
  %numbers=false,                   % where to put the line-numbers
  numbersep=5pt,
  numberstyle=\tiny\color{black},
  numberfirstline=true,
  stepnumber=1,                   % the step between two line-numbers.        
  numbersep=10pt,                  % how far the line-numbers are from the code
  backgroundcolor=\color{white},  % choose the background color. You must add \usepackage{color}
  showstringspaces=false,         % underline spaces within strings
  captionpos=b,                   % sets the caption-position to bottom
  breaklines=true,                % sets automatic line breaking
  breakatwhitespace=true,         % sets if automatic breaks should only happen at whitespace
  xleftmargin=.2in,
  extendedchars=\true,
  keepspaces = true,
}
\lstset{literate=%
   *{0}{{{\color{red!20!violet}0}}}1
    {1}{{{\color{red!20!violet}1}}}1
    {2}{{{\color{red!20!violet}2}}}1
    {3}{{{\color{red!20!violet}3}}}1
    {4}{{{\color{red!20!violet}4}}}1
    {5}{{{\color{red!20!violet}5}}}1
    {6}{{{\color{red!20!violet}6}}}1
    {7}{{{\color{red!20!violet}7}}}1
    {8}{{{\color{red!20!violet}8}}}1
    {9}{{{\color{red!20!violet}9}}}1
}


\begin{document}
\pagenumbering{gobble}

\section*{Модуль "Утилиты". Вопросы.}
\begin{enumerate}



\item \textbf{Cmake I}
\begin{enumerate}[a.]

\item \textbf{Основы Cmake.}\\
Что такое Cmake и для чего он нужен? Основы работы с CMake. Структура CMake-проекта. Файл CMakeListis.txt. Генерация файлов проекта для данной среды. 
Как скомпилировать проект с помощью CMake? Что такое таргет (target)?\\
Что делают следующие команды CMake:
\begin{itemize}
\item \texttt{cmake\_minimum\_required}
\item \texttt{project}
\item \texttt{set}
\item \texttt{option}
\item \texttt{message}
\item \texttt{add\_executable}
\item \texttt{add\_library} и её опции \texttt{STATIC}, \texttt{SHARED} и \texttt{MODULE}
\item \texttt{target\_link\_libraries} и её опции \texttt{PUBLIC}, \texttt{PRIVATE} и \texttt{INTERFACE}
\item \texttt{target\_compile\_options}
\item \texttt{add\_subdirectory}
\item \texttt{find\_package}
\end{itemize}


\item \textbf{Переменные Cmake.}\\
Переменные Cmake. Как создавать переменные и как их использовать в CMake? Какие типы переменных есть в CMake?
Переменные среды. Кешированные переменные. Чем кешированные переменные отличаются от обычных переменных? Поле \texttt{type} при создании кешированной переменной и какие значения оно может принимать. Изменение кешированных переменных. Задание кешированных переменных внутри CMake-скрипта, в командной строке и путём изменения файла CmakeCache.txt.\\
Переменные:
\begin{itemize}
\item \texttt{CMAKE\_SOURCE\_DIR}
\item \texttt{CMAKE\_BUILD\_DIR}
\item \texttt{CMAKE\_CURRENT\_SOURCE\_DIR}
\item \texttt{CMAKE\_CURRENT\_BUILD\_DIR}
\end{itemize}

\end{enumerate}
\newpage



\item \textbf{Cmake II}

\item \textbf{Git I}

\item \textbf{Git II}

\item \textbf{Тестирование}


\item \textbf{Qt?}
\item \textbf{Boost Asio ?}




\end{enumerate}

\end{document}