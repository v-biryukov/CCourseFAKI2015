\documentclass{article}
\usepackage[utf8x]{inputenc}
\usepackage{ucs}
\usepackage{amsmath} 
\usepackage{amsfonts}
\usepackage{upgreek}
\usepackage[english,russian]{babel}
\usepackage{graphicx}
\usepackage{float}
\usepackage{textcomp}
\usepackage{hyperref}
\usepackage{geometry}
  \geometry{left=2cm}
  \geometry{right=1.5cm}
  \geometry{top=1cm}
  \geometry{bottom=2cm}
\usepackage{tikz}
\usepackage{ccaption}
\usepackage{mathrsfs}
\usepackage[shortlabels]{enumitem}
\usepackage{multicol}
\usepackage{listings}
\lstset{
  language=C,                % choose the language of the code
  basicstyle=\linespread{1.1}\ttfamily,
  columns=fixed,
  fontadjust=true,
  basewidth=0.5em,
  keywordstyle=\color{blue}\bfseries,
  commentstyle=\color{gray},
  stringstyle=\ttfamily\color{orange!50!black},
  showstringspaces=false,
  %numbers=false,                   % where to put the line-numbers
  numbersep=5pt,
  numberstyle=\tiny\color{black},
  numberfirstline=true,
  stepnumber=1,                   % the step between two line-numbers.        
  numbersep=10pt,                  % how far the line-numbers are from the code
  backgroundcolor=\color{white},  % choose the background color. You must add \usepackage{color}
  showstringspaces=false,         % underline spaces within strings
  captionpos=b,                   % sets the caption-position to bottom
  breaklines=true,                % sets automatic line breaking
  breakatwhitespace=true,         % sets if automatic breaks should only happen at whitespace
  xleftmargin=.2in,
  extendedchars=\true,
  keepspaces = true,
}
\lstset{literate=%
   *{0}{{{\color{red!20!violet}0}}}1
    {1}{{{\color{red!20!violet}1}}}1
    {2}{{{\color{red!20!violet}2}}}1
    {3}{{{\color{red!20!violet}3}}}1
    {4}{{{\color{red!20!violet}4}}}1
    {5}{{{\color{red!20!violet}5}}}1
    {6}{{{\color{red!20!violet}6}}}1
    {7}{{{\color{red!20!violet}7}}}1
    {8}{{{\color{red!20!violet}8}}}1
    {9}{{{\color{red!20!violet}9}}}1
}


\begin{document}
\pagenumbering{gobble}

\section*{Модуль "Утилиты". Вопросы.}
\begin{enumerate}

\item \textbf{Сборка}
\begin{enumerate}[a.]
\item \textbf{Раздельная компиляция}\\
Что такое файл исходного кода и исполняемый файл? Этап сборки программы: препроцессинг, ассемблирование, компиляция и линковка. Что такое заголовочные файлы (header-файлы)? Что делает директива препроцессора \texttt{\#include}? Что такое единица трансляции? Компиляция программы с помощью \texttt{g++}. Опции компиляции \texttt{-E}, \texttt{-S} и \texttt{-c}. Что такое раздельная компиляция и в чём её преемущества?


\item \textbf{Библиотеки}\\
Что такое библиотека? Виды библиотек: header-only библиотеки, open-source библиотеки, статические библиотеки, динамические библиотеки. В чём различия между этими видами библиотек? В чём преимущества и недостатки каждого из видов библиотек? Как подключить библиотеки к своему проекту? 

\item \textbf{Статические библиотеки}\\
Как создать статическую библиотеку? Как подключить статическую библиотеку? Опции компилятора \texttt{-I}, \texttt{-L} и \texttt{-l}. Характерные расширения файлов статических библиотек на Linux и Windows. 

\item \textbf{Динамические библиотеки}\\
В чём главная разница между статическими и динамическими библиотеками? Как создать динамическую библиотеку? Как подключить динамическую библиотеку? Характерные расширения файлов динамических библиотек на Linux и Windows. 

\item \textbf{Опции компилятора \texttt{g++}}
\begin{itemize} 
\item Опции для указания стандарта языка, например \texttt{-std=c++20}
\item Опции для включения/отключения предупреждений: \texttt{-Wall}, \texttt{-Wextra}, \texttt{-Werror}.
\item Опция для указания директорий заголовочных файлов, необходимых для компиляции \texttt{-I}
\item Опция для указания директорий библиотек, необходимых для компиляции \texttt{-L}
\item Опция для указания названий библиотек, необходимых для компиляции \texttt{-l}
\item Опция для включения возможности проведения дебага: \texttt{-g}
\item Опции для включения оптимизаций: \texttt{-O0}, \texttt{-O1}, \texttt{-O2}, \texttt{-O3}, \texttt{-Os}
\item Опция \texttt{-DNDEBUG}
\item Опция \texttt{-D} для задания \texttt{\#define}-макросов. Как ёё использовать? Пример использования данной опции.
\end{itemize}
\end{enumerate}







\item \textbf{CMake как система сборки}
\begin{enumerate}[a.]

\item \textbf{Основы CMake}\\
Что такое Cmake и для чего он нужен? Основы работы с CMake. Структура CMake-проекта. Файл CMakeListis.txt.
Как скомпилировать проект с помощью CMake? Что делают следующие команды CMake:
\begin{itemize}
\item \texttt{cmake\_minimum\_required}
\item \texttt{project}
\item \texttt{add\_executable}
\item \texttt{message}
\end{itemize}

Как собрать проект с использованием CMake? Генерация файлов проекта для данной среды. Выбор генератора.
Опции программы \texttt{cmake}: \texttt{-S}, \texttt{-B}, \texttt{-G}, \texttt{-{}-build}.


\item \textbf{Таргеты}\\
Что такое таргет (target)? Что делают следующие команды CMake:
\begin{itemize}
\item \texttt{add\_executable}
\item \texttt{add\_library} и её опции \texttt{STATIC} и \texttt{SHARED}
\item \texttt{target\_link\_libraries} (если аргумент является таргетом)
\end{itemize}


\item \textbf{Свойства таргетов}\\
Что делают следующие команды CMake:
\begin{itemize}
\item \texttt{target\_include\_directories}
\item \texttt{target\_link\_directories}
\item \texttt{target\_link\_libraries} (если аргумент не является таргетом)
\item \texttt{target\_compile\_features}
\item \texttt{target\_compile\_definitions}
\item \texttt{target\_compile\_options}
\end{itemize}


\item \textbf{Типы зависимостей между таргетам}\\
Типы связей между двумя таргетам \texttt{PRIVATE}, \texttt{PUBLIC} и \texttt{INTERFACE}.
Типы связей между таргетом и его свойством \texttt{PRIVATE}, \texttt{PUBLIC} и \texttt{INTERFACE}.
В чём отличия между этими типами зависимостей? Зачем нужно указывать тип для каждой связи? 
Примеры ситуаций когда нужно использовать ту или иную связь.


\item \textbf{Простые переменные CMake}\\
Простые переменные CMake. Какие бывают типы у переменных языка CMake? Как создать простую переменную в CMake? Команда \texttt{set}. Как напечатать значение переменной на экран? Основные стандартные переменные:
\begin{itemize}
\item \texttt{CXX\_STANDARD}
\item \texttt{CMAKE\_CXX\_COMPILER}
\item \texttt{CMAKE\_SOURCE\_DIR}
\item \texttt{CMAKE\_BUILD\_DIR}
\item \texttt{BUILD\_SHARED\_LIBS}
\item \texttt{WIN32}, \texttt{LINUX}, \texttt{APPLE}, \texttt{MSVC}, \texttt{MINGW}
\end{itemize}

\item \textbf{Поддиректории}\\
Как добавить новую поддиректорию в CMake проект? Команда \texttt{add\_subdirectory}. Что происходит при выполнении этой команды?
Область видимости переменных. Видны ли переменные, созданные в родительской Cmake-директории, в поддиректории? Видны ли переменные, созданные в поддиректории, в родительской Cmake-директории? Опция \texttt{PARENT\_SCOPE} команды \texttt{set}. Переменные:
\begin{itemize}
\item \texttt{CMAKE\_CURRENT\_SOURCE\_DIR}
\item \texttt{CMAKE\_CURRENT\_BUILD\_DIR}
\end{itemize}

\end{enumerate}





\item \textbf{CMake как язык программирования}
\begin{enumerate}[a.]

\item \textbf{Переменные}\\
Какие бывают типы у переменных языка CMake? Как создать простую переменную в CMake? Команда \texttt{set}. Как получить значение переменной по её названию?

\item \textbf{Условная команда \texttt{if}}\\
Как пользоваться командой \texttt{if} и сопутствующими командами в языке CMake? Какие строки команда \texttt{if} воспринимает как истинные, а какие как ложные? Использование переменных как аргуметы команды \texttt{if}.

\item \textbf{Списки}\\
Что представляет собой список в языке CMake? Как создать список? Как работать со списком? Передача списка в функцию. Команда \texttt{list}. Опции этой команды: \texttt{LENGTH}, \texttt{GET}, \texttt{FIND}, \texttt{APPEND}, \texttt{SORT}. 

\item \textbf{Циклы}\\
Команда \texttt{while}. Команда \texttt{foreach}. Опции команды \texttt{foreach}: \texttt{RANGE} и \texttt{IN LISTS}. Итерирование по списку с помощью команды \texttt{foreach}.

\item \textbf{Функции}\\
Функции в языке CMake. Как создать функцию с помощью команды \texttt{function}? Как передавать в функцию? Переменные \texttt{ARGC}, \texttt{ARGV}, \texttt{ARGN}. Как возвращать из функции. Опция \texttt{PARENT\_SCOPE} команды \texttt{set}. Команда \texttt{return}. Области видимости функций.

\item \textbf{Манипуляции со строками}\\
Команда \texttt{string} и её опции:
\begin{multicols}{3}
\begin{itemize}
\item \texttt{FIND}
\item \texttt{REPLACE}
\item \texttt{APPEND}
\item \texttt{JOIN}
\item \texttt{TOLOWER}
\item \texttt{TOUPPER}
\item \texttt{LENGTH}
\item \texttt{SUBSTRING}
\item \texttt{COMPARE}
\end{itemize}
\end{multicols}

\newpage
\item \textbf{Файлы}\\
Команда \texttt{file} и её опции:
\begin{multicols}{3}
\begin{itemize}
\item \texttt{READ}
\item \texttt{STRINGS}
\item \texttt{WRITE}
\item \texttt{MAKE\_DIRECTORY}
\item \texttt{REMOVE}
\item \texttt{RENAME}
\item \texttt{COPY}
\item \texttt{SIZE}
\item \texttt{CHMOD}
\item \texttt{REAL\_PATH}
\item \texttt{DOWNLOAD}
\item \texttt{GLOB}
\end{itemize}
\end{multicols}
Является ли хорошей идеей использование команды \texttt{file} с опцией \texttt{GLOB}, чтобы найти названия всех файлов исходного кода некоторого таргета?

\item \textbf{Модули}\\
Что представляет собой модуль в языке CMake. Подключение модулей. Команда \texttt{include}. В каких папках ищутся модули? Переменная \texttt{CMAKE\_MODULE\_PATH}. Область видимости переменных. Переменная \texttt{CMAKE\_CURRENT\_LIST\_FILE}. Чем команда \texttt{include} отличается от команды \texttt{add\_subdirectory}?

\end{enumerate}






\item \textbf{CMake - дополнительные возможности}

\begin{enumerate}[a.]
\item \textbf{Кешированные переменные CMake.}\\
Переменные среды. Кешированные переменные. Чем кешированные переменные отличаются от обычных переменных? Поле \texttt{type} при создании кешированной переменной и какие значения оно может принимать. Изменение кешированных переменных. Задание кешированных переменных внутри CMake-скрипта, в командной строке и путём изменения файла CMakeCache.txt.\\
\end{enumerate}

\item \textbf{CMake - подключение сторонних библиотек}





\item \textbf{Git - локально}





\item \textbf{Git - удалённые репозитории}





\item \textbf{Тестирование}





\end{enumerate}

\end{document}