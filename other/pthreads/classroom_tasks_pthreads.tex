\documentclass{article}
\usepackage[utf8x]{inputenc}
\usepackage{ucs}
\usepackage{amsmath} 
\usepackage{amsfonts}
\usepackage{upgreek}
\usepackage[english,russian]{babel}
\usepackage{graphicx}
\usepackage{float}
\usepackage{textcomp}
\usepackage{hyperref}
\usepackage{geometry}
  \geometry{left=2cm}
  \geometry{right=1.5cm}
  \geometry{top=1cm}
  \geometry{bottom=2cm}
\usepackage{tikz}
\usepackage{ccaption}
\usepackage{multicol}
\setlength{\columnsep}{1.5cm}
\setlength{\columnseprule}{0.2pt}

\usepackage{listings}


\begin{document}
\pagestyle{plain}
\lstset{
  language=C,                % choose the language of the code
  basicstyle=\linespread{1.1}\ttfamily,
  columns=fixed,
  fontadjust=true,
  basewidth=0.5em,
  keywordstyle=\color{blue}\bfseries,
  commentstyle=\color{gray},
  stringstyle=\ttfamily\color{orange!50!black},
  showstringspaces=false,
  numbersep=5pt,
  numberstyle=\tiny\color{black},
  numberfirstline=true,
  stepnumber=1,                   % the step between two line-numbers.        
  numbersep=10pt,                  % how far the line-numbers are from the code
  backgroundcolor=\color{white},  % choose the background color. You must add \usepackage{color}
  showstringspaces=false,         % underline spaces within strings
  captionpos=b,                   % sets the caption-position to bottom
  breaklines=true,                % sets automatic line breaking
  breakatwhitespace=true,         % sets if automatic breaks should only happen at whitespace
  xleftmargin=.2in,
  extendedchars=\true,
  keepspaces = true,
}
\lstset{literate=%
   *{0}{{{\color{red!20!violet}0}}}1
    {1}{{{\color{red!20!violet}1}}}1
    {2}{{{\color{red!20!violet}2}}}1
    {3}{{{\color{red!20!violet}3}}}1
    {4}{{{\color{red!20!violet}4}}}1
    {5}{{{\color{red!20!violet}5}}}1
    {6}{{{\color{red!20!violet}6}}}1
    {7}{{{\color{red!20!violet}7}}}1
    {8}{{{\color{red!20!violet}8}}}1
    {9}{{{\color{red!20!violet}9}}}1
}

\title{Семинар: POSIX Threads. Классные задачи.\vspace{-5ex}}\date{}\maketitle
\section*{Pthreads:}
Функция \texttt{pthread\_create} - запускает новую нить исполнения.
\begin{lstlisting}
int pthread_create(pthread_t* t, const pthread_attr_t *attr, void* (*f)(void*), void* arg);
\end{lstlisting}
\begin{itemize}
\item \texttt{pthread\_t* t} - указатель на объект потока исполнения
\item \texttt{void* (*f)(void*)} - функция, которую будет исполнять поток
\item \texttt{void* arg} - аргумент функции \texttt{f}
\end{itemize}
Функция \texttt{pthread\_exit(NULL)} - завершает исполнение вызвавшего потока.\\

Пример программы, запускающей 2 дополнительных потока исполнения:
\begin{lstlisting}
#include <pthread.h>
#include <stdio.h>

void* print_hello(void* threadid)
{
    printf("Hello World! It's me, thread #%ld!\n", (long)threadid);
}

int main(int argc, char** argv)
{
    pthread_t t;
    pthread_create(&t, NULL, print_hello, (void*)0);
    pthread_create(&t, NULL, print_hello, (void*)1);
    pthread_exit(NULL);
}
\end{lstlisting}

\subsection*{Задачи:}
\begin{enumerate}
\item \textbf{}
\end{enumerate}

\end{document}