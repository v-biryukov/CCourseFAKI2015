\documentclass{article}
\usepackage[utf8x]{inputenc}
\usepackage{ucs}
\usepackage{amsmath} 
\usepackage{mathtext}
\usepackage{amsfonts}
\usepackage{upgreek}
\usepackage[english,russian]{babel}
\usepackage{graphicx}
\usepackage{float}
\usepackage{textcomp}
\usepackage{hyperref}
\usepackage{geometry}
  \geometry{left=2cm}
  \geometry{right=1.5cm}
  \geometry{top=1cm}
  \geometry{bottom=2cm}
\usepackage{tikz}
\usepackage{ccaption}
\usepackage{multicol}
\setlength{\columnsep}{1cm}


\begin{document}
\pagenumbering{gobble}

\subsection*{Справочная информация:}
Типы данных в C. Приведенны размеры для 64-х битных систем. 
\begin{center}
\begin{tabular}{ l l l | l l l }
  Спецификатор & Тип & Размер (байт) & Спецификатор & Тип & Размер (байт) \\
  \%h & short & 2 &  \%c & char & 1 \\
  \%uh & unsigned short & 2 & \%c & unsigned char & 1 \\
  \%d или \%i & int & 4 &  \%f & float & 4 \\
  \%u & unsigned int & 4 & \%lf & double & 8 \\
  \%ld & long & 8 (или 4) & \%Lf & long double & обычно 10\\
  \%lu & unsigned long & 8 (или 4) & \%p & указатель (<имя типа>*) & 8 \\
  \%lld & long long & 8 & \%s & Строка &  \\
  \%llu & unsigned long long & 8 &  \\
\end{tabular}
\end{center}
Многие функции в языке C возвращают особый тип size\_t. Часто это просто unsigned long:
\begin{verbatim}
typedef unsigned long size_t;
\end{verbatim}

\subsection*{Задачи:}
Функции не должны ничего считывать и печатать.
\begin{enumerate}
\item \textbf{Остаток:} Написать программу, которая считывает 2 числа $a$ и $b$ и печатает их остаток. $0 \le a, b \le 2^{32}-1$. Использовать тип unsigned int.
\item \textbf{Произведение чисел:} Написать программу, которая считывает 2 числа $a$ и $b$ и печатает их произведение. $0 \le a, b \le 2^{32}-1$.
\item \textbf{mod 7:} Написать программу, которая печатает все числа делящиеся на 7 в интервале от 700 до 1000, используя цикл for.
\item \textbf{Часть года:} Написать функцию на вход которой подаётся целое число -- число дней прошедших с начала года. Она должна возвращать вещественное число типа float -- доля прошедшего года(от 0 до 1). В году 365 дней.
\item \textbf{Математическая функция:} Написать функцию, которая вычисляет выражение $\sin(\sqrt{|x|})$. Использовать числа двойной точности double. Функция для вычисления модуля вещественного числа -- fabs() из библиотеки math.h. 
\item \textbf{Делимость:} На вход программе подаётся целое число $n$ и $n$ вещественных чисел типа int. Нужно напечатать 1 если все числа делятся на 7 и 0 иначе.
\item \textbf{Нормализация:} На вход программе подаётся целое число $n$ и $n$ вещественных чисел типа float. Нужно эти числа нормировать (то есть разделить на их сумму) и напечатать.
\item \textbf{Sqared Matrix:} Написать функцию \textbf{void matrix\_square(int n, int arr[SIZE][SIZE])}, которая возводит двумерную матрицу в квадрат. SIZE -- максимально возможный размер массива, задаётся  так:
\begin{verbatim}
#define SIZE 100
\end{verbatim}
\end{enumerate}

\end{document}