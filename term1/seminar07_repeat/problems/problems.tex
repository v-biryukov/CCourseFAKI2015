\documentclass{article}
\usepackage[utf8x]{inputenc}
\usepackage{ucs}
\usepackage{amsmath} 
\usepackage{mathtext}
\usepackage{amsfonts}
\usepackage{upgreek}
\usepackage[english,russian]{babel}
\usepackage{graphicx}
\usepackage{float}
\usepackage{textcomp}
\usepackage{hyperref}
\usepackage{geometry}
  \geometry{left=2cm}
  \geometry{right=1.5cm}
  \geometry{top=1cm}
  \geometry{bottom=2cm}
\usepackage{tikz}
\usepackage{ccaption}
\usepackage{multicol}
\setlength{\columnsep}{1cm}


\begin{document}
\pagenumbering{gobble}

\subsection*{Справочная информация:}
\subsubsection*{Типы и их спецификаторы в printf/scanf:}
Типы данных в C. Приведенны размеры для 64-х битных систем. 
\begin{center}
\begin{tabular}{ l l l | l l l }
  Спецификатор & Тип & Размер (байт) & Спецификатор & Тип & Размер (байт) \\
  \%h & short & 2 &  \%c & char & 1 \\
  \%uh & unsigned short & 2 & \%c & unsigned char & 1 \\
  \%d или \%i & int & 4 &  \%f  & float & 4 \\
  \%u & unsigned int & 4 & \%lf & double & 8 \\
  \%ld & long & 8 (или 4) & \%Lf & long double & обычно 10\\
  \%lu & unsigned long & 8 (или 4) & \%g & \%f, без нулей на конце\\
  \%lld & long long & 8 & \%p & указатель (<имя типа>*) & 8 \\
  \%llu & unsigned long long & 8 & \%s & Строка &  \\
\end{tabular}
\end{center}
Многие функции в языке C возвращают особый тип size\_t. Часто это просто unsigned long:
\begin{verbatim}
typedef unsigned long size_t;
\end{verbatim}
\subsubsection*{Функции:}
Пример функции вычисляющей $\cos \big(|x| + |y|\big)$:
\begin{verbatim}
#include <math.h>
float cos_abs(float x, float y)
{
    float result = cos(fabs(x) + fabs(y));
    return result;
}
\end{verbatim}

\subsubsection*{Массивы:}
\begin{verbatim}
#include <stdio.h>

// Функция, которая вычисляет сумму элементов массива
// В функцию передаём n - размер массива и сам массив input_array
float get_sum(int n, float input_array[])
{
    float sum = 0.0;
    for (int i = 0; i < n; ++i)
        sum += input_array[i];
    return sum;
}

int main()
{
    int n;            // Объявляем n - это число будет хранить число элементов в массиве
    float arr[100];   // Объявляем массив вещественных чисел типа float
    scanf("%d", &n);  // Считываем n
    for (int i = 0; i < n; ++i)   // Считываем элементы массива
    {
        scanf("%f", &arr[i]);
    }
    printf("%f\n", get_sum(n, arr));  // Вычисляем и печатаем сумму
}
\end{verbatim}

\newpage
\subsubsection*{Двумерный массивы:}
\begin{verbatim}
#include <stdio.h>

#define SIZE 100

// Функция, которая вычисляет сумму 2-х матриц и записывает результат в третью
// Помните, что массивы могут меняться внутри функции
void sum(int n, int m, int m1[SIZE][SIZE], int m2[SIZE][SIZE], int result[SIZE][SIZE])
{
    for (int i = 0; i < n; ++i)
        for (int j = 0; j < m; ++j)
            result[i][j] = m1[i][j] + m2[i][j];
}

int main()
{
    int n, m;            // Объявляем n и m - размеры массива
    int A[SIZE][SIZE], B[SIZE][SIZE], C[SIZE][SIZE];   // Объявляем 3 массива
    scanf("%d%d", &n, &m);  // Считываем n и m
    
    for (int i = 0; i < n; ++i)   // Считываем элементы массива A
        for (int j = 0; j < m; ++j)
            scanf("%d", &A[i][j]);
            
    for (int i = 0; i < n; ++i)   // Считываем элементы массива B
        for (int j = 0; j < m; ++j)
            scanf("%d", &A[i][j]);
            
    sum(n, m, A, B, C); // Вычисляем сумму A и B и записываем результат в C
    
    for (int i = 0; i < n; ++i)   // Печатаем C
    {
        for (int j = 0; j < m; ++j)
            printf("%d ", C[i][j]);
        printf("\n");
    }
}
\end{verbatim}
\subsubsection*{Оператор switch:}
\begin{verbatim}
int a;
scanf("%d", &a);
switch ( c )  
{  
    case 1:  
        printf("One\n") 
        break; 
    case 2:  
        printf("Two\n")  
        break;  
    case 3:  
        printf("Thee\n")  
        break; 
    default:  
        printf("Other\n")  
}  

\end{verbatim}


\newpage

\subsection*{Задачи:}
Функции не должны ничего считывать и печатать. Во всех задачах с функциями нужно не только написать функцию, но и вызвать её из функции main().
\begin{enumerate}
\item \textbf{Остаток:} Написать программу, которая считывает 2 числа $a$ и $b$ и печатает остаток деления $a$ на $b$. $0 \le a, b \le 2^{32}-1$. Использовать тип unsigned int.
\item \textbf{Остаток + if:} Написать программу, которая считывает 2 числа $a$ и $b$ и печатает остаток деления большего числа на меньшее. $0 \le a, b \le 2^{32}-1$.
\item \textbf{Произведение чисел:} Написать программу, которая считывает 2 числа $a$ и $b$ и печатает их произведение. $0 \le a, b \le 2^{32}-1$. Обратите внимание на диапазон значений типов.
\item \textbf{mod 7:} Написать программу, которая печатает все числа делящиеся на 7 в интервале от 700 до 1000, используя цикл for.
\item \textbf{break:} Написать программу, которая считывает целые числа и печатает их до первого отрицательного. Например, если на вход поступает последовательность 5 0 74 -3 5 31 -7 -10, то программа должна напечатать 5 0 74. Использовать оператор break.
\item \textbf{Часть года:} Написать функцию на вход которой подаётся целое число -- число дней прошедших с начала года. Она должна возвращать вещественное число типа float -- доля прошедшего года(от 0 до 1). В году 365 дней.
\item \textbf{Математическая функция:} Написать функцию, которая вычисляет выражение $\sin(\sqrt{|x|})$. Использовать числа двойной точности double. Функция для вычисления модуля вещественного числа -- fabs() из библиотеки math.h. 
\item \textbf{Единичный круг:} Написать функцию, которая проверяет принадлежат ли 2 вещественных числа единичному кругу. Использовать тип double.
\item \textbf{Гипербола:} Написать функцию, которая проверяет принадлежат ли 2 вещественных числа области 
$\{y > \frac{1}{x}, x > 0\}$. Использовать тип double. Функция должна возвращать 0 или 1. Тесты: \\
(x = 2.0, y = 4.0) -> 1; \\
(x = 0.0, y = 2.0) -> 0; \\
(x = 0.0, y = 0.0) -> 0; \\
(x = -2.0, y = -5.0) -> 0;\\
\item \textbf{Делимость:} На вход программе подаётся целое число $n$ и $n$ целых чисел типа unsigned long. Нужно напечатать 1 если все числа делятся на 7, в ином случае нужно напечатать 0.
\item \textbf{Нормализация:} На вход программе подаётся целое число $n$ и $n$ вещественных чисел типа float. Нужно эти числа нормировать (то есть разделить на их сумму) и напечатать.
\item \textbf{Среднее и дисперсия:} На вход программе подаётся целое число $n$ и $n$ вещественных чисел типа double ${x_i}$. Нужно найти среднее этих чисел $\mu$ и дисперсию $D$: 
$$\mu = \frac{1}{n}\sum_{i=0}^{n-1}x_i$$
$$D = \frac{1}{n}\sum_{i=0}^{n-1}(x_i - \mu)^2$$
\item \textbf{Таблица умножения:} Написать программу, которая распечатывает таблицу умножения.
\item \textbf{Sqared Matrix:} Написать функцию \textbf{void matrix\_square(int n, int in[SIZE][SIZE], int out[SIZE][SIZE])}, которая возводит матрицу in в квадрат и записывает результат в матрицу out. SIZE -- максимально возможный размер массива, задаётся  так:
\begin{verbatim}
#define SIZE 100
\end{verbatim}
\item \textbf{A\_n:} Написать функцию \textbf{void matrix\_square(int n, int in[SIZE][SIZE], int out[SIZE][SIZE])}, которая возводит матрицу in в n-ю степень и записывает результат в матрицу out. Использовать функцию, написанную в предыдущей задаче.
\item \textbf{Matrix Multiply:} Написать функцию \\ 
\textbf{void matrix\_multiply(int k, int m, int n, int m1[SIZE][SIZE], int m2[SIZE][SIZE], int out[SIZE][SIZE])}, которая умножает матрицу m1 размера $k \times m$ на матрицу m2 размера $m \times n$ и записывает результат в матрицу out.

\item \textbf{switch:} Написать программу, которая по номеру дня недели (от 1 до 7) печатает название этого дня. Если число не входит в этот диапазон, то программа должна напечатать Error!\\
\begin{tabular}{ l | l }
  input & output \\
  \hline
  1 & Monday  \\
  2 & Tuesday  \\
  7 & Sunday  \\
  100500 & Error!  \\
\end{tabular}\\

Что будет, если убрать все операторы break?

\item \textbf{Сколько дорог:}
(Задача с ejudge -> Динамическое программирование) Из верхнего левого угла в правый нижний угол сетки 2x2 можно пройти 6 разными путями (без возвратов, т.е. если идти только вниз или вправо). Сколько таких разных путей можно найти в сетке N×M?
\begin{center}
\includegraphics[width=0.2\linewidth]{../images/ways.png}
\end{center}
Пусть $C_{i, j}$ -- число путей от точки (0, 0) до точки (i, j). Как $C_{i, j}$ зависит от $C_{i-1, j}$ и $C_{i, j-1}$?

\item \textbf{Черепашка:}
В левом верхнем углу прямоугольной таблицы размером $N \times M$ находится черепашка. В каждой клетке таблицы записано некоторое число. Черепашка может перемещаться вправо или вниз, при этом маршрут черепашки заканчивается в правом нижнем углу таблицы.

Подсчитаем сумму чисел, записанных в клетках, через которую проползла черепашка (включая начальную и конечную клетку). Найдите наибольшее возможное значение этой суммы.

Подсказка: Пусть $C_{i, j}$ -- наибольшее возможное значение суммы, если черепашка доползла до точки (i, j). Как $C_{i, j}$ зависит от $C_{i-1, j}$ и $C_{i, j-1}$? \\
Тест: \\
\begin{tabular}{ p{15mm} | p{15mm} }
  input & output \\
  \hline
  3 4 \newline 1 1 2 1 \newline 2 2 1 1 \newline 2 1 2 1 & 9  \\
\end{tabular}

\end{enumerate}

\end{document}