\documentclass{article}
\usepackage[utf8x]{inputenc}
\usepackage{ucs}
\usepackage{amsmath} 
\usepackage{mathtext}
\usepackage{amsfonts}
\usepackage{upgreek}
\usepackage[english,russian]{babel}
\usepackage{graphicx}
\usepackage{float}
\usepackage{textcomp}
\usepackage{hyperref}
\usepackage{geometry}
  \geometry{left=2cm}
  \geometry{right=1.5cm}
  \geometry{top=1cm}
  \geometry{bottom=2cm}
\usepackage{tikz}
\usepackage{ccaption}
\usepackage{multicol}
\setlength{\columnsep}{1cm}


\begin{document}
\pagenumbering{gobble}

\subsection*{Справочная информация:}
\begin{center}
\begin{tabular}{ l l l | l l l }
  Спецификатор & Тип & Размер (байт) & Спецификатор & Тип & Размер (байт) \\
  \%d или \%i & int & 4 &  \%f & float & 4 \\
  \%u & unsigned int & 4 & \%f & double & 8 \\
  \%l & long & 8 & \%p & указатель (<имя типа>*) & 8\\
  \%ul & unsigned long & 8 & \%c & char & 1 \\
  \%ll & long long & 8 & \%s & Строка & \\
  \%ull & unsigned long long & 8 \\
\end{tabular}
\end{center}
\subsection*{Задачи:}
Функции не должны ничего считывать и печатать.
\begin{enumerate}
\item \textbf{Произведение чисел:} Написать программу, которая считывает 2 числа $a$ и $b$ и печатает их произведение. $0 \le a, b \le 2^{32}-1$. Обратите внимание на диапазон значений типа.
\item \textbf{mod 7:} Написать программу, которая печатает все числа делящиеся на 7 в интервале от 700 до 1000, используя цикл for.
\item \textbf{Часть года:} Написать функцию на вход которой подаётся целое число -- число дней прошедших с начала года. Она должна возвращать вещественное число типа float -- доля прошедшего года(от 0 до 1). В году 365 дней.
\item \textbf{Математическая функция:} Написать функцию, которая вычисляет выражение $\sin(\sqrt{x})$ от положительного числа $x$.
\item \textbf{Нормализация:} На вход программе подаётся целое число $n$ и $n$ вещественных чисел типа float. Нужно эти числа нормировать (то есть разделить на их сумму) и напечатать.
\end{enumerate}


\newpage
\subsection*{Справочная информация по указателям:}
Каждая переменная в языке C хранится где-то в памяти и имеет адрес. Адрес переменной это просто номер первого байта соответствующей области памяти. Чтобы получить адрес переменной нужно просто перед переменной поставить \&(амперсанд).
Указатель это переменная, которая хранит адреса переменных. Тип указателя такой: <тип переменной>*. Пример:
\begin{verbatim}
int a = 42;  // Переменная, которая хранит число 42
int* p = &a; // Указатель, который будет хранить адрес переменной a
\end{verbatim}
Чтобы доступиться к переменной по указателю нужно поставить символ * перед указателем:
\begin{verbatim}
*p = *p + 10;
printf("%d", a);  // Напечатает 52
printf("%d", *p); // Напечатает 52
\end{verbatim}
Указатели часто используются чтобы изменять передаваемые значения в функциях:
\begin{multicols}{2}
\begin{verbatim}
// Неправильно:
void normalize(float x, float y)
{
    float sum = x + y;
    x = x / sum;
    y = y / sum; 
    // Изменятся x и y - копии a и b
}
// ...
float a = 20.0, b = 80.0;
normalize(a, b);
// a и b не изменятся: a=20.0, b=80.0
\end{verbatim}
\begin{verbatim}
// Правильно:
void normalize(float* x, float* y)
{
    float sum = *x + *y;
    *x = *x / sum;
    *y = *y / sum; 
    // Изменятся переменные a и b
}
// ...
float a = 20.0, b = 80.0;
normalize(&a, &b);
// a и b изменятся:a=0.2, b=0.8
\end{verbatim}
\end{multicols}

\subsection*{Задачи:}
\begin{enumerate}
\item \textbf{Работа с указателями}
	\begin{enumerate}
	\item Объявить переменную типа int и инициализировать её какими-либо значениями
	\item Напечатать значение и адрес переменной, используя эту переменную (чтобы напечатать адрес используйте спецификатор \%p)
	\item Объявить указатель типа int* и присвоить ему адрес переменной
	\item Напечатать значение и адрес переменной, используя только указатель
	\item Изменить значение переменной используя только указатель и напечатать это значение
	\end{enumerate}

\item \textbf{Modify1:} Написать функцию \textbf{void add10(int* p)}, которая добавляет 10 к переменной типа int. Используйте эту функцию в функции main().

\item \textbf{Modify2:} Написать функцию \textbf{void cube(double* p)}, которая возводит значение переменной типа double в куб, используя указатель на эту переменную. Используйте эту функцию в функции main() следующим образом:
\begin{verbatim}
double x = 0.99;
while (x)
{
    cube(&x);
    printf("%.200f\n", x);
}

\end{verbatim}
\item \textbf{Swap:} Написать функцию \textbf{swap}, которая меняет значения 2-х переменных типа int местами. Используйте эту функцию в функции main().
\item \textbf{Sqared Matrix:} Написать функцию \textbf{void matrix\_square(int n, int arr[SIZE][SIZE])}, которая возводит двумерную матрицу в квадрат. SIZE -- максимально возможный размер массива, задаётся  так:
\begin{verbatim}
#define SIZE 100
\end{verbatim}
\end{enumerate}

\end{document}