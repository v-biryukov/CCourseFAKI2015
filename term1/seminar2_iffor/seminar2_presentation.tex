\documentclass[14pt,pdf,hyperref={unicode}]{beamer}

% \documentclass[aspectratio=43]{beamer}
% \documentclass[aspectratio=1610]{beamer}
% \documentclass[aspectratio=169]{beamer}

\usepackage{lmodern}

% подключаем кириллицу 
\usepackage[T2A]{fontenc}
\usepackage[utf8]{inputenc}
\usepackage{listings}
\usepackage{graphicx}
\usepackage{hyperref}

% отключить клавиши навигации
\setbeamertemplate{navigation symbols}{}

% тема оформления
\usetheme{CambridgeUS}

% цветовая схема
\usecolortheme{seahorse}

\definecolor{light-gray}{gray}{0.90}

\lstset{basicstyle=\ttfamily,breaklines=true}

\title{Семинар №2}   
\subtitle{ФАКИ \the\year}
\author{Бирюков В. А.} 
\date{\today} 
% \logo{\includegraphics[height=5mm]{images/logo.png}\vspace{-7pt}}

\begin{document}

\lstset{language=C}

% титульный слайд
\begin{frame}
\titlepage
\end{frame} 

\defverbatim[colored]\makeset{
\begin{lstlisting}[language=C++,basicstyle=\ttfamily,keywordstyle=\color{blue}]
void make_set(int X) {
  parent[X] = X;
}
\end{lstlisting}
}



\begin{frame}
\frametitle{Переменные} 
\begin{center}
\begin{itemize}
\item В языке C все переменные нужно объявить перед использованием
\item При объявлении -- выделяется память под переменную
\item Области видимости переменной
\item Название переменной может содержать латинские буквы, цифры и \_
\item Название переменной не может начинаться с цифры
\end{itemize}
\end{center}
\end{frame}


\begin{frame}
\frametitle{Объявление переменных} 
\begin{center}
\begin{itemize}
\item Переменную нужно объявить перед использованием
\item Примеры объявления:\\
\textcolor{blue}{\textbf{int}} n;\\
\textcolor{blue}{\textbf{float}} p;
\item int -- целочисленный тип \\
\item float -- тип чисел с плавающей точкой
\end{itemize}
\end{center}
\end{frame}

\begin{frame}
\frametitle{Инициализация переменных} 
\begin{center}
\begin{itemize}
\item Переменные инициализируются с помощью оператора присваивания =
\item Примеры:\\
n = 3; \\
\textcolor{blue}{\textbf{float}} p = 5.4; \\
\textcolor{blue}{\textbf{int}} a, b, c = 9;
\end{itemize}
\end{center}
\end{frame}



\begin{frame}
\frametitle{Базовые операторы}
\frametitle{Логические операторы}
Возвращают тип bool
\begin{center}
\begin{tabular}{ l l}
  ! & не \\
  || & или \\
  \&\& & и \\
\end{tabular}
\end{center}
\end{frame}



\begin{frame}
\frametitle{Базовые операторы}
\frametitle{Приоритет операторов}
\begin{center}
\begin{enumerate}
\item (), []
\item ++, --, +, -(унарные), sizeof
\item *, /, \%
\item +, -
\item >,<,<=,>=
\item ==, !=
\item \&, |, \&\&, ||
\item =, +=, и т.д.
\end{enumerate}
\end{center}
Приоритет операторов C подробнее:\\
\href{http://ru.cppreference.com/w/c/language/operator_precedence}
{\textcolor{red}{ru.cppreference.com/w/c/language/operator\_precedence}}
\end{frame}






\section{Управляющие конструкции}
\begin{frame}
\begin{center}
\begin{beamercolorbox}[sep=8pt,center]{part
title}
\usebeamerfont{part title}\insertsection
\end{beamercolorbox}
\end{center}
\end{frame}


\begin{frame}[fragile]
\frametitle{Базовые управляющие конструкции} 
\framesubtitle{if, if else}

\lstset{
  language=C,                % choose the language of the code
  keywordstyle=\color{blue},
  numbers=none,                   % where to put the line-numbers
  stepnumber=1,                   % the step between two line-numbers.        
  numbersep=5pt,                  % how far the line-numbers are from the code
  backgroundcolor=\color{light-gray},  % choose the background color. You must add \usepackage{color}
  showspaces=false,               % show spaces adding particular underscores
  showstringspaces=false,         % underline spaces within strings
  showtabs=false,                 % show tabs within strings adding particular underscores
  tabsize=2,                      % sets default tabsize to 2 spaces
  captionpos=b,                   % sets the caption-position to bottom
  breaklines=true,                % sets automatic line breaking
  breakatwhitespace=true,         % sets if automatic breaks should only happen at whitespace
}

\lstinputlisting{./programms/code_if1.c}



\end{frame}

\begin{frame}[fragile]
\frametitle{Базовые управляющие конструкции} 
\framesubtitle{if, if else}
\lstset{
  language=C,                % choose the language of the code
  keywordstyle=\color{blue},
  numbers=none,                   % where to put the line-numbers
  stepnumber=1,                   % the step between two line-numbers.        
  numbersep=5pt,                  % how far the line-numbers are from the code
  backgroundcolor=\color{light-gray},  % choose the background color. You must add \usepackage{color}
  showspaces=false,               % show spaces adding particular underscores
  showstringspaces=false,         % underline spaces within strings
  showtabs=false,                 % show tabs within strings adding particular underscores
  tabsize=2,                      % sets default tabsize to 2 spaces
  captionpos=b,                   % sets the caption-position to bottom
  breaklines=true,                % sets automatic line breaking
  breakatwhitespace=true,         % sets if automatic breaks should only happen at whitespace
}

\lstinputlisting{./programms/code_if2.c}

\end{frame}


\begin{frame}
\frametitle{Базовые управляющие конструкции} 
\framesubtitle{Тернарный оператор :?}

\lstset{
  language=C,                % choose the language of the code
  keywordstyle=\color{blue},
  numbers=none,                   % where to put the line-numbers
  stepnumber=1,                   % the step between two line-numbers.        
  numbersep=5pt,                  % how far the line-numbers are from the code
  backgroundcolor=\color{light-gray},  % choose the background color. You must add \usepackage{color}
  showspaces=false,               % show spaces adding particular underscores
  showstringspaces=false,         % underline spaces within strings
  showtabs=false,                 % show tabs within strings adding particular underscores
  tabsize=2,                      % sets default tabsize to 2 spaces
  captionpos=b,                   % sets the caption-position to bottom
  breaklines=true,                % sets automatic line breaking
  breakatwhitespace=true,         % sets if automatic breaks should only happen at whitespace
}

\lstinputlisting{./programms/code_ternar.c}
\end{frame}

\begin{frame}
\frametitle{Базовые управляющие конструкции} 
\framesubtitle{Цикл while}

\lstset{
  language=C,                % choose the language of the code
  keywordstyle=\color{blue},
  numbers=none,                   % where to put the line-numbers
  stepnumber=1,                   % the step between two line-numbers.        
  numbersep=5pt,                  % how far the line-numbers are from the code
  backgroundcolor=\color{light-gray},  % choose the background color. You must add \usepackage{color}
  showspaces=false,               % show spaces adding particular underscores
  showstringspaces=false,         % underline spaces within strings
  showtabs=false,                 % show tabs within strings adding particular underscores
  tabsize=2,                      % sets default tabsize to 2 spaces
  captionpos=b,                   % sets the caption-position to bottom
  breaklines=true,                % sets automatic line breaking
  breakatwhitespace=true,         % sets if automatic breaks should only happen at whitespace
}

\lstinputlisting{./programms/code_while.c}

Напечатает 1 2 3 

\end{frame}

\begin{frame}
\frametitle{Базовые управляющие конструкции} 
\framesubtitle{Цикл do while}

\lstset{
  language=C,                % choose the language of the code
  keywordstyle=\color{blue},
  numbers=none,                   % where to put the line-numbers
  stepnumber=1,                   % the step between two line-numbers.        
  numbersep=5pt,                  % how far the line-numbers are from the code
  backgroundcolor=\color{light-gray},  % choose the background color. You must add \usepackage{color}
  showspaces=false,               % show spaces adding particular underscores
  showstringspaces=false,         % underline spaces within strings
  showtabs=false,                 % show tabs within strings adding particular underscores
  tabsize=2,                      % sets default tabsize to 2 spaces
  captionpos=b,                   % sets the caption-position to bottom
  breaklines=true,                % sets automatic line breaking
  breakatwhitespace=true,         % sets if automatic breaks should only happen at whitespace
}

\lstinputlisting{./programms/code_do_while.c}

Напечатает 1 2 3 

\end{frame}

\begin{frame}
\frametitle{Базовые управляющие конструкции} 
\framesubtitle{Цикл for}

\lstset{
  language=C,                % choose the language of the code
  keywordstyle=\color{blue},
  numbers=none,                   % where to put the line-numbers
  stepnumber=1,                   % the step between two line-numbers.        
  numbersep=5pt,                  % how far the line-numbers are from the code
  backgroundcolor=\color{light-gray},  % choose the background color. You must add \usepackage{color}
  showspaces=false,               % show spaces adding particular underscores
  showstringspaces=false,         % underline spaces within strings
  showtabs=false,                 % show tabs within strings adding particular underscores
  tabsize=2,                      % sets default tabsize to 2 spaces
  captionpos=b,                   % sets the caption-position to bottom
  breaklines=true,                % sets automatic line breaking
  breakatwhitespace=true,         % sets if automatic breaks should only happen at whitespace
}

\lstinputlisting{./programms/code_for.c}

Напечатает 1 2 3 

\end{frame}

\begin{frame}
\frametitle{Базовые управляющие конструкции} 
\framesubtitle{Замечания}

В условии циклов может стоять любой оператор:

\lstset{
  language=C,                % choose the language of the code
  keywordstyle=\color{blue},
  numbers=none,                   % where to put the line-numbers
  stepnumber=1,                   % the step between two line-numbers.        
  numbersep=5pt,                  % how far the line-numbers are from the code
  backgroundcolor=\color{light-gray},  % choose the background color. You must add \usepackage{color}
  showspaces=false,               % show spaces adding particular underscores
  showstringspaces=false,         % underline spaces within strings
  showtabs=false,                 % show tabs within strings adding particular underscores
  tabsize=2,                      % sets default tabsize to 2 spaces
  captionpos=b,                   % sets the caption-position to bottom
  breaklines=true,                % sets automatic line breaking
  breakatwhitespace=true,         % sets if automatic breaks should only happen at whitespace
}

\lstinputlisting{./programms/code_loop_remarks.c}


\end{frame}






\begin{frame}
\frametitle{Управляющие конструкции} 
\framesubtitle{Оператор break}

\lstset{
  language=C,                % choose the language of the code
  keywordstyle=\color{blue},
  numbers=none,                   % where to put the line-numbers
  stepnumber=1,                   % the step between two line-numbers.        
  numbersep=5pt,                  % how far the line-numbers are from the code
  backgroundcolor=\color{light-gray},  % choose the background color. You must add \usepackage{color}
  showspaces=false,               % show spaces adding particular underscores
  showstringspaces=false,         % underline spaces within strings
  showtabs=false,                 % show tabs within strings adding particular underscores
  tabsize=2,                      % sets default tabsize to 2 spaces
  captionpos=b,                   % sets the caption-position to bottom
  breaklines=true,                % sets automatic line breaking
  breakatwhitespace=true,         % sets if automatic breaks should only happen at whitespace
}

\lstinputlisting{./programms/code_break.c}
\end{frame}

\begin{frame}
\frametitle{Управляющие конструкции} 
\framesubtitle{Оператор continue}

\lstset{
  language=C,                % choose the language of the code
  keywordstyle=\color{blue},
  numbers=none,                   % where to put the line-numbers
  stepnumber=1,                   % the step between two line-numbers.        
  numbersep=5pt,                  % how far the line-numbers are from the code
  backgroundcolor=\color{light-gray},  % choose the background color. You must add \usepackage{color}
  showspaces=false,               % show spaces adding particular underscores
  showstringspaces=false,         % underline spaces within strings
  showtabs=false,                 % show tabs within strings adding particular underscores
  tabsize=2,                      % sets default tabsize to 2 spaces
  captionpos=b,                   % sets the caption-position to bottom
  breaklines=true,                % sets automatic line breaking
  breakatwhitespace=true,         % sets if automatic breaks should only happen at whitespace
}

\lstinputlisting{./programms/code_continue.c}
\end{frame}

\begin{frame}
\frametitle{Управляющие конструкции} 
\framesubtitle{Оператор выбора switch}

\lstset{
  language=C,                % choose the language of the code
  keywordstyle=\color{blue},
  numbers=none,                   % where to put the line-numbers
  stepnumber=1,                   % the step between two line-numbers.        
  numbersep=5pt,                  % how far the line-numbers are from the code
  backgroundcolor=\color{light-gray},  % choose the background color. You must add \usepackage{color}
  showspaces=false,               % show spaces adding particular underscores
  showstringspaces=false,         % underline spaces within strings
  showtabs=false,                 % show tabs within strings adding particular underscores
  tabsize=2,                      % sets default tabsize to 2 spaces
  captionpos=b,                   % sets the caption-position to bottom
  breaklines=true,                % sets automatic line breaking
  breakatwhitespace=true,         % sets if automatic breaks should only happen at whitespace
}

\lstinputlisting{./programms/code_switch.c}
\end{frame}

\begin{frame}
\frametitle{Управляющие конструкции} 
\framesubtitle{Оператор безусловного перехода goto}

\begin{itemize}
\item Оператор goto передает управление на оператор, помеченный меткой
\item Оператор goto в языках высокого уровня является объектом критики, поскольку чрезмерное его применение приводит к созданию нечитаемого кода
\item Использование goto в практике программирования на языке C настоятельно не рекомендуется
\end{itemize}


\end{frame}

\begin{frame}
\frametitle{Литература} 
\begin{enumerate}
\item Брайан У. Керниган, Деннис М. Ритчи "Язык программирования C"
\item Томас Кормен, Чарльз Лейзерстон, Рональд Ривест, Клиффорд Штайн "Алгоритмы: построение и анализ"
\end{enumerate}

\end{frame}



\end{document}
