\documentclass{article}
\usepackage[utf8x]{inputenc}
\usepackage{ucs}
\usepackage{amsmath} 
\usepackage{amsfonts}
\usepackage{upgreek}
\usepackage[english,russian]{babel}
\usepackage{graphicx}
\usepackage{float}
\usepackage{textcomp}
\usepackage{hyperref}
\usepackage{geometry}
  \geometry{left=2cm}
  \geometry{right=1.5cm}
  \geometry{top=1cm}
  \geometry{bottom=2cm}
\usepackage{tikz}
\usepackage{ccaption}
\usepackage{mathrsfs}


\begin{document}
\pagenumbering{gobble}

\section*{Теория:}
\begin{enumerate}
\item \textbf{Основные команды командной строки linux (\texttt{bash})}:\\
 \texttt{cd}, \texttt{ls} (опции \texttt{-l} и \texttt{-a}), \texttt{pwd}, \texttt{cp}, \texttt{mv}, \texttt{rm} (опция \texttt{-r}), \texttt{mkdir}, \texttt{find}(опция \texttt{-name}), \texttt{du}(опции \texttt{-s} и \texttt{-h}), программа \texttt{top}, текстовый редактор \texttt{nano} (или текстовый редактор \texttt{vim}), компилятор \texttt{gcc} (опции \texttt{-o}, \texttt{-std=c99}, \texttt{-lm} и \texttt{-S}). Перенаправление вывода \texttt{>}.


\item \textbf{Управляющие конструкции. Ветвление и циклы}:
Оператор ветвления \texttt{if-else}. Ислпользования логических операторов в условии оператора ветвления. Чем отличается тернарный оператор от \texttt{if-else}? Цикл \texttt{while}. Цикл \texttt{for}. Цикл \texttt{do while}. Операторы \texttt{break} и \texttt{continue}. Оператор \texttt{switch}. Оператор \texttt{break} в \texttt{switch} выражениях. Чем отличается \texttt{switch} от многократного вызова \texttt{if else}.

\item \textbf{Операторы}
Арифметические операторы(\texttt{+ - * / \%}). Что делает оператор деления если аргументы целочисленные и если аргументы -- числа с плавающей точкой. Оператор присваивания (\texttt{=}). Оператор присваивания сложения и подобные ему (\texttt{+= -= *= /= \%=}). Операторы инкремента и декремента (\texttt{++  $--$}). Префиксный  и постфиксный инкремент/декремент, чем они отличаются. Операторы сравнения (\texttt{== != > < >= <=}). Что возвращают операторы сравнения? Логические операторы (\texttt{!  ||  \&\&}). Побитовые операторы (\texttt{$\sim$  \&  |  \textasciicircum \quad <{}<  >{}>}). Тернарный оператор (\texttt{? :}). Оператор нахождения адреса (\texttt{\&}). Оператор нахождения размера переменной (\texttt{sizeof}). Оператор разыменования (\texttt{*}). Оператор обращения к элементу массива (\texttt{[]}) и его связь с оператором разыменования. Оператор доступа к полю структуры (\texttt{.}). Оператор доступа к полю структуры через указатель на структуру (\texttt{->}). Приоритет операторов. Создание новых названий для типов с помощью ключевого слова \texttt{typedef}.


\item \textbf{Переменные и базовые типы}
Переменные. Понятия объявления, определения, в чём различие между ними. Инициализация и присваивание, в чём различие между ними. Типы целочисленных переменных: \texttt{char}, \texttt{short}, \texttt{int}, \texttt{long}, \texttt{long long} и их \texttt{unsigned}-аналоги. Размеры этих типов на современных системах и диапазоны значений, которые могут принимать данные типы. Представление целочисленных переменных в памяти. Как хранятся в памяти отрицательные числа? Дополнительный код. Что такое тип \texttt{size\_t}. Когда он используется? Типы чисел с плавающей точкой: \texttt{float}, \texttt{double} и \texttt{long double}. Представление этих типов в памяти. Стандарт IEEE 754. Размеры этих типов и количество значащих цифр, которые могут хранить переменные этих типов. Неявное приведение типов. Когда оно происходит? Явное приведение типов, как привести один тип в другой. Перечисляемый тип. Константы Квалификатор типа \texttt{const}. Разница между определение константы с помощью директивы \texttt{\#define} и квалификатора \texttt{const}.


\item \textbf{Основные функции стандартной библиотеки}\\
Ввод и вывод в языке C. Функции \texttt{printf} и \texttt{scanf} из библиотеки \texttt{stdio.h}. Что принимает и возвращает функция \texttt{printf}? Что принимает и возвращает функция \texttt{scanf}? Математическая библиотека \texttt{math.h}. Функции \texttt{sqrt}, \texttt{exp}, \texttt{sin}, \texttt{cos}, \texttt{tan}, \texttt{asin}, \texttt{acos}, \texttt{atan}, \texttt{atan2}, \texttt{fabs}, \texttt{floor}, \texttt{log}, \texttt{pow}. Сравнение двух чисел с плавающей точкой с помощью функции \texttt{fabs}. Библиотека \texttt{stdlib.h}. Функции \texttt{exit}, \texttt{malloc}, \texttt{calloc}, \texttt{realloc}, \texttt{free}, \texttt{rand}, \texttt{srand}, \texttt{abs}. Основные функции для работы с файлами: \texttt{fopen}, \texttt{fscanf}, \texttt{fprintf} и \texttt{fclose}.

\item \textbf{Массивы}\\
Массивы. Элемент массива и индекс массива. Как хранятся массивы в памяти? Объявление и определение массивов. Инициализация массивов. Можно ли присваивать массив другому массиву с помощью оператора \texttt{=}? Как распечатать массив? Размер массивов. Как узнать размер массива? Как передаются массивы в функции. Array to pointer decay. Как вернуть массив из функции. Двумерный массив. Объявление, определение и инициализация двумерного массива. Как двумерный массив хранится в памяти? Как двумерный массив передаётся внутрь функции?


\item \textbf{Символы и строки}\\
Символы. Кодировка ASCII. Использование целочисленного типа \texttt{char} для хранения кодов символов. Чтение и запись символов (спецификатор \texttt{\%с}). Строки в языке C. Символ завершения строки. Чтение и запись строк (спецификатор \texttt{\%s}). Библиотека \texttt{string.h}. Функции \texttt{strlen}, \texttt{strcpy}, \texttt{strncpy}, \texttt{strcmp}, \texttt{strcat}, \texttt{strchr}, \texttt{strstr}. Функции \texttt{sprintf} и \texttt{sscanf}, использование этих функций для конвертации числа в строку и наоборот.


\item \textbf{Функции}\\
Функции. Аргументы функции. Возвращаемое значение функции. Объявления функции. Прототип функции. Определение функции. Как переменные базовых типов и структуры передаются в функции? Как массивы передаюся в функции? Три типа передачи аргументов в функцию (по значению, через указатель, через указатель на константу). Передача одномерных и многомерных массивов в функции. Возврат из функции. Ключевое слово \texttt{return}. Ключевое слово \texttt{void}. Рекурсия. Алгоритмы вычисления факториала, чисел Фибоначчи и бинарного возведения в степень с помощью рекурсии. Реализация вызова функций с помощью сегмента памяти стек. Адрес возрата. Стековый кадр. Переполнение стека при рекурсии. 

\item \textbf{Структуры}\\
Структуры. Объявление структуры. Определение структуры. Инициализация структуры. Доступ к полю структуры. Размер структуры. Выравнивание полей структур. Указатели на структуры. Доступ к полю по указателю на структуру. Передача структур в функции и возврат их из функций. 

\item \textbf{Память и указатели}\\
Шестнадцатеричная и восьмеричная системы счисления. Печать и считывания целочисленных переменных в восьмеричной и шестнадцатеричной системах с помощью спецификаторов \texttt{\%o} и \texttt{\%x}. Порядок байт. Little endian и big endian. Указатель. Объявление указателя. Инициализация указателя. Размер указателя на 64-х битных системах. Адресная арифметика. Операторы, применимые к указателям и что они делают: \texttt{++}, \texttt{--}, прибавление целого числа, вычитание двух указателей, разымкнование и оператор взятие индекса (\texttt{[]}). Операции нахождения адреса (\texttt{\&}) переменной и операция разыменования \texttt{*} указателя. Указатели разных типов, чем они различаются. Указатель \texttt{void*}. Константный указатель и указатель на константу. Указатели на указатели. Функции \texttt{memset}, \texttt{memcpy}, \texttt{memmove} из стандартной библиотеки.

\item \textbf{Сегмент памяти Стек}\\
Что такое сегменты памяти? Ошибка Segmentation Fault. Виртуальная память. Сегмент памяти стек(не путать с абстрактным типом данных -- стек). Выделение и освобождение памяти в стеке. Размер стека. Переполнение стека. Как можно переполнить стек?

\item \textbf{Сегмент памяти Куча. Динамическое выделение памяти}\\
Сегмент памяти куча (heap -- не путать со структурой данных -- куча).
Динамическое выделение и освобождение памяти в куче: \texttt{malloc}, \texttt{calloc}, \texttt{realloc} и \texttt{free}. Преимущества и недостатки кучи перед стеком Организация массива в памяти. Организация связного списка в памяти. Хранение двумерного массива в виде одномерного массива. Динамический двумерный массив. Утечки памяти. Основы работы с valgrind.

\item \textbf{Сегмент памяти Текст}\\
Сегмент памяти text. Преобразование кода программы в код на языке ассемблера и в двоичный код. Указатели на функции. Объявление указателей на функции. Передача указателей на функции в другие функции. Стандартная функция \texttt{qsort} и передача в ней компаратора.

\item \textbf{Сегмент памяти Данные}\\
Сегменты памяти data и bss. Чем они различаются? Что такое глобальные переменные. Что такое статические переменные. Где хранятся глобальные и статические переменные? Когда и как инициализируются глобальные и статические переменные? Строковые литералы. Где хранятся строковые литералы?

\item  \textbf{Вычислительная сложность, $O(n)$ нотация}\\
Что такое $O(n)$, $\Omega(n)$ и $\Theta(n)$ нотации, математическое определение и смысл. Машина Тьюринга (детерминистическая и недетерминистическая). Классы сложности задач: $P$, $NP$, $PSPACE$, $EXPTIME$ и $EXPSPACE$.
Знать(или уметь выводить) сложности всех пройденных алгоритмов: алгоритм бинарного поиска в отсортированном массиве, алгоритмы сортировки (пузырьком, вставками, выбором, быстрая, сортировка слиянием, подсчётом, цифровая). Знать или уметь выводить сложности операций с пройденными структурами данных: поиск элемента по индексу в массиве и списке, вставка в массив(статический и динамический) и список; удаление из массива и списка; поиск по массиву и списку.

\item \textbf{Сортировка}\\
Парадигма разделяй и властвуй. Как возникает логарифмическая сложность в задачах, решаемых с помощью метода разделяй и властвуй. Алгоритм бинарного поиска в отсортированном массиве.
Сортировки пузырьком, вставками, выбором, быстрая, слиянием, подсчётом, цифровая и $Bogosort$. Вычислительные сложности всех этих алгоритмов. Уметь писать сортировку выбором, быструю сортировку и сортировку подсчётом. Cтандартная функция \texttt{qsort}. Функция компаратор \texttt{cmp}. Сортировка с помощью бинарного дерева поиска и с помощью бинарной кучи.


\item  \textbf{Абстрактные типы данных. Стек и очередь}\\
Структуры данных и абстрактный типы данных. Абстрактный тип данных Стек. Операции push и pop. Реализация стека на основе динамического массива. Динамическое выделение памяти для стека. Стратегии перевыделения памяти: аддитивная и мультипликативная. Абстрактный тип данных Очередь. Реализация очереди на основе динамического массива. Реализация стека и очереди на основе связного списка. Алгоритмические сложности операций со стеком.

\item \textbf{Связный список}\\
Связный список. Узел связного списка. Реализация связного списка на языке C. Функции нахождения длины списка, вставки элемента в начало/конец, удаления элемента из начала/конца, поиска в списке, обращения списка. Алгоритмические сложности операций со списком. Двусвязный список.


\item \textbf{Деревья}\\
Определение графа. Определение дерева. Бинарное дерево. Бинарное дерево поиска(bst). Высота дерева. Представление бинарного дерева в языке C. Операции поиска, добавления и удаления элемента в бинарном дереве поиска. Вычислительные сложности этих операций. Сбалансированное дерево. Два распространённых приёма для балансировки бинарных деревьев поиска (AVL-деревья и красно-чёрные деревья). Определение AVL-дерева. Операции вращения. Операции поиска, добавления и удаления элемента в AVL-дерево. Вычислительные сложности этих операций. Абстрактные типы данных множество (set) и словарь (dictionary).


\item \textbf{Хеш-таблица}\\
Что такое хеш-функция? Свойства хорошей хеш-функции. Что такое хеш-таблица? Добавление/удаление элемента в хеш-таблицу. Поиск элемента в хеш-таблице. Вычислительная сложность операций с хеш-таблицей. Абстрактные типы данных множество и словарь на основе хеш-таблицы. Чем различие между реализацие множества и словаря на основе сбалансированного дереве и на основе хеш-таблицы.

\item \textbf{Файлы и аргументы командной строки}\\
Системные вызовы. Системные вызовы для работы с файлами в Linux (\texttt{open}, \texttt{read}, \texttt{write} и \texttt{lseek}). Буферизация. Работа с файлами в языке C - библиотека \texttt{stdio.h}. Структура \texttt{FILE}. Функции \texttt{fopen} и \texttt{fclose}. Режимы открытия файлов (\texttt{"r"\quad "w"\quad "a"\quad "r+"\quad "w+"\quad "a+"}). Функции \texttt{fgetc}, \texttt{fputc}, \texttt{fscanf}, \texttt{fprintf}, \texttt{fgets}, \texttt{fputs}, \texttt{rewind} и \texttt{fseek}. Посимвольное чтение файла в цикле. Константа \texttt{EOF}. Чтение и запись в бинарные файлы. Функции \texttt{fread} и \texttt{fwrite}. Формат файла \texttt{.csv}. Форматы файлов для хранения изображений \texttt{.ppm} (текстовый и бинарный) и \texttt{.jpg}. Аргументы командной строки (\texttt{argc} и \texttt{argv}). Преобразование строки в число и обратно с помощью \texttt{sprintf} и \texttt{sscanf}.


\item \textbf{Этапы компиляции}\\
Что такое файл исходного кода и исполняемый файл. Этап компиляции: препроцессинг, компиляция и линковка. Директивы препроцессора \texttt{\#include}, \texttt{\#define}, \texttt{\#ifdef}, \texttt{\#else} и \texttt{\#endif}. Компиляция программы с помощью \texttt{gcc}. Опции \texttt{gcc: -E, -c, -S}.


\item \textbf{* Очередь с приоритетом. Двоичная куча}\\
Хранение бинарных деревьев в массиве. Структура данных - двоичная куча. Операции добавления элемента в кучу. Операция удаления максимального элемента из кучи. Вычислительные сложности этих операций. Абстрактный тип данных - очередь с приоритетом. Реализация очереди с приоритетом с помощью двоичной кучи. Пирамидальная сортировка.

\item \textbf{* Динамическое программирование}\\
Суть динамического программирования. Задача о вычислении чисел Фибоначчи. Задача о поиске подмассива с максимальной суммой. Задача о наибольшей общей подпоследовательности. Задача о рюкзаке. 

\end{enumerate}



\newpage
\section*{Материалы для подготовки:}
\begin{enumerate}
\item Кормен, Лейзерсон, Ривест. Алгоритмы: построение и анализ.\\ 
\hspace*{16pt} \href{https://lib.mipt.ru/book/16736/}{lib.mipt.ru/book/16736/}
\item Керниган Ритчи Язык программирования C \\
\hspace*{16pt} \href{https://lib.mipt.ru/book/266005/}{lib.mipt.ru/book/266005/}
\item Лекторий МФТИ: \\ \hspace*{16pt} 
\href{http://lectoriy.mipt.ru/course/ComputerTechnology-Informatics-14L#lectures}
{lectoriy.mipt.ru/course/ComputerTechnology-Informatics-14L\#lectures}
\item Фоксфорд\\
\href{https://www.youtube.com/watch?v=J-7XmpSUeQE&list=PL66kIi3dt8A5sa_qBur8uxmtuuwuJQGS1&index=28}
{www.youtube.com/watch?v=J-7XmpSUeQE\&list=PL66kIi3dt8A5sa\_qBur8uxmtuuwuJQGS1\&index=28}
\item Язык C и структуры данных в Йелле.\\ \hspace*{16pt} \href{http://www.cs.yale.edu/homes/aspnes/classes/223/notes.html}{www.cs.yale.edu/homes/aspnes/classes/223/notes.html}
\end{enumerate}

\subsubsection*{Дополнительные материалы по некоторым вопросам:}
\begin{itemize}

\item Сегменты памяти:
\begin{itemize}
\item \href{https://www.youtube.com/watch?v=_8-ht2AKyH4}{youtube.com/watch?v=\_8-ht2AKyH4}  (англ.)
\end{itemize}
\item Этапы компиляции:
\begin{itemize}
\item \href{https://www.youtube.com/watch?v=Je9FFb2zldk}{youtube.com/watch?v=Je9FFb2zldk}
\item \href{https://www.youtube.com/watch?v=ylA55D4B4_M}{youtube.com/watch?v=ylA55D4B4\_M}
\end{itemize}
\item Стек вызовов:
\begin{itemize}
\item \href{https://www.youtube.com/watch?v=hU3ONaqZzD8}{youtube.com/watch?v=hU3ONaqZzD8}
\end{itemize}


\end{itemize}
\end{document}