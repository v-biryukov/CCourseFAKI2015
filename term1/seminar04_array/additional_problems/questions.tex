\documentclass{article}
\usepackage[utf8x]{inputenc}
\usepackage{ucs}
\usepackage{amsmath} 
\usepackage{mathtext}
\usepackage{amsfonts}
\usepackage{upgreek}
\usepackage[english,russian]{babel}
\usepackage{graphicx}
\usepackage{float}
\usepackage{textcomp}
\usepackage{hyperref}
\usepackage{geometry}
  \geometry{left=2cm}
  \geometry{right=1.5cm}
  \geometry{top=1cm}
  \geometry{bottom=2cm}
\usepackage{tikz}
\usepackage{ccaption}



\begin{document}
\pagenumbering{gobble}

\section*{Задачи на массивы:}

\begin{enumerate}
\item \textbf{Аrray declaration:} Объявить следующие массивы:
\begin{itemize}
\item массив из 10 элементов типа \textbf{int}
\item массив из 20 элементов типа \textbf{float}
\item массив из 30 элементов типа \textbf{unsigned long long}
\end{itemize}
\item \textbf{Array initialization:} Объявить массив под названием days\_in\_month с 12 элементами типа \textbf{int} и инициализировать его следующими значениями: 31, 28, 31, 30, 31, 30, 31, 31, 30, 31, 30, 31. Отдельной операцией изменить второе значение с массиве с 28 на 29.
\item \textbf{Print array:} Напечатать содержимое массива days\_in\_month на экран. Числа должны быть напечатаны в одну строку через пробел.
\item \textbf{Print array function:} Написать функцию \textbf{void print\_array(int n, int* arr)}, которая будет печатать содержимое массива с элементами типа \textbf{int}. На вход эта функция должна принимать целое число n -- число элементов в массиве и arr -- массив из целых чисел типа \textbf{int} (или указатель на первый элемент в этом массиве). Использовать эту функцию в функции main(), чтобы вывести на экран массив days\_in\_month.
\item \textbf{Sum:} Написать функцию \textbf{int sum(int n, int* arr)}, которая будет возвращать сумму массива целых чисел. Входные параметры такие же как и предыдущей задаче. Функция не должна ничего печатать и считывать. Использовать эту функцию в функции main() чтобы найти сумму чисел массива days\_in\_month.
\item \textbf{Min index:} Написать функцию \textbf{int min\_index(int n, int* arr)}, которая будет возвращать индекс наименьшего числа в массиве. Входные параметры такие же как и предыдущей задаче. Функция не должна ничего печатать и считывать. Если в массиве есть несколько минимальных элементов, то функция должна вывести индекс первого из них. Использовать эту функцию в функции main() чтобы найти номер месяца с наименьшим числом дней.

\item \textbf{Scan array:} Считать массив целых чисел из стандартного входа. Сначала на вход подаётся одно целое число n -- количество чисел в массиве. Затем подаются n чисел через пробел.
 
\item \textbf{Scan array function:} Написать функцию \textbf{int scan\_array(int* arr)}, которая будет считывать массив. На вход этой функции подаётся массив, причём известно, что размер этого массива достаточен для хранения всех считываемых чисел. Функция должна сначала считать целое число -- количество элементов в массиве. Затем она должна считать все элементы. Функция должна вернуть количество элементов в массиве. Использовать эту функцию в функции main() чтобы считать массив.

\item \textbf{Summary:} Используя написанные функции, создать программу которая будет считывать массив и печатать все элементы массива, идущие до минимального.



\end{enumerate}

\end{document}