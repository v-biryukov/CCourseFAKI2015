\documentclass{article}
\usepackage[utf8x]{inputenc}
\usepackage{ucs}
\usepackage{amsmath} 
\usepackage{amsfonts}
\usepackage{marvosym}
\usepackage{wasysym}
\usepackage{upgreek}
\usepackage[english,russian]{babel}
\usepackage{graphicx}
\usepackage{float}
\usepackage{textcomp}
\usepackage{hyperref}
\usepackage{geometry}
  \geometry{left=2cm}
  \geometry{right=1.5cm}
  \geometry{top=1cm}
  \geometry{bottom=2cm}
\usepackage{tikz}
\usepackage{ccaption}
\usepackage{multicol}

\hypersetup{
   colorlinks=true,
   citecolor=blue,
   linkcolor=black,
   urlcolor=blue
}

\usepackage{listings}
%\setlength{\columnsep}{1.5cm}
%\setlength{\columnseprule}{0.2pt}

\usepackage[absolute]{textpos}

\renewcommand{\thesubsection}{\arabic{subsection}}

\begin{document}
\pagenumbering{gobble}
\lstset{
  language=C,                % choose the language of the code
  basicstyle=\linespread{1.1}\ttfamily,
  columns=fixed,
  fontadjust=true,
  basewidth=0.5em,
  keywordstyle=\color{blue}\bfseries,
  commentstyle=\color{gray},
  stringstyle=\ttfamily\color{orange!50!black},
  showstringspaces=false,
  numbersep=5pt,
  numberstyle=\tiny\color{black},
  numberfirstline=true,
  stepnumber=1,                   % the step between two line-numbers.        
  numbersep=10pt,                  % how far the line-numbers are from the code
  backgroundcolor=\color{white},  % choose the background color. You must add \usepackage{color}
  showstringspaces=false,         % underline spaces within strings
  captionpos=b,                   % sets the caption-position to bottom
  breaklines=true,                % sets automatic line breaking
  breakatwhitespace=true,         % sets if automatic breaks should only happen at whitespace
  xleftmargin=.2in,
  extendedchars=\true,
  keepspaces = true,
}
\lstset{literate=%
   *{0}{{{\color{red!20!violet}0}}}1
    {1}{{{\color{red!20!violet}1}}}1
    {2}{{{\color{red!20!violet}2}}}1
    {3}{{{\color{red!20!violet}3}}}1
    {4}{{{\color{red!20!violet}4}}}1
    {5}{{{\color{red!20!violet}5}}}1
    {6}{{{\color{red!20!violet}6}}}1
    {7}{{{\color{red!20!violet}7}}}1
    {8}{{{\color{red!20!violet}8}}}1
    {9}{{{\color{red!20!violet}9}}}1
}


\section*{Массивы:}

Эти задачи нужно оформить в соответствии с правилами оформления: \\ 
\href{http://style.vdi.mipt.ru/CodeStyle.html}{http://style.vdi.mipt.ru/CodeStyle.html}\\
\href{https://www.youtube.com/watch?v=NSNvfr_KpDc}{youtube.com/watch?v=NSNvfr\_KpDc}\\
 и прислать мне на почту \href{mailto:vladimir.biryukov@phystech.edu}{vladimir.biryukov@phystech.edu}.\\
 
В некоторых задачах потребуются входные файлы, которые можно найти по адресу:


\subsection{Простая работа с файлами. fprintf и fscanf}
Для простейшей работы с файлами мы будем использовать функции \texttt{fscanf} и \texttt{fprintf}, которые работают аналогично функциям \texttt{printf} и \texttt{scanf}. Подробней файлы будем проходить позднее. Библиотека для работы с файлами - \texttt{stdio.h} (та же самая, что и для \texttt{printf} и \texttt{scanf}). \\
\subsubsection*{Пример записи в файл:}
\begin{lstlisting}
#include <stdio.h>
int main()
{
	// Открываем файл под названием result.txt
	// "w" = write - открываем файл на запись
	// Так как файл открывается на запись, то необязательно чтобы он существовал
	FILE* fout = fopen("result.txt", "w");
	
	fprintf(fout, "Hello world of files\n");	
		
	// Закрываем файл
	fclose(fout);
}
\end{lstlisting}
\subsubsection*{Пример чтения чисел из файла:}
\begin{lstlisting}
#include <stdio.h>
int main()
{
	// Открываем файл под названием numbers.txt
	// "r" = read - открываем файл на чтение
	FILE* fin = fopen("numbers.txt", "r");
	
	int n;
	int a[100];
	
	fscanf(fin, "%d", &n);
	for (int i = 0; i < n; i++)
		fscanf(fin, "%d", &a[i]);	
		
	fclose(fin);
}
\end{lstlisting}

\begin{itemize}
\item \textbf{Задача 1. Чтение/запись:} Создайте файл \texttt{input.txt} в котором будут храниться входные числа в следующем виде:
\begin{lstlisting}
15
54 32 53 64 1 21 77 4 6 81 34 10 92 17 42
\end{lstlisting}
Сначала идёт число n - количевство чисел последовательности, а потом эти n чисел. Ваша задача - написать программу, которая будет считывать эти числа, возводить их в квадрат и записывать результат в файл \texttt{output.txt}.
\end{itemize}
\newpage

\subsection{Сортировки}

\begin{lstlisting}
#include <stdio.h>
void print_array(int lo, int hi, int* arr)
{
	for (int i = lo; i < hi; i++)
		printf("%d ", arr[i]);
	printf("\n");	
}

// Сортировка выбором, которая сортирует числа массива с индексами от lo до hi-1
// Выбор таких индексов удобен для реализации реккурсивных алгоритмов сортировок
void selection_sort(int lo, int hi, int* arr)
{
	for (int j = lo; j < hi; j++)
	{
		// Находим индекс минимального элемента на отрезке [j:hi-1]
		int min_index = j;
		for (int i = j+1; i < hi; i++)
			if (arr[i] < arr[min_index])
				min_index = i;
		
		// Меняем местами элемент номер j и минимальный элемент
		int temp = arr[j];
		arr[j] = arr[min_index];
		arr[min_index] = temp;
	}
}

int main()
{
	int a[15] = {45, 73, 34, 82, 31, 1, 64, 54, 47, 73, 62, 11, 16, 7, 26};
	selection_sort(0, 15, a);
	print_array(0, 15, a);
}
\end{lstlisting}
\begin{itemize}
\item \textbf{Задача 2. Чтение/запись сортировки:} Считайте числа из файла \texttt{input.txt} и запишите их в файл \texttt{output.txt} в отсортированом виде.
\item \textbf{Задача 3. Сортировка пузырьком (2 балла):} Напишите функцию \texttt{void bubble\_sort(int lo, int hi, int* arr)}, реализующую алгоритм сортировки пузырьком. О сортировке пузырьком можно посмотреть, например, тут:\\
\href{https://www.youtube.com/watch?v=oqpICiM165I}{youtube.com/watch?v=oqpICiM165I} \\
\href{https://www.youtube.com/watch?v=xli_FI7CuzA}{youtube.com/watch?v=xli\_FI7CuzA}

\item \textbf{Задача 4. Быстрая сортировка (2 балла):} Напишите функцию \texttt{void quick\_sort(int lo, int hi, int* arr)}, реализующую алгоритм быстрой сортировки (советую использовать разбиение Ломуто, так как оно немного проще). О быстрой сортировке можно посмотреть, например, тут:\\
\href{https://ru.wikipedia.org/wiki/%D0%91%D1%8B%D1%81%D1%82%D1%80%D0%B0%D1%8F_%D1%81%D0%BE%D1%80%D1%82%D0%B8%D1%80%D0%BE%D0%B2%D0%BA%D0%B0}{Википедия}\\
\href{https://www.youtube.com/watch?v=MZaf_9IZCrc}{www.youtube.com/watch?v=MZaf\_9IZCrc}

\item \textbf{Задача 5. Сортировка большого количества чисел:} В файле \texttt{numbers.txt} хранятся 300000 случайных чисел. Отсортируйте их с помощью 3-х различных методов сортировки (выбором, пузырьком и быстрой). Результат запишите в файл \texttt{sorted\_numbers.txt}.
\end{itemize}
\newpage
\subsection{Двумерные массивы}

\begin{lstlisting}
#include <stdio.h>
// Зададим константу MAX = 200 - максимальныый возможный размер матрицы
#define MAX 200

// В отличии от одномерного массива, в двумерном массиве при передаче
// в функцию обязательно нужно указывать размер ( количество столбцов = MAX )
// %g - печатает вещественные числа также как и %f, но без нулей на конце
void print_array(int n, float arr[MAX][MAX]) 
{
    for (int i = 0; i < n; i++) 
    {
        for (int j = 0; j < n; j++)
            printf("%5g ", arr[i][j]);
        printf("\n");
    }
}

// Суммируем 2 квадратные матрицы A и B размера n на n и записываем результат в C
void sum(int n, float A[MAX][MAX], float B[MAX][MAX], float C[MAX][MAX])
{
    for (int i = 0; i < n; i++) 
        for (int j = 0; j < n; j++)
            C[i][j] = A[i][j] + B[i][j];
}
int main() 
{
    // Создаём массивы вещественных чисел ( с запасом )
    float a[MAX][MAX] = {{7, 7, 2}, {1, 8, 3}, {2, 1, 6}};
    float b[MAX][MAX] = {{5, 2, 9}, {-4, 2, 11}, {7, 1, -5}};
    
    // Мы создали матрицы с 200 на 200 ( c запасом ), но будем использовать только 
    // маленькую часть 3 на 3. В будущем мы научимся как создавать матрицы нужного
    // размера во время выполнения программы 
    
    printf("a = \n");
    print_array(3, a);
    
    printf("b = \n");
    print_array(3, b);
    
    float c[MAX][MAX];
    sum(3, a, b, c);    
    printf("a + b = \n");
    print_array(3, c);
}

\end{lstlisting}
\newpage
\begin{itemize}
\item \textbf{Задача 6. Умножение на число:} Написать функцию \texttt{void multiply\_by\_number(int n, float A[MAX][MAX], float x)}, которая умножает квадратную матрицу A (n на n) на число x.
\item \textbf{Задача 7. Присвоение:} Написать функцию \texttt{void assign(int n, float A[MAX][MAX], float B[MAX][MAX])}, которая присваивает элементам матрицы A соответствующие элементы матрицы B (т. е. A = B).
\item \textbf{Задача 8. Умножение матриц (2 балла):} Написать функцию \texttt{void multiply(int n, float A[MAX][MAX], float B[MAX][MAX], float C[MAX][MAX])}, которая перемножает матрицы A и B (строка на столбец), а результат записывает в матрицу C. \\
Проверьте ваш код на следующих тестах:

\begin{center}
$\begin{pmatrix}
7 & 7 & 2 \\
1 & 8 & 3 \\
2 & 1 & 6 \\
\end{pmatrix} * 
\begin{pmatrix}
5 & 2 & 9 \\
-4 & 2 & 11 \\
7 & 1 & -5 \\
\end{pmatrix}=
\begin{pmatrix}
21 & 30 & 130 \\
-6 & 21 & 82 \\
48 & 12 & -1 \\
\end{pmatrix}
$
\end{center}

\begin{center}

$
\begin{pmatrix}
5 & 2 & 9 \\
-4 & 2 & 11 \\
7 & 1 & -5 \\
\end{pmatrix} *
\begin{pmatrix}
7 & 7 & 2 \\
1 & 8 & 3 \\
2 & 1 & 6 \\
\end{pmatrix}
=
\begin{pmatrix}
55 & 60 & 70 \\
-4 & -1 & 64 \\
40 & 52 & -13 \\
\end{pmatrix}
$
\end{center}

\begin{center}

$
\begin{pmatrix}
7 & 7 & 2 \\
1 & 8 & 3 \\
2 & 1 & 6 \\
\end{pmatrix} *
\begin{pmatrix}
0 & 0 & 1 \\
0 & 1 & 0 \\
1 & 0 & 0 \\
\end{pmatrix}
=
\begin{pmatrix}
2 & 7 & 7 \\
3 & 8 & 1 \\
6 & 1 & 2 \\
\end{pmatrix}
$
\end{center}
В файлах \texttt{mat\_A10.txt} и \texttt{mat\_B10.txt} лежат матрицы 10 на 10. считайте эти матрицы с помощью \texttt{fscanf}, перемножьте (A на B) и запишите результат в другой файл с помощью \texttt{fprintf}. В результате должно получиться:
\begin{center}
$
\begin{pmatrix}
259 & -15 & 237 & 257 &  231 &  67  & 237  & -64  & 152  & 363 \\
555 & 233 & 539 & 188 &  356 &  325 &  423 &  -47 &  123 &  387 \\
497 & 512 & 572 & 95  & 619  & 155  & 414  & 207  & 203  & 217 \\
455 & 280 & 675 & 354 &  664 &  346 &  483 &  177 &  168 &  404 \\
264 & 182 & 272 & 290 &  474 &  -33 &  234 &  99  & 379  & 156 \\
272 & 180 & 469 & 286 &  326 &  282 &  325 &  215 &  195 &  231 \\
421 & 363 & 475 & 506 &  359 &  481 &  468 &  101 &  325 &  328 \\
384 & 218 & 567 & 395 &  475 &  488 &  361 &  168 &  291 &  298 \\
387 & 297 & 480 & 170 &  318 &  423 &  483 &  10  & -17  & 406 \\
193 & 241 & 486 & 38  & 403  & 146  & 286  & 326  & 212  & 172 \\
\end{pmatrix}
$
\end{center}

\item \textbf{Задача 9. Матрица в степени:} Написать функцию \texttt{void power(int n, float A[MAX][MAX], float C[MAX][MAX], int k)}, которая вычисляет $A^k$, т.е. возводит матрицу A в k-ю степень, а результат записывает в матрицу C. Используйте функции \texttt{multiply} и \texttt{assign}. \\ Псевдокод функции:
\begin{lstlisting}
float B[MAX][MAX]
B = A
C = A
for ( k - 1 раз )
{
	C = A*B
	B = C
}
\end{lstlisting}
Проверьте ваш код на следующих тестах:
\begin{center}

$
\begin{pmatrix}
7 & 7 & 2 \\
1 & 8 & 3 \\
2 & 1 & 6 \\
\end{pmatrix}^4 = 
\begin{pmatrix}
7116 & 15654 & 9549 \\
4002 & 8955 & 6135 \\
3369 & 6165 & 4350 \\
\end{pmatrix}
$
\end{center}

\begin{center}
$
\begin{pmatrix}
0 & 1 & 0 \\
1 & 0 & 1 \\
1 & 0 & 0 \\
\end{pmatrix}^{50} = 
\begin{pmatrix}
525456 & 396655 & 299426 \\
696081 & 525456 & 396655 \\
396655 & 299426 & 226030 \\
\end{pmatrix}
$
\end{center}
\item \textbf{Задача 10. Метод Гаусса (3 балла):} 
Написать программу, которая бы решала линейную систему уравнений $Ax = b$ методом Гаусса. Главная функция этой программы должна иметь вид: \\ \texttt{void solve\_linear\_system(int n, float A[MAX][MAX], float b[], float x[]).} \\
Программа должна считывать 
\begin{lstlisting}
#include <stdio.h>
#define MAX 200

// Возможно понадобится вспомогательная функция для перестановки строк матрицы A
void swap_rows(int n, float A[MAX][MAX], int k, int m)
{
    // Ваш код
}

void solve_linear_system(int n, float A[MAX][MAX], float b[], float x[])
{
    // Ваш код
}

int main()
{
    int n;
    float A[MAX][MAX];
    float b[MAX];
    float x[MAX];
    
    // Считываем n, A и b из файла
	
    solve_linear_system(n, A, b, x);
    
    // Записываем x в новый файл
}
\end{lstlisting}
Проверьте вашу программу на следующих тестах: \\
\begin{enumerate}
\item Следующая система:
$$
\begin{cases} 
x_1 + x_2 - x_3 = 9 \\ 
x_2 + 3x_3 = 3 \\ 
-x_1 - 2x_3 = 2 
\end{cases}
$$
Файл для считывания должен выглядеть следующим образом:
\begin{lstlisting}
 3
 1 1 -1
 0 1  3
-1 0 -2
9
3
2
\end{lstlisting}
Решение этой системы: $x = (\frac{2}{3}, 7, -\frac{4}{3}) \approx (0.67, 7, -1.33)$
\item Система из файла \texttt{system1.txt}. Решение в файле \texttt{x1.txt}.
\item Система из файла \texttt{system2.txt}. Решение в файле \texttt{x2.txt}.
\end{enumerate}
\end{itemize}
\end{document}