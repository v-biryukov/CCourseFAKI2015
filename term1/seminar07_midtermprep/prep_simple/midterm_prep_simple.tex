\documentclass{article}
\usepackage[utf8x]{inputenc}
\usepackage{ucs}
\usepackage{amsmath} 
\usepackage{amsfonts}
\usepackage{upgreek}
\usepackage[english,russian]{babel}
\usepackage{graphicx}
\usepackage{float}
\usepackage{textcomp}
\usepackage{hyperref}
\usepackage{geometry}
  \geometry{left=2cm}
  \geometry{right=1.5cm}
  \geometry{top=1cm}
  \geometry{bottom=2cm}
\usepackage{tikz}
\usepackage{ccaption}
\usepackage{multicol}
\setlength{\columnsep}{1cm}
\usepackage{listings}


\begin{document}
\pagenumbering{gobble}
\lstset{
  language=C,                % choose the language of the code
  basicstyle=\linespread{1.1}\ttfamily,
  columns=fixed,
  fontadjust=true,
  basewidth=0.5em,
  keywordstyle=\color{blue}\bfseries,
  commentstyle=\color{gray},
  stringstyle=\ttfamily\color{orange!50!black},
  showstringspaces=false,
  numbersep=5pt,
  numberstyle=\tiny\color{black},
  numberfirstline=true,
  stepnumber=1,                   % the step between two line-numbers.        
  numbersep=10pt,                  % how far the line-numbers are from the code
  backgroundcolor=\color{white},  % choose the background color. You must add \usepackage{color}
  showstringspaces=false,         % underline spaces within strings
  captionpos=b,                   % sets the caption-position to bottom
  breaklines=true,                % sets automatic line breaking
  breakatwhitespace=true,         % sets if automatic breaks should only happen at whitespace
  xleftmargin=.2in,
  extendedchars=\true,
  keepspaces = true,
}
\lstset{literate=%
   *{0}{{{\color{red!20!violet}0}}}1
    {1}{{{\color{red!20!violet}1}}}1
    {2}{{{\color{red!20!violet}2}}}1
    {3}{{{\color{red!20!violet}3}}}1
    {4}{{{\color{red!20!violet}4}}}1
    {5}{{{\color{red!20!violet}5}}}1
    {6}{{{\color{red!20!violet}6}}}1
    {7}{{{\color{red!20!violet}7}}}1
    {8}{{{\color{red!20!violet}8}}}1
    {9}{{{\color{red!20!violet}9}}}1
}

\section*{Задания для подготовки к КР1. Информатика МФТИ 1 курс.}
\subsection*{Типы и их спецификаторы в printf/scanf:}
\begin{center}
\begin{tabular}{ c c c c }
 тип & размер (байт) & диапазон значений ($2^{\# bits}$) & спецификатор \\ \hline
 char & 1 & от -128 до 127 & \texttt{\%hhd} \\ 
 short & 2 & от -32768 до 32767 & \texttt{\%hd}  \\  
 int & 4 & примерно от -2-х миллиардов до 2-х миллиардов & \texttt{\%d}  \\  
 long & 4 или 8 & такой же как у int или long long в зависимости от системы & \texttt{\%ld}  \\  
 long long & 8 & примерно от $-10^{19}$ до $10^{19}$ & \texttt{\%lld}  \\  
 unsigned char & 1 & от 0 до 255 & \texttt{\%hhu} \\ 
 unsigned short & 2 & от 0 до 65535 & \texttt{\%hu}  \\  
 unsigned int & 4 & примерно от 0 до 4-х миллиардов & \texttt{\%u}  \\  
 unsigned long & 4 или 8 & такой же как у unsigned int или unsigned long long & \texttt{\%lu}  \\  
 unsigned long long & 8 & от 0 до $2^{64} \approx 2*10^{19}$  & \texttt{\%llu}  \\  
 16-ричная система & - & - & \texttt{\%x}  \\   \\
 указатель & 8 & $2^{64} \approx 2*10^{19}$ & \texttt{\%p}  \\   \\
 float & 4 & 6 значащих цифр, степень - от $10^{-38}$ до $10^{38}$ & \texttt{\%f} \\
 double & 8 & 15 значащих цифр, степень - от $10^{-308}$ до $10^{308}$ & \texttt{\%lf} \\
 long double & 10 или 12 & зависит от системы & \texttt{\%Lf} \\ 
 печать без 0 на конце & - & - & \texttt{\%g} \\ \\
 char (как символ) & 1 & от -128 до 128 & \texttt{\%c}\\
 строка = массив char-ов & размер масива & - & \texttt{\%s}\\ \hline
\end{tabular}
\end{center}
Многие функции в языке C возвращают особый тип size\_t. Часто это просто unsigned long.

\subsection*{Переменные и операторы:}
\begin{itemize}
\item \textbf{Остаток:} Ниже - пример программы, которая считывает 2 числа и печатает сумму этих чисел:
\begin{lstlisting}
#include <stdio.h>
int main()
{
	int x, y;
	scanf("%d%d");
	printf("%d\n", x + y);
}
\end{lstlisting}
Написать программу, которая будет считывать 2 числа и печатать остаток деления первого числа на второе.
\item \textbf{Значение, адрес и размер:} Пример программы, которая создаёт и инициализирует переменную \texttt{a} типа \texttt{int} и печатает её значение, размер и адрес.
\begin{lstlisting}
#include <stdio.h>
int main()
{
	int a = 753;
	printf("int: Value = %d; Size = %d; Address = %p\n", a, sizeof(a), &a);
}
\end{lstlisting}
Напишите одну программу, которая делает то же самое для следующих типов : \texttt{char}, \texttt{short}, \texttt{int}, \texttt{long}, \texttt{long long}, \texttt{float}, \texttt{double}, \texttt{char*}, \texttt{int*}, \texttt{float*}, \texttt{double*}, \texttt{int array[80]}. Программа должна создать переменные всех этих типов и напечатать значения, адреса и размеры каждой переменной.
\end{itemize}
\newpage

\subsection*{Оператор if:}
Пример программы которая проверяет делится ли число на 2 при этом не лежит на отрезке \texttt{[15, 100]}:
\begin{lstlisting}
#include <stdio.h>
int main()
{
	int a;
	scanf("%d", &a)
	if (a % 2 == 0 && (a < 15 || a > 100))
		printf("YES\n");
	else
		printf("NO\n");
}
\end{lstlisting}
\begin{itemize}
\item \textbf{Единичный круг:} Написать программу, которая проверяет лежит ли точка, задаваемая двумя целыми числами \texttt{x} и \texttt{y} внутри круга радиуса \texttt{R}. Числа \texttt{x}, \texttt{y} и \texttt{R} считать с помощью \texttt{scanf}. Программа должна печатать \texttt{YES} или \texttt{NO}.
\item \textbf{Гипербола:} Написать программу, которая проверяет принадлежат ли точка, задаваемая двумя вещественными числами, области\\  $\{y > \frac{1}{x}, x > 0\}$. Использовать тип \texttt{double} (спецификатор для \texttt{double} - \texttt{\%lf}).
\end{itemize}

\subsection*{Циклы while и for. Операторы break и continue:}
\begin{itemize}
\item \textbf{mod 7:} Написать программу, которая печатает все числа делящиеся на 7 в интервале от 700 до 1000, используя цикл \texttt{for}. (и не используя оператор \texttt{if}).
\item \textbf{break:} Написать программу, которая считывает целые числа и печатает их до первого отрицательного. Например, если на вход поступает последовательность 5 0 74 -3 5 31 -7 -10, то программа должна напечатать 5 0 74. Использовать оператор break.
\end{itemize}

\subsection*{Математическая библиотека math.h:}
Ниже - пример функции вычисляющей $\cos \big(|x| + |y|\big)$. ($x$ и $y$ задаются в градусах).
\begin{lstlisting}
#include <stdio.h>
#include <math.h>

double func(double x, double y)
{
	return cos(3.14159265/180*(fabs(x) + fabs(y)));
}
int main()
{
	double x, y;
	scanf("%lf%lf", &x, &y);
	printf("%g\n", func(x, y));  // вызываем функцию
}
\end{lstlisting}
Чтобы подключить математическую библиотеку нужно использовать опцию \texttt{-lm} при компиляции в \texttt{gcc}:\\
\texttt{\$ gcc -std=c99 -lm <имя файла>}
\begin{itemize}
\item \textbf{Математическая функция:} Написать функцию, которая вычисляет выражение $\sqrt{|\sin(x)|^7}$. Использовать числа двойной точности double. x подаётся на вход в градусах. Вызвать эту функцию из \texttt{main}.
\item \textbf{Сравнение вещественных чисел:} Написать функцию \texttt{int is\_equal(double a double b, double eps)}, которая сравнивает 2 числа a и b с точностью eps. Вызвать эту функцию из \texttt{main}.
\end{itemize}
\subsection*{Массивы:}
Пример программы, которая показывает как работать с массивами:
\begin{lstlisting}
#include <stdio.h>
int main()
{
	int n;
	int array[1000]; // Выделяем память под 1000 элементов ( с запасом )
	// Считываем размер массива и его элементы:
	scanf("%d", &n);
	for (int i = 0; i < n; i++)
		scanf("%d", &array[i]);
	
	// Возводим каждый элемент в квадрат:
	for (int i = 0; i < n; i++)
		array[i] *= array[i];
	
	// Печатаем:
	for (int i = 0; i < n; i++)
		printf("%d ", array[i]);
}
\end{lstlisting}

\begin{itemize}

\item \textbf{Нормализация:} На вход программе подаётся целое число $n$ и $n$ вещественных чисел типа float. Нужно эти числа нормировать (то есть разделить на их сумму) и напечатать.
\item \textbf{Среднее и дисперсия:} На вход программе подаётся целое число $n$ и $n$ вещественных чисел типа double ${x_i}$. Нужно найти среднее этих чисел $\mu$ и дисперсию $D$: 
$$\mu = \frac{1}{n}\sum_{i=0}^{n-1}x_i \quad \quad \quad \quad D = \frac{1}{n}\sum_{i=0}^{n-1}(x_i - \mu)^2$$
\end{itemize}

\subsection*{Массивы и функции:}
Пример функции, которая принимает на вход массив:
\begin{lstlisting}
#include <stdio.h>
int sum(int* array, int n)
{
	int result = 0;
	for (int i = 0; i < n; i++)
		result += array[i];
	return result;
}
int main()
{
	int numbers[6] = {4, 8, 15, 16, 23, 42};
	printf("Sum = %d\n", sum(numbers, 6));
}
\end{lstlisting}
Помните что, в отличии от обычных переменных, массив внутри функции может меняться.
\begin{itemize}
\item \textbf{Делимость:} На вход функции подаётся целое число $n$ - размер массива и сам массив. Функция должна вернуть число 1 если все числа массива делятся на 7. Иначе, функция должна вернуть число 0. Проверить работу этой функции, вызвав её из функции \texttt{main}.
\end{itemize}

\newpage
\subsection*{Сортировка вставками:}
\begin{lstlisting}
#include <stdio.h>
// Сортировка вставками:
void selection_sort(int array[], int n)
{
	for (int i = 0; i < n; i++)
	{
		// Находим минимальный элемент
		int min_index = i;
		for (int j = i + 1; j < n; j++)
			if (array[j] < array[min_index])
				min_index = j;
				
		// Меняем первый и минимальный
		int temp = array[i];
		array[i] = array[min_index];
		array[min_index] = temp;
	}
}
int main()
{
	int numbers[10] = {54, 76, 83, 26, 17, 95, 43, 6, 54, 61};
	selection_sort(numbers, 10);
	for (int i = 0; i < 10; i++)
		printf("%d ", numbers[i]);
}
\end{lstlisting}
\begin{itemize}
\item \textbf{Сортировка по последней цифре:} Измените функцию \texttt{selections\_sort} в примере выше так, чтобы она сортировала по последней цифре числа. То есть в результате должно получиться: $61, 83, 43, 54, 54, 95, 76, 26, 6, 17$ (числа с одинаковой последней цифрой могут идти в любом порядке). Подсказка: нужно поменять всего 1 строчку.
\end{itemize}

\subsection*{Структуры:}
\begin{lstlisting}
#include <stdio.h>
struct point
{
	float x, y;
};
typedef struct point Point;

int main()
{
	Point points[5] = {{5.2, 3.1}, {7.3, -3.4}, {-1.2, 6.4}, {3.8, 8.7}, {-0.6, 2.1}};
	for (int i = 0; i < 5; i++)
		printf("%g %g\n", points[i].x, points[i].y);
}
\end{lstlisting}
\begin{itemize}
\item \textbf{Структура Triangle:} Определите структуру, которая будет описывать треугольник на плоскости.
\item \textbf{Сортировка структур:} Измените функцию \texttt{selections\_sort} в примере выше так, чтобы она принимала на вход массив структур \texttt{Point} и сортировала их по расстоянию от начала координат. Используйте эту функцию, чтобы отсортировать массив \texttt{points} в функции \texttt{main}.
\end{itemize}


\subsection*{Структуры и функции:}
Структуры передаются в функции и возвращаются из функций по тем же правилам, что и обычные переменные:
\begin{lstlisting}
#include <stdio.h>
struct point
{
	float x, y;
};
typedef struct point Point;

// Функция, которая будет находить точку, лежащую посередине между a и b
Point midpoint(Point a, Point b)
{
	Point result;
	result.x = (a.x + b.x) / 2;
	result.y = (a.y + b.y) / 2;
	return result;
}
int main()
{
	Point points[2] = {{5.2, 3.1}, {7.3, -3.4}};
	Point m = midpoint(points[0], points[1]);
	printf("%g %g\n", m.x, m.y);
}
\end{lstlisting}

\begin{itemize}
\item \textbf{Расстояние:} Написать функцию \texttt{float distance(Point a, Point b)}, которая вычисляет расстояние между двумя точками. Вызовите эту функцию из \texttt{main}.
\item \textbf{Периметр :} Написать функцию \texttt{float perimeter(Triangle t)}, которая вычисляет периметр треугольника. Вызовите эту функцию из \texttt{main}.
\item \textbf{Центр масс :} Написать функцию \texttt{Point center\_of\_mass(Triangle t)}, которая вычисляет центр масс треугольника. Вызовите эту функцию из \texttt{main}. Центр масс треугольника вычисляется просто как среднее арифметическое по всем координатам.

\item \textbf{Перенос :} Написать функцию \texttt{void transfer(Triangle* pt, Point r)}, которая передвигает треугольник на вектор \texttt{r}. Протестируйте эту функцию в  \texttt{main}.

\item \textbf{Вращение :} Написать функцию \texttt{void transfer(Triangle* pt, float angle)}, которая вращает треугольник на угол \texttt{angle}. Протестируйте эту функцию в  \texttt{main}.
\end{itemize}

\newpage
\subsection*{Двумерные массивы:}
Пример программы, вычисляющей сумму двух двумерных массивов:
\begin{lstlisting}
#include <stdio.h>
#define MAX 100

// Функция, которая вычисляет сумму двух матриц и записывает результат в третью
// В отличии от одномерных массивов, тут обязательно нужно указывать максимальный размер
void sum(int n, int A[MAX][MAX], int B[MAX][MAX], int result[MAX][MAX])
{
    for (int i = 0; i < n; ++i)
        for (int j = 0; j < n; ++j)
            result[i][j] = A[i][j] + B[i][j];
}
int main()
{
    int n;
    int A[MAX][MAX], B[MAX][MAX], C[MAX][MAX];
    scanf("%d", &n);
    
    for (int i = 0; i < n; ++i)
        for (int j = 0; j < n; ++j)
            scanf("%d", &A[i][j]);
         
    for (int i = 0; i < n; ++i)
        for (int j = 0; j < n; ++j)
            scanf("%d", &B[i][j]);
            
    sum(n, A, B, C); // Вычисляем сумму A и B и записываем результат в C
    
    for (int i = 0; i < n; ++i)   // Печатаем C
    {
        for (int j = 0; j < n; ++j)
            printf("%d ", C[i][j]);
        printf("\n");
    }
}
\end{lstlisting}
\begin{itemize}
\item \textbf{Перемножение матриц (1 балл):} Написать функцию \texttt{void mult(int n, int A[MAX][MAX], int B[MAX][MAX], int result[MAX][MAX])}, которая вычисляет произведение двух квадратных матриц (строка на столбец) и записывает результат в третью. Вызовите эту функцию из \texttt{main}.
\end{itemize}

\end{document}