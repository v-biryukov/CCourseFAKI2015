\documentclass[12pt,pdf,hyperref={unicode}]{beamer}
\usetheme[progressbar=frametitle]{metropolis}

\usepackage{booktabs}
\usepackage[scale=2]{ccicons}
\usepackage{pgfplots}
\usepgfplotslibrary{dateplot}
\usepackage{xspace}
\newcommand{\themename}{\textbf{\textsc{metropolis}}\xspace}
% подключаем кириллицу 
\usepackage[T2A]{fontenc}
\usepackage[utf8]{inputenc}
\usepackage{listings}
\usepackage{hyperref}

% отключить клавиши навигации
\setbeamertemplate{navigation symbols}{}
% тема оформления
\usetheme{Pittsburgh}
% цветовая схема
\usecolortheme{default}

\definecolor{light-gray}{gray}{0.90}
\definecolor{code-background}{gray}{0.98}
\definecolor{bashcolor}{RGB}{203,221,236}


\title{Презентация по информатике.\\ 1 курс МФТИ. Часть 1.}   
\subtitle{Переменные, операторы, функции, массивы, простые сортировки, строки, структуры, динамическое программирование.}
\author{Бирюков В. А.} 
\date{\today} 

\begin{document}
\lstset{
  language=C,                % choose the language of the code
  basicstyle=\linespread{1.1}\ttfamily,
  columns=fullflexible,
  fontadjust=true,
  basewidth=0.5em,
  keywordstyle=\color{blue}\bfseries,
  commentstyle=\color{gray},
  stringstyle=\ttfamily\color{orange!50!black},
  showstringspaces=false,
  numbersep=5pt,
  numberstyle=\tiny\color{black},
  numberfirstline=true,
  stepnumber=1,                   % the step between two line-numbers.        
  numbersep=10pt,                  % how far the line-numbers are from the code
  backgroundcolor=\color{code-background},  % choose the background color. You must add \usepackage{color}
  showstringspaces=false,         % underline spaces within strings
  captionpos=b,                   % sets the caption-position to bottom
  breaklines=true,                % sets automatic line breaking
  breakatwhitespace=true,         % sets if automatic breaks should only happen at whitespace
  xleftmargin=.2in,
  extendedchars=\true,
  keepspaces = true,
  escapeinside={<@}{@>}
}
\lstset{literate=%
   *{0}{{{\color{red!20!violet}0}}}1
    {1}{{{\color{red!20!violet}1}}}1
    {2}{{{\color{red!20!violet}2}}}1
    {3}{{{\color{red!20!violet}3}}}1
    {4}{{{\color{red!20!violet}4}}}1
    {5}{{{\color{red!20!violet}5}}}1
    {6}{{{\color{red!20!violet}6}}}1
    {7}{{{\color{red!20!violet}7}}}1
    {8}{{{\color{red!20!violet}8}}}1
    {9}{{{\color{red!20!violet}9}}}1
}

% титульный слайд
\begin{frame}
\titlepage
\end{frame} 
\begin{frame}{Содержание}
\tableofcontents
\end{frame}

\defverbatim[colored]\makeset{
\begin{lstlisting}[language=C++,basicstyle=\ttfamily,keywordstyle=\color{blue}]
void make_set(int X) {
  parent[X] = X;
}
\end{lstlisting}
}


\section{Терминал}
\begin{frame}[fragile]
\frametitle{Среда программирования на Linux}
Для этого курса вам понадобится:
\begin{itemize}
\item Хороший терминал - на Linux уже установлен.
\item Компилятор - gcc или clang. gcc скорей всего уже установлен, но если нет, то:
\begin{lstlisting}[language=bash,backgroundcolor=\color{bashcolor}]
$ sudo apt install gcc
\end{lstlisting}
\item Хороший текстовый редактор. Советую Sublime Text:\\
\href{https://www.sublimetext.com/}{\color{blue}{sublimetext.com}} \\
Другие редакторы, которые можно попробовать - \\VS Code и Atom.
\end{itemize}
\end{frame}

\begin{frame}[fragile]
\frametitle{Среда программирования на Windows}
Для этого курса вам понадобится:
\begin{itemize}
\item Хороший терминал - стандартный терминал Windows очень плох, поэтому советую установить cmder:\\
\href{https://cmder.net/}{\color{blue}{cmder.net}} \\
\item Компилятор - MinGW (аналог gcc на Windows).
\href{http://www.mingw.org/}{\color{blue}{mingw.org}}\\
Не забудьте установить переменную PATH.
\item Хороший текстовый редактор. Советую Sublime Text:\\
\href{https://www.sublimetext.com/}{\color{blue}{sublimetext.com}} \\
Другие редакторы, которые можно попробовать - \\VS Code и Atom.
\end{itemize}
\end{frame}

\begin{frame}[fragile]
\frametitle{Работа с терминалом}
\begin{itemize}
\item pwd - печатает текущую директорию
\begin{lstlisting}[language=bash,backgroundcolor=\color{bashcolor}]
$ pwd
> /home-local/student
\end{lstlisting}
\item cd - переходим в другую директорию.\\
Перейти в папку workspace текущей директории:
\begin{lstlisting}[language=bash,backgroundcolor=\color{bashcolor}]
$ cd workspace
\end{lstlisting}
Перейти в папку выше:
\begin{lstlisting}[language=bash,backgroundcolor=\color{bashcolor}]
$ cd ..
\end{lstlisting}
\end{itemize}
\end{frame}

\begin{frame}[fragile]
\frametitle{Работа с терминалом}
\begin{itemize}
\item ls - печатает все файлы в текущей директории.
\begin{lstlisting}[language=bash,backgroundcolor=\color{bashcolor}]
$ ls
<@\textcolor{blue}{workspace/}@> prog.c data.txt <@\textcolor{teal}{a.out}@>
\end{lstlisting}
\item ls -l - то же самое, но больше информации
\begin{lstlisting}[language=bash,backgroundcolor=\color{bashcolor}]
$ ls -l
drwxr-xr-x student 0 Oct 3 <@\textcolor{blue}{workspace/}@>
-rw-r--r-- student 450 Nov 16 prog.c
-rw-r--r-- student 620 Nov 16 data.txt
-rwxr-xr-x student 12654 Nov 16 a.out
\end{lstlisting}
Слева направо - права доступа (read, write, execute), владелец, размер в байтах, дата изменения и название файла.
\end{itemize}
\end{frame}

\begin{frame}[fragile]
\frametitle{Работа с терминалом}
\begin{itemize}
\item cp - копирование\\
Копируем файл \texttt{prog.c} в папку \texttt{workspace}:
\begin{lstlisting}[language=bash,backgroundcolor=\color{bashcolor}]
$ cp prog.c workspace/
\end{lstlisting}
\item mv - перемещение.\\
Перемещаем файл \texttt{prog.c} в папку \texttt{workspace}:
\begin{lstlisting}[language=bash,backgroundcolor=\color{bashcolor}]
$ mv prog.c workspace/
\end{lstlisting}
Переименовываем файл \texttt{prog.c} в \texttt{new.c}:
\begin{lstlisting}[language=bash,backgroundcolor=\color{bashcolor}]
$ mv prog.c new.c
\end{lstlisting}
\end{itemize}
\end{frame}


\begin{frame}[fragile]
\frametitle{Компилятор gcc.}
\begin{itemize}
\item gcc - компилятор\\
Компилируем файл \texttt{prog.c}. В результате создаётся исполняемый файл \texttt{a.out}:
\begin{lstlisting}[language=bash,backgroundcolor=\color{bashcolor}]
$ gcc prog.c
\end{lstlisting}
Запускаем созданную программу:
\begin{lstlisting}[language=bash,backgroundcolor=\color{bashcolor}]
$ ./a.out
\end{lstlisting}
Удобно объединить эти операции:
\begin{lstlisting}[language=bash,backgroundcolor=\color{bashcolor}]
$ gcc prog.c && ./a.out
\end{lstlisting}
\end{itemize}
\end{frame}

\begin{frame}[fragile]
\frametitle{Компилятор gcc. Опции.}
\begin{itemize}
\item Опция \texttt{-o}\\
Задаём имя выдаваемого исполняемого файла:
\begin{lstlisting}[language=bash,backgroundcolor=\color{bashcolor}]
$ gcc -o prog prog.c
\end{lstlisting}
Будет создан файл \texttt{prog}, а не \texttt{a.out}.\\
\item Опция \texttt{-std=c99}\\
Используем более новую версию языка C (99-го года):
\begin{lstlisting}[language=bash,backgroundcolor=\color{bashcolor}]
$ gcc -std=c99 prog.c
\end{lstlisting}
\item Опция \texttt{-lm}\\
Эта опция нужна для линковки математической библиотеки \texttt{math.h}.
\begin{lstlisting}[language=bash,backgroundcolor=\color{bashcolor}]
$ gcc -lm prog.c
\end{lstlisting}
\end{itemize}
\end{frame}

\section{Функция main(), переменные, printf и scanf}


\begin{frame}[fragile]
\frametitle{Функция main}
Простейшая программа на языке C:
\begin{lstlisting}
int main() {
}
\end{lstlisting}
Любая программа на языке C должна содержать main.\\
Выполнение программы начинается с этой функции. \\
Данная программа ничего не делает.
\end{frame}

\begin{frame}[fragile]
\frametitle{Код возврата}
Функция \texttt{main} возвращает код возврата.\\ 
\begin{lstlisting}
int main() {
    return 0;
}
\end{lstlisting}
\texttt{0} - программа завершилась без ошибок. \\
не \texttt{0} - программа завершилась с ошибками. \\
Многие компиляторы не требуют писать \texttt{return 0;} на конце \texttt{main}. В этой презентации эта строка будет опускаться.
\end{frame}



\begin{frame}[fragile]
\frametitle{Hello world}
Программа, которая печатает Hello world:
\begin{lstlisting}
#include <stdio.h>
int main() {
    printf("Hello world");
}
\end{lstlisting}
Библиотека \texttt{stdio.h} (standard input/output) содержит функции для работы вводом данных в программу или выводом их из программы.\\
В частности, \texttt{printf} печатает на экран.\\
\texttt{printf} = \texttt{print formatted} = форматированная печать.
\end{frame}


\begin{frame}[fragile]
\frametitle{Hello world}
Программа, которая печатает Hello world:
\begin{lstlisting}
#include <stdio.h>
int main() {
    printf("Hello world\n");
}
\end{lstlisting}
c переносом на новую строку на конце.\\
$"\backslash n"$ = перенос на новую строку (Enter) \\
$"\backslash t"$ = табуляция (Tab) \\
$"\backslash b"$ = передвигаем каретку влево (стрелочка влево) \\
$"\backslash b\quad\backslash b"$ = удалить 1 символ (backspace) \\
\end{frame}


\begin{frame}[fragile]
\frametitle{Комментарии}
В код можно вставлять комментарии.
\begin{lstlisting}
int main() {
	// Комментарий в 1 строку
	
	/* Комментарий
	          в несколько
	                    строк   */ 
}
\end{lstlisting}
При компиляции этот текст отбрасывается и никак не влияет на работу программы. \\
Комментарий нужны чтобы сделать код более понятным.
\end{frame}

\begin{frame}[fragile]
\frametitle{Переменные типа \texttt{int}}
\texttt{int} = integer = целые числа.
\texttt{int} - переменная, которая хранит целые числа. \\
Создадим переменную под именем \texttt{a}, предназначенную для хранения целых чисел: \\
\begin{lstlisting}
int main() {
    int a; 
}
\end{lstlisting}
Кстати, функция \texttt{main} также возвращает переменную типа \texttt{int} - код возврата.
\end{frame}

\begin{frame}[fragile]
\frametitle{Переменные типа \texttt{int}}
\texttt{int} - переменная, которая хранит целые числа.
\begin{lstlisting}
int main() {
    // Объявление:
    int a; 
    // Объявление и инициализация:
    int b = 451; 
    // Присваивание:
    b = 233;
}
\end{lstlisting}
\end{frame}

\begin{frame}[fragile]
\frametitle{Имена переменных}
\begin{itemize}
\item Имена переменных могут содержать любые латинские буквы и верхнего и нижнего регистров, цифры и символ подчёркивания \texttt{\_}
\item Но не могут начинаться с цифры
\end{itemize}
\begin{lstlisting}
int main() {
    int ab$c; // Ошибка
    int 5x;   // Ошибка
    int _fj374834oJR_394t;  // Ок
    // ( но лучше выбирать осмысленные имена )
}
\end{lstlisting}
\end{frame}

\begin{frame}[fragile]
\frametitle{Печать значений переменных типа \texttt{int}}
\begin{lstlisting}
#include <stdio.h>
int main() {
    int a = 451;
    printf("The password is %d.\n", a);
}
\end{lstlisting}
Вместо \texttt{\%d} напечатает \texttt{451}
\end{frame}

\begin{frame}[fragile]
\frametitle{Печать значений переменных типа \texttt{int}}
\begin{lstlisting}
#include <stdio.h>
int main() {
    int a = 451;
    // Часто нужно напечатать только число:
    printf("%d\n", a);
    // %5d - печатает как минимум 5 символов
    // Если число меньше, то дополнит пробелами
    printf("%5d\n", a);
    // %05d - печатает как минимум 5 символов
    // Если число меньше, то дополнит нулями
    printf("%05d\n", a);
}
\end{lstlisting}
\end{frame}

\begin{frame}[fragile]
\frametitle{Печать значений переменных типа \texttt{int}}

\begin{lstlisting}
#include <stdio.h>
int main() {
    int a = 451;
    // Печать в шестнадцатеричной системе:
    printf("%x\n", a);
}
\end{lstlisting}
\end{frame}

\begin{frame}[fragile]
\frametitle{Адрес и размер переменной}
Расположение переменных в памяти характеризуется их адресом(номером первого байта) и размером переменной.
\begin{lstlisting}
#include <stdio.h>
int main() {
    int a = 451;
    &a; // так можно найти адрес переменной
    sizeof(a); // так можно найти размер
    // Напечатаем их:
    printf("Address of a = %x.\n", &a);
    printf("Size of a = %d.\n", sizeof(a));
}
\end{lstlisting}
\end{frame}

\begin{frame}
\frametitle{Базовые операторы}
\frametitle{Приоритет операторов}
\begin{center}
\begin{enumerate}
\item (), []
\item ++, --, +, -(унарные), sizeof
\item *, /, \%
\item +, -
\item >,<,<=,>=
\item ==, !=
\item \&, |, \&\&, ||
\item =, +=, и т.д.
\end{enumerate}
\end{center}
Приоритет операторов C подробнее:\\
\href{http://ru.cppreference.com/w/c/language/operator_precedence}
{\textcolor{red}{ru.cppreference.com/w/c/language/operator\_precedence}}
\end{frame}







\section{Управляющие конструкции: if else, циклы while и for}


\begin{frame}[fragile]
\frametitle{if} 



\begin{lstlisting}
#include <stdio.h>
int main() {
    int x;
    scanf("%d", &x);
    if (x < 5)
        printf("%d is less than five!!!\n", x);
}
\end{lstlisting}



\end{frame}

\begin{frame}[fragile]
\frametitle{if else}

\begin{lstlisting}
#include <stdio.h>
int main() {
    int x;
    scanf("%d", &x);
    if (x < 5)
       printf("%d\n", x);
    else if (x == 5)
        printf("Equal to 5\n");
    else
        printf("More than 5\n");
}
\end{lstlisting}
\end{frame}


\begin{frame}[fragile]
\frametitle{Базовые управляющие конструкции} 
\framesubtitle{Тернарный оператор :?}

\begin{lstlisting}
z = (x > 0) ? x : -x;
min = (x < y) ? x : y;
\end{lstlisting}
\end{frame}


\begin{frame}[fragile]
\frametitle{Базовые управляющие конструкции} 
\framesubtitle{Цикл while}

\begin{lstlisting}
#include <stdio.h>
int main() {
    int i = 1;

    while (i < 10) {
        printf("%d ", i);
        i++;
    }
}
\end{lstlisting}

Напечатает 1 2 3 4 5 6 7 8 9

\end{frame}

\begin{frame}[fragile]
\frametitle{Базовые управляющие конструкции} 
\framesubtitle{Цикл do while}

\begin{lstlisting}
int i = 1;

do
{
	printf("%d ", i);
	i++;
} while (i < 4)
\end{lstlisting}

Напечатает 1 2 3 

\end{frame}

\begin{frame}[fragile]
\frametitle{Базовые управляющие конструкции} 
\framesubtitle{Цикл for}

\begin{lstlisting}
for (int i = 0; i < 3; ++i)
	printf("%d ", i);
\end{lstlisting}
Напечатает 1 2 3 

\end{frame}

\begin{frame}[fragile]
\frametitle{Базовые управляющие конструкции} 
\framesubtitle{Замечания}

В условии циклов может стоять любой оператор:

\begin{lstlisting}
int i = 0;

while (i++ < 3)
	printf("%d ", i);
\end{lstlisting}

\end{frame}



\begin{frame}
\frametitle{Управляющие конструкции} 
\framesubtitle{Оператор break}

\lstinputlisting{./programms/code_break.c}
\end{frame}

\begin{frame}
\frametitle{Управляющие конструкции} 
\framesubtitle{Оператор continue}

\lstinputlisting{./programms/code_continue.c}
\end{frame}

\begin{frame}
\frametitle{Управляющие конструкции} 
\framesubtitle{Оператор выбора switch}

\lstinputlisting{./programms/code_switch.c}
\end{frame}


\begin{frame}
\frametitle{Управляющие конструкции} 
\framesubtitle{Оператор безусловного перехода goto}

\begin{itemize}
\item Оператор goto передает управление на оператор, помеченный меткой
\item Оператор goto в языках высокого уровня является объектом критики, поскольку чрезмерное его применение приводит к созданию нечитаемого кода
\item Использование goto в практике программирования на языке C настоятельно не рекомендуется
\end{itemize}


\end{frame}


\section{Массивы}


\section{Строки}

\section{Динамическое программирование}


\section{Указатели}

\section{Стиль кода}


\begin{frame}
\frametitle{Комментарии}
\lstinputlisting{./programms/comments.c}

\end{frame}

\begin{frame}
\frametitle{Стиль кода} 

\begin{itemize}
\item Отступы. В программе должна быть структура.
Количество отступов соответствует уровню вложенности. Уровень вложенности увеличивается внутри { }, а также в телах операторов if, for, while, do-while, switch
\item Каждый отступ - это ЛИБО TAB, ЛИБО n пробелов (лучше всего n = 4). Мешать их нельзя.
\item Скобка \{ должна быть на следующей строке, под началом ключевого слова if/for/.
\end{itemize}


\end{frame}


\begin{frame}
\frametitle{Стиль кода} 

\begin{itemize}
\item Скобка \} должна быть СТРОГО под соответствующей \{. После неё не должно быть ничего, за исключением комментариев.
\item Каждый оператор (особенно, содержащий ключевое слово) должен быть с новой строки, после оператора и знака ; не должно быть ничего, кроме комментариев.
\item Пробелы должны быть ПОСЛЕ , и ;(в цикле for) , а до них они НЕ нужны.
\end{itemize}
\end{frame}


\end{document}
