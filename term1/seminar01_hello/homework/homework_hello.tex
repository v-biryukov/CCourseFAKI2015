\documentclass{article}
\usepackage[utf8x]{inputenc}
\usepackage{ucs}
\usepackage{amsmath} 
\usepackage{amsfonts}
\usepackage{marvosym}
\usepackage{wasysym}
\usepackage{upgreek}
\usepackage[english,russian]{babel}
\usepackage{graphicx}
\usepackage{float}
\usepackage{textcomp}
\usepackage{hyperref}
\usepackage{geometry}
  \geometry{left=2cm}
  \geometry{right=1.5cm}
  \geometry{top=1cm}
  \geometry{bottom=2cm}
\usepackage{tikz}
\usepackage{ccaption}
\usepackage{multicol}
\usepackage{hyperref}
\usepackage{empheq}


\makeatletter
\def\@bignumber#1#2{%
  \ifx#2\end
    #1\let\next\@gobble
  \else
    #1\hspace{0pt plus 1pt}\let\next\@bignumber
  \fi
  \next#2}
\newcommand{\bignumber}[1]{\@bignumber#1\end}
\makeatother

\usepackage{listings}
%\setlength{\columnsep}{1.5cm}
%\setlength{\columnseprule}{0.2pt}

\usepackage[absolute]{textpos}

\begin{document}
\pagenumbering{gobble}

\lstset{
  language=C,                % choose the language of the code
  basicstyle=\linespread{1.1}\ttfamily,
  columns=fixed,
  fontadjust=true,
  basewidth=0.5em,
  keywordstyle=\color{blue}\bfseries,
  commentstyle=\color{gray},
  stringstyle=\ttfamily\color{orange!50!black},
  showstringspaces=false,
  numbersep=5pt,
  numberstyle=\tiny\color{black},
  numberfirstline=true,
  stepnumber=1,                   % the step between two line-numbers.        
  numbersep=10pt,                  % how far the line-numbers are from the code
  backgroundcolor=\color{white},  % choose the background color. You must add \usepackage{color}
  showstringspaces=false,         % underline spaces within strings
  captionpos=b,                   % sets the caption-position to bottom
  breaklines=true,                % sets automatic line breaking
  breakatwhitespace=true,         % sets if automatic breaks should only happen at whitespace
  xleftmargin=.2in,
  extendedchars=\true,
  keepspaces = true,
}
\lstset{literate=%
   *{0}{{{\color{red!20!violet}0}}}1
    {1}{{{\color{red!20!violet}1}}}1
    {2}{{{\color{red!20!violet}2}}}1
    {3}{{{\color{red!20!violet}3}}}1
    {4}{{{\color{red!20!violet}4}}}1
    {5}{{{\color{red!20!violet}5}}}1
    {6}{{{\color{red!20!violet}6}}}1
    {7}{{{\color{red!20!violet}7}}}1
    {8}{{{\color{red!20!violet}8}}}1
    {9}{{{\color{red!20!violet}9}}}1
}

\renewcommand{\thesubsection}{\arabic{subsection}}
\makeatletter
\def\@seccntformat#1{\@ifundefined{#1@cntformat}%
   {\csname the#1\endcsname\quad}%    default
   {\csname #1@cntformat\endcsname}}% enable individual control
\newcommand\section@cntformat{}     % section level 
\newcommand\subsection@cntformat{Задача \thesubsection.\space} % subsection level
\newcommand\subsubsection@cntformat{\thesubsubsection.\space} % subsubsection level
\makeatother


\title{Семинар \#1: Основы. Домашнее задание.\vspace{-5ex}}\date{}\maketitle


\subsection{Условие:} 
Напишите программу, которая будет считывать число и проверять, является ли число чётным и принадлежащим следующему множеству 
$[0, 20] \cup (100, 200)$ и печатать \texttt{Yes} или \texttt{No}. Используйте один оператор \texttt{if}.
\begin{center}
\begin{tabular}{ l l }
 вход & выход \\ \hline
 \texttt{4} & \texttt{Yes}  \\ 
 \texttt{5} & \texttt{No}  \\ 
 \texttt{20} & \texttt{Yes} \\ 
 \texttt{22} & \texttt{No} \\  
 \texttt{100} & \texttt{No} \\
 \texttt{102} & \texttt{Yes} \\    
 \texttt{202} & \texttt{No} \\  
\end{tabular}
\end{center}

\subsection{Три числа:}
На вход программе подаются три числа: \texttt{a}, \texttt{b} и \texttt{c}. Нужно проверить следующие условия:
\begin{enumerate}
\item Если числа \texttt{a}, \texttt{b} и \texttt{c} являются последовательными, то нужно напечатать \texttt{Consecutive}.
\item Если последовательность \texttt{a}, \texttt{b}, \texttt{c} является возрастающей, то нужно напечатать \texttt{Increasing}.
\item Если последовательность \texttt{a}, \texttt{b}, \texttt{c} является убывающей, то нужно напечатать \texttt{Decreasing}.
\item Если все три числа равны, то нужно напечатать \texttt{Equal}.
\item В ином случае нужно напечатать \texttt{None}.
\end{enumerate}

\begin{center}
\begin{tabular}{ l l }
 вход & выход \\ \hline
 \texttt{1 2 3} & \texttt{Consecutive Increasing}  \\ 
 \texttt{1 2 4} & \texttt{Increasing}  \\
 \texttt{1 1 2} & \texttt{None} \\ 
 \texttt{1 2 1} & \texttt{None} \\ 
 \texttt{1 5 9} & \texttt{Increasing}  \\ 
 \texttt{1 0 -1} & \texttt{Consecutive Decreasing}  \\ 
 \texttt{1 5 4} & \texttt{None} \\ 
 \texttt{7 7 7} & \texttt{Equal} \\  
 \texttt{20 15 5} & \texttt{Decreasing} \\
\end{tabular}
\end{center}


\subsection{Число, квадрат и куб:} 
Напишите программу, которая будет печатать само число, его квадрат и его куб от \texttt{1} до \texttt{n}, разделённые стрелочкой.
Число \texttt{n} считывается с помощью \texttt{scanf}. 
Например, при \texttt{n = 5}, программа должна напечатать следующее:
\begin{verbatim}
1 ->  1 ->   1
2 ->  4 ->   8
3 ->  9 ->  27
4 -> 16 ->  64
5 -> 25 -> 125
\end{verbatim}
Для того чтобы все числа печатались выровнено, можно использовать спецификатор \texttt{\%3i} за место \texttt{\%i} в \texttt{printf}. В этом случае, если число имеет в записи меньше 3-х цифр, то \texttt{printf} напечатает необходимое число пробелов перед числом.



\subsection{Последовательность:} 
Пример программы, которая считывает число \texttt{n}. Затем считывает \texttt{n} чисел и находит среди них максимум. 

\begin{lstlisting}
#include <stdio.h>
#include <limits.h>
int main() 
{
    int n;
    scanf("%i", &n);
    int max = INT_MIN;
    for (int i = 0; i < n; ++i) 
    {
        int a;
        scanf("%i", &a);
        if (a > max)
            max = a;
    }
    printf("Max = %i\n", max);
}
\end{lstlisting}
В этой программе используется константа \texttt{INT\_MIN} из библиотеки \texttt{limits.h}. Эта константа равна минимальному возможному значению чисел типа \texttt{int}, то есть \texttt{INT\_MIN = -2147483648}.
\subsubsection*{Подзадачи:}
Измените программу выше так чтобы:
\begin{enumerate}
\item Программа находила минимум, а не максимум. Может понадобиться константа \texttt{INT\_MAX = 2147483647}.
\item Программа находила минимальное чётное число и максимальное нечётное. Если чётных или нечётных чисел нет, то программа должна печатать \texttt{None} за место числа.
\begin{center}
\begin{tabular}{ l l }
 вход & выход \\ \hline
 \texttt{3 4 5 6} & \texttt{4 5}  \\ 
 \texttt{3 7 7 7} & \texttt{None 7}  \\
 \texttt{10 1 8 2 4 8 8 1 5 2 8} & \texttt{2 5} \\
 \texttt{4 10 8 6 8} & \texttt{6 None}
\end{tabular}
\end{center}
\item Программа находила максимум и количество элементов, равных этому максимуму.
\begin{center}
\begin{tabular}{ l l }
 вход & выход \\ \hline
 \texttt{3 1 2 3} & \texttt{3 1}  \\ 
 \texttt{3 7 7 7} & \texttt{7 3}  \\
 \texttt{10 1 8 2 4 8 8 1 5 2 8} & \texttt{8 4}
\end{tabular}
\end{center}

\item Программа печатала \texttt{Increasing} если последовательность чисел строго возрастает, \texttt{Decreasing}, если последовательность чисел строго убывает и \texttt{Equal}, если все члены последовательности равны. В любом ином случае программа должна печатать \texttt{None}.
\begin{center}
\begin{tabular}{ l l }
 вход & выход \\ \hline
 \texttt{3 1 2 3} & \texttt{Increasing}  \\ 
 \texttt{3 7 7 7} & \texttt{Equal}  \\
 \texttt{5 20 15 10 7 5} & \texttt{Decreasing}  \\ 
 \texttt{4 1 1 4 5} & \texttt{None}
\end{tabular}
\end{center}
\end{enumerate}

\newpage
\subsection{Числа-градины I:}
Пусть нам на вход поступает число \texttt{n}. Мы преобразуем это число следующим образом $n = f(n)$, где
\begin{equation*}
f(n) = 
\left\{
\begin{alignedat}{2}
 &3 n + 1, &\quad \textup{если } n - \textup{нечётное}\\
 &n / 2,   & \textup{если } n - \textup{чётное}
\end{alignedat}
\right.
\end{equation*}
Затем повторяем этот алгоритм до тех пор пока число не достигнет единицы. Получится некоторая последовательность. Например, если изначально 
\texttt{n = 7}, то последовательность будет выглядеть следующим образом:
\begin{verbatim}
7 22 11 34 17 52 26 13 40 20 10 5 16 8 4 2 1
\end{verbatim}
Ваша задача заключается в том, чтобы напечатать эту последовательность, её длину и максимальный элемент этой последовательность по изначальному числу \texttt{n}.
\begin{center}
\begin{tabular}{ l | l }
 вход & выход \\ \hline
 \texttt{3} & \texttt{3 10 5 16 8 4 2 1}  \\ 
   & \texttt{Length = 8, Max = 16}  \\ \hline
\texttt{256} & \texttt{256 128 64 32 16 8 4 2 1}  \\ 
   & \texttt{Length = 9, Max = 256}  \\ \hline
 \texttt{7} & \texttt{7 22 11 34 17 52 26 13 40 20 10 5 16 8 4 2 1}  \\ 
   & \texttt{Length = 17, Max = 52}  \\
\end{tabular}
\end{center}

\subsection{Числа-градины II:}
На вход поступает 2 числа \texttt{a} и \texttt{b}. Нужно найти такое число \texttt{n} ($a \le n \le b$), для которого последовательность чисел-градин будет самой длинной. Нужно напечатать число \texttt{n}, а также длину последовательности, которая начинается с \texttt{n}.
\begin{center}
\begin{tabular}{ l l }
 вход & выход \\ \hline
 \texttt{1 5}  & \texttt{3 8}  \\ 
 \texttt{1 8} & \texttt{7 17}  \\ 
 \texttt{1 10} & \texttt{9 20}  \\ 
 \texttt{10 15} & \texttt{14 18}  \\ 
 \texttt{1 100} & \texttt{97 119}  \\ 
 \texttt{1 500} & \texttt{327 144}  \\ 
 \texttt{400 500} & \texttt{487 142}  \\ 
 \texttt{1 1000} & \texttt{871 179}  \\ 
 \texttt{1 10000} & \texttt{6171 261}  \\ 
 \texttt{1 100000} & \texttt{77031 351}  \\ 
\end{tabular}
\end{center}

\subsection{Сумма:}
На вход программе подаются два целых числа \texttt{n} и \texttt{m}. Нужно посчитать следующую сумму:
\begin{align*}
  S_{n,m} &= \sum_{i=1}^{n} \sum_{j=1}^{m} (-1)^{i + j} i \cdot j \\
\end{align*}
Например, если \texttt{n = 3}, а \texttt{m = 4}, то сумма будет равна:
\begin{align*}
  S_{3,4} &= 1 - 2 + 3 - 4 - 2 + 4 - 6 + 8 + 3 - 6 + 9 - 12 = -4\\
\end{align*}

\begin{center}
\begin{tabular}{ l l }
 вход & выход \\ \hline
 \texttt{1 1}  & \texttt{1}  \\ 
 \texttt{2 2} & \texttt{1}  \\ 
 \texttt{3 3} & \texttt{4}  \\ 
 \texttt{3 4} & \texttt{-4}  \\  
 \texttt{5 7} & \texttt{12}  \\  
 \texttt{10 10} & \texttt{25}  \\  
 \texttt{77 107} & \texttt{2106}  \\ 
\end{tabular}
\end{center}


\subsection{Печать всех делимых:}
На вход программе подаются числа \texttt{a}, \texttt{b}, \texttt{c}. Программа должна напечатать все числа, делящиеся на \texttt{c} на отрезке \texttt{[a, b]} через пробел.

\begin{center}
\begin{tabular}{ l l }
 вход & выход \\ \hline
 \texttt{1 20 4}  & \texttt{4 8 12 16 20}  \\ 
 \texttt{1 20 7} &  \texttt{7 14}  \\ 
 \texttt{1 10000 9500} & \texttt{9500}  \\ 
 \texttt{1 1000000000 500000000} & \texttt{500000000 1000000000} \\
 \texttt{1 1000000000 123456789} & \texttt{123456789 246913578 370370367 493827156}  \\ 
                                 & \texttt{617283945 740740734 864197523 987654312}  \\   
\end{tabular}
\end{center}


\subsection{Пифагоровы тройки:}
На вход приходит целое число \texttt{n}. Нужно напечатать все возможные пифагоровы тройки $a$, $b$ и $c$, такие что $a \le n$, $b \le n$ и $c \le n$. Пифагорова тройка -- это тройка натуральных чисел, для которых верно:
$$
a^2 + b^2 = c^2
$$

Пифагоровы тройки, получаемые из некоторой пифагоровой тройки путём обмена местами чисел $a$ и $b$ считаются дублирующими. Пифагоровы тройки, получаемые из некоторой пифагоровой тройки путём умножения всех чисел на некоторое натуральное число, также считаются дублирующими. Печатать дублирующие тройки не нужно.

\textit{Подсказка:} Просто переберите все возможные значения $a$, $b$ и $c$.

\begin{center}
\begin{tabular}{ l l }
 вход & выход \\ \hline
 \texttt{15}  & \texttt{3 4 5}  \\ 
              & \texttt{5 12 13}  \\ 
 \texttt{50}  & \texttt{3 4 5}  \\
              & \texttt{5 12 13}  \\ 
              & \texttt{8 15 17}  \\ 
              & \texttt{7 24 25}  \\ 
              & \texttt{20 21 29}  \\ 
              & \texttt{12 35 37}  \\ 
              & \texttt{9 40 41} 
\end{tabular}
\end{center}
\end{document}