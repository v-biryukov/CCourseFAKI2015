\documentclass{article}
\usepackage[utf8x]{inputenc}
\usepackage{ucs}
\usepackage{amsmath} 
\usepackage{amsfonts}
\usepackage{upgreek}
\usepackage[english,russian]{babel}
\usepackage{graphicx}
\usepackage{float}
\usepackage{textcomp}
\usepackage{hyperref}
\usepackage{geometry}
  \geometry{left=2cm}
  \geometry{right=1.5cm}
  \geometry{top=1cm}
  \geometry{bottom=2cm}
\usepackage{tikz}
\usepackage{ccaption}
\usepackage{multicol}

\usepackage{listings}
%\setlength{\columnsep}{1.5cm}
%\setlength{\columnseprule}{0.2pt}


\begin{document}
\pagenumbering{gobble}

\lstset{
  language=C,                % choose the language of the code
  basicstyle=\linespread{1.1}\ttfamily,
  columns=fixed,
  fontadjust=true,
  basewidth=0.5em,
  keywordstyle=\color{blue}\bfseries,
  commentstyle=\color{gray},
  stringstyle=\ttfamily\color{orange!50!black},
  showstringspaces=false,
  %numbers=false,                   % where to put the line-numbers
  numbersep=5pt,
  numberstyle=\tiny\color{black},
  numberfirstline=true,
  stepnumber=1,                   % the step between two line-numbers.        
  numbersep=10pt,                  % how far the line-numbers are from the code
  backgroundcolor=\color{white},  % choose the background color. You must add \usepackage{color}
  showstringspaces=false,         % underline spaces within strings
  captionpos=b,                   % sets the caption-position to bottom
  breaklines=true,                % sets automatic line breaking
  breakatwhitespace=true,         % sets if automatic breaks should only happen at whitespace
  xleftmargin=.2in,
  extendedchars=\true,
  keepspaces = true,
}
\lstset{literate=%
   *{0}{{{\color{red!20!violet}0}}}1
    {1}{{{\color{red!20!violet}1}}}1
    {2}{{{\color{red!20!violet}2}}}1
    {3}{{{\color{red!20!violet}3}}}1
    {4}{{{\color{red!20!violet}4}}}1
    {5}{{{\color{red!20!violet}5}}}1
    {6}{{{\color{red!20!violet}6}}}1
    {7}{{{\color{red!20!violet}7}}}1
    {8}{{{\color{red!20!violet}8}}}1
    {9}{{{\color{red!20!violet}9}}}1
}


\section*{Добро пожаловать в язык программирования C:}
Простейшая программа на языке C выглядит следующим образом:
\begin{lstlisting}
#include <stdio.h>
int main() 
{
    printf("Hello world!");
}
\end{lstlisting}

Эта программа печатает на экран строку "Hello world!".
\begin{itemize}
\item \texttt{\#include <stdio.h>}  - включаем библиотеку stdio (standard input/output), которая содержит функцию \texttt{printf}.
\item \texttt{int main() \{ ... \}} - основная функция программы, с неё начинается исполнение любой программы.
\item \texttt{printf("Hello world!");} - печатаем на экран.
\end{itemize}

Любая программа на языке C должна содержать особую функцию под названием main. По аналогии с обычными математическими функциями, функции в языке C могут принимать и возвращать значения. Принимаемые значения указываются в круглых скобкая(в данном случае там ничего нет так как функция ничего не принимает) а тип возвращаемого значения указывается перед функцией (для функции main это всегда тип int, т.е. Integer - т.е. целое число). В фигурных скобках описываются операции, которые совершает функция. 
\subsubsection*{Задание 1:}
\begin{enumerate}
\item Скомпилируйте данную программу и запустите.
\item Напишите программу, которая печатает на экран Hello MIPT!
\item В строке функции \texttt{printf()} можно использовать некоторые специальные символы \textbackslash n и \textbackslash t. Добавьте эти символы в строку функции printf и выясните, что они делают.
\end{enumerate}

\section*{Переменные:}

Пример работы с переменными:
\begin{lstlisting}
#include <stdio.h>
int main() 
{
    int a;
    int b = 5;
    a = 3;
    int res = a * b + (b / a);
    printf("Result = %i\n", res);
}
\end{lstlisting}

\begin{itemize}
\item \texttt{int a} - Объявляем, что у нас есть переменная a, которая будет хранить целые числа (от англ. integer - целое число).
\item \texttt{int b = 5} - Объявляем, что у нас есть переменная b, которая будет хранить целые числа и присваиваем ей число 5.
\item \texttt{a = 3} - Присваиваем переменной a число 3.
\item \texttt{res = a * b + (b / a)} - Сохраняем в переменной res результат вычислений.
\item \texttt{printf(``Result = \%i \textbackslash n '', res)} Печатаем, за место \%i подставится значение переменной res.
\end{itemize}
\subsubsection*{Задание 2:}
\begin{enumerate}
\item Пусть \texttt{a = 436596}, а \texttt{b = 7361}. Найти и напечатать остаток деления \texttt{a} на \texttt{b}. Остаток вычисляется с помощью оператора \texttt{\%}.
\end{enumerate}

\section*{Считывание переменных:}
Считывание переменных из терминала осуществляется с помощью функции \texttt{scanf} из библиотеки \texttt{stdio}.
Пример программы, которая считывает переменные a и b и печатает их на экран:
\begin{lstlisting}
#include <stdio.h>
int main() 
{
    int a, b;
    scanf("%i", &a); // <-- не забудьте тут амперсанд &
    scanf("%i", &b); // <-- не забудьте тут амперсанд &
    printf("Multiplication = %i\n", a * b);
}
\end{lstlisting}
\subsubsection*{Задание 3:}
\begin{enumerate}
\item Считать 2 целых числа и напечатать остаток деления первого на второе.
\end{enumerate}



\section*{Вещественные числа:}
Считывание переменных из терминала осуществляется с помощью функции \texttt{scanf} из библиотеки \texttt{stdio}.
Пример программы, которая считывает 2 вещественных числа и вычисляет среднее геометрическое:
\begin{lstlisting}
#include <stdio.h>
#include <math.h>
int main() 
{
    float a, b;
    scanf("%f", &a); // <-- не забудьте тут амперсанд & и %f
    scanf("%f", &b); // <-- не забудьте тут амперсанд & и %f
    printf("Geometric average = %f\n", sqrt(a * b));
}
\end{lstlisting}
В библиотеке \texttt{math.h} хранятся матемачисчие функции, такие как \texttt{sqrt} (корень), \texttt{sin}, \texttt{cos}, \texttt{exp}, \texttt{log} (натуральный логарифм), \texttt{fabs} (модуль вещ. числа) и другие. 
\subsubsection*{Задание 4:}
\begin{enumerate}
\item На вход программе подаются 2 положительных вещественных числа - катеты треугольника. Найти гипотенузу.
\item На вход программе подаются 2 положительных вещественных числа \texttt{a} и \texttt{b}. Вычислить значение выражения \texttt{sin(|a - b|) + log(a + b)}.
\end{enumerate}
\section*{Логические операторы:}
Пример программы, использующие логические операторы:
\begin{lstlisting}
#include <stdio.h>
int main() 
{
    int age;
    scanf("%i", &age);
    if (age >= 18 && age < 28)
    	printf("Yes\n")
    else
    	printf("No\n")
}
\end{lstlisting}

\begin{center}
\texttt{\begin{multicols}{2}
\begin{tabular}{ c c }
 == & равно \\ 
 != & не равно \\  
 > & больше \\  
 >= & больше и равно \\ 
 < & меньше \\
 <= & меньше и равно \\
\end{tabular}
\begin{tabular}{ c c }
 \&\& & логическое И \\ 
 || & логическое ИЛИ \\  
 !  & логическое НЕ \\  
\end{tabular}
\end{multicols}
}
\end{center}
\subsubsection*{Задание 5:}
\begin{enumerate}
\item Написать программу, которая принимает на вход число и печатает \texttt{Positive}, если число положительное, \texttt{Negative}, если число отрицательное и \texttt{Zero}, если число равно нулю.
\item Написать программу, которая принимает на вход число и печатает \texttt{Yes}, если число принадлежит множеству $(-\infty, -12] \cup (97, +\infty)$.
\item Написать программу, которая принимает на вход число и печатает \texttt{Even}, если число четное и \texttt{Odd}, если число нечетное. Подсказка: \texttt{\%}.
\end{enumerate}


\end{document}