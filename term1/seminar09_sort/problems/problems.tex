\documentclass{article}
\usepackage[utf8x]{inputenc}
\usepackage{ucs}
\usepackage{amsmath} 
\usepackage{mathtext}
\usepackage{amsfonts}
\usepackage{upgreek}
\usepackage[english,russian]{babel}
\usepackage{graphicx}
\usepackage{float}
\usepackage{textcomp}
\usepackage{hyperref}
\usepackage{geometry}
  \geometry{left=2cm}
  \geometry{right=1.5cm}
  \geometry{top=1cm}
  \geometry{bottom=2cm}
\usepackage{tikz}
\usepackage{ccaption}
\usepackage{multicol}
%\setlength{\columnsep}{1.5cm}
%\setlength{\columnseprule}{0.2pt}


\begin{document}
\pagenumbering{gobble}

\newpage

\subsection*{Сложность алгоритмов}
\begin{enumerate}

\item Расположите следующие функции в порядке увеличения скорости роста при больших $n$:
\begin{multicols}{3}
\begin{enumerate}
\item $\log{n}$
\item $1$
\item $\sqrt{n}$
\item $n$
\item $1.01^n$
\item $\log{\log{n}}$
\item $\sqrt{2}^{log n}$
\item $(log (n))^{log(n)}$
\item $2^{2^n}$
\item $n!$
\item $n^n$
\item $n\log{n}$
\item $n^2$
\item $2^n$
\end{enumerate}
\end{multicols}

\item Отметьте все функции, равные $\Theta(n^2)$ и все функции, равные $O(n^2)$
\begin{multicols}{2}
\begin{itemize}
\item $1000 n^2$
\item $e^n$
\item $4 n^2 + 10 n + 50$
\item $\log{n}$
\item $\frac{n^3}{1000} + 5000 n^2$
\item $log(n^9 + n^5)$
\item $n\log{n}$
\item $n^3 / (1 + n)$
\end{itemize}
\end{multicols}


\item \textbf{} Чему равна алгоритмическая сложность следующих операций?
\begin{itemize}
\item Поиск элемента в массиве размера N
\item Добавление элемента в начало массива размера N
\item Сортировка пузырьком массива размера N
\item Быстрая сортировка массива размера N
\item Добавление элемента в стек размера N
\item Сложение матриц размера  $N \times N$
\item Простой алгоритм умножения матриц размера  $N \times N$
\item Следующий участок кода:
\begin{verbatim}
int sum = 0;
for (int i = 0; i < N; i++)
    for (int j = i+1; j < N; j++)
        for (int k = j+1; k < N; k++)
                sum++;
\end{verbatim}
\end{itemize}

\item Алиса и Боб любят игры и соревнования. И сейчас они готовы приступить к новой игре. Всего у них есть $X$ плиток шоколада. По правилам игры они могут есть этот шоколад по очереди(первой начинает Алиса). Известно, что Алиса съедает $7$ плиток шоколада за ход, а Боб -- $5$ плиток шоколада. Выйгрывает тот, кто съест последнюю плитку. При заданном X, определить победителя. \\
Предложено 2 алгоритма решения этой задачи:
\begin{itemize}
\item Плохой: Вычитаем сначала $7$, затем $5$ и так до тех пор пока не дойдём до $0$ (или отрицательного числа). Чему равна сложность данного решения?
\item Хороший: Сначала находим остаток от деления X на 12. В зависимости от остатка определяем победителя. Чему равна сложность данного решения?
\end{itemize}
\end{enumerate}

\newpage
\subsection*{Сортировки}
\begin{enumerate}
\item Создайте массив со следующими элементами: \{163, 623, 7345, 545, 43, 73, 5, 536, 963, 1571\}
\item \textbf{Сортировка выбором:} Написать функцию сортировки выбором \texttt{void selection\_sort(int n, int arr[])}.\\
\texttt{arr} -- массив чисел, которые нужно отсортировать, \texttt{n} -- количество чисел в этом массиве.
Будем обозначать подмассивы так: arr[k:m] -- подмассив массива arr с элементами под номерами от k до m-1 (не включая m). Таким образом, весь массив можно обозначить как arr[0:n]. \\

Алгоритм сортировки выбором:
\begin{itemize}
\item Найти минимальный элемент в массиве.
\item Поменять местами минимальный элемент и первый элемент массива.
\item Повторить эти операции для подмассива arr[1:n], затем для подмассива arr[2:n] и т.д.
\end{itemize}
\item \textbf{Реккурсивная сортировка выбором:} Написать реккурсивную функцию сортировки выбором \\
\texttt{void rec\_selection\_sort(int start, int n, int arr[])}.\\
\texttt{arr} -- массив чисел, которые нужно отсортировать, 
\texttt{n} -- количество чисел в массиве arr,
\texttt{start} -- начальный индекс подмассива в массиве arr.\\
\item \textbf{Быстрая сортировка:}

\end{enumerate}
\end{document}