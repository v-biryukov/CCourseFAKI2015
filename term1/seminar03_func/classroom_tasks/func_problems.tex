\documentclass{article}
\usepackage[utf8x]{inputenc}
\usepackage{ucs}
\usepackage{amsmath} 
\usepackage{amsfonts}
\usepackage{marvosym}
\usepackage{wasysym}
\usepackage{upgreek}
\usepackage[english,russian]{babel}
\usepackage{graphicx}
\usepackage{float}
\usepackage{textcomp}
\usepackage{hyperref}
\usepackage{geometry}
  \geometry{left=2cm}
  \geometry{right=1.5cm}
  \geometry{top=1cm}
  \geometry{bottom=2cm}
\usepackage{tikz}
\usepackage{ccaption}
\usepackage{multicol}
\usepackage{hyperref}



\usepackage{listings}
%\setlength{\columnsep}{1.5cm}
%\setlength{\columnseprule}{0.2pt}

\usepackage[absolute]{textpos}

\begin{document}
\pagenumbering{gobble}

\lstset{
  language=C,                % choose the language of the code
  basicstyle=\linespread{1.1}\ttfamily,
  columns=fixed,
  fontadjust=true,
  basewidth=0.5em,
  keywordstyle=\color{blue}\bfseries,
  commentstyle=\color{gray},
  stringstyle=\ttfamily\color{orange!50!black},
  showstringspaces=false,
  numbersep=5pt,
  numberstyle=\tiny\color{black},
  numberfirstline=true,
  stepnumber=1,                   % the step between two line-numbers.        
  numbersep=10pt,                  % how far the line-numbers are from the code
  backgroundcolor=\color{white},  % choose the background color. You must add \usepackage{color}
  showstringspaces=false,         % underline spaces within strings
  captionpos=b,                   % sets the caption-position to bottom
  breaklines=true,                % sets automatic line breaking
  breakatwhitespace=true,         % sets if automatic breaks should only happen at whitespace
  xleftmargin=.2in,
  extendedchars=\true,
  keepspaces = true,
}
\lstset{literate=%
   *{0}{{{\color{red!20!violet}0}}}1
    {1}{{{\color{red!20!violet}1}}}1
    {2}{{{\color{red!20!violet}2}}}1
    {3}{{{\color{red!20!violet}3}}}1
    {4}{{{\color{red!20!violet}4}}}1
    {5}{{{\color{red!20!violet}5}}}1
    {6}{{{\color{red!20!violet}6}}}1
    {7}{{{\color{red!20!violet}7}}}1
    {8}{{{\color{red!20!violet}8}}}1
    {9}{{{\color{red!20!violet}9}}}1
}

\section*{Задачи на функции:}
\begin{enumerate}
\item \textbf{Фаренгейты в цельсии:} На вход подаются температуры плавления и кипения в Фаренгейтах некого вещества. Вам нужно узнать будет ли оно жидкостью при комнатной температуре (25 градусов цельсия) и напечатать температуры плавления и кипения в цельсиях.\\ Формула для перевода Цельсий в Фаренгейты: $T_c = \frac{5}{9}(T_f-32)$. Была написана следующую программу:
\begin{lstlisting}
#include <stdio.h>
int main()
{
	float Tf1, Tf2;
	scanf("%f%f", &Tf1, &Tf2);
	float Tc_room = 25;
	
	if (Tc_room > 5.0/9.0*(Tf1-32) && Tc_room < 5.0/9.0*(Tf2-32))
		printf("YES\n");
	else
		printf("NO\n");
	
	printf("Temepratures in Celsius %f %f\n", 5.0/9.0*(Tf1-32), 5.0/9.0*(Tf2-32));
}
\end{lstlisting}
Упросите программу, добавив функцию \texttt{float ftoc(float tf)}, которая будет принимать на вход температуру в фаренгейтах и возвращать её в значение в цельсиях.

\item \textbf{Максимум:} У вас есть функция max, которая принимает 2 числа и возвращает их максимум:
\begin{lstlisting}
#include <stdio.h>
int max(int a, int b)
{
	if (a > b)
		return a;
	else
		return b;
}
int main()
{
	int a, b, c;
	scanf("%d%d%d", &a, &b, &c);
	printf("Max of b and c = %d\n", << Ваш код >>);
	printf("Max of a, b, c = %d\n", << Ваш код >>);
}

\end{lstlisting}
Используя эту функцию, найти максимум 2-х чисел и максимум 3-х чисел.
\item \textbf{Функция, которая ничего не делает:} Написать функцию, которая ничего не принимает, не возвращает и вообще ничего не делает. Используйте \texttt{void}.
\item \textbf{Функция, которая только возвращает:} Написать функцию \textbf{int the\_answer()}, которая ничего не принимает, но возвращает целое число равное 42. Вызвать эту функцию в функции \textbf{main()}, чтобы вывести на экран строку ``The answer is 42''.
\item \textbf{Функция, которая только печатает:} Написать функцию \textbf{void hello()}, которая ничего не принимает и не возвращает, но выводит на экран строку ``Hello''. Вызвать эту функцию в функции \textbf{main()} 100 раз в цикле.
\item \textbf{Counter iterative:} Написать функцию \texttt{void counter\_iterative(int n)}, которая принимает число n, ничего не возвращает и печатает числа от 0 до n. Вызвать эту функцию в функции \texttt{main()}.

\newpage
\item \textbf{Doubler:} Написать функцию doubler, которая принимает число и возвращает это число, умноженное на 2. Вызвать эту функцию из функции main:
\begin{lstlisting}
// Тут нужно написать функцию doubler
int main()
{
	int a;
	scanf("%d", &a);
	printf("%d\n", doubler(a));
}
\end{lstlisting}


\item \textbf{Функция, которая принимает, возвращает, считывает и печатает:} Напишите функцию \texttt{float combiner(float a)}, которая принимает вещественное число a, считывает вещественное число b с помощью scanf, печатает их среднее арифметическое($(a+b)/2$) и возвращает их среднее геометрическое ($\sqrt{a b}$). Используйте эту функцию в функции main, чтобы найти среднее арифметическое и среднее геометрическое чисел 42 и 256. \\ Для нахождения корня вещественного числа вам понадобится функция \textbf{sqrt} из библиотеки math.h. Чтобы подключить эту библиотеку вам нужно добавить в исходный код соответствующую директиву include и добавить опцию компилятора \textbf{-lm}. \\ \textbf{\$ gcc -lm <имя исходного файла>}

\item \textbf{Print time:} Написать функцию \texttt{void print\_time(int h, int m)}, которая принимает на вход два целых числа (часы и минуты), и выводит строку вида \textbf{hh:mm}. Для печати времени в таком формате нужно использовать функцию printf с опцией ``\%02i:\%02i''. Вызвать эту функцию в функции \texttt{main()}.

\item \textbf{Is prime?:} Написать функцию \textbf{int is\_prime(int n)}, которая будет проверять является ли число n простым и возвращать 1 если число n простое либо 0 если число n не является простым.
\item \textbf{Print primes:} Написать функцию \textbf{void print\_primes(int a, int b)}, которая будет печатать все простые числа из отрезка [a, b]. Используйте функцию \texttt{is\_prime} из предыдущей задачи! Вызвать эту функцию в функции \textbf{main()}.


\subsection*{Рекурсия:}
\item \textbf{Counter:} Написана рекурсивная функция \textbf{void counter(int n)}, которая печатает числа от n до 1.
\begin{lstlisting}
void counter(int n)
{
	if (n > 0)
	{
		printf("%d ", n);
		counter(n-1);
	}
}
int main()
{
	counter(42);
}
\end{lstlisting}
Видоизмените эту функцию так, чтобы она печатала числа от 1 до n. (в нормальном порядке).

\item \textbf{Sum recursive:} Напишите функцию \textbf{int sumrec(int n)}, которая рекурсивно вычисляет сумму первых n натуральных чисел. Вызовите эту функцию из main.


\newpage
\subsection*{Задачи на функции. Функции и массивы:}
На вход подаётся натуральное число n и n целых чисел. считайте этот массив в функции main:
\begin{lstlisting}
int main()
{
	int n;
	int arr[1000];
	scanf("%d", &n);
	for (int i = 0; i < n; ++i)
		scanf("%d", &arr[i]);
}
\end{lstlisting}
\item \textbf{Array print:} Напишите функцию \texttt{void print\_array(int n, int arr[])}, которая принимает на вход массив и печатает все его числа через пробел. В конце строки напечатайте перенос на новую строку.
\item \textbf{Array sum:} Напишите функцию \texttt{int sum\_array(int n, int arr[])}, которая принимает на вход массив и возвращает сумму массива. Вызовите эту функцию в main(), чтобы найти сумму массива. Сама функция не должна ничего печатать.
\item \textbf{Array max:} Напишите функцию \texttt{int max\_array(int n, int arr[])}, которая принимает на вход массив и возвращает максимальный элемент массива. Вызовите эту функцию в main(), чтобы найти сумму массива. Сама функция не должна ничего печатать.
\item \textbf{Double array:} Напишите функцию \texttt{void double\_array(int n, int arr[])}, которая увеличивает каждый элемент массива в 2 раза. Вызовите эту функцию в main(). Напечатайте массив до и после увеличения в 2 раза, используя функцию print\_array.
\item \textbf{Sort array:} Напишите функцию \texttt{void sort\_array(int n, int arr[])}, которая сортирует массив методом выбора. Вызовите эту функцию в main(). Напечатайте массив до и после сортировки, используя print\_array.

\subsection*{Задачи на функции. Передача по ссылке (дополнительно):}
\item \textbf{Modify:} Написать функцию \textbf{void doublerm(int*)}, которая удваивает число, поступающее на вход, используя указатель на эту переменную. Используйте эту функцию, чтобы вывести на экран кубы чисел от 1 до 100.
\item \textbf{Swap:} Написать функцию \textbf{swap}, которая меняет значения 2-х переменных типа int местами. Используйте эту функцию в функции \textbf{main()}.



\end{enumerate}

\end{document}