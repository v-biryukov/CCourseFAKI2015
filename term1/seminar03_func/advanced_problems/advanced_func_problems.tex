\documentclass{article}
\usepackage[utf8x]{inputenc}
\usepackage{ucs}
\usepackage{amsmath} 
\usepackage{amsfonts}
\usepackage{marvosym}
\usepackage{wasysym}
\usepackage{upgreek}
\usepackage[english,russian]{babel}
\usepackage{graphicx}
\usepackage{float}
\usepackage{textcomp}
\usepackage{hyperref}
\usepackage{geometry}
  \geometry{left=2cm}
  \geometry{right=1.5cm}
  \geometry{top=1cm}
  \geometry{bottom=2cm}
\usepackage{tikz}
\usepackage{ccaption}
\usepackage{multicol}
\usepackage{hyperref}



\usepackage{listings}
%\setlength{\columnsep}{1.5cm}
%\setlength{\columnseprule}{0.2pt}

\usepackage[absolute]{textpos}

\begin{document}
\pagenumbering{gobble}

\lstset{
  language=C,                % choose the language of the code
  basicstyle=\linespread{1.1}\ttfamily,
  columns=fixed,
  fontadjust=true,
  basewidth=0.5em,
  keywordstyle=\color{blue}\bfseries,
  commentstyle=\color{gray},
  stringstyle=\ttfamily\color{orange!50!black},
  showstringspaces=false,
  numbersep=5pt,
  numberstyle=\tiny\color{black},
  numberfirstline=true,
  stepnumber=1,                   % the step between two line-numbers.        
  numbersep=10pt,                  % how far the line-numbers are from the code
  backgroundcolor=\color{white},  % choose the background color. You must add \usepackage{color}
  showstringspaces=false,         % underline spaces within strings
  captionpos=b,                   % sets the caption-position to bottom
  breaklines=true,                % sets automatic line breaking
  breakatwhitespace=true,         % sets if automatic breaks should only happen at whitespace
  xleftmargin=.2in,
  extendedchars=\true,
  keepspaces = true,
}
\lstset{literate=%
   *{0}{{{\color{red!20!violet}0}}}1
    {1}{{{\color{red!20!violet}1}}}1
    {2}{{{\color{red!20!violet}2}}}1
    {3}{{{\color{red!20!violet}3}}}1
    {4}{{{\color{red!20!violet}4}}}1
    {5}{{{\color{red!20!violet}5}}}1
    {6}{{{\color{red!20!violet}6}}}1
    {7}{{{\color{red!20!violet}7}}}1
    {8}{{{\color{red!20!violet}8}}}1
    {9}{{{\color{red!20!violet}9}}}1
}

\section*{Продвинутые задачи:}
\begin{enumerate}
\item \textbf{Задача 1 - Продвинутый helloworld:} 
Вывести на экран строку \texttt{!\textbackslash@\#\$\textasciicircum\&\%} Если возникнут вопросы по этой или по другим задачам, то ответы можно найти на stackoverflow. Просто загуглите, например, "how to print backslash c" или "how to print backslash c stackoverflow".


\item \textbf{Задача 2 - Целочисленные переменные:} Различные целочисленные типы языка C представлены в следующей таблице:

\begin{center}
\begin{tabular}{ c c c c }
 тип & размер (байт) & диапазон значений ($2^{\# bits}$) & модификатор \\ \hline
 char & 1 & от -128 до 127 & \%hhd \\ 
 short & 2 & от -32768 до 32767 & \%hd  \\  
 int & 4 & примерно от -2-х миллиардов до 2-х миллиардов & \%d  \\  
 long & 4 или 8 & такой же как у int или long long в зависимости от системы & \%ld  \\  
 long long & 8 & примерно от $-10^{19}$ до $10^{19}$ & \%lld  \\  
 unsigned char & 1 & от 0 до 255 & \%hhu \\ 
 unsigned short & 2 & от 0 до 65535 & \%hu  \\  
 unsigned int & 4 & примерно от 0 до 4-х миллиардов & \%u  \\  
 unsigned long & 4 или 8 & такой же как у unsigned int или unsigned long long & \%lu  \\  
 unsigned long long & 8 & от 0 до $2^{64} \approx 2*10^{19}$  & \%llu  \\  
 16-ричная система & - & - & \%x  \\ 
 указатель & 8 & $2^{64} \approx 2*10^{19}$ & \%p  \\  
\end{tabular}
\end{center}


\begin{enumerate}
\item \textbf{Произведение чисел:} Напишите функцию, которая вычисляет произведение 2-х положительный чисел $a < 2^{32}$ и $b < 2^{32}$. Проверьте вашу функцию на следующих значениях:
\begin{center}
\begin{tabular}{ c c }
 вход & выход \\ \hline
 2 2 & 4  \\ 
 2000000000 2 & 4000000000  \\ 
 1444444444 777777777 & 1123456788654320988 \\ 
 4222222222 3777777777 & 15950617279827160494 \\   
\end{tabular}
\end{center}
\item \textbf{Факториал:} Напишите функцию, которая вычисляет факториал числа $n \leq 20$. Проверьте вашу функцию на следующих значениях:
\begin{center}
\begin{tabular}{ c c }
 вход & выход \\ \hline
 0 & 1  \\ 
 1 & 1 \\  
 5 & 120   \\  
 10 & 3628800   \\  
 20 & 2432902008176640000   \\  
\end{tabular}
\end{center}

\item \textbf{Размещения:} В комбинаторике размещением (из n по k) $A_n^k$ называется упорядоченный набор из k различных элементов из некоторого множества различных n элементов. Размещения вычисляются следующим образом: $A_n^k = \frac{n!}{(n-k)!}$. Напишите функцию, которая будет вычислять размещения при условии, что  $A_n^k < 2^{64}$. Проверьте вашу функцию на следующих значениях:
\begin{center}
\begin{tabular}{ c c }
 вход & выход \\ \hline
 5 2 & 20  \\ 
 20 10 & 670442572800  \\ 
 30 12 & 41430393164160000 \\ 
 60 11 & 13679492361575040000 \\   
\end{tabular}
\end{center}
\end{enumerate}



\newpage
\item \textbf{Задача 3 - Передача в функцию по адресу:} \\
Краткое введение в указатели (указатели это очень просто):
\begin{lstlisting}
int main() { 
    // Предположим у нас есть переменная:
    int x = 42; 
    // Положение этой переменной в памяти характеризуется двумя числами - адресом и 
    // размером переменной. Узнать их можно, используя операторы & и sizeof:
    printf("Size and address of x = %d and %llu\n",  sizeof(x), &x);
    // Обратите внимание, что для отображения адреса использовался модификатор %llu, так 
    // как в 64 битных системах адрес это 64 битное число. Также можно было бы 
    // использовать модификатор %p.
    
    // Для работы с адресами в языке C вводится специальный тип, который называется 
    // указатель. Введём переменную для хранения адреса переменной x:
    int* address_of_x = &x;
    // Теперь в переменной address_of_x типа int* будет храниться число - адрес 
    // переменной x. Если бы x был бы не int, а float, то для хранения адреса x нужно было 
    // бы использовать тип float* .
    
    // Ну хорошо, у нас есть переменная, которая хранит адрес x. Как её дальше 
    // использовать? Очень просто - поставьте звёздочку перед адресом, чтобы получить 
    // переменную x.
    // *address_of_x это то же самое, что и  x
    *address_of_x += 10;
    printf("%d\n", x);
    // Запомните: &  -  по переменной получить адрес
    //            *  -  по адресу получить переменную
} 
\end{lstlisting}

Как использовать указатели для передачи в функции адреса переменной: 
\begin{lstlisting}
#include <stdio.h>
// Эта функция не удвоит значение
void doubler_naive(int x) {
	x *= 2;
}

// Эта функция сработает, но таким образом можно изменить только одну переменную за раз.
// К тому же, тут происходит 2 лишних копирования переменной x. Что может быть плохо, если переменная x будет не типа int, а, например, структурой большого размера.
int doubler(int x) {
	return 2*x;
}

// Лучший способ - передача по адресу
void doubler_by_address(int* address_of_x) {
	*address_of_x *= 2;
}

int main() {
	int x = 79;
	printf("%d\n", x);
	doubler_naive(x);
	printf("%d\n", x);
	x = doubler(x);
	printf("%d\n", x);
	doubler_by_address(&x);
	printf("%d\n", x);
}
\end{lstlisting}

\begin{enumerate}
\item \textbf{Меняем переменную по адресу:} Пусть в функции main() определена переменная \texttt{float\ x = 4.53}. Вам нужно ввести переменную типа \texttt{float*} и сохранить в ней адрес x. А затем увеличить x в 2 раза, используя только указатель.
\item \textbf{Куб:} Напишите функции \texttt{float cube1(float x)} и \texttt{void cube2(float* address\_of\_x)}, которые будут возводить значение переменной в куб двумя разными методами. Вызовите обе функции в функции main(). 
\item \textbf{Swap:} Напишите функцию \texttt{void swap(int* address\_of\_a, int* address\_of\_b)}, которая меняет значения 2-х переменных типа int местами. Используйте эту функцию в функции \texttt{main()}. 

\item \textbf{Меняем указатель:} 
В приведённой ниже программе программист хотел написать функцию, которая бы меняла указатель p таким образом, чтобы он хранил адрес глобальной переменной. Но, к сожалению, он сделал ошибку.
\begin{lstlisting}
int global_x = 22; 

void change(int* address) { 
    address = &global_x; 
} 
  
int main() { 
    int x = 11; 
    int* p = &x;
  
    // Пытаемся изменить указатель, чтобы он хранил адрес глобальной переменной
    printf("before: %d\n", *p);
    change(p);
    printf("after: %d\n", *p);
} 
\end{lstlisting}
Исправьте этот код.
\end{enumerate}


\end{enumerate}

\end{document}