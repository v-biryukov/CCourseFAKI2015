\documentclass{article}
\usepackage[utf8x]{inputenc}
\usepackage{ucs}
\usepackage{amsmath} 
\usepackage{mathtext}
\usepackage{amsfonts}
\usepackage{marvosym}
\usepackage{wasysym}
\usepackage{upgreek}
\usepackage[english,russian]{babel}
\usepackage{graphicx}
\usepackage{float}
\usepackage{textcomp}
\usepackage{hyperref}
\usepackage{geometry}
  \geometry{left=2cm}
  \geometry{right=1.5cm}
  \geometry{top=1cm}
  \geometry{bottom=2cm}
\usepackage{tikz}
\usepackage{ccaption}
\usepackage{multicol}
\usepackage{hyperref}



\usepackage{listings}
%\setlength{\columnsep}{1.5cm}
%\setlength{\columnseprule}{0.2pt}

\usepackage[absolute]{textpos}

\begin{document}
\pagenumbering{gobble}

\lstset{
  language=C,                % choose the language of the code
  basicstyle=\linespread{1.1}\ttfamily,
  columns=fixed,
  fontadjust=true,
  basewidth=0.5em,
  keywordstyle=\color{blue}\bfseries,
  commentstyle=\color{gray},
  stringstyle=\ttfamily\color{orange!50!black},
  showstringspaces=false,
  %numbers=false,                   % where to put the line-numbers
  numbersep=5pt,
  numberstyle=\tiny\color{black},
  numberfirstline=true,
  stepnumber=1,                   % the step between two line-numbers.        
  numbersep=10pt,                  % how far the line-numbers are from the code
  backgroundcolor=\color{white},  % choose the background color. You must add \usepackage{color}
  showstringspaces=false,         % underline spaces within strings
  captionpos=b,                   % sets the caption-position to bottom
  breaklines=true,                % sets automatic line breaking
  breakatwhitespace=true,         % sets if automatic breaks should only happen at whitespace
  xleftmargin=.2in,
  extendedchars=\true,
  keepspaces = true,
}
\lstset{literate=%
   *{0}{{{\color{red!20!violet}0}}}1
    {1}{{{\color{red!20!violet}1}}}1
    {2}{{{\color{red!20!violet}2}}}1
    {3}{{{\color{red!20!violet}3}}}1
    {4}{{{\color{red!20!violet}4}}}1
    {5}{{{\color{red!20!violet}5}}}1
    {6}{{{\color{red!20!violet}6}}}1
    {7}{{{\color{red!20!violet}7}}}1
    {8}{{{\color{red!20!violet}8}}}1
    {9}{{{\color{red!20!violet}9}}}1
}


\subsection*{Добро пожаловать в язык программирования C:}
Простейшая программа на языке C выглядит следующим образом:
\begin{lstlisting}
int main() {}
\end{lstlisting}
Любая программа на языке C должна содержать особую функцию под названием main. По аналогии с обычными математическими функциями, функции в языке C могут принимать и возвращать значения. Принимаемые значения указываются в круглых скобкая(в данном случае там ничего нет так как функция ничего не принимает) а тип возвращаемого значения указывается перед функцией (для функции main это всегда тип int, т.е. Integer - т.е. целое число). В фигурных скобках описываются операции, которые совершает функция. Скомпилируйте данную программу и запустите.


Для ввода с экрана и вывода на экран используются функции printf и scanf из библиотеки stdio.h:
\begin{lstlisting}
#include <stdio.h>  // Подключаем библиотеку, содержащую функции printf и scanf
int main() 
{
    printf("Hello world!"); // Используем функцию printf
}
\end{lstlisting}
Скомпилируйте данную программу и запустите. \\

\subsubsection*{Задание 1:}
В строке функции printf() можно использовать некоторые специальные символы \textbackslash n и \textbackslash t. Добавьте эти символы в строку функции printf и выясните, что они делают.

\subsubsection*{Переменные:}
Работа с переменными
\begin{lstlisting}
int main() 
{
    int a;
    int b = 5;
    a = 3;
    c = a * b + (b / a);
}
\end{lstlisting}


\begin{lstlisting}
#include <stdio.h>
int main() 
{
    int age = 15; // Обьявляем и инициализируем переменную типа целое число
    age = 20;
    printf("I am %i years old\n", my_age);
}
\end{lstlisting}

\begin{lstlisting}
#include <stdio.h>
int main() 
{
    int a;     // Обьявляем переменную типа целое число
    int b = 5; // Обьявляем и инициализируем переменную типа целое число
    a = 4;
    b = 6;
    printf("I am %i years old\n", a);
}
\end{lstlisting}

\subsubsection*{Задание 1:}
В строке функции printf() можно использовать некоторые специальные символы \textbackslash n и \textbackslash t. Добавьте эти символы в строку функции printf и выясните, что они делают.

\end{document}