\documentclass{article}
\usepackage[utf8x]{inputenc}
\usepackage{ucs}
\usepackage{amsmath} 
\usepackage{mathtext}
\usepackage{amsfonts}
\usepackage{upgreek}
\usepackage[english,russian]{babel}
\usepackage{graphicx}
\usepackage{textcomp}
\usepackage{geometry}
  \geometry{left=2cm}
  \geometry{right=1.5cm}
  \geometry{top=1cm}
  \geometry{bottom=2cm}
\usepackage{tikz}
\usepackage{ccaption}



\begin{document}
\pagenumbering{gobble}
\section*{Вопросы:}

\begin{enumerate}

\item Назовите базовые команды командной строки linux
\item Что такое препроцессор?
\item Что делает директива \#include?
\item Что делает директива \#define?
\item Назовите все стандартные целочисленные типы C и их размер.
\item Назовите все стандартные типы с плавающей точкой в C и их размер.
\item Операторы break и continue
\item Что такое прототип функции?
\item Как функция может возвращать значения?
\item Спецификатор типа void
\item Области видимости переменных в функции
\item Передача по ссылке и по значению
\item Аргументы функции main
\item Как объявить константу в C?
\item Объявление массива
\item Как хранятся массивы в памяти
\item Строки в стиле C
\item Функции работы со строками
\item Алгоритмы сортировки вставками, выбором, пузырьком (1 из 3-х)
\item Описание структуры, typedef
\item Что такое указатель? Какой размер переменной указателя? 
\item Как получить адрес переменной? Как получить переменную по адресу?
\item Указатель на 1-й элемент массива, на 5-й.
\item Стэк и куча. Что такое, преемущества и недостатки.
\item Что делает malloc? Выделите память на массив из n элементов типа int.
\item Как освобождать память? Почему плохо не освобождать память?
\item Есть массив A состоящий из 100 элементов. Что будет, если исполнить A[110].
\item Принцип "разделяй и властвуй"
\item Сортировка слиянием или быстрая сортировка
\item Что такое связный список.
\item Сложность добавления и удаления элементов в связный список и в массив.
\item Что такое дерево?
\item Что такое двоичное дерево поиска? 
\item Сложность поиска элемента с помощью двоичного дерева поиска. Сравнение с бинарным поиском.





\end{enumerate}


\end{document}