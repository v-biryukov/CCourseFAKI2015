\documentclass{article}
\usepackage[utf8x]{inputenc}
\usepackage{ucs}
\usepackage{amsmath} 
\usepackage{mathtext}
\usepackage{amsfonts}
\usepackage{upgreek}
\usepackage[english,russian]{babel}
\usepackage{graphicx}
\usepackage{float}
\usepackage{textcomp}
\usepackage{hyperref}
\usepackage{geometry}
  \geometry{left=2cm}
  \geometry{right=1.5cm}
  \geometry{top=1cm}
  \geometry{bottom=2cm}
\usepackage{tikz}
\usepackage{ccaption}



\begin{document}
\pagenumbering{gobble}

\section*{Теория:}
\begin{enumerate}

\item  \textbf{O(n) нотация}\\
Что такое $O(n)$, $\Omega(n)$ и $\Theta(n)$ нотации, математическое определение и смысл.
Знать(или уметь выводить) сложности всех пройденных алгоритмов: алгоритм бинарного поиска в отсортированном массиве, алгоритмы сортировки (пузырьком, вставками, выбором, быстрая, сортировка слиянием, подсчётом, цифровая). Знать или уметь выводить сложности операций с пройденными структурами данных: поиск массива по индексу в массиве и списке, вставка в массив(статический и динамический) и список; удаление из массива и списка; поиск по массиву и списку.\\

\item  \textbf{Стек и очередь}\\
Структуры данных и абстрактный типы данных. Стек. Операции push и pop. Реализация стека на основе массива. Динамическое выделение памяти для стека. Стратегии перевыделения памяти(аддитивная и мультипликативная).  Очередь. Реализация стека и очереди на основе связного списка. Алгоритмические сложности операций со стеком.\\

\item \textbf{Связный список}\\
Связный список. Как реализуются функции нахождения длины списка, вставки элемента в начало/конец, удаления элемента из начала/конца, поиска в списке, обращения списка. Алгоритмические сложности операций со списком. Двусвязный список.\\


\item \textbf{Cортировка}\\
Знать сортировки пузырьком, вставками, выбором, быструю сортировку, сортировку слиянием, подсчётом и сортировка выбором. Алгоритмические сложности всех этих алгоритмов. Уметь писать сортировку выбором, быструю сортировку и сортировку подсчётом. Уметь использовать стандартный функцию сортировки qsort.\\

\item \textbf{Разделяй и властвуй}\\
Парадигма разделяй и властвуй. Логарифмическая сложность. Алгоритм бинарного поиска в отсортированном массиве, сортировка слиянием, быстрая сортировка. Уметь писать бинарный поиск и быструю сортировку.\\

\item \textbf{Память}\\
Что такое переменная. Адреса переменных. Указатели. Определение адреса переменной. Разыменование. Основные типы данных и их размеры: \texttt{int, float double, char, void*, int*, char*}. Адресная арифметика. Динамическое выделение и освобождение памяти в куче: malloc(), free() и realloc(). Утечки памяти. Указатель \texttt{void*}. Преобразование типа указателя. Организация массива в памяти. Организация связного списка в памяти.\\

\item \textbf{Передача аргументов в функцию}\\
Все типы передачи аргументов в функцию. Передача по ссылке и передача по значению. Передача через адрес переменной. Передача через указатель на константу. Передача массивов, строк и двумерных массивов в функцию. Передача структур в функцию. Возвращение аргумента. Возвращение нескольких аргументов из функции. Возвращение массива из функции. Ключевое слово void.\\

\item \textbf{Сегменты памяти}\\
Что такое сегменты памяти. Сегмент памяти стек вызовов(или просто стек). Выделение памяти в стеке. Переполнение стека. Сегмент памяти куча(heap). Динамическое выделение и освобождение памяти в куче: malloc(), free() и realloc(). Динамическое выделение двумерного массива. Преимущества и недостатки кучи перед стеком. Сегмент памяти text. Ошибка Segmentation Fault.\\

\item \textbf{Этапы компиляции}\\
Что такое файл исходного кода и исполняемый файл. Этап компиляции: препроцессинг, компиляция и линковка. Директивы препроцессора \#include и \#define. Компиляция программы с помощью gcc. Опции gcc: -E, -c.

\end{enumerate}



\newpage
\section*{Практика:}
Из того, что не было на контрольной работе:

\begin{enumerate}
\item \textbf{Основные команды командной строки linux}: cd, ls, pwd, cp, mv, mkdir, текстовый редактор nano (или vi), компилятор gcc. Создать файл исходного кода программы, которая будет считывать числа из входного файла, складывать их и записывать результат в другой файл. Создать файл входных данных. Скомпилировать программу. Всё это, пользуясь только командами командной строки. Для тех, у кого нет возможности использовать linux:\\
\href{https://www.tutorialspoint.com/unix_terminal_online.php}{www.tutorialspoint.com/unix\_terminal\_online.php}\\
Там нет редактора nano, но есть редактор vi (или vim). \\
Основы работы с vi вы можете посмотреть тут:
\href{https://www.youtube.com/watch?v=R33F0EDivwk}
{youtube.com/watch?v=R33F0EDivwk}

\item \textbf{Аргументы командной строки, бинарное чтение/запись} \\
Уметь использовать в программе аргументы командной строки: argc, argv. Бинарное чтение/запись -- fread, fwrite.\\
\end{enumerate}


\section*{Материалы для подготовки:}
\begin{enumerate}
\item Кормен, Лейзерсон, Ривест. Алгоритмы: построение и анализ.\\ 
\hspace*{16pt} \href{https://lib.mipt.ru/book/16736/}{lib.mipt.ru/book/16736/}
\item Керниган Ритчи Язык программирования C \\
\hspace*{16pt} \href{https://lib.mipt.ru/book/266005/}{lib.mipt.ru/book/266005/}
\item Простое введение в алгоритмы hexlet.io. \\ \hspace*{16pt} \href{https://www.youtube.com/watch?v=8JlTwMg1dyw&list=PLwwk4BHih4fjIT5cT4i1s93b99aJScUGB}{youtube.com/watch?v=8JlTwMg1dyw\&list=PLwwk4BHih4fjIT5cT4i1s93b99aJScUGB}
\item Лекторий МФТИ: \\ \hspace*{16pt} 
\href{http://lectoriy.mipt.ru/course/ComputerTechnology-Informatics-14L#lectures}
{lectoriy.mipt.ru/course/ComputerTechnology-Informatics-14L\#lectures}
\item Курс по структурам данных stepic.org  \\
\hspace*{16pt} \href{https://www.youtube.com/watch?v=vRvSdWVst54}{youtube.com/watch?v=vRvSdWVst54}
\item Продвинутый уровень. Язык C и структуры данных в Йелле.\\ \hspace*{16pt} \href{http://www.cs.yale.edu/homes/aspnes/classes/223/notes.html}{www.cs.yale.edu/homes/aspnes/classes/223/notes.html}
\end{enumerate}

\subsubsection*{Дополнительные материалы по некоторым вопросам:}
\begin{itemize}

\item Вопрос №8. Сегменты памяти:\\
\href{https://www.youtube.com/watch?v=_8-ht2AKyH4}{youtube.com/watch?v=\_8-ht2AKyH4}  (англ.)\\
\href{https://tproger.ru/translations/programming-concepts-stack-and-heap/}{tproger.ru/translations/programming-concepts-stack-and-heap/}
\item Вопрос №9. Этапы компиляции:\\
\href{https://www.youtube.com/watch?v=ylA55D4B4_M}{youtube.com/watch?v=ylA55D4B4\_M}

\end{itemize}
\end{document}