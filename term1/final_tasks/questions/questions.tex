\documentclass{article}
\usepackage[utf8x]{inputenc}
\usepackage{ucs}
\usepackage{amsmath} 
\usepackage{amsfonts}
\usepackage{upgreek}
\usepackage[english,russian]{babel}
\usepackage{graphicx}
\usepackage{float}
\usepackage{textcomp}
\usepackage{hyperref}
\usepackage{geometry}
  \geometry{left=2cm}
  \geometry{right=1.5cm}
  \geometry{top=1cm}
  \geometry{bottom=2cm}
\usepackage{tikz}
\usepackage{ccaption}
\usepackage{mathrsfs}


\begin{document}
\pagenumbering{gobble}

\section*{Теория:}
\begin{enumerate}
\item \textbf{Основные команды командной строки linux (\texttt{bash})}:\\
 \texttt{cd}, \texttt{ls} (опции \texttt{-l} и \texttt{-a}), \texttt{pwd}, \texttt{cp}, \texttt{mv}, \texttt{rm} (опция \texttt{-r}), \texttt{mkdir}, \texttt{find}(опция \texttt{-name}), \texttt{du}(опции \texttt{-s} и \texttt{-h}), программа \texttt{top}, текстовый редактор \texttt{nano} (или текстовый редактор \texttt{vim}), компилятор \texttt{gcc} (опции \texttt{-o}, \texttt{-std=c99}, \texttt{-lm} и \texttt{-S}). Перенаправление вывода \texttt{>}. Если у вас Windows, то советую использовать консоль cmder.

\item \textbf{Основы C: переменные, операторы, функции из стандартной библиотеки}\\
Переменные. Типы целочисленных переменных (\texttt{char}, \texttt{short}, \texttt{int}, \texttt{long}, \texttt{long long}, \texttt{unsigned char}, \texttt{unsigned short}, \texttt{unsigned int}, \texttt{unsigned long}, \texttt{unsigned long long}, \texttt{size\_t}). Размеры этих типов. Представление целочисленных переменных в памяти. Типы чисел, для хранения вещественных чисел (\texttt{float} и \texttt{double})  Размеры этих типов и представление их в памяти. Операторы. Арифметические операторы(\texttt{+ - * / \%}). Операторы присваивания (\texttt{= += -= *= /= \%=}). Операторы инкремента и декремента (\texttt{++  $--$}). Операторы сравнения (\texttt{== != > < >= <=}). Логические операторы (\texttt{!  ||  \&\&}). Побитовые операторы (\texttt{$\sim$  \&  |  \textasciicircum \quad $\ll$  $\gg$}). Тернарный оператор (\texttt{? :}). Оператор нахождения адреса (\texttt{\&}). Оператор нахождения размера переменной (\texttt{sizeof}). Оператор разыменования (\texttt{*}). Оператор обращения к элементу массива (\texttt{[]}) и его связь с оператором разыменования. Оператор доступа к полю структуры (\texttt{.}). Оператор доступа к полю структуры через указатель на структуру (\texttt{->}). Приоритеты операторов. Явное и неявное приведение типов. Ввод и вывод в языке C. Функции \texttt{printf} и \texttt{scanf} из библиотеки \texttt{stdio.h}. Математическая библиотека \texttt{math.h}. Функции \texttt{sqrt}, \texttt{exp}, \texttt{sin}, \texttt{cos}, \texttt{tan}, \texttt{asin}, \texttt{acos}, \texttt{atan}, \texttt{atan2}, \texttt{fabs}, \texttt{floor}, \texttt{log}, \texttt{pow}. Сравнение двух чисел с плавающей точкой с помощью функции \texttt{fabs}. Библиотека \texttt{stdlib.h}. Функции \texttt{exit}, \texttt{malloc}, \texttt{calloc}, \texttt{realloc}, \texttt{free}, \texttt{rand}, \texttt{srand}, \texttt{abs}.

\item \textbf{Основы C: указатели, массивы и строки}\\
Указатель. Объявление указателя. Размер указателя. Адресная арифметика. Операции нахождения адреса (\texttt{\&}) и операция разыменования \texttt{*}. Указатели на указатели. Массивы. Объявление массивов. Инициализация массивов. Размер массивов. Связь между массивами и указателями. Статический двумерный массив. Строки. Связь между строками, массивами и указателями. Кодировка ASCII. Переменные \texttt{char} для хранения символов. Чтение и запись символов (\texttt{\%с}) и строк(\texttt{\%s}). Библиотека \texttt{string.h}. Функции \texttt{strlen}, \texttt{strcpy}, \texttt{strcmp}, \texttt{strcat}, \texttt{strstr}, \texttt{memset}, \texttt{memcpy}.

\item \textbf{Основы C: управляющие конструкции, функции и структуры}\\
Управляющие конструкции \texttt{if else}, \texttt{for}, \texttt{while}, \texttt{do while}, \texttt{switch}. Операторы \texttt{break} и \texttt{continue}. Функции. Объявления функций. Прототип функции. Три типа передачи аргументов в функцию (по значению, через указатель, через указатель на константу). Передача одномерных и многомерных массивов в функции. Возврат из функции. Ключевое слово \texttt{void}. Реализация вызова функций с помощью сегмента памяти стек. Адрес возрата. Стековый кадр. Рекурсия. Переполнение стека при рекурсии. Структуры. Инициализация структур. Доступ к полю структуры. Размер структуры. Указатели на структуры. Доступ к полю по указателю на структуру. Передача структур в функции и возврат их из функций. Оператор \texttt{typedef}.

\item \textbf{Сегменты памяти}\\
Что такое сегменты памяти. Сегмент памяти стек(не путать с абстрактным типом данных - стек). Выделение и освобождение памяти в стеке. Переполнение стека. Сегмент памяти куча (heap - не путать со структурой данных - куча). Динамическое выделение и освобождение памяти в куче: \texttt{malloc}, \texttt{calloc}, \texttt{realloc} и \texttt{free}. Преимущества и недостатки кучи перед стеком. Ошибка Segmentation Fault. Сегмент памяти text. Преобразование кода программы в код на языке ассемблера и в двоичный код. Указатели на функции.

\item \textbf{Динамическое выделение памяти}\\
Динамическое выделение и освобождение памяти в куче: \texttt{malloc}, \texttt{calloc}, \texttt{realloc} и \texttt{free}.  Указатель \texttt{void*}. Преобразование типа указателя. Организация массива в памяти. Организация связного списка в памяти. Хранение двумерного массива в виде одномерного массива. Динамический двумерный массив. Утечки памяти. Основы работы с valgrind.

\item  \textbf{Вычислительная сложность, $O(n)$ нотация}\\
Что такое $O(n)$, $\Omega(n)$ и $\Theta(n)$ нотации, математическое определение и смысл. Машина Тьюринга (детерминистическая и недетерминистическая). Классы сложности задач: $P$, $NP$, $PSPACE$, $EXPTIME$ и $EXPSPACE$.
Знать(или уметь выводить) сложности всех пройденных алгоритмов: алгоритм бинарного поиска в отсортированном массиве, алгоритмы сортировки (пузырьком, вставками, выбором, быстрая, сортировка слиянием, подсчётом, цифровая). Знать или уметь выводить сложности операций с пройденными структурами данных: поиск элемента по индексу в массиве и списке, вставка в массив(статический и динамический) и список; удаление из массива и списка; поиск по массиву и списку.


\item  \textbf{Стек и очередь}\\
Структуры данных и абстрактный типы данных. Абстрактный тип данных Стек. Операции push и pop. Реализация стека на основе динамического массива. Динамическое выделение памяти для стека. Стратегии перевыделения памяти: аддитивная и мультипликативная. Абстрактный тип данных Очередь. Реализация очереди на основе динамического массива. Реализация стека и очереди на основе связного списка. Алгоритмические сложности операций со стеком.

\item \textbf{Связный список}\\
Связный список. Узел связного списка. Реализация связного списка на языке C. Функции нахождения длины списка, вставки элемента в начало/конец, удаления элемента из начала/конца, поиска в списке, обращения списка. Алгоритмические сложности операций со списком. Двусвязный список.


\item \textbf{Деревья}\\
Определение графа. Определение дерева. Бинарное дерево. Бинарное дерево поиска(bst). Высота дерева. Представление бинарного дерева в языке C. Операции поиска, добавления и удаления элемента в бинарном дереве поиска. Вычислительные сложности этих операций. Сбалансированное дерево. Два распространённых приёма для балансировки бинарных деревьев поиска (AVL-деревья и красно-чёрные деревья). Определение AVL-дерева. Операции вращения. Операции поиска, добавления и удаления элемента в AVL-дерево. Вычислительные сложности этих операций.


\item \textbf{Сортировка}\\
Парадигма разделяй и властвуй. Как возникает логарифмическая сложность в задачах, решаемых с помощью метода разделяй и властвуй. Алгоритм бинарного поиска в отсортированном массиве.
Сортировки пузырьком, вставками, выбором, быстрая, слиянием, подсчётом, цифровая и $\mathbb{BOGOSORT}$. Вычислительные сложности всех этих алгоритмов. Уметь писать сортировку выбором, быструю сортировку и сортировку подсчётом. Cтандартная функция \texttt{qsort}. Функция компаратор \texttt{cmp}. Сортировка с помощью бинарного дерева поиска и с помощью бинарной кучи.


\item \textbf{Передача аргументов в функцию}\\
Три типа передачи аргументов в функцию. Передача по значению. Передача через указатель. Передача через указатель на константу. Передача массивов, строк и двумерных массивов в функцию. Передача структур в функцию. Возвращение переменных из функции. Возвращение массива из функции.


\item \textbf{Файлы и аргументы командной строки}\\
Системные вызовы. Системные вызовы для работы с файлами в Linux (\texttt{open}, \texttt{read}, \texttt{write} и \texttt{lseek}). Буферизация. Работа с файлами в языке C - библиотека \texttt{stdio.h}. Структура \texttt{FILE}. Функции \texttt{fopen} и \texttt{fclose}. Режимы открытия файлов (\texttt{"r"\quad "w"\quad "a"\quad "r+"\quad "w+"\quad "a+"}). Функции \texttt{fgetc}, \texttt{fputc}, \texttt{fscanf}, \texttt{fprintf}, \texttt{fgets}, \texttt{fputs}, \texttt{rewind} и \texttt{fseek}. Посимвольное чтение файла в цикле. Константа \texttt{EOF}. Чтение и запись в бинарные файлы. Функции \texttt{fread} и \texttt{fwrite}. Формат файла \texttt{.csv}. Форматы файлов для хранения изображений \texttt{.ppm} (текстовый и бинарный) и \texttt{.jpg}. Аргументы командной строки (\texttt{argc} и \texttt{argv}). Преобразование строки в число и обратно с помощью \texttt{sprintf} и \texttt{sscanf}.


\item \textbf{Этапы компиляции}\\
Что такое файл исходного кода и исполняемый файл. Этап компиляции: препроцессинг, компиляция и линковка. Директивы препроцессора \texttt{\#include}, \texttt{\#define}, \texttt{\#ifdef}, \texttt{\#else} и \texttt{\#endif}. Компиляция программы с помощью \texttt{gcc}. Опции \texttt{gcc: -E, -c, -S}.


\item \textbf{* Очередь с приоритетом. Двоичная куча}\\
Хранение бинарных деревьев в массиве. Структура данных - двоичная куча. Операции добавления элемента в кучу. Операция удаления максимального элемента из кучи. Вычислительные сложности этих операций. Абстрактный тип данных - очередь с приоритетом. Реализация очереди с приоритетом с помощью двоичной кучи. Пирамидальная сортировка.

\item \textbf{* Динамическое программирование}\\
Суть динамического программирования. Задача о вычислении чисел Фибоначчи. Задача о поиске подмассива с максимальной суммой. Задача о наибольшей общей подпоследовательности. Задача о рюкзаке. 

\end{enumerate}



\newpage
\section*{Материалы для подготовки:}
\begin{enumerate}
\item Кормен, Лейзерсон, Ривест. Алгоритмы: построение и анализ.\\ 
\hspace*{16pt} \href{https://lib.mipt.ru/book/16736/}{lib.mipt.ru/book/16736/}
\item Керниган Ритчи Язык программирования C \\
\hspace*{16pt} \href{https://lib.mipt.ru/book/266005/}{lib.mipt.ru/book/266005/}
\item Лекторий МФТИ: \\ \hspace*{16pt} 
\href{http://lectoriy.mipt.ru/course/ComputerTechnology-Informatics-14L#lectures}
{lectoriy.mipt.ru/course/ComputerTechnology-Informatics-14L\#lectures}
\item Фоксфорд\\
\href{https://www.youtube.com/watch?v=J-7XmpSUeQE&list=PL66kIi3dt8A5sa_qBur8uxmtuuwuJQGS1&index=28}
{www.youtube.com/watch?v=J-7XmpSUeQE\&list=PL66kIi3dt8A5sa\_qBur8uxmtuuwuJQGS1\&index=28}
\item Язык C и структуры данных в Йелле.\\ \hspace*{16pt} \href{http://www.cs.yale.edu/homes/aspnes/classes/223/notes.html}{www.cs.yale.edu/homes/aspnes/classes/223/notes.html}
\end{enumerate}

\subsubsection*{Дополнительные материалы по некоторым вопросам:}
\begin{itemize}

\item Сегменты памяти:
\begin{itemize}
\item \href{https://www.youtube.com/watch?v=_8-ht2AKyH4}{youtube.com/watch?v=\_8-ht2AKyH4}  (англ.)
\end{itemize}
\item Этапы компиляции:
\begin{itemize}
\item \href{https://www.youtube.com/watch?v=Je9FFb2zldk}{youtube.com/watch?v=Je9FFb2zldk}
\item \href{https://www.youtube.com/watch?v=ylA55D4B4_M}{youtube.com/watch?v=ylA55D4B4\_M}
\end{itemize}
\item Стек вызовов:
\begin{itemize}
\item \href{https://www.youtube.com/watch?v=hU3ONaqZzD8}{youtube.com/watch?v=hU3ONaqZzD8}
\end{itemize}


\end{itemize}
\end{document}