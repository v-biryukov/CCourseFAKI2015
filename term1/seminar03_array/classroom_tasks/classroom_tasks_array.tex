\documentclass{article}
\usepackage[utf8x]{inputenc}
\usepackage{ucs}
\usepackage{amsmath} 
\usepackage{amsfonts}
\usepackage{marvosym}
\usepackage{wasysym}
\usepackage{upgreek}
\usepackage[english,russian]{babel}
\usepackage{graphicx}
\usepackage{float}
\usepackage{textcomp}
\usepackage{hyperref}
\usepackage{geometry}
  \geometry{left=2cm}
  \geometry{right=1.5cm}
  \geometry{top=1cm}
  \geometry{bottom=2cm}
\usepackage{tikz}
\usepackage{ccaption}
\usepackage{multicol}
\usepackage{hyperref}


\usepackage{listings}
%\setlength{\columnsep}{1.5cm}
%\setlength{\columnseprule}{0.2pt}

\usepackage[absolute]{textpos}



\begin{document}
\pagenumbering{gobble}
\lstset{
  language=C,                % choose the language of the code
  basicstyle=\linespread{1.1}\ttfamily,
  columns=fixed,
  fontadjust=true,
  basewidth=0.5em,
  keywordstyle=\color{blue}\bfseries,
  commentstyle=\color{gray},
  stringstyle=\ttfamily\color{orange!50!black},
  showstringspaces=false,
  numbersep=5pt,
  numberstyle=\tiny\color{black},
  numberfirstline=true,
  stepnumber=1,                   % the step between two line-numbers.        
  numbersep=10pt,                  % how far the line-numbers are from the code
  backgroundcolor=\color{white},  % choose the background color. You must add \usepackage{color}
  showstringspaces=false,         % underline spaces within strings
  captionpos=b,                   % sets the caption-position to bottom
  breaklines=true,                % sets automatic line breaking
  breakatwhitespace=true,         % sets if automatic breaks should only happen at whitespace
  xleftmargin=.2in,
  extendedchars=\true,
  keepspaces = true,
}
\lstset{literate=%
   *{0}{{{\color{red!20!violet}0}}}1
    {1}{{{\color{red!20!violet}1}}}1
    {2}{{{\color{red!20!violet}2}}}1
    {3}{{{\color{red!20!violet}3}}}1
    {4}{{{\color{red!20!violet}4}}}1
    {5}{{{\color{red!20!violet}5}}}1
    {6}{{{\color{red!20!violet}6}}}1
    {7}{{{\color{red!20!violet}7}}}1
    {8}{{{\color{red!20!violet}8}}}1
    {9}{{{\color{red!20!violet}9}}}1
}


\section*{Массивы:}
\subsection*{Одномерные массивы}

\begin{lstlisting}
#include <stdio.h>
int main() {
	// Массив из 10 чисел типа int. Без инициализации ( значения могут быть какими угодно )
	int a[10];
	// Можно считать, что мы создали сразу 10 переменных типа int под именами
	// a[0], a[1], a[2], ... a[9]
	
	// Считаем числа в массив с помощью scanf
	for (int i = 0; i < 10; i++)
		scanf("%d", &a[i]);
	// Если вы знаете, что будет хранится в массиве, то можно сразу его инициализировать
	int b[10] = {4, 8, 15, 16, 23, 42}; // (оставшиеся 4 числа будут установлены нулями)
}
\end{lstlisting}
\begin{enumerate}
\item \textbf{Объявление:} Объявить следующие массивы и напечатайте их адрес и размер:
\begin{itemize}
\item массив из 10 элементов типа \texttt{char}
\item массив из 20 элементов типа \texttt{float}
\item массив из 30 элементов типа \texttt{unsigned long long}
\end{itemize}
\item \textbf{Инициализация:} Объявить массив под названием \texttt{days\_in\_month} с 12 элементами типа \texttt{int} и инициализировать его следующими значениями: 31, 28, 31, 30, 31, 30, 31, 31, 30, 31, 30, 31. Отдельной операцией изменить второе значение с массиве с 28 на 29.
\item \textbf{Печать:} Напечатать содержимое массива days\_in\_month на экран. Числа должны быть напечатаны в одну строку через пробел.


\subsection*{Передача массивов в функцию}
\begin{lstlisting}
// Массивы в функцию всегда передаются через указатель (int arr[] и int* arr одно и то же)
void print_array(int n, int* arr) {
	for (int i = 0; i < n; i++)
		printf("%d ", arr[i]);
	printf("\n");
}
// Следовательно, при изменении массива в функции, он меняется и вне функции
void square_array(int n, int arr[]) {
	for (int i = 0; i < n; i++)
		arr[i] = arr[i] * arr[i];
}
int main() {
	// Часто, мы не знаем размер массива заранее, поэтому берём с запасом 
	// (1000 элементов) и работаем только с n первыми элементами
	int a[1000];
	int n;
	scanf("%d", &n);
	for (int i = 0; i < n; i++)
		scanf("%d", &a[i]);
	print_array(n, a);
	square_array(n, a);
	print_array(n, a);
}
\end{lstlisting}
\item \textbf{Сумма массива:} Написать функцию \texttt{int sum(int n, int* arr)}, которая будет возвращать сумму массива целых чисел. Входные параметры такие же как и предыдущей задаче. Функция не должна ничего печатать и считывать. Использовать эту функцию в функции main() чтобы найти сумму чисел массива \texttt{days\_in\_month}.

\item \textbf{Умножение массива:} Написать функцию \texttt{void multiply(int n, int* arr, int k)}, которая будет умножать все элементы массива на число k. Протестируйте эту функцию в main().

\item \textbf{Индекс минимального элемента:} Написать функцию \texttt{int min\_index(int n, int* arr)}, которая будет возвращать индекс наименьшего числа в массиве. Входные параметры такие же как и предыдущей задаче. Функция не должна ничего печатать и считывать. Если в массиве есть несколько минимальных элементов, то функция должна вывести индекс первого из них.

\item \textbf{Сортировка выбором:} Написать функцию \texttt{void selection\_sort(int n, int* arr)}, которая будет сортировать массив методом выбора. \\
Алгоритм сортировки методом выбора для сортировки массива \texttt{[0:n]}:
\begin{itemize}
\item Находим индекс минимального элемента
\item Переставляем местами 0-й элемент с минимальным
\item Повторяем то же самое для подмассива \texttt{[1:n]} 
\end{itemize}
Протестируйте работу функции на массиве из хотя бы 30-ти элементов.

\item \textbf{Сортировка выбором по убыванию:} Написать функцию \texttt{void selection\_sort\_descend(int n, int* arr)}, которая будет сортировать массив методом выбора по убыванию.

\item \textbf{Реккурсивная сортировка выбором:} Написать функцию \texttt{void selection\_sort\_rec(int lo, int hi, int* arr)}, которая будет сортировать часть массива от индекса lo до hi (не всключая hi) методом выбора. Используйте реккурсию. \\


\subsection*{Вложенные циклы}
\begin{lstlisting}
for (int i = 0; i < n; i++)
	for (int j = 0; j < m; j++) {
		printf("%d : %d\n", i, j);
	}
\end{lstlisting}
\item \textbf{Таблица умножения:} Напечатать таблицу умножения с помощью вложенных циклов. Таблица должна быть квадратной, примерно такой, как печатают на тетрадках. Используйте модификатор \texttt{"\%5d"}.

\newpage

\subsection*{Двумерные циклы и массивы}
\begin{lstlisting}
// Зададим константу MAX = 100
#define MAX 100
void print_array(int n, int m, int arr[MAX][MAX]) {
	for (int i = 0; i < n; i++) {
		for (int j = 0; j < m; j++)
			printf("%d ", arr[i][j]);
		printf("\n");
	}
}
int main() {
    // Создаём массивы с запасом
	int a[MAX][MAX] = {{7, 7, 2}, {1, 8, 3}, {2, 1, 6}};
	int b[MAX][MAX] = {{5, 2, 9}, {-4, 2, 11}, {7, 1, -5}};
	// Печатаем только 9 элементов
	print_array(3, 3, a);
}
\end{lstlisting}
\item \textbf{Умножение двумерного массива на число:} Написать функцию \texttt{void multiply\_num(int n, int m, int arr[MAX][MAX], int x)}, которая будет умножать двумерный массив на число. Протестируйте работу функции в main().
\item \textbf{Сложение двумерных массивов:} Написать функцию \texttt{void sum(int n, int m, int A[MAX][MAX], int B[MAX][MAX], int C[MAX][MAX])}, которая будет складывать двумерные массивы и записывать их в массив \texttt{C}. Протестируйте работу функции в main().
\item \textbf{Перемножение двумерных массивов:} Написать функцию \texttt{void mult(int n, int m, int A[MAX][MAX], int B[MAX][MAX], int C[MAX][MAX])}, которая будет перемножать двумерные массивы и записывать их в массив \texttt{C}. Протестируйте работу функции в main().
\item \textbf{N-я степень:} Написать функцию \texttt{void power(int n, int m, int A[MAX][MAX], int N, int C[MAX][MAX])}, которая будет возводить матрицу A в степень N и записывать результат в массив \texttt{C}. Протестируйте работу функции в main().


\end{enumerate}

\end{document}