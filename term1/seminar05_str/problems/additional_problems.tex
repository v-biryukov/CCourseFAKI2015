\documentclass{article}
\usepackage[utf8x]{inputenc}
\usepackage{ucs}
\usepackage{amsmath} 
\usepackage{mathtext}
\usepackage{amsfonts}
\usepackage{upgreek}
\usepackage[english,russian]{babel}
\usepackage{graphicx}
\usepackage{float}
\usepackage{textcomp}
\usepackage{hyperref}
\usepackage{geometry}
  \geometry{left=2cm}
  \geometry{right=1.5cm}
  \geometry{top=1cm}
  \geometry{bottom=2cm}
\usepackage{tikz}
\usepackage{ccaption}



\begin{document}
\pagenumbering{gobble}

\section*{Дополнительные задачи на строки:}
\begin{enumerate}
\item \textbf{Сумма цифр в строке:} На вход подаётся строка. Нужно найти сумму всех цифр в этой строке. Например, если подаётся строка ``abc12def8gh'', то программа должна печатать 11.
\item \textbf{Сортировка строки:} Отсортировать символы строки по их коду ASCII. ``Something321'' $\rightarrow$ ``123Seghimnot''.
\item \textbf{Количество слов:} Написать функцию \textbf{int} word\_count(\textbf{char*} str), которая будет считать количество слов в строке. Слова могут быть разделены одним или несколькими пробелами. В этой задаче строка не считывается, а просто задаётся в функции main() и подаётся на вход функции word\_count().
\end{enumerate}

\end{document}