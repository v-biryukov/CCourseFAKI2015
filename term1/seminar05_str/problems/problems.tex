\documentclass{article}
\usepackage[utf8x]{inputenc}
\usepackage{ucs}
\usepackage{amsmath} 
\usepackage{mathtext}
\usepackage{amsfonts}
\usepackage{upgreek}
\usepackage[english,russian]{babel}
\usepackage{graphicx}
\usepackage{float}
\usepackage{textcomp}
\usepackage{hyperref}
\usepackage{geometry}
  \geometry{left=2cm}
  \geometry{right=1.5cm}
  \geometry{top=1cm}
  \geometry{bottom=2cm}
\usepackage{tikz}
\usepackage{ccaption}



\begin{document}
\pagenumbering{gobble}

\section*{Задачи на строки:}
\begin{enumerate}
\item \textbf{ASCII:} Вывести на экран все символы таблицы ASCII. Символы выводятся с помощью функции printf с опцией "\%c"
\item \textbf{Модификация строки:} Объявите строку которая может содержать до 50 символов. Проинициализировать её как ``a bunch of characters''. Отдельными командами изменить строку на ``a Bunch of Characters''
\item \textbf{Усечение строки:} На вход подаётся строка. Вывести на экран содержимое строки до первого символа запятой ``,''. Можно использовать только один вызов функции printf(). Строки выводятся/считываются с помощью функции printf()/scanf() с опцией "\%s"
\item \textbf{for\_6:} Решить задачу for\_6
\item \textbf{Обратная строка:} Считать строку, "перевернуть" и вывести на экран. То есть если на вход поступает строка ``example''  то вывести нужно ``elpmaxe''.
\item \textbf{Поиск подстроки:} Считать 2 строки и проверить является ли вторая строка подстрокой первой строки. Нужно использовать функцию char* strstr(const char* str, const char* substr). Вывести на экран YES или NO соответственно.
\item \textbf{Uppercase:} Написать функцию, которая будет поданную на вход строку переводить из lowercase в UPPERCASE.
\item \textbf{for\_14:} Решить задачу for\_14 . (Подсказка: Введите переменную-счётчик, которая будет увеличиваться, когда встречается открывающая скобка, и уменьшаться, когда встречается закрывающая)
\end{enumerate}


\end{document}