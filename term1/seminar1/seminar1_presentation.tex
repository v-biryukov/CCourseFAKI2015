\documentclass[14pt,pdf,hyperref={unicode}]{beamer}

% \documentclass[aspectratio=43]{beamer}
% \documentclass[aspectratio=1610]{beamer}

\usepackage{lmodern}

% подключаем кириллицу 
\usepackage[T2A]{fontenc}
\usepackage[utf8]{inputenc}
\usepackage{listings}
\usepackage{graphicx}
\usepackage{hyperref}

% отключить клавиши навигации
\setbeamertemplate{navigation symbols}{}

% тема оформления
\usetheme{CambridgeUS}

% цветовая схема
\usecolortheme{seahorse}

\definecolor{light-gray}{gray}{0.90}

\title{Семинар №1}   
\subtitle{ФАКИ 2015}
\author{Бирюков В. А.} 
\date{\today} 
% \logo{\includegraphics[height=5mm]{images/logo.png}\vspace{-7pt}}

\begin{document}

% титульный слайд
\begin{frame}
\titlepage
\end{frame} 

\section{Основы командной строки}
\begin{frame}
\begin{center}
\begin{beamercolorbox}[sep=8pt,center]{part
title}
\usebeamerfont{part title}\insertsection
\end{beamercolorbox}
\end{center}
\end{frame}

\begin{frame}
\frametitle{Основы командной строки} 
\framesubtitle{Основные команды}
\begin{description}
  \item[pwd (сокращение от personal working directory)] \hfill \\
  \item[ls (сокращение от list)] \hfill \\
  Опции: -l, -a
  \item[cd (change directory)] \hfill \\
  Применение: cd <имя директории> \\
  Особые директории: . ..  $\sim$
  \item[man (manual)] \hfill \\
  Применение: man <имя команды>\\
  Например: man ls
\end{description}
\end{frame}

\begin{frame}
\frametitle{Основы командной строки} 
\framesubtitle{Основные команды}
\begin{description}
  \item[cp (copy)] \hfill \\
  Применение: cp <источник> <назначение>
  \item[mv (move)] \hfill \\
  Применение: mv <источник> <назначение>\\
  Можно переименовывать файлы
\end{description}
\end{frame}

\begin{frame}
\frametitle{Основы командной строки} 
\framesubtitle{Основные команды}
\begin{description}
  \item[mkdir (make directory)] \hfill \\
  Применение: mkdir <название директории>
  \item[nano - текстовый редактор] \hfill \\
  Создание текстового файла.\\
  Ctrl + X - закрыть редактор\\
  Ctrl + O - сохранение файла
\end{description}
\end{frame}

\begin{frame}
\frametitle{Основы командной строки} 
\framesubtitle{Основные команды}
\begin{description}
  \item[rm (remove)] \hfill \\
  Применение: rm <имя файла> \\
  Чтобы удалить директорию: опция -r\\
  Будьте осторожны!
\end{description}
\end{frame}

\section{Компилятор gcc}
\begin{frame}
\begin{center}
\begin{beamercolorbox}[sep=8pt,center]{part
title}
\usebeamerfont{part title}\insertsection
\end{beamercolorbox}
\end{center}
\end{frame}

\begin{frame}
\frametitle{Пример простейшей программы на языке C}
\lstset{
  language=C,                % choose the language of the code
  numbers=none,                   % where to put the line-numbers
  stepnumber=1,                   % the step between two line-numbers.        
  numbersep=5pt,                  % how far the line-numbers are from the code
  backgroundcolor=\color{light-gray},  % choose the background color. You must add \usepackage{color}
  showspaces=false,               % show spaces adding particular underscores
  showstringspaces=false,         % underline spaces within strings
  showtabs=false,                 % show tabs within strings adding particular underscores
  tabsize=2,                      % sets default tabsize to 2 spaces
  captionpos=b,                   % sets the caption-position to bottom
  breaklines=true,                % sets automatic line breaking
  breakatwhitespace=true,         % sets if automatic breaks should only happen at whitespace
}

\lstinputlisting{./programms/first_programm.c}

Так выглядит синтаксис функции на C \\
Функция main -- начальная точка выполнения для всех С и C++ программ

\end{frame}


\begin{frame}
\frametitle{Пример программы helloworld}
\lstset{
  language=C,                % choose the language of the code
  numbers=none,                   % where to put the line-numbers
  stepnumber=1,                   % the step between two line-numbers.        
  numbersep=5pt,                  % how far the line-numbers are from the code
  backgroundcolor=\color{light-gray},  % choose the background color. You must add \usepackage{color}
  showspaces=false,               % show spaces adding particular underscores
  showstringspaces=false,         % underline spaces within strings
  showtabs=false,                 % show tabs within strings adding particular underscores
  tabsize=2,                      % sets default tabsize to 2 spaces
  captionpos=b,                   % sets the caption-position to bottom
  breaklines=true,                % sets automatic line breaking
  breakatwhitespace=true,         % sets if automatic breaks should only happen at whitespace
}

\lstinputlisting{./programms/hello_world.c}

\#include - директива препроцессора, всталяет содержимое указанного файла на своё место \\
printf (print formatted) -- функция для вывода \\
\textbackslash n - символ переноса строки

\end{frame}


\begin{frame}
\frametitle{Компилятор gcc}
\begin{description}
  \item[gcc (GNU Compiler Collection)] \hfill \\
  gcc <имя файла для компиляции>\\
  -o -- задать имя файла \\
  Имя файла по умолчанию: a.out \\
  Пример: \\
  \quad  gcc -o hello hello.c \\
  Скомпилирует файл hello.c и создаст исполняемый файл hello, который можно будет запустить исполнив ./hello
\end{description}
\end{frame}


\begin{frame}
\frametitle{Задание} 
\begin{center}
\begin{itemize}
\item Создать свою папку \\
\item Внутри этой папки создать текстовый файл с помощью nano\\
\item Написать простую программу \\
\item Скомпилировать её с помощью gcc
\end{itemize}
\end{center}
\end{frame}

\section{Основы языка C. Базовые типы и операторы.}
\begin{frame}
\begin{center}
\begin{beamercolorbox}[sep=8pt,center]{part
title}
\usebeamerfont{part title}\insertsection
\end{beamercolorbox}
\end{center}
\end{frame}

\begin{frame}
\frametitle{Особенности языка C:} 
\begin{center}
\begin{itemize}
\item Простой синтаксис
\item Малое количество зарезервированных слов
\item Простой доступ к памяти, указатели
\item Низкоуровневый
\item Очень быстрый
\item Небезопасный
\item Сложно писать большие программы
\end{itemize}
\end{center}
\end{frame}

\begin{frame}
\frametitle{Где язык C используется:} 
\begin{center}
\begin{itemize}
\item Системное программирование, операционные системы
\item Микроконтроллеры
\item Встраиваемые системы, драйвера и т.д.
\end{itemize}
\end{center}
\end{frame}

\begin{frame}
\frametitle{Переменные} 
\begin{center}
\begin{itemize}
\item В языке C все переменные нужно объявить перед использованием
\item При объявлении -- выделяется память под переменную
\item Области видимости переменной
\item Название переменной может содержать латинские буквы, цифры и \_
\item Название переменной не может начинаться с цифры
\end{itemize}
\end{center}
\end{frame}


\begin{frame}
\frametitle{Объявление переменных} 
\begin{center}
\begin{itemize}
\item Переменную нужно объявить перед использованием
\item Примеры объявления:\\
\textcolor{blue}{\textbf{int}} n;\\
\textcolor{blue}{\textbf{float}} p;
\item int -- целочисленный тип \\
\item float -- тип чисел с плавающей точкой
\end{itemize}
\end{center}
\end{frame}

\begin{frame}
\frametitle{Инициализация переменных} 
\begin{center}
\begin{itemize}
\item Переменные инициализируются с помощью оператора присваивания =
\item Примеры:\\
n = 3; \\
\textcolor{blue}{\textbf{float}} p = 5.4; \\
\textcolor{blue}{\textbf{int}} a, b, c = 9;
\end{itemize}
\end{center}
\end{frame}

\begin{frame}
\frametitle{Комментарии}
\lstset{
  language=C,                % choose the language of the code
  numbers=none,                   % where to put the line-numbers
  stepnumber=1,                   % the step between two line-numbers.        
  numbersep=5pt,                  % how far the line-numbers are from the code
  backgroundcolor=\color{light-gray},  % choose the background color. You must add \usepackage{color}
  showspaces=false,               % show spaces adding particular underscores
  showstringspaces=false,         % underline spaces within strings
  showtabs=false,                 % show tabs within strings adding particular underscores
  tabsize=2,                      % sets default tabsize to 2 spaces
  captionpos=b,                   % sets the caption-position to bottom
  breaklines=true,                % sets automatic line breaking
  breakatwhitespace=true,         % sets if automatic breaks should only happen at whitespace
}

\lstinputlisting{./programms/comments.c}

\end{frame}

\begin{frame}
\frametitle{Вывод в stdout. Функция printf.}
printf(строка форматирования, пер1, пер2, ...)
\begin{center}
\begin{tabular}{ l l l }
  Обозначение & Типы & Пример \\
  d или i & Целочисленные типы & 392 \\
  f & Типы с плавающей точкой & 392.5 \\
  e & Научная нотация & 3.9265e+2 \\
  c & Символ & a \\
  s & Строка & HelloMipt! \\
\end{tabular}
\end{center}

\end{frame}

\begin{frame}
\frametitle{Чтение из stdin. Функция scanf.}
scanf(строка форматирования, \&пер1, \&пер2, ...)
\begin{center}
\begin{tabular}{ l l l }
  Обозначение & Типы\\
  d или i & Целочисленные типы  \\
  f & Типы с плавающей точкой \\
  e & Научная нотация \\
  c & Символ\\
  s & Строка\\
\end{tabular}
\end{center}
\end{frame}

\begin{frame}
\frametitle{Базовые типы}
\frametitle{Целочисленные типы} 
\begin{center}
\includegraphics[scale=0.5]{bit_positions.png}
\end{center}
Число бит на тип зависит от компилятора. Обычные значения такие:
\begin{center}
\begin{tabular}{ l c l }
  Название типа & Число бит & Макс. значения \\
  char & 8 & 0..255 \\
  short & 16 & -32768..32767 \\
  int & 32 & $-2 \cdot 10^9$ ..$+2 \cdot 10^9$ \\
  long & 32& $-2 \cdot 10^9$ ..$+2 \cdot 10^9$ \\
  long long & 64 & $-2^{64}$ ..$+2^{64}-1$ \\
\end{tabular}
\end{center}
\end{frame}

\begin{frame}
\frametitle{Базовые типы}
\frametitle{Беззнаковые целочисленные типы} 
Число бит на тип зависит от компилятора. Обычные значения такие:
\begin{center}
\begin{tabular}{ l c l }
  Название типа & Число бит & Макс. значения \\
  unsigned short & 16 & 0..65535 \\
  unsigned int & 32 & $0$ ..$+4 \cdot 10^9$ \\
  unsigned long & 32& $0$ ..$+4 \cdot 10^9$ \\
  unsigned long long & 64 & $0$ ..$+2^{65}-1$ \\
\end{tabular}
\end{center}
sizeof() -- размер файла в байтах
\end{frame}

\begin{frame}
\frametitle{Базовые типы}
\frametitle{Тип bool} 
\begin{itemize}

\item Принимает значения true(1) или false(0) \\
\item Занимает 1 байт\\
\item 0 соответствует false\\
\item все остальные значения соответсвуют true

\end{itemize}
\end{frame}

\begin{frame}
\frametitle{Базовые типы}
\frametitle{Типы чисел с плавающей точкой} 
\includegraphics[scale=0.6]{floats.png}
\begin{center}
\begin{tabular}{ l c l }
  Название типа & Число бит & Макс. значения \\
  float & 32 & $10^{-38}$..$10^{+38}$ \\
  double & 64 & $10^{-308}$..$10^{+308}$ \\
\end{tabular}
\end{center}
Обычно используется double, так как float может недостаточно точен
\end{frame}

\begin{frame}
\frametitle{Базовые операторы}
\frametitle{Оператор присваивания =} 
Присваивает переменной значение:\\
Пример:\\
\quad a = 1;\\
\quad b = a + 1;\\
\quad \textcolor{blue}{float} c = 5.6;\\
\quad \textcolor{blue}{float} d = 19;\\

\end{frame}


\begin{frame}
\frametitle{Базовые операторы}
\frametitle{Математические операторы: + - / * \%} 
Примеры:\\
\quad a = 1 + 1; \\
\quad b = 5.0 / 2.0;\\
\quad c = 5 / 2;\\
\quad d = 5 \% 2;\\
\end{frame}

\begin{frame}
\frametitle{Базовые операторы}
\frametitle{Унарные операторы: + - ++ - -}
Оператор инкремента ++ -- увеличивает значение переменной на 1 и присваевает переменной  \\
Примеры:\\
\quad a = +5; \\
\quad b = -a;\\
\quad c = ++a;\\
\quad d = c++;\\
\end{frame}

\begin{frame}
\frametitle{Базовые операторы}
\frametitle{Операторы отношения}
Возвращают тип bool
\begin{center}
\begin{tabular}{ l l}
  == & равно \\
  != & не равно \\
  > & больше \\
  >= & больше или равно \\
  < & меньше \\
  <= & меньше или равно \\
\end{tabular}
\end{center}

Не путайте оператор присваивания = и оператор сравнения == !! 
\end{frame}

\begin{frame}
\frametitle{Базовые операторы}
\frametitle{Логические операторы}
Возвращают тип bool
\begin{center}
\begin{tabular}{ l l}
  ! & не \\
  || & или \\
  \&\& & и \\
\end{tabular}
\end{center}
\end{frame}

\begin{frame}
\frametitle{Базовые операторы}
\frametitle{Побитовые операторы}
\begin{center}
\includegraphics[scale=0.5]{bitwise_or.png}
\end{center}
\begin{center}
\begin{tabular}{ l l}
  $\sim$ & побитовое не \\
  | & побитовое или \\
  \& & побитовое и \\
  \^{} & побитовое исключающее или \\
\end{tabular}
\end{center}
\end{frame}

\begin{frame}
\frametitle{Базовые операторы}
\frametitle{Другие операторы присваивания}
\begin{center}
\begin{tabular}{ l l}
  += & a += b тоже что и a = a + b  \\
  -= & a -= b тоже что и a = a - b  \\
  *= & a *= b тоже что и a = a * b  \\
  |= & a |= b тоже что и a = a | b  \\
  и другие
\end{tabular}
\end{center}
\end{frame}

\begin{frame}
\frametitle{Базовые операторы}
\frametitle{Приоритет операторов}
\begin{center}
\begin{enumerate}
\item (), []
\item ++, --, +, -(унарные), sizeof
\item *, /, \%
\item +, -
\item >,<,<=,>=
\item ==, !=
\item \&, |, \&\&, ||
\item =, +=, и т.д.
\end{enumerate}
\end{center}
Приоритет операторов C подробнее:\\
\href{http://ru.cppreference.com/w/c/language/operator_precedence}
{\textcolor{red}{ru.cppreference.com/w/c/language/operator\_precedence}}
\end{frame}

\begin{frame}
\frametitle{Задание 1} 
\framesubtitle{Конвертер температуры из Фаренгейт в Цельсии} 
\begin{center}
\begin{itemize}
\item Создать переменную типа double \\
\item Считать её значение из стандартного ввода с помощью scanf\\
\item Создать новую переменную типа double и записать преобразованное выражение \\
\item $T_C = \frac{5}{9}(T_F-32)$\\
\item Вывести значение в стандартный вывод с помощью printf\\
\item Скомпилировать программу с помощью gcc и запустить
\end{itemize}
\end{center}
\end{frame}

\begin{frame}
\frametitle{Задание 2} 
\framesubtitle{Знаки после запятой} 
\begin{center}
\begin{itemize}
\item Пользователь вводит вещественное число $x$ и целое число $n$ \\
\item Нужно вывести $x^5$ c $n$ знаками после запятой\\
\end{itemize}
\end{center}
\begin{center}
\includegraphics[scale=0.5]{Printf_illustration.png}
\end{center}
\end{frame}

\begin{frame}
\frametitle{Задание 3} 
\framesubtitle{Площадь треугольника} 
\begin{center}
\begin{itemize}
\item Пользователь вводит три пары вещественных чисел -- координаты точек треугольника\\
\item Нужно вывести площадь этого треугольника\\
\item Площадь треугольника можно найти так: $S = \frac{1}{2}|[\vec{r_1} \vec{r_2}]|$\\
\item Где $\vec{r_1}$ и $\vec{r_2}$ -- вектора соответствующие сторонам треугольника.
\item Могут понадобиться функции abs() из stdlib.h или sin() и sqrt() из math.h
\end{itemize}
\end{center}
\end{frame}

\begin{frame}
\frametitle{Задание 4} 
\framesubtitle{Индекс символа} 
\begin{center}
\begin{itemize}
\item Пользователь вводит символ\\
\item Нужно напечатать номер этого символа в кодировке ASCII\\
\item Также нужно напечатать символы и индексы соседних символов в кодировке ASCII \\
\end{itemize}
\end{center}
\end{frame}


\end{document}
