\documentclass{article}
\usepackage[utf8x]{inputenc}
\usepackage{ucs}
\usepackage{amsmath} 
\usepackage{amsfonts}
\usepackage{upgreek}
\usepackage[english,russian]{babel}
\usepackage{graphicx}
\usepackage{float}
\usepackage{textcomp}
\usepackage{hyperref}
\usepackage{geometry}
  \geometry{left=2cm}
  \geometry{right=1.5cm}
  \geometry{top=1cm}
  \geometry{bottom=2cm}
\usepackage{tikz}
\usepackage{ccaption}
\usepackage{multicol}


\usepackage{listings}


\begin{document}
\pagestyle{plain}
\lstset{
  language=C,                % choose the language of the code
  basicstyle=\linespread{1.1}\ttfamily,
  columns=fixed,
  fontadjust=true,
  basewidth=0.5em,
  keywordstyle=\color{blue}\bfseries,
  commentstyle=\color{gray},
  stringstyle=\ttfamily\color{orange!50!black},
  showstringspaces=false,
  numbersep=5pt,
  numberstyle=\tiny\color{black},
  numberfirstline=true,
  stepnumber=1,                   % the step between two line-numbers.        
  numbersep=10pt,                  % how far the line-numbers are from the code
  backgroundcolor=\color{white},  % choose the background color. You must add \usepackage{color}
  showstringspaces=false,         % underline spaces within strings
  captionpos=b,                   % sets the caption-position to bottom
  breaklines=true,                % sets automatic line breaking
  breakatwhitespace=true,         % sets if automatic breaks should only happen at whitespace
  xleftmargin=.2in,
  extendedchars=\true,
  keepspaces = true,
}
\lstset{literate=%
   *{0}{{{\color{red!20!violet}0}}}1
    {1}{{{\color{red!20!violet}1}}}1
    {2}{{{\color{red!20!violet}2}}}1
    {3}{{{\color{red!20!violet}3}}}1
    {4}{{{\color{red!20!violet}4}}}1
    {5}{{{\color{red!20!violet}5}}}1
    {6}{{{\color{red!20!violet}6}}}1
    {7}{{{\color{red!20!violet}7}}}1
    {8}{{{\color{red!20!violet}8}}}1
    {9}{{{\color{red!20!violet}9}}}1
}
\renewcommand{\thesubsection}{\arabic{subsection}}
\makeatletter
\def\@seccntformat#1{\@ifundefined{#1@cntformat}%
   {\csname the#1\endcsname\quad}%    default
   {\csname #1@cntformat\endcsname}}% enable individual control
\newcommand\section@cntformat{}     % section level 
\newcommand\subsection@cntformat{Задача \thesubsection.\space} % subsection level
\newcommand\subsubsection@cntformat{\thesubsubsection.\space} % subsubsection level
\makeatother


\title{Семинар \#7: Структуры. Домашнее задание.\vspace{-5ex}}\date{}\maketitle

\subsection{Треугольник}
Для описания треугольников на плоскости были определены структуры \texttt{Point} и \texttt{Triangle}:
\begin{lstlisting}
struct point 
{
    double x, y;
};
typedef struct point Point;

struct triangle 
{
    Point a, b, c;
};
typedef struct triangle Triangle;
\end{lstlisting}
Напишите следующие функции для работы с этими структурами:
\begin{itemize}
\item Функцию \texttt{print\_point}, которая будет принимать точку и печатать её в формате \texttt{(1.23, 4.56)}. То есть в круглых скобках, через запятую и с двумя знаками после запятой.

\item Функцию \texttt{print\_triangle}, которая будет принимать на вход треугольник и печатать координаты треугольника в следующем формате: \texttt{\{(1.00, 0.00), (0.50, 2.00), (0.00, 1.50)\}}.

\item Функцию \texttt{distance}, которая будет принимать на вход 2 точки и возвращать расстояние между ними.

\item Функцию \texttt{get\_triangle\_perimeter}, которая будет принимать треугольник по константному указателю и возвращать его периметр.

\item Функцию \texttt{get\_triangle\_area}, которая будет принимать на вход треугольник по константному указателю и возвращать его площадь. Можно использовать формулу Герона.

\item Функцию \texttt{moved\_triangle}, которая будет принимать на вход треугольник по константному указателю и одну точку (она будет играть роль вектора-перемещения). Функция должна возвращать новый треугольник, у которого все координаты будут передвинуты на вектор-перемещение.

\item Функцию \texttt{move\_triangle}, которая будет принимать на вход треугольник по указателю и одну точку (она будет играть роль вектора-перемещения). Функция должна менять передаваемый ей треугольник.
\end{itemize}

\subsection{Рецензии на компьютерные игры}
На вход программе приходит информация о рецензиях компьютерных игр. В первой строке содержится число \texttt{n} - количество игр. Далее идут \texttt{n} строк. В каждой строке содержится название игры, заканчивающееся двоеточием сразу после идёт целое число \texttt{k} -- количество оценок, которые эта игра получила, затем идут \texttt{k} оценок. Оценка, это число от 1 до 10. Нужно отсортировать все игры по средней оценке и напечатать название игр и их среднюю оценку.

\begin{center}
\begin{tabular}{ l | l }
 вход & выход \\ \hline
 \texttt{5} & 								  \texttt{The Cube, 8.286} \\
 \texttt{Need For Speed: 6 6 1 2 7 5 4} &      \texttt{Metal Power, 5.900 } \\
 \texttt{Sector: 3 1 4 2} & 					  \texttt{Principle Of Chaos 2, 5.200} \\
 \texttt{The Cube: 7 9 8 7 9 8 10 7} &          \texttt{Need For Speed, 4.667 } \\
 \texttt{Principle Of Chaos 2: 5 4 3 6 5 7} &     \texttt{Sector, 2.333} \\
 \texttt{Metal Power: 10 8 5 3 9 6 2 6 7 5 8} & \\
\end{tabular}
\end{center}
Протестировать программу можно на файле \texttt{videogames.txt}.


\newpage

\subsection{Создание указателей}
Решения всех подзадач этой части -- одна строка. Результат выполнения задания -- \texttt{.txt} файл, который содержит все эти строки.

\begin{enumerate}
\item В следующей программе создаётся структура \texttt{Book} из семинара про структуры:
\begin{lstlisting}
struct book 
{
    char title[50];
    int pages;
    float price;
};
typedef struct book Book;

int main() 
{
    Book b = {"Fahrenheit 451", 400, 700.0};

}
\end{lstlisting}

\begin{enumerate}
\item Создайте указатель \texttt{pb} и сделайте так, чтобы он указывал на структуру \texttt{b}.
\item Создайте указатель \texttt{pprice} и сделайте так, чтобы он указывал на поле \texttt{price} структуры \texttt{b}.
\item Создайте указатель \texttt{pc} и сделайте так, чтобы он указывал символ \texttt{'t'} поля \texttt{title} структуры \texttt{b}.
\end{enumerate} 


\item В следующей программе создаётся переменная \texttt{a} типа \texttt{float} и \texttt{p} указатель, который хранит её адрес:
\begin{lstlisting}
int main() 
{
    float a = 1.2;
    float* p = &a;
    // Тут нужно написать 1 строку кода
}
\end{lstlisting}
Создайте указатель \texttt{pp} и сделайте так, чтобы он указывал на указатель \texttt{p}.


\item В следующей программе создаётся структура \texttt{Book} из семинара про структуры и указатель на неё:
\begin{lstlisting}
struct book 
{
    char title[50];
    int pages;
    float price;
};
typedef struct book Book;

int main() 
{
    Book b = {"Fahrenheit 451", 400, 700.0};
    Book* pb = &b;
    // Тут нужно написать 1 строку кода
}
\end{lstlisting}
Создайте указатель \texttt{ppb} и сделайте так, чтобы он указывал на указатель \texttt{pb}.


\end{enumerate}



\subsection{Использование указателей}
Решения всех подзадач этой части -- одна строка. Результат выполнения задания -- \texttt{.txt} файл, который содержит все эти строки.
\begin{enumerate}
\item В следующей программе была создана структура \texttt{a} типа \texttt{Date} и указатель на эту структуру. Добавьте \texttt{1} к значению поля \texttt{year}, используя только указатель \texttt{p}.
\begin{multicols}{2}
\begin{lstlisting}
#include <stdio.h>
struct date 
{
    int day, month, year;
};
typedef struct date Date;
int main() 
{
    Date a = {15, 5, 1970};
    Date* p = &a;
    // Тут нужно написать 1 строку кода
    printf("%d %d %d\n", 
           a.day, a.month, a.year);
}
\end{lstlisting}

\vfill \null    
\columnbreak
\vfill \null 

\includegraphics[scale=1]{../images/pointer_schemes/pointer_to_struct_date.png}
\end{multicols}

\item В следующей программе была создана структура \texttt{a} типа \texttt{Movie} и указатель на неё.
\begin{multicols}{2}
\begin{lstlisting}
#include <stdio.h>
struct date 
{
    int day, month, year;
};
typedef struct date Date;

struct movie 
{
    char title[50];
    float rating;
    Date release_date;
};
typedef struct movie Movie;
\end{lstlisting}

\vfill \null  
\columnbreak
\vfill \null  

\begin{center}
\includegraphics[scale=1]{../images/pointer_schemes/pointer_to_struct_movie.png}
\end{center}
\end{multicols}

\vspace{-10ex}
\begin{lstlisting}
int main() 
{
    Movie a = {"Inception", 8.661, {8, 6, 2010}};
    Movie* p = &a;
    // Тут нужно написать 1 строку кода
}
\end{lstlisting}
\begin{enumerate}
\item Увеличьте на \texttt{1} значение поля \texttt{rating}, используя только указатель \texttt{p}.
\item Увеличьте на \texttt{1} значение поля месяца выхода фильма, используя только указатель \texttt{p}.
\end{enumerate}

\newpage
\item В следующей программе был создан массив \texttt{array} из структур типа \texttt{Movie} и указатель \texttt{p}, который указывает на второй элемент массива (\texttt{array[1]}).
\begin{lstlisting}
#include <stdio.h>
struct date 
{
    int day, month, year;
};
typedef struct date Date;

struct movie 
{
    char title[50];
    float rating;
    struct date release_date;
};
typedef struct movie Movie;

int main() 
{
    Movie array[3] = {{"Inception", 8.661, {8, 6, 2010}}, 
                      {"Green Mile", 9.062, {6, 12, 1999}}, 
                      {"Leon", 8.679, {14, 9, 1994}}};
    Movie* p = &array[1];
}
\end{lstlisting}

\vspace{-59ex}
\begin{center}
\quad\quad\quad\quad\quad\quad\quad\quad\quad\quad\quad\quad\quad\quad\quad\quad\quad\quad\quad\quad\quad\quad\quad
\includegraphics[scale=1]{../images/pointer_schemes/pointer_to_array_of_struct_movie.png}
\end{center}

\begin{enumerate}
\item Увеличьте на \texttt{1} значение рейтинга фильма \texttt{Inception}, используя только указатель \texttt{p}. При этом менять \texttt{p} нельзя, он  должен указывать на \texttt{array[1]}.
\item Удвойте значение года выхода фильма \texttt{Leon}, используя только указатель \texttt{p}. При этом менять \texttt{p} нельзя, он  должен указывать на \texttt{array[1]}.
\end{enumerate}

\end{enumerate}

\newpage
\subsection{Передача в функцию по указателю}
\subsection*{Передача в функцию по значению}
\begin{multicols}{2}
\begin{lstlisting}
#include <stdio.h>
struct movie 
{
    char title[50];
    float rating;
    struct date release_date;
};
typedef struct movie Movie;

void change_rating(Movie m) 
{
    m.rating += 1;
}
int main() 
{
    Movie a = {"Inception", 8.661, 
                {8, 6, 2010}};
    change_rating(a);
}
\end{lstlisting}
\columnbreak
\begin{center}
\includegraphics[scale=0.8]{../images/pointer_schemes/function_by_value.png}
\end{center}
\end{multicols}
Всё, что передаётся в функцию, копируется (кроме массивов). Поэтому функция \texttt{change\_rating} будет менять
поле \texttt{rating} у копии структуры \texttt{a}, а изначальная структура не изменится.

\subsection*{Передача в функцию по указателю:}
\begin{multicols}{2}
\begin{lstlisting}
#include <stdio.h>
struct movie 
{
    char title[50];
    float rating;
    struct date release_date;
};
typedef struct movie Movie;

void change_rating(Movie* pm) 
{
    pm->rating += 1;
}
int main() 
{
    Movie a = {"Inception", 8.661, 
              {8, 6, 2010}};
    Movie* p = &a;
    change_rating(&a);
}
\end{lstlisting}
\columnbreak
\begin{center}
\includegraphics[scale=0.8]{../images/pointer_schemes/function_by_pointer.png}
\end{center}
\end{multicols}
Теперь в функцию копируется указатель, который содержит
адрес стурктуры \texttt{a}. Используя этот указатель, мы можем изменить изначальную структуру. Более того, так как указатель занимает меньше памяти, его копирования в функцию происходит быстрее, чем копирование всей структуры.

\subsection*{Подзадачи:}
\begin{enumerate}

\item Напишите функцию \texttt{void increase\_rating(Movie* p)}, которая будет принимать указатель типа \texttt{Movie*} и увеличивать рейтинг фильма, на которой указывает \texttt{p}, на 1.

\item Напишите функцию \texttt{void change\_year\_of\_movies(Movie* p, int size)}, которая принимает на вход указатель на первый элемент массива структур типа \texttt{Movie} и размер этого массива. Функция должна увеличивать год выхода всех фильмов на \texttt{1}. Протестируйте функции, вызвав их из функции \texttt{main} с помощью следующего кода:

\begin{lstlisting}
#include <stdio.h>
struct date 
{
    int day, month, year;
};
typedef struct date Date;
struct movie 
{
    char title[50];
    float rating;
    struct date release_date;
};
typedef struct movie Movie;

void print_date(const Date* pd) 
{
    printf("%02d.%02d.%04d", pd->day, pd->month, pd->year);
}
void print_movie(const Movie* pm) 
{
    printf("Title: %s\nRating: %.2f\nDate: ", pm->title, pm->rating);
    print_date(&pm->release_date);
    printf("\n");
}
// Тут вам нужно написать функции increase_rating и change_year_of_movies

int main() 
{
    Movie a[3] = {{"Inception", 8.661, {8, 6, 2010}}, 
                  {"Green Mile", 9.062, {6, 12, 1999}}, 
                  {"Leon", 8.679, {14, 9, 1994}}};
    increase_rating()
    change_year_of_movies(a, 3);
    for (int i = 0; i < 3; ++i)
        print_movie(&a[i]);
}
\end{lstlisting}

\end{enumerate}
\end{document}