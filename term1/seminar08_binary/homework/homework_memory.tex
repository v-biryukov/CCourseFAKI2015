\documentclass{article}
\usepackage[utf8x]{inputenc}
\usepackage{ucs}
\usepackage{amsmath} 
\usepackage{amsfonts}
\usepackage{marvosym}
\usepackage{wasysym}
\usepackage{upgreek}
\usepackage[english,russian]{babel}
\usepackage{graphicx}
\usepackage{float}
\usepackage{textcomp}
\usepackage{hyperref}
\usepackage{geometry}
  \geometry{left=2cm}
  \geometry{right=1.5cm}
  \geometry{top=1cm}
  \geometry{bottom=2cm}
\usepackage{tikz}
\usepackage{ccaption}
\usepackage{multicol}

\hypersetup{
   colorlinks=true,
   citecolor=blue,
   linkcolor=black,
   urlcolor=blue
}

\usepackage{listings}
%\setlength{\columnsep}{1.5cm}
%\setlength{\columnseprule}{0.2pt}

\usepackage[absolute]{textpos}

\usepackage{colortbl,graphicx,tikz}
\definecolor{X}{rgb}{.5,.5,.5}


\begin{document}
\pagenumbering{gobble}
\lstset{
  language=C,                % choose the language of the code
  basicstyle=\linespread{1.1}\ttfamily,
  columns=fixed,
  fontadjust=true,
  basewidth=0.5em,
  keywordstyle=\color{blue}\bfseries,
  commentstyle=\color{gray},
  stringstyle=\ttfamily\color{orange!50!black},
  showstringspaces=false,
  numbersep=5pt,
  numberstyle=\tiny\color{black},
  numberfirstline=true,
  stepnumber=1,                   % the step between two line-numbers.        
  numbersep=10pt,                  % how far the line-numbers are from the code
  backgroundcolor=\color{white},  % choose the background color. You must add \usepackage{color}
  showstringspaces=false,         % underline spaces within strings
  captionpos=b,                   % sets the caption-position to bottom
  breaklines=true,                % sets automatic line breaking
  breakatwhitespace=true,         % sets if automatic breaks should only happen at whitespace
  xleftmargin=.2in,
  extendedchars=\true,
  keepspaces = true,
}
\lstset{literate=%
   *{0}{{{\color{red!20!violet}0}}}1
    {1}{{{\color{red!20!violet}1}}}1
    {2}{{{\color{red!20!violet}2}}}1
    {3}{{{\color{red!20!violet}3}}}1
    {4}{{{\color{red!20!violet}4}}}1
    {5}{{{\color{red!20!violet}5}}}1
    {6}{{{\color{red!20!violet}6}}}1
    {7}{{{\color{red!20!violet}7}}}1
    {8}{{{\color{red!20!violet}8}}}1
    {9}{{{\color{red!20!violet}9}}}1
}

\renewcommand{\thesubsection}{\arabic{subsection}}
\makeatletter
\def\@seccntformat#1{\@ifundefined{#1@cntformat}%
   {\csname the#1\endcsname\quad}%    default
   {\csname #1@cntformat\endcsname}}% enable individual control
\newcommand\section@cntformat{}     % section level 
\newcommand\subsection@cntformat{Задача \thesubsection.\space} % subsection level
\newcommand\subsubsection@cntformat{\thesubsubsection.\space} % subsubsection level
\makeatother

\title{Семинар \#8: Память. Домашнее задание.\vspace{-5ex}}\date{}\maketitle
\subsection{Печать разных типов:}
Напишите функцию \texttt{void polyprint(const char* type, void* p)}, которая должна будет печатать то, на что указывает указатель \texttt{p}. Тип того, на что указывает \texttt{p}, задаётся с помощью первой переменной и может принимать следующие значения:
\begin{itemize}
\item Если \texttt{type == "Integer"}, то \texttt{p} указывает на целое число типа \texttt{int}.
\item Если \texttt{type == "Float"}, то \texttt{p} указывает на вещественное число типа \texttt{float}.
\item Если \texttt{type == "Character"}, то \texttt{p} указывает на символ (тип \texttt{char}).
\item Если \texttt{type == "Date"}, то \texttt{p} указывает на структуру \texttt{Date} (определение этой структуры смотрите выше).
\item Если \texttt{type == "Movie"}, то \texttt{p} указывает на структуру \texttt{Movie} (определение этой структуры смотрите выше).
\item Если \texttt{type == "String"}, то \texttt{p} указывает на первый символ строки.
\item Если \texttt{type == "IntegerArray 15"}, то \texttt{p} указывает на первый элемент массива размером 15. Элементы этого массива имеют тип \texttt{int}. Нужно распечатать все элементы через пробел. Тут нужно использовать функцию \texttt{sscanf}, для того чтобы распарсить строку \texttt{type}.
\item В ином случае функция должна печатать \texttt{Error!}
\end{itemize}
В любом случае, в конце функция должна печатать символ перехода на новую строку. Для сравнения строк нужно пользоваться функцией \texttt{strcmp}. Протестируйте функцию с помощью следующего кода:

\begin{lstlisting}
#include <stdio.h>
struct date 
{
    int day, month, year;
};
typedef struct date Date;
struct movie 
{
    char title[50];
    float rating;
    struct date release_date;
};
typedef struct movie Movie;
// Тут нужно написать функцию polyprint

int main() 
{
    int a = 123;
    polyprint("Integer", &a);
    float b = 1.5;
    polyprint("Float", &b);
    char c = 'T';
    polyprint("Character", &c);
    
    Date d = {15, 5, 1970};
    polyprint("Date", &d);
    Movie e = {"Inception", 8.661, {8, 6, 2010}};
    polyprint("Movie", &e);
    
    char f[] = "Sapere Aude";
    polyprint("String", f);
    int g[] = {10, 20, 30, 40, 50};
    polyprint("IngerArray 5", g);
}
\end{lstlisting}

\subsection{Изменить символы:}
Напишите функцию \texttt{void set\_characters(char* begin, char* end, char c)}, которая задаёт символы в строке символом \texttt{c}. Начиная с символа, на который указывает \texttt{begin} и заканчивая символом на который указывает \texttt{end} (но не включая его). Гарантируется, что \texttt{end} указывает на символ, находящийся в этой же строке и не левее символа, на который указывает \texttt{begin}. Протестируйте функцию с помощью следующего кода:
\begin{lstlisting}
#include <stdio.h>
// Тут нужно написать функции set_characters

int main() 
{
    char s[] = "Sapere Aude";
    set_characters(&s[2], &s[8], 'b');
    printf("%s\n", s); // Должно напечатать Sabbbbbbude
    set_characters(s, &s[4], 'a');
    printf("%s\n", s); // Должно напечатать aaaabbbbude
}
\end{lstlisting}


\subsection{Просмотр памяти:}
Как выглядит память, инициализируемая при создании следующих переменных (в системе с порядком байт Little Endian):
\begin{multicols}{2}
\begin{itemize}
\item \texttt{int a = 0x11223344;}
\item \texttt{int b = 65535;}
\item \texttt{int c = -1;}
\item \texttt{int array[3] = \{10, 2000, 65535}\};
\item \texttt{char str[8] = '{}'Hello'{}'};
\item \texttt{float x = 1.0f};
\item
\begin{verbatim}
struct data {
    char str[5];
    int number;
};
struct data c = {"Cat", 100000};
\end{verbatim}
\end{itemize}
\end{multicols}
Память представить в виде последовательности 2-значных шестнадцатеричных чисел. Например число \\
$123456 = 1e240_{16}$ будет храниться в памяти как \texttt{40 E2 01 00}. \\

\textit{Подсказка:} Чтобы проверить, как будет выглядеть память, можно создать указатель типа \texttt{char*} на эту память и распечатать каждый байт в виде шестнадцатеричного числа.





\end{document}
