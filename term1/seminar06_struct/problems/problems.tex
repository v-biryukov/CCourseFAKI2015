\documentclass{article}
\usepackage[utf8x]{inputenc}
\usepackage{ucs}
\usepackage{amsmath} 
\usepackage{mathtext}
\usepackage{amsfonts}
\usepackage{upgreek}
\usepackage[english,russian]{babel}
\usepackage{graphicx}
\usepackage{float}
\usepackage{textcomp}
\usepackage{hyperref}
\usepackage{geometry}
  \geometry{left=2cm}
  \geometry{right=1.5cm}
  \geometry{top=1cm}
  \geometry{bottom=2cm}
\usepackage{tikz}
\usepackage{ccaption}
\usepackage{multicol}
\setlength{\columnsep}{1.5cm}
\setlength{\columnseprule}{0.2pt}


\begin{document}
\pagenumbering{gobble}

\newpage
\subsection*{Справочная информация по указателям:}
Каждая переменная в языке C хранится где-то в памяти и имеет адрес. Адрес переменной это просто номер первого байта соответствующей области памяти. Чтобы получить адрес переменной нужно перед переменной поставить \&(амперсанд).
Указатель это переменная, которая хранит адреса переменных. Тип указателя такой: <тип переменной>*. Пример:
\begin{verbatim}
int a = 42;  // Переменная, которая хранит число 42
int* p = &a; // Указатель, который будет хранить адрес переменной a
\end{verbatim}
Чтобы доступиться к переменной по указателю нужно поставить символ * перед указателем:
\begin{verbatim}
*p = *p + 10;
printf("%d", a);  // Напечатает 52
printf("%d", *p); // Напечатает 52
\end{verbatim}
Указатели часто используются чтобы изменять передаваемые значения в функциях:
\begin{multicols}{2}
\begin{verbatim}
// Неправильно:
void normalize(float x, float y)
{
    float sum = x + y;
    x = x / sum;
    y = y / sum; 
    // Изменятся x и y - копии a и b
}
// ...
float a = 20.0, b = 80.0;
normalize(a, b);
// a и b не изменятся: a=20.0, b=80.0
\end{verbatim}
\begin{verbatim}
// Правильно:
void normalize(float* x, float* y)
{
    float sum = *x + *y;
    *x = *x / sum;
    *y = *y / sum; 
    // Изменятся переменные a и b
}
// ...
float a = 20.0, b = 80.0;
normalize(&a, &b);
// a и b изменятся:a=0.2, b=0.8
\end{verbatim}
\end{multicols}

\subsection*{Задачи:}
\begin{enumerate}
\item \textbf{Работа с указателями}
	\begin{enumerate}
	\item Объявить переменную типа float и инициализировать её какими-либо значениями
	\item Напечатать значение и адрес переменной, используя эту переменную (чтобы напечатать адрес используйте спецификатор \%p)
	\item Объявить указатель типа float* и присвоить ему адрес переменной
	\item Напечатать значение и адрес переменной, используя только указатель
	\item Изменить значение переменной используя только указатель и напечатать это значение
	\end{enumerate}

\item \textbf{Modify1:} Написать функцию \textbf{void add10(int* p)}, которая добавляет 10 к переменной типа int. Используйте эту функцию в функции main() следующим образом:
\begin{verbatim}
int a = 50;
add10(&a);
printf("%d\n", a);
\end{verbatim}
\item \textbf{Cube:} Написать функцию \textbf{double cube(double p)}, которая возвращает куб числа. Используйте эту функцию в функции main() чтобы возвести в куб переменную типа double.
\item \textbf{Modify2:} Написать функцию \textbf{void cube(double* p)}, которая возводит значение переменной типа double в куб, используя указатель на эту переменную. Используйте эту функцию в функции main() чтобы возвести в куб переменную типа double.
\item \textbf{Swap:} Написать функцию \textbf{swap}, которая меняет значения 2-х переменных типа int местами. Используйте эту функцию в функции main().
\item \textbf{Умножение массива на число:} Написать функцию \textbf{mult\_array}, которая принимает на вход массив и некоторое число и умножает все элементы массива на это число.
\end{enumerate}

\newpage
\subsection*{Справочная информация по структурам:}
\begin{multicols}{2}
\begin{verbatim}
// Описываем структуру книги
struct book
{
    char title[50];
    int pages;
    float price;
};
// Чтобы писать Book вместо struct book
typedef struct book Book;

// Функция для печати информации о книге
// Происходит передача по значению
// Используется оператор . (точка)
void print_book_info(Book b)
{
    printf("\nBook info:\n");
    printf("Title: %s\n", b.title);
    printf("Pages: %d\n", b.pages);
    printf("Price: %f\n", b.price);
}

// Функция, которая изменяет цену книги
// Происходит передача через указатель
// Используется оператор -> (стрелочка)
void change_price(Book* pb, float new_price)
{
    pb->price = new_price;
}


\end{verbatim}
\begin{verbatim}
int main()
{
    // Создадим переменную типа Book по имени a
    // Её поля не заданы, там может быть мусор
    // Но их можно будет задать позднее
    Book a; 

    // Создадим переменную типа Book по имени b
    // и сразу её инициализируем
    Book b = {"The Martian", 10, 550.0};
    // К полям структуры можно получить доступ
    // с помощью оператора . (точка)
    b.pages = 369;

    // Массив книг, может содержать до 100 книг
    // Сейчас там 3 книги, остальное -- мусор
    Book scifi_books[100] = {
        {"The Dark Tower", 300, 500.0},
        {"Fahrenheit 451", 400, 700.0},
        {"The Day of the Triffids", 304, 450.0}
    };
    // Используем функцию print_book_info()
    print_book_info(scifi_books[2]);
        
    // Используем функцию change_price()
    // Обратите внимание на амперсанд
    change_price(&scifi_books[0], 2000.0);
    
    // Конечно, можно было сделать и так:
    scifi_books[0].price = 2000.0;
}
\end{verbatim}
\end{multicols}

\subsection*{Задачи:}
\begin{enumerate}
\item \textbf{Date:} Описать структуру Date, с полями: day, month и year. Написать функцию void print\_date(Date x) для печати этой структуры в формате DD.MM.YYYY
\item \textbf{Movie:}
\begin{enumerate}
\item Описать структуру Movie с полями: 
\begin{itemize}
\item title -- название фильма
\item running\_time -- длительность в минутах
\item rating -- оценка на Кинопоиске
\item release\_date -- дата выхода
\end{itemize}
\item Объявить переменную типа Movie в функции main и инициализировать её следующими значениями:\\
title -- ``Blade Runner 2049'', running\_time -- 163, rating -- 7.98, release\_date -- \{3, 10, 2017\}.
\item Отдельными командами изменить рейтинг и месяц выхода фильма.
\item Создать указатель Movie* и присвоить ему адрес созданой переменной.  Изменить поле running\_time, используя только указатель.
\item Написать функцию print\_movie\_info(Movie m) и вызвать её в функции main().
\item Написать функцию change\_rating(Movie* pm, float new\_rating) и вызвать её в функции main().
\item Объявить и инициализировать массив, содержащий 4 различных фильма.
\item Написать функцию, которая по массиву фильмов находит средний рейтинг.
\item Написать функцию, которая принимает на вход массив фильмов и возвращает указатель на фильм с самым высоким рейтингом.
\end{enumerate}
\end{enumerate}


\end{document}