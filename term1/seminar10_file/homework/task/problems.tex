\documentclass{article}
\usepackage[utf8x]{inputenc}
\usepackage{ucs}
\usepackage{amsmath} 
\usepackage{amsfonts}
\usepackage{upgreek}
\usepackage[english,russian]{babel}
\usepackage{graphicx}
\usepackage{float}
\usepackage{textcomp}
\usepackage{hyperref}
\usepackage{geometry}
\geometry{left=2cm}
\geometry{right=1.5cm}
\geometry{top=1cm}
\geometry{bottom=2cm}
\usepackage{tikz}
\usepackage{ccaption}
\usepackage{multicol}
\usepackage{listings}


\begin{document}
\pagenumbering{gobble}

\lstset{
  language=C,                % choose the language of the code
  basicstyle=\linespread{1.1}\ttfamily,
  columns=fixed,
  fontadjust=true,
  basewidth=0.5em,
  keywordstyle=\color{blue}\bfseries,
  commentstyle=\color{gray},
  stringstyle=\ttfamily\color{orange!50!black},
  showstringspaces=false,
  %numbers=false,                   % where to put the line-numbers
  numbersep=5pt,
  numberstyle=\tiny\color{black},
  numberfirstline=true,
  stepnumber=1,                   % the step between two line-numbers.        
  numbersep=10pt,                  % how far the line-numbers are from the code
  backgroundcolor=\color{white},  % choose the background color. You must add \usepackage{color}
  showstringspaces=false,         % underline spaces within strings
  captionpos=b,                   % sets the caption-position to bottom
  breaklines=true,                % sets automatic line breaking
  breakatwhitespace=true,         % sets if automatic breaks should only happen at whitespace
  xleftmargin=.2in,
  extendedchars=\true,
  keepspaces = true,
}
\lstset{literate=%
   *{0}{{{\color{red!20!violet}0}}}1
    {1}{{{\color{red!20!violet}1}}}1
    {2}{{{\color{red!20!violet}2}}}1
    {3}{{{\color{red!20!violet}3}}}1
    {4}{{{\color{red!20!violet}4}}}1
    {5}{{{\color{red!20!violet}5}}}1
    {6}{{{\color{red!20!violet}6}}}1
    {7}{{{\color{red!20!violet}7}}}1
    {8}{{{\color{red!20!violet}8}}}1
    {9}{{{\color{red!20!violet}9}}}1
}


\subsection*{Простые чтение и запись}

Пример использования файлов для решения задачи обращения массива чисел. \\
В файле input.txt хранятся числа в следующем формате: сначала идёт число n -- количество чисел, а затем идут n чисел. В файл output.txt нужно записать эти же числа в обратном порядке. Программа, которая решает эту задачу выглядит так:
\begin{lstlisting}
#include <stdio.h>
#include <stdlib.h>

int main()
{
    // Открываем 2 файла - один на чтение(read - r), другой на запись (write - w)
    FILE* in  = fopen("input.txt", "r");
    FILE* out = fopen("output.txt", "w");
    // Считываем число элементов
    int n;
    fscanf(in, "%d", &n);

    // Выделяем память
    int* arr = (int*)malloc(n*sizeof(int));
    // Считываем числа в массив arr
    for (int i = 0; i < n; ++i)
        fscanf(in, "%d", &arr[i]);
    // Записываем числа в файл output.txt
    for (int i = 0; i < n; ++i)
	    fprintf(out, "%d ", arr[n-i-1]);


    free(arr);
    fclose(in);
    fclose(out);
}
\end{lstlisting}


\begin{itemize}
\item \textbf{Reverse} Протестируйте программу выше. Не забудьте создать файл input.txt и заполнить его числами. (Эту задачу присылать не нужно)
\item \textbf{Сумма чисел} В файле input.txt лежат числа (сначала идёт число n -- количество чисел, а затем идут n чисел). Вам нужно записать в файл output.txt сумму этих чисел(гарантируется, что эта сумма помещается в int).
\item \textbf{Сортировка} В файле input.txt лежат числа (сначала идёт число n -- количество чисел, а затем идут n чисел). Вам нужно отсортировать эти числа (используйте qsort) и записать их в файл output.txt. Протестируйте вашу программу на файле numbers.txt.
\end{itemize}

\newpage

\subsection*{Посимвольное считывание из файла}
Посимвольное считывание из файла с помощью функции fgetc позволяет просто решать некоторые задачи. Пример программы, которая находит количество цифр в файле:
\begin{lstlisting}
FILE * f = fopen("input.txt", "r");
int c, number_of_digits = 0;

while ((c = fgetc(f)) != EOF)
{
    if (c >= '0' && c <= '9')
        number_of_digits += 1;
}
fclose(f);
\end{lstlisting}
\begin{itemize}
\item Написать программу, которая находит количество символов и строк в файле. Протестировать работу программы на книгах из папки \path{books/}.

\item Написать программу, которая находит количество слов в файле. Слово это любая последовательность символов, разделённая пробелом, символом табуляции('\textbackslash t') или символом перевода строки('\textbackslash n'). Протестировать работу программы на книгах из папки \path{books/}.

\item Изменить программу из предыдущей задачи так, чтобы она принимала название файла через аргумент командной строки. Протестировать работу программы на книгах из папки \path{books/}.
\end{itemize}

\end{document}