\documentclass{article}
\usepackage[utf8x]{inputenc}
\usepackage{ucs}
\usepackage{amsmath} 
\usepackage{amsfonts}
\usepackage{upgreek}
\usepackage[english,russian]{babel}
\usepackage{graphicx}
\usepackage{float}
\usepackage{textcomp}
\usepackage{hyperref}
\usepackage{geometry}
  \geometry{left=2cm}
  \geometry{right=1.5cm}
  \geometry{top=1cm}
  \geometry{bottom=2cm}
\usepackage{tikz}
\usepackage{ccaption}
\usepackage{multicol}

\usepackage{listings}
%\setlength{\columnsep}{1.5cm}
%\setlength{\columnseprule}{0.2pt}


\begin{document}
\pagenumbering{gobble}

\lstset{
  language=C,                % choose the language of the code
  basicstyle=\linespread{1.1}\ttfamily,
  columns=fixed,
  fontadjust=true,
  basewidth=0.5em,
  keywordstyle=\color{blue}\bfseries,
  commentstyle=\color{gray},
  stringstyle=\ttfamily\color{orange!50!black},
  showstringspaces=false,
  %numbers=false,                   % where to put the line-numbers
  numbersep=5pt,
  numberstyle=\tiny\color{black},
  numberfirstline=true,
  stepnumber=1,                   % the step between two line-numbers.        
  numbersep=10pt,                  % how far the line-numbers are from the code
  backgroundcolor=\color{white},  % choose the background color. You must add \usepackage{color}
  showstringspaces=false,         % underline spaces within strings
  captionpos=b,                   % sets the caption-position to bottom
  breaklines=true,                % sets automatic line breaking
  breakatwhitespace=true,         % sets if automatic breaks should only happen at whitespace
  xleftmargin=.2in,
  extendedchars=\true,
  keepspaces = true,
}
\lstset{literate=%
   *{0}{{{\color{red!20!violet}0}}}1
    {1}{{{\color{red!20!violet}1}}}1
    {2}{{{\color{red!20!violet}2}}}1
    {3}{{{\color{red!20!violet}3}}}1
    {4}{{{\color{red!20!violet}4}}}1
    {5}{{{\color{red!20!violet}5}}}1
    {6}{{{\color{red!20!violet}6}}}1
    {7}{{{\color{red!20!violet}7}}}1
    {8}{{{\color{red!20!violet}8}}}1
    {9}{{{\color{red!20!violet}9}}}1
}


\section*{Теория}

\subsection*{Форматированный ввод вывод в строку}
Считывание и печать даты в формате: YYYY-MM-DD. Например "2008-10-5".\\
\begin{lstlisting}
char str1[20];
char str2[20] = "2010-5-2";
sprintf(str1, "%d-%2d-%2d", 2008, 10, 5);
sscanf(str2, "%d-%d-%d", &year, &month, &day);
\end{lstlisting}

\begin{enumerate}
\item \textbf{Конвертация переменных в строки и обратно:} 
Объявить и инициализировать переменную типа float. Конвертировать это число в строку, используя функцию sprintf().
\item \textbf{Форматированное считывание из строки:} Создать строку представляющую некоторое время и равную "14:54:25.734". Считать количество часов, минут, секунд и миллисекунд в переменные соответствующего типа.
\end{enumerate}


\subsection*{Работа с файлами}
\begin{itemize}
\item \textbf{Запись в файл:} Создаём и открываем файл "myfile.txt" на запись ("w" означает "write"). Проверяем получилось ли открыть файл, если нет, то пишем сообщение об ошибке и выходим.  В дальнейших примерах эта проверка будет опускаться. Если получилось открыть, то записываем в файл строку с помощью fprintf().
\begin{lstlisting}
FILE* fp = fopen("myfile.txt", "w");
if (fp == NULL)
{
    printf("Error!\n");
    exit(1);
}
fprintf(fp,"What 1 programmer do in one month, 2 programmers can do in %d months\n", 2);
fclose(fp);
\end{lstlisting}
\item \textbf{Чтение из файла:} Открываем файл на чтение ("r" означает "read"). И считываем 3 целых числа.
\begin{lstlisting}
FILE* fp = fopen("numbers.txt", "r");
int x, y, z;
fscanf(fp,"%d%d%d", &x, &y, &z);
fclose(fp);
\end{lstlisting}
\item \textbf{Бинарное чтение/запись:} Записываем содержимое массива arr в файл "arr\_binary.data". Считываем числа из файла в новый массив.
\begin{lstlisting}
FILE* fp1 = fopen("arr_binary.data", "w");
int arr[5] = {55, 66, 77, 88, 99};
fwrite(arr, sizeof(int), 5, fp1);
fclose(fp1);
// ...
FILE* fp2 = fopen("arr_binary.data", "r");
int another_arr[5];
fread(arr, sizeof(int), 5, fp2);
fclose(fp2);
\end{lstlisting}
\item \textbf{Режимы открытия файла:} \\
\begin{tabular}{ | l || l |}
\hline
  r & открыть существующий файл для чтения \\
  w & создать новый файл и открыть его для записи \\
  a & открыть для записи в конец файла \\
  r+ & открыть для чтения/записи, с начала файла  \\
  w+ & создать новый файл и открыть его для чтения/записи \\
  a+ & открыть для чтения/записи в конец файла \\
\hline
\end{tabular}\\
\end{itemize}

\newpage
\section*{Задачи}
\subsection*{Файлы}
Для работы с файлами вам нужно знать в какой директории находится ваш исполняемый файл. Для этого создайте новую директорию в папку \path{/home-local/Student/workspace/}, а в этой директории создайте исходный файл формата .c. Его можно открыть с помощью geany или nano. Если вы хотите компилировать с помощью gcc, то используйте опцию -o, чтобы задать имя исполняемого файла: 
\begin{verbatim}
gcc -std=c99 -o myexe ./source_file.c
\end{verbatim}
\begin{enumerate}
\item \textbf{Создать файл:} Напишите программу, которая будет создавать файл, используя функцию fopen().
\item \textbf{Запись в файл:} Записать в файл по имени \path{hello.txt} фразу "Hello files".
\item \textbf{Запись массива чисел:} Записать в файл по имени \path{numbers.txt} все числа от 0 до 100000, которые делятся на 17.
\item \textbf{Input/output:} Создать файл \path{input.txt} следующего формата: сначала идёт число $n$, а затем $n$ целых чисел, например:
\begin{verbatim}
7
9 3 5 10 43 52 5
\end{verbatim}
Ограничений на число $n$ нет, поэтому для выделения памяти под массив нужно использовать malloc(). Считать эти числа, найти их среднее и записать все числа большие, чем среднее, в файл по имени \path{output.txt}.
\item \textbf{Little or big endian:} Число 6242983 в шестнадцатеричной системе счисления имеет вид 0x005f42a7. Запишите это число в файл в бинарном виде, используя функцию fwrite(). Проверьте, что записалось в файл. Чтобы посмотреть файл в шестнадцатеричном виде можно использовать утилиту xxd:
\begin{verbatim}
xxd <имя вашего файла>
\end{verbatim}
Определить какой порядок байт используется в вашей системе: прямой(big endian) или обратный(little endian).
\end{enumerate}

\subsection*{Аргументы командной строки}
\begin{itemize}
\item \textbf{Аргументы ls:} Программы могут принимать аргументы. Простейший пример -- утилита ls. Если запустить ls без аргументов, то она просто напечатает содержимое текущей директории. Однако этим возможности утилиты не ограничиваются. Запустите утилиту ls с разными параметрами (это лучше делать в директории содержащей хотя бы несколько различных файлов и папок):
\begin{multicols}{2}
\begin{enumerate}
\item \texttt{ls -m}
\item \texttt{ls -a}
\item \texttt{ls -l}
\item \texttt{ls -R}
\item \texttt{ls -l -R}
\item \texttt{ls -lR}
\item \texttt{ls -l <путь до другой директории>}
\item \texttt{ls -{}-help}
\end{enumerate}
\end{multicols}
\item \textbf{argc:} Следующая простейшая программа печатает количество аргументов командной строки. Скомпилируйте эту программу и протестируйте её, запуская с разным количеством аргументов.
\begin{lstlisting}
int main(int argc, char** argv)
{
	// argc - количество аргументов командной строки
	// argv - массив массивов символов ( или массив строк ) - сами аргументы
	printf("%d\n", argc);
}
\end{lstlisting}

\item \textbf{Bogosort:} В папке \path{~/workspace/seminar10_file/programms/bogosort} лежит исходный код известной (но не очень практичной) сортировки Bogosort. Алгоритм этой сортировки состоит из двух шагов:
\begin{itemize}
\item Проверяем не отсортирован ли массив
\item Случайно перемешиваем массив и переходим к 1-му шагу
\end{itemize}
Скомпилируйте программу с помощью команды \texttt{gcc ./bogosort.c -o bogosort}. Обратите внимание, что был использован аргумент командной строки \texttt{-o} компилятора gcc, задающий имя результирующего исполняемого файла. Если его не использовать, то исполняемый файл будет иметь имя по умолчанию \texttt{a.out}.\\
Запустите исполняемый файл \texttt{bogosort}. Массив, который нужно отсортировать задаётся через аргументы командной строки.
\begin{lstlisting}
./bogosort 5 4 3 2 1      // Сортируем массив {5, 4, 3, 2, 1}
\end{lstlisting}
Попробуйте отсортировать массив из 10-ти элементов.
\item \textbf{Сравнение скорости сортировок:} В папке \path{~/workspace/seminar10_file/programms/sorting} содержатся исходные коды нескольких сортировок: сортировки пузырьком (bubblesort.c), быстрой сортировки (самописная -- quicksort.c, с использование qsort() -- qsort.c) и цифровой сортировки (radixsort.c). Вам предлагается сравнить скорость работы этих сортировок.
\begin{itemize}
\item Чтобы сравнить эти сортировки, нужны числа для сортировки. Скомпилируйте файл \texttt{generate\_numbers.c} с помощью команды \texttt{gcc -std=c99 -o generate ./generate\_numbers.c}. Появится исполняемый файл \texttt{generate}. Если вызвать эту программу следующим образом: \texttt{./generate 100}, то программа сгенерирует 100 случайных чисел и запишет их в файл \texttt{numbers.txt}.
\item Скомпилируйте файлы bubblesort.c,  qsort.c, radixsort.c, countsort.c в исполняемые файлы bubblesort, qsort, radixsort, countsort соответственно. Каждая из этих программ считывает числа из файла \texttt{numbers.txt} сортирует их соответствующим образом и записывает отсортированный массив в файл \texttt{output.txt}.
\item Сгенерируйте 1000 случайных чисел и протеструйте скорость работы различных сортировок на этих числах. Для этого можно использовать утилиту \texttt{time}:
\begin{verbatim}
time ./bubblesort
\end{verbatim}
Сгенерируйте $10^5$ случайных чисел и протеструйте скорость работы различных сортировок на этих числах. То же самое для $10^7$ чисел.
\item У программы \texttt{generate} есть ещё один параметр командной строки, который задаёт максимальное значение сгенерированных чисел. Сгенерируйте $10^7$ чисел от $0$ до $9$ и протестируйте скорость работы различных сортировок.
\end{itemize}
\end{itemize}

\subsection*{Посимвольное считывание из файла}
Пример программы, которая находит количество цифр в файле:
\begin{lstlisting}
FILE * f = fopen("input.txt", "r");
int c, number_of_digits = 0;

while ((c = fgetc(f)) != EOF)
{
    if (c >= '0' && c <= '9')
        number_of_digits += 1;
}
fclose(f);
\end{lstlisting}
\begin{itemize}
\item Написать программу, которая находит количество символов, слов и строк в файле. Слово это любая последовательность символов, разделённая пробелом, символом табуляции('\textbackslash t') или символом перевода строки('\textbackslash n'). Рекомендуется написать отдельные функции под каждую из этих подзадач.
\item Изменить программу из предыдущей задачи так, чтобы она принимала название файла через аргумент командной строки. Протестировать работу программы на книгах из папки \path{~/workspace/seminar10_file/programms/books/}.
\end{itemize}

\end{document}