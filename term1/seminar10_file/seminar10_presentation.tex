\documentclass[12pt,pdf,hyperref={unicode}]{beamer}


%\documentclass[10pt]{beamer}

\usetheme[progressbar=frametitle]{metropolis}

\usepackage{booktabs}
\usepackage[scale=2]{ccicons}

\usepackage{pgfplots}
\usepgfplotslibrary{dateplot}

\usepackage{xspace}
\newcommand{\themename}{\textbf{\textsc{metropolis}}\xspace}


%\usepackage{lmodern}

% подключаем кириллицу 
\usepackage[T2A]{fontenc}
\usepackage[utf8]{inputenc}
\usepackage{listings}
%\usepackage{graphicx}
\usepackage{hyperref}

% отключить клавиши навигации
\setbeamertemplate{navigation symbols}{}

% тема оформления
\usetheme{Pittsburgh}

% цветовая схема
\usecolortheme{default}

\definecolor{light-gray}{gray}{0.90}

\title{Семинар №10}   
\subtitle{ФАКИ \the\year}
\author{Бирюков В. А.} 
\date{\today}
% \logo{\includegraphics[height=5mm]{images/logo.png}\vspace{-7pt}}

\begin{document}

\lstset{language=C}

% титульный слайд
\begin{frame}
\titlepage
\end{frame} 



\defverbatim[colored]\makeset{
\begin{lstlisting}[language=C++,basicstyle=\ttfamily,keywordstyle=\color{blue},
                stringstyle=\color{red}\ttfamily]
void make_set(int X) {
  parent[X] = X;
}
\end{lstlisting}
}

\lstset{
  language=C,                % choose the language of the code
  keywordstyle=\color{blue},
  numbers=none,                   % where to put the line-numbers
  stepnumber=1,                   % the step between two line-numbers.        
  numbersep=5pt,                  % how far the line-numbers are from the code
  backgroundcolor=\color{light-gray},  % choose the background color. You must add \usepackage{color}
  showspaces=false,               % show spaces adding particular underscores
  showstringspaces=false,         % underline spaces within strings
  showtabs=false,                 % show tabs within strings adding particular underscores
  tabsize=2,                      % sets default tabsize to 2 spaces
  captionpos=b,                   % sets the caption-position to bottom
  breaklines=true,                % sets automatic line breaking
  breakatwhitespace=true,         % sets if automatic breaks should only happen at whitespace
}





\section{Функция main}


\begin{frame}[fragile]
\frametitle{Функция main} 
\begin{center}
\includegraphics[width=0.35\linewidth]{images/function_syntax_main.png}
\end{center}
\begin{itemize}
\item Точка входа в программу
\item Возвращает 0, если программа завершилась нормально
\item В простейшем виде не принимает аргументов
\end{itemize}
\end{frame}


\begin{frame}[fragile]
\frametitle{Функция main}  
\begin{center}
\includegraphics[width=0.7\linewidth]{images/function_syntax_main_args.png}
\end{center}
\begin{itemize}
\item argv -- параметры, передаваемые в функцию main
\item argc -- количество этих параметров
\end{itemize}
\end{frame}

\begin{frame}[fragile]
\frametitle{Функции} 
\framesubtitle{Функция main}
\begin{center}
\includegraphics[width=1.0\linewidth]{images/function_argcargv.png}
\end{center}
\end{frame}




\section{Запись/чтение файлов}

\begin{frame}[fragile]
\frametitle{Запись/чтение файлов} 
\begin{itemize}
\item Для работы с файлами нужно подключить заголовочный файл <stdio.h>
\item Создание указателя на файл:
\begin{lstlisting}
FILE * pFile;
\end{lstlisting}
\end{itemize}
\end{frame}

\begin{frame}[fragile]
\frametitle{Запись/чтение файлов} 
\frametitle{fopen и fclose} 
\begin{itemize}
\item Открыть файл:
\begin{lstlisting}
FILE *fopen(const char *filename, const char *mode);
\end{lstlisting}
\item Закрыть файл:
\begin{lstlisting}
int fclose(FILE *a_file);
\end{lstlisting}
\end{itemize}
\end{frame}

\begin{frame}[fragile]
\frametitle{Режмы работы с файлом} 
\begin{lstlisting}
FILE *fopen(const char *filename, const char *mode);
\end{lstlisting}
\begin{tabular}{ l || l }
  r & открыть существующий файл для чтения \\
  w & создать файл и открыть для записи \\
  a & открыть для записи в конец файла \\
  r+ & открыть для чтения/записи, с начала файла  \\
  w+ & создать файл и его открыть для чтения/записи \\
  a+ & открыть для чтения/записи в конец файла \\
\end{tabular}
\end{frame}

\begin{frame}[fragile]
\frametitle{fprintf, fscanf}  
\begin{lstlisting}[language=C++,basicstyle=\ttfamily,keywordstyle=\color{blue},
                stringstyle=\color{orange}\ttfamily]
   #include <stdio.h>
   FILE *fptr;
   fptr=fopen("output.txt", "w");
   if(fptr==NULL){
      printf("Error!");   
      exit(1);             
   }
   fprintf(fptr, "%d", n);   
   fclose(fptr);
\end{lstlisting}
\end{frame}


\begin{frame}[fragile]
\frametitle{fputc, fgetc}  
\begin{lstlisting}[language=C++,basicstyle=\ttfamily,keywordstyle=\color{blue},stringstyle=\color{orange}\ttfamily]
   FILE * f = fopen("input.txt", "r");
   
   int number_of_chars = 0;
   int c;
   
   while ((c = fgetc(f)) != EOF)
   {
       number_of_chars++
   }
   fclose(f);
\end{lstlisting}
\end{frame}

\begin{frame}[fragile]
\frametitle{Бинарные чтение/запись}  
\framesubtitle{fread, fwrite}  
\begin{lstlisting}[language=C++,basicstyle=\ttfamily,keywordstyle=\color{blue},stringstyle=\color{orange}\ttfamily]
   char c[] = "some string data";
   char buffer[100];
   FILE *fp = fopen("output.txt", "w+");
   fwrite(c, strlen(c) + 1, 1, fp);
   fclose(fp);
\end{lstlisting}
\end{frame}





\section{Задание}

\begin{frame}[fragile]
\frametitle{Задание} 
\begin{itemize}
\item Посчитать число строк, слов, символов в файле.
\item Написать аналог программы wc.
\item Контест Fail not found (задачи arguments, D и cp*)
\end{itemize}
\end{frame}

\end{document}
