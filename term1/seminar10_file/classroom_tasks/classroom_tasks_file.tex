\documentclass{article}
\usepackage[utf8x]{inputenc}
\usepackage{ucs}
\usepackage{amsmath} 
\usepackage{amsfonts}
\usepackage{upgreek}
\usepackage[english,russian]{babel}
\usepackage{graphicx}
\usepackage{float}
\usepackage{textcomp}
\usepackage{hyperref}
\usepackage{geometry}
  \geometry{left=2cm}
  \geometry{right=1.5cm}
  \geometry{top=1cm}
  \geometry{bottom=2cm}
\usepackage{tikz}
\usepackage{ccaption}
\usepackage{multicol}

\usepackage{listings}
%\setlength{\columnsep}{1.5cm}
%\setlength{\columnseprule}{0.2pt}


\begin{document}

\lstset{
  language=C,                % choose the language of the code
  basicstyle=\linespread{1.1}\ttfamily,
  columns=fixed,
  fontadjust=true,
  basewidth=0.5em,
  keywordstyle=\color{blue}\bfseries,
  commentstyle=\color{gray},
  stringstyle=\ttfamily\color{orange!50!black},
  showstringspaces=false,
  %numbers=false,                   % where to put the line-numbers
  numbersep=5pt,
  numberstyle=\tiny\color{black},
  numberfirstline=true,
  stepnumber=1,                   % the step between two line-numbers.        
  numbersep=10pt,                  % how far the line-numbers are from the code
  backgroundcolor=\color{white},  % choose the background color. You must add \usepackage{color}
  showstringspaces=false,         % underline spaces within strings
  captionpos=b,                   % sets the caption-position to bottom
  breaklines=true,                % sets automatic line breaking
  breakatwhitespace=true,         % sets if automatic breaks should only happen at whitespace
  xleftmargin=.2in,
  extendedchars=\true,
  keepspaces = true,
}
\lstset{literate=%
   *{0}{{{\color{red!20!violet}0}}}1
    {1}{{{\color{red!20!violet}1}}}1
    {2}{{{\color{red!20!violet}2}}}1
    {3}{{{\color{red!20!violet}3}}}1
    {4}{{{\color{red!20!violet}4}}}1
    {5}{{{\color{red!20!violet}5}}}1
    {6}{{{\color{red!20!violet}6}}}1
    {7}{{{\color{red!20!violet}7}}}1
    {8}{{{\color{red!20!violet}8}}}1
    {9}{{{\color{red!20!violet}9}}}1
}

\title{Семинар \#9: Аргументы командной строки. Файлы.\vspace{-5ex}}\date{}\maketitle

\section*{Функции \texttt{sprintf} и \texttt{sscanf}.}
На предыдущих занятиях мы прошли функции ввода/вывода из стандартных потоков: \texttt{printf} и \texttt{scanf}. А также функции для ввода/вывода из файлов: \texttt{fprintf} и \texttt{fscanf}. Функции \texttt{sprintf} и \texttt{sscanf} предназначены для ввода/вывода в строку или из строки. Пример использования можно посмотреть в файле \texttt{0sscanf.c}.
\begin{itemize}
\item \textbf{Задача \#1: Конвертация переменных в строки и обратно} 
\begin{lstlisting}
char str1[10] = "79";
char str2[10] = "435";
\end{lstlisting}
Конвертирутуйте эти 2 строки в числа, сложите и напечатайте результат
\end{itemize}



\section*{Аргументы командной строки}
Программы могут принимать аргументы. Простейший пример -- утилита \texttt{ls}. Если запустить \texttt{ls} без аргументов, то она просто напечатает содержимое текущей директории.  Если же использовать эту программу с опцией \texttt{-l}: \texttt{ls -l}, то на экран выведется подробное описание файлов и папок в текущей директории. Другой пример - опция для компилятора \texttt{gcc -std=c99}.\\
\begin{itemize}

\item \textbf{Задача \#2: \texttt{xxd}} - это простая программа, которая выводит на экран всё содержимое файла побайтово. Если, например, запустить программму следующим образом:
\texttt{xxd a.out}, то она выведет на экран всё содержимое этого исполняемого файла. Часто используемые опции командной строки: \texttt{-h} (сокращение от help) и \texttt{-v} (сокращение от version).
\begin{itemize}
\item Запустите \texttt{xxd} с аргументом -- именем файла \texttt{hello.txt}. Этот файл содержит лишь строку \texttt{Hello}.\\ 
\texttt{xxd} покажет вам содержимое этого файла в шестнадцатеричном виде и в виде \texttt{ASCII}.
\item Запустите \texttt{xxd} с опцией \texttt{-h}.
\item Запустите \texttt{xxd} с нужной опцией, чтобы вывод файла \texttt{hello.txt} был представлен в двоичном виде.
\item \textbf{*} Если файл большой, то весь вывод \texttt{xxd} не поместится на экран. Перенаправить вывод в нужный файл можно следующим образом:
\texttt{xxd a.out > temp.txt}. После этого в файле \texttt{temp.txt} будет хранится всё, что было бы напечатано на экран.
\item \textbf{*} Создайте программу Hello World и скомпилируйте её в файл \texttt{a.out}. Сохраните вывод \texttt{xxd ./a.out} в отдельном файле \texttt{hw.txt}. Измените файл \texttt{hw.txt}, так чтобы программа печатала Hello MIPT. Создайте исполняемый файл из файла \texttt{hw.txt}, используя \texttt{xxd} с опцией \texttt{-r}.
\end{itemize}

\item \textbf{Задача \#3: argc:} Простейшая программа \texttt{1argc.c} печатает количество аргументов командной строки. Скомпилируйте эту программу и протестируйте её, запуская с разным количеством аргументов.
\item \textbf{Задача \#4: argv:} Простейшая программа \texttt{2argv.c} печатает аргументы командной строки. Скомпилируйте эту программу и протестируйте её, запуская с разным количеством аргументов.
\begin{lstlisting}
#include <stdio.h>
int main(int argc, char** argv)
{
	printf("Number of arguments = %d\n", argc);
	for (int i = 0; i < argc; ++i)
		printf("Argument #%d = %s\n", i, argv[i]);
}
\end{lstlisting}
\item \textbf{Задача \#5: Сумма аргументов:} Создайте программу \texttt{sum}, которая будет печатать сумму всех аргументов. Например, при вызове \\
\texttt{./sum 4 8 15 16 23 42}\\
программа должна напечатать \texttt{108}
\end{itemize}

\newpage
\section*{Функции для работы с файлами}
\begin{itemize}
\item \textbf{\texttt{fopen}:} Открывает файл для чтения/записи\\
\textbf{Режимы открытия файла:} \\
\begin{tabular}{ | l || l |}
\hline
  r & открыть существующий файл для чтения (read)\\
  w & создать новый файл и открыть его для записи (write)\\
    & если файл уже существует, то он удалится перед записью\\
  a & открыть для записи в конец файла (append)\\
  r+ & открыть для чтения/записи, с начала файла  \\
  w+ & создать новый файл и открыть его для чтения/записи \\
  a+ & открыть для чтения/записи в конец файла \\
\hline
\end{tabular}\\\\
Для бинарных файлов в Windows нужно добавить символ \texttt{b}.
\item \textbf{\texttt{fopen}:} Закрывает файл
\item \textbf{\texttt{fprintf/fscanf}:} Функции работают аналогично \texttt{printf/scanf}, только первым аргументом нужно передать файл (указатель на структуру \texttt{FILE}).
\begin{lstlisting}
#include <stdio.h>
#include <stdlib.h>
int main()
{
    FILE* fp = fopen("myfile.txt", "w");
    if (fp == NULL)
    {
        printf("Error!\n");
        exit(1);
    }
    fprintf(fp, "Hello world!");
    fclose(fp);
}
\end{lstlisting}
\begin{itemize}
\item Создаём и открываем файл \texttt{"myfile.txt"} на запись.
\item Проверяем получилось ли открыть файл, если нет, то пишем сообщение об ошибке и выходим.  В дальнейших примерах эта проверка будет опускаться для экономии места.
\item Если получилось открыть, то записываем в файл строку с помощью \texttt{fprintf}.
\item Закрываем файл.
\end{itemize}
 
\textbf{Задача \#6:} Скомпилируйте программу \texttt{3fprintf.c} и запустите. В результате выполнения программы должен появиться файл \texttt{myfile.txt} с содержимым \texttt{Hello world!}.


\item \textbf{\texttt{fwrite/fread} - бинарное чтение/запись:} \texttt{fwrite} записывает некоторый участок памяти в файл без обработки. \texttt{fread} считывает данные из файла в память без обработки.

Пример. Записываем 4 байта памяти переменной \texttt{a} в файл \texttt{binary.dat}:
\begin{lstlisting}
#include <stdio.h>
int main()
{
	int a = 287454020;  // Это число = 0x11223344 в шестнадцатеричной системе
	FILE* fb = fopen("binary.dat", "w");
	fwrite(&a, sizeof(int), 1, fb);
	fclose(fb);
}
\end{lstlisting}
\textbf{Задача \#7: Печать в текстовом и бинарном виде:}\\
В файле \texttt{4text\_and\_binary.c} содержится пример записи числа в текстовом и бинарном виде. Скомпилируйте эту программу и запустите. Должно появиться 2 файла (\texttt{number.txt} и \texttt{number.bin}). Изучите оба эти файла, открывая их в текстовом редакторе, а также с помощью утилиты \texttt{xxd}. Объясните результат.


\textbf{Задача \#8: Печать массива в бинарном виде:}\\
Пусть есть массив из чисел типа \texttt{int}: \texttt{int array[5] = \{111, 222, 333, 444, 555\};}\\
Запишите эти числа в текстовый файл \texttt{array.txt}, используя \texttt{fprintf}. Изучите содержимое этого файла побайтово с помощью \texttt{xxd}.\\
Запишите эти числа в бинарный файл \texttt{array.bin}, используя \texttt{fwrite}. Изучите содержимое этого файла побайтово с помощью \texttt{xxd}.



\item \textbf{\texttt{fgetc} - посимвольное чтение из файла} - возвращает ASCII код следующего символа из файла. Если символов не осталось, то она возвращает константу \texttt{EOF} равную \texttt{-1}.\\
Пример программы, которая находит количество цифр в файле:
\begin{lstlisting}
#include <stdio.h>
int main()
{
    FILE* f = fopen("input.txt", "r");
    int c; 
    int num_of_digits = 0;

    while ((c = fgetc(f)) != EOF)
    {
        if (c >= '0' && c <= '9')
            num_of_digits += 1;
    }
    printf("Number of digits = %d\n", num_of_digits);
    fclose(f);
}
\end{lstlisting}
Эта программа содержится в файле \texttt{5number\_of\_digits.c}
\begin{itemize}
\item \textbf{Задача \#9:} Написать программу \texttt{symbolcount}, которая считает количество символов в файле. название файла должно передаваться через аргумент командной строки:\\
\begin{verbatim}
gcc -o symbolcount main.c
./symbolcount war_and_peace.txt
3332371
\end{verbatim}
\item \textbf{Задача \#10:} Написать программу \texttt{linecount}, которая находит количество строк в файле.
\item \textbf{Задача \#11:} Написать программу \texttt{wordcount}, которая находит количество слов в файле. Слово это любая последовательность символов, разделённая \textit{одним или несколькими} пробельными, символами. Пробельные символы это пробел, перенос на новую строку(\texttt{\textbackslash n}) либо табуляция(\texttt{\textbackslash t}).


\end{itemize}
\item \textbf{\texttt{ftell}}: Функция, которая возвращает текущее положение в файле. Например, если мы начали считать с начала файла и считали 10 символов, то эта функция вернёт 10.
\item \textbf{\texttt{fseek}}: Функция, которая устанавливает положение в файле. Например, следующая строка устанавливает положение на 100-й символ:
\begin{lstlisting}
fseek (fout, 100, SEEK_SET);
\end{lstlisting}
Следующие функции будут считывать всё начиная с 101-го символа.
\end{itemize}
\newpage
\section*{Работа с изображениями формата \texttt{.ppm}}
Простейший формат для изображение имеет следующую структуру
\begin{verbatim}
P3
3 2
255
255 0 0 
0 255 0  
0 0 255 
255 255 0 
255 255 255 
0 0 0
\end{verbatim}
\begin{itemize}
\item В первой строке задаётся тип файла \texttt{P3} - означает, что в этом файле будет храниться цветное изображение, причём значения пикселей будет задаваться в текстовом формате.
\item Во второй строке задаются размеры картинки - 3 на 2 пикселя.
\item Во третьей строке задаётся максимальное значение RGB компоненты цвета.
\item Дальше идут RGB компоненты цветов каждого пикселя в текстовом формате.
\end{itemize}
Картинка имеет следующий вид:
\center{\includegraphics[scale=0.5]{../images/tiny.png}}

\begin{itemize}
\item \textbf{Задача \#12:} Написать программу, которая генерирует одноцветную картинку (500 на 500) в формате \texttt{.ppm}. Цвет должен передаваться через аргументы командной строки.
\item \textbf{Задача \#13: Белый шум:} Написать программу, которая случайное изображение в формате \texttt{.ppm}. Цвет каждого пикселя задаётся случайно.
\item \textbf{Задача \#14: Градиент:} Написать программу, которая генерирует градиентную картинку в формате \texttt{.ppm}. Два цвета должны передаваться через аргументы командной строки.
\item \textbf{Задача \#15: Черно-белая картинка:} Написать программу, которая считывает изображение в формате \texttt{.ppm} и сохраняет его в черно-белом виде. Файл изображения должен передаваться через аргументы командной строки. Считайте файл \texttt{russian\_peasants\_1909.ppm} и сделайте его черно-белым.
\end{itemize}

\end{document}