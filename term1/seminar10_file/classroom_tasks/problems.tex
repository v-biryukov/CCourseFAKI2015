\documentclass{article}
\usepackage[utf8x]{inputenc}
\usepackage{ucs}
\usepackage{amsmath} 
\usepackage{amsfonts}
\usepackage{upgreek}
\usepackage[english,russian]{babel}
\usepackage{graphicx}
\usepackage{float}
\usepackage{textcomp}
\usepackage{hyperref}
\usepackage{geometry}
  \geometry{left=2cm}
  \geometry{right=1.5cm}
  \geometry{top=1cm}
  \geometry{bottom=2cm}
\usepackage{tikz}
\usepackage{ccaption}
\usepackage{multicol}

\usepackage{listings}
%\setlength{\columnsep}{1.5cm}
%\setlength{\columnseprule}{0.2pt}


\begin{document}
\pagenumbering{gobble}

\lstset{
  language=C,                % choose the language of the code
  basicstyle=\linespread{1.1}\ttfamily,
  columns=fixed,
  fontadjust=true,
  basewidth=0.5em,
  keywordstyle=\color{blue}\bfseries,
  commentstyle=\color{gray},
  stringstyle=\ttfamily\color{orange!50!black},
  showstringspaces=false,
  %numbers=false,                   % where to put the line-numbers
  numbersep=5pt,
  numberstyle=\tiny\color{black},
  numberfirstline=true,
  stepnumber=1,                   % the step between two line-numbers.        
  numbersep=10pt,                  % how far the line-numbers are from the code
  backgroundcolor=\color{white},  % choose the background color. You must add \usepackage{color}
  showstringspaces=false,         % underline spaces within strings
  captionpos=b,                   % sets the caption-position to bottom
  breaklines=true,                % sets automatic line breaking
  breakatwhitespace=true,         % sets if automatic breaks should only happen at whitespace
  xleftmargin=.2in,
  extendedchars=\true,
  keepspaces = true,
}
\lstset{literate=%
   *{0}{{{\color{red!20!violet}0}}}1
    {1}{{{\color{red!20!violet}1}}}1
    {2}{{{\color{red!20!violet}2}}}1
    {3}{{{\color{red!20!violet}3}}}1
    {4}{{{\color{red!20!violet}4}}}1
    {5}{{{\color{red!20!violet}5}}}1
    {6}{{{\color{red!20!violet}6}}}1
    {7}{{{\color{red!20!violet}7}}}1
    {8}{{{\color{red!20!violet}8}}}1
    {9}{{{\color{red!20!violet}9}}}1
}


\section*{Аргументы командной строки. Файлы.}

\subsection*{Функции \texttt{sprintf} и \texttt{sscanf}.}
\begin{lstlisting}
#include <stdio.h>
int main(){
    char str[20] = "2010-5-2";
    // Считываем из строки:
    int year, month, day;
    sscanf(str, "%d-%d-%d", &year, &month, &day);
}
\end{lstlisting}

\begin{itemize}
\item \textbf{Конвертация переменных в строки и обратно:} 
\begin{lstlisting}
char str1[10] = "435";
char str2[10] = "73";
\end{lstlisting}
Конвертирутуйте эти 2 строки в числа, сложите и напечатайте результат
\end{itemize}



\subsection*{Аргументы командной строки}
Программы могут принимать аргументы. Простейший пример -- утилита \texttt{ls}. Если запустить \texttt{ls} без аргументов, то она просто напечатает содержимое текущей директории.  Если же использовать эту программу с опцией \texttt{-l}: \texttt{ls -l}, то на экран выведется подробное описание файлов и папок в текущей директории. Другой пример - опция для компилятора \texttt{gcc -std=c99}.\\
\begin{itemize}

\item \textbf{\texttt{xxd}} - это простая программа, которая выводит на экран всё содержимое файла побайтово. Если, например, запустить программму следующим образом:
\texttt{xxd a.out}, то она выведет на экран всё содержимое этого исполняемого файла. Часто используемые опции командной строки: \texttt{-h} (сокращение от help) и \texttt{-v} (сокращение от version).
\begin{itemize}
\item Запустите \texttt{xxd} с опцией \texttt{-h}.
\item Запустите \texttt{xxd} с нужной опцией, чтобы вывод был представлен в двоичном виде.
\item Перенаправить вывод в нужный файл можно следующим образом:
\texttt{xxd a.out > temp.txt}
\item \textbf{*} Создайте программу Hello World и скомпилируйте её в файл \texttt{a.out}. Сохраните вывод \texttt{xxd ./a.out} в отдельном файле \texttt{hw.txt}. Измените файл \texttt{hw.txt}, так чтобы программа печатала Hello MIPT. Создайте исполняемый файл из файла \texttt{hw.txt}, используя \texttt{xxd} с опцией \texttt{-r}.
\end{itemize}

\item \textbf{argc:} Следующая простейшая программа печатает количество аргументов командной строки. Скомпилируйте эту программу и протестируйте её, запуская с разным количеством аргументов.
\begin{lstlisting}
int main(int argc, char** argv)
{
	// argc - количество аргументов командной строки
	printf("%d\n", argc);
}
\end{lstlisting}
\item \textbf{argv:} Следующая программа печатает аргументы командной строки. Скомпилируйте эту программу и протестируйте её, запуская с разным количеством аргументов.
\begin{lstlisting}
int main(int argc, char** argv)
{
	// argv - массив массивов символов ( или массив строк ) - сами аргументы
	for (int i = 0; i < argc; i++)
		printf("%d: %s\n", i, argv[i]);
}
\end{lstlisting}
Создайте программу \texttt{sum}, которая будет печатать сумму всех аргументов. Например, при вызове \\
\texttt{./sum 4 8 15 16 23 42}\\
программа должна напечатать 108


\end{itemize}

\subsection*{Функции для работы с файлами}
\begin{itemize}
\item \textbf{\texttt{fprintf/fscanf}:} Создаём и открываем файл "myfile.txt" на запись ("w" означает "write"\,, "r" означает "read"). Проверяем получилось ли открыть файл, если нет, то пишем сообщение об ошибке и выходим.  В дальнейших примерах эта проверка будет опускаться. Если получилось открыть, то записываем в файл строку с помощью fprintf().
\begin{lstlisting}
#include <stdio.h>
#include <stdlib.h> // для функции exit
int main(){
    FILE* fp = fopen("myfile.txt", "w");
    if (fp == NULL)
    {
        printf("Error!\n");
        exit(1);
    }
    fprintf(fp,"Hello world!\n");
    fclose(fp);
}
\end{lstlisting}

\textbf{Режимы открытия файла:} \\
\begin{tabular}{ | l || l |}
\hline
  r & открыть существующий файл для чтения \\
  w & создать новый файл и открыть его для записи \\
  a & открыть для записи в конец файла \\
  r+ & открыть для чтения/записи, с начала файла  \\
  w+ & создать новый файл и открыть его для чтения/записи \\
  a+ & открыть для чтения/записи в конец файла \\
\hline
\end{tabular}\\
Для бинарных файлов нужно добавить символ \texttt{b}.

\item \textbf{\texttt{fread/fwrite} - бинарное чтение/запись:} Записываем содержимое массива arr в файл \texttt{numbers.dat}.
\begin{lstlisting}
#include <stdio.h>
int main(){
    FILE* fp = fopen("numbers.dat", "wb");
    int arr[5] = {55, 66, 77, 88, 99};
    fwrite(arr, sizeof(int), 5, fp1);
    fclose(fp1);
}
\end{lstlisting}
\begin{itemize}
\item Пусть есть массив из чисел типа \texttt{unsigned char}: \\
\texttt{unsigned char array[] = \{97, 255, 4, 145\};}\\
Запишите эти числа в текстовый файл \texttt{text.txt}, используя \texttt{fprintf}. Изучите содержимое этого файла побайтово с помощью \texttt{xxd}.
\item Запишите эти числа в бинарный файл \texttt{data.dat}, используя \texttt{fwrite}. Изучите содержимое этого файла побайтово с помощью \texttt{xxd}.
\item \textbf{Little or big endian:} Число \texttt{int a = 6242983;} в шестнадцатеричной системе счисления имеет вид \texttt{005f42a7}. Запишите это число в файл в бинарном виде, используя функцию fwrite(). Проверьте, что записалось в файл. Чтобы посмотреть файл в шестнадцатеричном виде можно использовать утилиту xxd.Определить какой порядок байт используется в вашей системе: прямой(big endian) или обратный(little endian).
\end{itemize}

\newpage
\item \textbf{\texttt{fgetc} - посимвольное чтение из файла} - возвращает ASCII код следующего символа из файла. Если символов не осталось, то она возвращает константу \texttt{EOF} равную \texttt{-1}.\\
Пример программы, которая находит количество цифр в файле:
\begin{lstlisting}
#include <stdio.h>
int main(){
    FILE * f = fopen("input.txt", "r");
    int c, num_of_digits = 0;

    while ((c = fgetc(f)) != EOF)
    {
        if (c >= '0' && c <= '9')
            num_of_digits += 1;
    }
    printf("Number of digits = %d\n", num_of_digits);
    fclose(f);
}
\end{lstlisting}
\begin{itemize}
\item Написать программу \texttt{symbolcount}, которая считает количество символов в файле. название файла должно передаваться через аргумент командной строки:\\
\begin{verbatim}
gcc -o symbolcount main.c
./symbolcount war_and_peace.txt
3332371
\end{verbatim}
\item Написать программу \texttt{linecount}, которая находит количество строк в файле.
\item Написать программу \texttt{wordcount}, которая находит количество слов в файле. Слово это любая последовательность символов, разделённая одним или несколькими пробельными, символами. Пробельные символы это пробел, перенос на новую строку(\texttt{\textbackslash n}) либо табуляция(\texttt{\textbackslash t}).
\end{itemize}

\item \textbf{Работа с изображениями формата \texttt{.ppm}}\\
Простейший формат для изображение имеет следующую структуру
\begin{verbatim}
P3
3 2
255
255 0 0 
0 255 0  
0 0 255 
255 255 0 
255 255 255 
0 0 0
\end{verbatim}
\begin{itemize}
\item В первой строке задаётся тип файла \texttt{P3} - означает, что в этом файле будет храниться цветное изображение, причём значения пикселей будет задаваться в текстовом формате.
\item Во второй строке задаются размеры картинки - 3 на 2 пикселя.
\item Во третьей строке задаётся максимальное значение RGB компоненты цвета.
\item Дальше идут RGB компоненты цветов каждого пикселя в текстовом формате.
\end{itemize}
Картинка имеет следующий вид:
\center{\includegraphics[scale=0.5]{tiny.png}}

\newpage
\begin{itemize}
\item Написать программу, которая генерирует одноцветную картинку (500 на 500) в формате ppm. Цвет должен передаваться через аргументы командной строки.
\item \textbf{Белый шум:} Написать программу, которая случайное изображение в формате ppm. Цвет каждого пикселя задаётся случайно.
\item \textbf{Градиент:} Написать программу, которая генерирует градиентную картинку в формате ppm. Два цвета должны передаваться через аргументы командной строки.
\item \textbf{Черно-белая картинка:} Написать программу, которая считывает изображение в формате \texttt{ppm} и сохраняет его в черно-белом виде. Файл изображения должен передаваться через аргументы командной строки.
\end{itemize}

\end{itemize}
\end{document}