\documentclass[12pt,pdf,hyperref={unicode}]{beamer}


%\documentclass[10pt]{beamer}

\usetheme[progressbar=frametitle]{metropolis}

\usepackage{booktabs}
\usepackage[scale=2]{ccicons}

\usepackage{pgfplots}
\usepgfplotslibrary{dateplot}

\usepackage{xspace}
\newcommand{\themename}{\textbf{\textsc{metropolis}}\xspace}


%\usepackage{lmodern}

% подключаем кириллицу 
\usepackage[T2A]{fontenc}
\usepackage[utf8]{inputenc}
\usepackage{listings}
%\usepackage{graphicx}
\usepackage{hyperref}

% отключить клавиши навигации
\setbeamertemplate{navigation symbols}{}

% тема оформления
\usetheme{Pittsburgh}

% цветовая схема
\usecolortheme{default}

\definecolor{light-gray}{gray}{0.90}

\title{Семинар №2}   
\subtitle{ФАКИ \the\year}
\author{Бирюков В. А.} 
\date{\today} 
% \logo{\includegraphics[height=5mm]{images/logo.png}\vspace{-7pt}}

\begin{document}

\lstset{language=C}

% титульный слайд
\begin{frame}
\titlepage
\end{frame} 

\defverbatim[colored]\makeset{
\begin{lstlisting}[language=C++,basicstyle=\ttfamily,keywordstyle=\color{blue}]
void make_set(int X) {
  parent[X] = X;
}
\end{lstlisting}
}

\lstset{
  language=C,                % choose the language of the code
  basicstyle=\ttfamily,
  columns=fixed,
  fontadjust=true,
  basewidth=0.5em,
  keywordstyle=\color{blue}\bfseries,
  commentstyle=\color{gray},
  stringstyle=\ttfamily\color{yellow!40!red},
  showstringspaces=false,
  %numbers=false,                   % where to put the line-numbers
  numbersep=5pt,
  numberstyle=\tiny\color{gray},
  numberfirstline=true,
  stepnumber=1,                   % the step between two line-numbers.        
  numbersep=5pt,                  % how far the line-numbers are from the code
  backgroundcolor=\color{white!90!gray},  % choose the background color. You must add \usepackage{color}
  showstringspaces=false,         % underline spaces within strings
  captionpos=b,                   % sets the caption-position to bottom
  breaklines=true,                % sets automatic line breaking
  breakatwhitespace=true,         % sets if automatic breaks should only happen at whitespace
}
\lstset{literate=%
   *{0}{{{\color{red!20!violet}0}}}1
    {1}{{{\color{red!20!violet}1}}}1
    {2}{{{\color{red!20!violet}2}}}1
    {3}{{{\color{red!20!violet}3}}}1
    {4}{{{\color{red!20!violet}4}}}1
    {5}{{{\color{red!20!violet}5}}}1
    {6}{{{\color{red!20!violet}6}}}1
    {7}{{{\color{red!20!violet}7}}}1
    {8}{{{\color{red!20!violet}8}}}1
    {9}{{{\color{red!20!violet}9}}}1
}




\begin{frame}
\frametitle{Тип данных boolean}
Тип данных, который принимает 2 значения: \textbf{False} и \textbf{True} (либо 0 и 1). \\
В языке C роль типа данных boolean часто играет тип int: \\
\begin{center}
\begin{tabular}{ l | l}
  int & bool \\\hline
  0 & False \\
  любое другое значение & True \\
\end{tabular}
\end{center}

\end{frame}


\begin{frame}
\frametitle{Логические операторы}
Возвращают тип bool
\begin{center}
\begin{tabular}{ l | l}
  ! & не \\
  || & или \\
  \&\& & и \\
\end{tabular}
\end{center}
\end{frame}


\begin{frame}
\frametitle{Базовые операторы}
\frametitle{Приоритет операторов}
\begin{center}
\begin{enumerate}
\item (), []
\item ++, --, +, -(унарные), sizeof
\item *, /, \%
\item +, -
\item >,<,<=,>=
\item ==, !=
\item \&, |, \&\&, ||
\item =, +=, и т.д.
\end{enumerate}
\end{center}
Приоритет операторов C подробнее:\\
\href{http://ru.cppreference.com/w/c/language/operator_precedence}
{\textcolor{red}{ru.cppreference.com/w/c/language/operator\_precedence}}
\end{frame}







\section{Управляющие конструкции}

\section{Оператор ветвления: if else}

\begin{frame}[fragile]
\frametitle{Базовые управляющие конструкции} 
\framesubtitle{if, if else}



\lstinputlisting{./programms/code_if1.c}



\end{frame}

\begin{frame}[fragile]
\frametitle{Базовые управляющие конструкции} 
\framesubtitle{if, if else}

\lstinputlisting{./programms/code_if2.c}

\end{frame}


\begin{frame}
\frametitle{Базовые управляющие конструкции} 
\framesubtitle{Тернарный оператор :?}

\lstinputlisting{./programms/code_ternar.c}
\end{frame}



\section{Циклы: while, do while, for}


\begin{frame}
\frametitle{Базовые управляющие конструкции} 
\framesubtitle{Цикл while}

\lstinputlisting{./programms/code_while.c}

Напечатает 1 2 3 

\end{frame}

\begin{frame}
\frametitle{Базовые управляющие конструкции} 
\framesubtitle{Цикл do while}

\lstinputlisting{./programms/code_do_while.c}

Напечатает 1 2 3 

\end{frame}

\begin{frame}
\frametitle{Базовые управляющие конструкции} 
\framesubtitle{Цикл for}

\lstinputlisting{./programms/code_for.c}

Напечатает 1 2 3 

\end{frame}

\begin{frame}
\frametitle{Базовые управляющие конструкции} 
\framesubtitle{Замечания}

В условии циклов может стоять любой оператор:

\lstinputlisting{./programms/code_loop_remarks.c}


\end{frame}



\section{Другие управляющие конструкции: break, continue, switch}

\begin{frame}
\frametitle{Управляющие конструкции} 
\framesubtitle{Оператор break}

\lstinputlisting{./programms/code_break.c}
\end{frame}

\begin{frame}
\frametitle{Управляющие конструкции} 
\framesubtitle{Оператор continue}

\lstinputlisting{./programms/code_continue.c}
\end{frame}

\begin{frame}
\frametitle{Управляющие конструкции} 
\framesubtitle{Оператор выбора switch}

\lstinputlisting{./programms/code_switch.c}
\end{frame}



\section{Оператор goto}


\begin{frame}
\frametitle{Управляющие конструкции} 
\framesubtitle{Оператор безусловного перехода goto}

\begin{itemize}
\item Оператор goto передает управление на оператор, помеченный меткой
\item Оператор goto в языках высокого уровня является объектом критики, поскольку чрезмерное его применение приводит к созданию нечитаемого кода
\item Использование goto в практике программирования на языке C настоятельно не рекомендуется
\end{itemize}


\end{frame}

\section{Стиль кода}


\begin{frame}
\frametitle{Комментарии}
\lstinputlisting{./programms/comments.c}

\end{frame}

\begin{frame}
\frametitle{Стиль кода} 

\begin{itemize}
\item Отступы. В программе должна быть структура.
Количество отступов соответствует уровню вложенности. Уровень вложенности увеличивается внутри { }, а также в телах операторов if, for, while, do-while, switch
\item Каждый отступ - это ЛИБО TAB, ЛИБО n пробелов (лучше всего n = 4). Мешать их нельзя.
\item Скобка \{ должна быть на следующей строке, под началом ключевого слова if/for/.
\end{itemize}


\end{frame}


\begin{frame}
\frametitle{Стиль кода} 

\begin{itemize}
\item Скобка \} должна быть СТРОГО под соответствующей \{. После неё не должно быть ничего, за исключением комментариев.
\item Каждый оператор (особенно, содержащий ключевое слово) должен быть с новой строки, после оператора и знака ; не должно быть ничего, кроме комментариев.
\item Пробелы должны быть ПОСЛЕ , и ;(в цикле for) , а до них они НЕ нужны.
\end{itemize}
\end{frame}


\end{document}
