\documentclass{article}
\usepackage[utf8x]{inputenc}
\usepackage{ucs}
\usepackage{amsmath} 
\usepackage{amsfonts}
\usepackage{marvosym}
\usepackage{wasysym}
\usepackage{upgreek}
\usepackage[english,russian]{babel}
\usepackage{graphicx}
\usepackage{float}
\usepackage{textcomp}
\usepackage{hyperref}
\usepackage{geometry}
  \geometry{left=2cm}
  \geometry{right=1.5cm}
  \geometry{top=1cm}
  \geometry{bottom=2cm}
\usepackage{tikz}
\usepackage{ccaption}
\usepackage{multicol}
\usepackage{hyperref}



\usepackage{listings}
%\setlength{\columnsep}{1.5cm}
%\setlength{\columnseprule}{0.2pt}

\usepackage[absolute]{textpos}

\begin{document}
\pagenumbering{gobble}

\lstset{
  language=C,                % choose the language of the code
  basicstyle=\linespread{1.1}\ttfamily,
  columns=fixed,
  fontadjust=true,
  basewidth=0.5em,
  keywordstyle=\color{blue}\bfseries,
  commentstyle=\color{gray},
  stringstyle=\ttfamily\color{orange!50!black},
  showstringspaces=false,
  numbersep=5pt,
  numberstyle=\tiny\color{black},
  numberfirstline=true,
  stepnumber=1,                   % the step between two line-numbers.        
  numbersep=10pt,                  % how far the line-numbers are from the code
  backgroundcolor=\color{white},  % choose the background color. You must add \usepackage{color}
  showstringspaces=false,         % underline spaces within strings
  captionpos=b,                   % sets the caption-position to bottom
  breaklines=true,                % sets automatic line breaking
  breakatwhitespace=true,         % sets if automatic breaks should only happen at whitespace
  xleftmargin=.2in,
  extendedchars=\true,
  keepspaces = true,
}
\lstset{literate=%
   *{0}{{{\color{red!20!violet}0}}}1
    {1}{{{\color{red!20!violet}1}}}1
    {2}{{{\color{red!20!violet}2}}}1
    {3}{{{\color{red!20!violet}3}}}1
    {4}{{{\color{red!20!violet}4}}}1
    {5}{{{\color{red!20!violet}5}}}1
    {6}{{{\color{red!20!violet}6}}}1
    {7}{{{\color{red!20!violet}7}}}1
    {8}{{{\color{red!20!violet}8}}}1
    {9}{{{\color{red!20!violet}9}}}1
}

\title{Семинар \#2: Функции. Классные задачи.\vspace{-5ex}}\date{}\maketitle
\section*{Работа с терминалом:}
\subsection*{Основные команды терминала:}
\texttt{
\begin{tabular}{ c | c }
 pwd                     & напечатать имя текущей директории \\ 
 ls                      &  напечатать все файлы и папки текущей директории \\ 
 ls -l                   &  то же, что и ls, но больше информации о файлах \\
 cd <имя папки>          & перейти в соответствующую папку \\
                         & например:  cd /home-local/student \\
 mkdir <имя новой папки> & создать новую папку \\
 rm <путь до файла>                 & удалить файл \\
 gcc <имя файла исходного кода>          & скомпилировать и сохранить в исполняемый файл a.out \\
 ./a.out                            & запустить исполняемый файл a.out \\
 gcc <имя исходного файла> \&\& ./a.out & скомпилировать и запустить \\
\end{tabular}
}
\subsection*{Задание 1:}
\begin{enumerate}
\item Перейдите в терминале в вашу рабочую папку
\item Создайте в ней файл \texttt{function.c} с содержимым:
\begin{lstlisting}
#include <stdio.h>
void hello()
{
	printf("Hello\n");
}	
int main()
{
	for (int i = 0; i < 10; i++)
		hello();
}
\end{lstlisting}
Можно создать с помощью \texttt{nano} или \texttt{gedit}.
\item Скомпилируйте этот файл и запустите.
\end{enumerate}

\section*{Функции без возвращаемого значения:}
Пример программы, которая печатает натуральные числа от \texttt{1} до \texttt{n}:
\begin{lstlisting}
#include <stdio.h>
void print_numbers(int n)
{
	for (int i = 0; i < n; i++)
		printf("%d ", i);
}	
int main()
{
	// Тут нужно дописать
}
\end{lstlisting}
\begin{enumerate}
\item Используйте функцию \texttt{print\_numbers}, чтобы вывести на экран все числа от \texttt{0} до \texttt{200}.
\item Напишите функцию \texttt{print\_even\_numbers(int a, int b)}, которая будет печатать все четные числа от \texttt{a} до \texttt{b}. Вызовите эту функцию из функции \texttt{main}.
\item Напишите функцию \texttt{print\_many\_times(int a, int b)}, которая будет печатать число \texttt{a} \texttt{b} раз. Вызовите эту функцию из функции \texttt{main}.
\item Напишите функцию \texttt{print\_rectangle(int a, int b)}, которая будет печатать прямоугольник из звёздочек \texttt{*}.  Например, если эта функция будет вызвана с аргументами \texttt{4} и \texttt{3}, то функция должна напечатать:
\begin{verbatim}
****
****
****
\end{verbatim}
Вызовите эту функцию из функции \texttt{main} с различными аргументами.
\item Напишите функцию \texttt{void multi(int type, int a, int b)}, которая, в зависимости от переменной \texttt{type}, должна делать различные вещи. При \texttt{type == 1}, она должна вызывать функцию  \texttt{print\_even\_numbers}. При \texttt{type == 2}, она должна вызывать функцию  \texttt{print\_many\_times}. При \texttt{type == 3}, она должна вызывать функцию  \texttt{print\_rectangle}. При ином другом значении \texttt{type}, она должна просто печатать \texttt{Error!}. Протестируйте вашу функцию.
\end{enumerate}

\section*{Функции c возвращаемым значением:}
Пример функции, которая принимает на вход 2 числа и возвращает максимальное из них:
\begin{lstlisting}
#include <stdio.h>
// Функция возвращает значение типа int
int max(int a, int b)
{
	if (a > b)
		return a;
	else
		return b;
}
int main()
{
	printf("%d\n", max(10, 40));  // На место max(10, 40) подставится результат вычисления
	printf("%d\n", max(5, -14));
}
\end{lstlisting}
\begin{enumerate}


\item \textbf{Минимум:} Напишите функцию \texttt{int min(int a, int b)} и протестируйте её.

\item \textbf{Минимум из трёх:} Напишите функцию \texttt{int min(int a, int b, int c)} и протестируйте её. Подсказка: можно использовать функцию из предыдущей задачи.

\item \textbf{Doubler:} Написать функцию \texttt{doubler}, которая принимает число и возвращает это число, умноженное на 2. Вызвать эту функцию из функции main:
\begin{lstlisting}
// Тут нужно написать функцию doubler
int main()
{
	int a;
	scanf("%d", &a);
	printf("%d\n", doubler(a));
}
\end{lstlisting}

\item \textbf{Sum:} Напишите функцию \texttt{int sum(int n)}, которая вычисляет сумму первых n натуральных чисел. Вызовите эту функцию из main и протестируйте на следующих значениях: \texttt{10} (ответ - \texttt{55}), \texttt{100} (ответ - \texttt{5050}) и \texttt{1234} (ответ - \texttt{761995}).

\newpage
\item \textbf{Среднее геометрическое}
Пример программы, которая вычисляет среднее геометрическое (при компиляции не забудьте флаг \texttt{-lm})
\begin{lstlisting}
#include <stdio.h>
#include <math.h>
float geometric_mean(float a, float b)
{
	return sqrt(a * b);
}
int main()
{
	float x = geometric_mean(10, 45);
	printf("%f\n", x);
}
\end{lstlisting}
Видоизмените эту программу так, чтобы она считывала 2 вещественных числа и печатала среднее геометрическое.




\item \textbf{Is prime?:} Написать функцию \texttt{int is\_prime(int n)}, которая будет проверять является ли число n простым и возвращать \texttt{1} если число \texttt{n} простое либо \texttt{0} если число \texttt{n} не является простым.
\item \textbf{Print primes:} Написать функцию \texttt{void print\_primes(int a, int b)}, которая будет печатать все простые числа из отрезка [a, b]. Используйте функцию \texttt{is\_prime} из предыдущей задачи! Вызвать эту функцию в функции \texttt{main}.
\end{enumerate}

\section*{Рекурсия:}
Что напечатает следующая программа?
\begin{lstlisting}
#include <stdio.h>
void hello(int n)
{
	if (n < 0)
		return;
	printf("Hello!\n");
	hello(n - 1);
}
int main()
{
	hello(10);
}
\end{lstlisting}
Что произойдёт если убрать условие \texttt{if (n < 0) return;}?

\begin{enumerate}
\item \textbf{Sum recursive:} Напишите функцию \texttt{int sumrec(int n)}, которая рекурсивно вычисляет сумму первых n натуральных чисел. Вызовите эту функцию из main.

\item \textbf{Counter:} Написана рекурсивная функция \texttt{void counter(int n)}, которая печатает числа от n до 1.
\begin{lstlisting}
#include <stdio.h>
void counter(int n) {
	if (n > 0) {
		printf("%d ", n);
		counter(n-1);
	}
}
int main() {
	counter(42);
}
\end{lstlisting}
Немного измените эту функцию так, чтобы она печатала числа от 1 до n. (в нормальном порядке).
\end{enumerate}


\section*{Указатели:}
\begin{lstlisting}
#include <stdio.h>
int main()
{
	int a = 100;
	int* address = &a;   // адрес переменной a, хранится в переменной address
	
	// Чтобы по адресу получить саму переменную, нужно поставить * перед
	// *address это то же самое, что и a
	*address = 321;
	
	printf("%d\n", a);     
}
\end{lstlisting}
\begin{enumerate}
\item \textbf{Pointer:} Создайте переменную \texttt{b} типа \texttt{float} и присвойте ей какое-либо значение. Создайте переменную \texttt{p} типа указатель на \texttt{float} (\texttt{p} - это сокращение от pointer - указатель)  и присвойте ей значение -- адрес переменной \texttt{b}. Измените переменную \texttt{b}, используя только переменную \texttt{p}.
\item \textbf{Pointer to pointer:} Создайте переменную \texttt{pp} и присвойте ей значение -- адрес переменной \texttt{p}. Измените переменную \texttt{b}, используя только переменную \texttt{pp}.
\end{enumerate}

\section*{Передача по указателю:}
\begin{lstlisting}
#include <stdio.h>
void add10_wrong(int a) // По значению
{
	a += 10;
}
void add10_right(int* address_of_a) // По указателю
{
	*address_of_a += 10;
}
int main()
{
	int a = 80;
	add10_wrong(a);
	printf("%d\n", a); 
	    
	add10_right(&a);
	printf("%d\n", a); 
}
\end{lstlisting}

\begin{enumerate}
\item \textbf{Modify:} Написать функцию \texttt{void doublerp(int* address)}, которая удваивает число, поступающее на вход, используя указатель на эту переменную. Протестируйте эту функцию в функции \texttt{main}.
\item \textbf{Swap:} Написать функцию \texttt{swap}, которая меняет значения 2-х переменных типа int местами. Используйте эту функцию в функции \texttt{main()}.
\item \textbf{Quadratic:} Написать функцию \\
\texttt{int solve\_quadratic(double a, double b, double c, double* px1, double* px2)},\\
которая будет решать квадратное уравнение. Результат решения функция должна записывать по адресам \texttt{px1} и \texttt{px2}. Функция должна возвращать целое число - количество корней данного уравнения. Протестировать вашу функцию.
\end{enumerate}

\end{document}



\iffalse
\item \textbf{Фаренгейты в цельсии:} На вход подаются температуры плавления и кипения в Фаренгейтах некого вещества. Вам нужно узнать будет ли оно жидкостью при комнатной температуре (25 градусов цельсия) и напечатать температуры плавления и кипения в цельсиях.\\ Формула для перевода Цельсий в Фаренгейты: $T_c = \frac{5}{9}(T_f-32)$. Была написана следующую программу:
\begin{lstlisting}
#include <stdio.h>
int main()
{
	float Tf1, Tf2;
	scanf("%f%f", &Tf1, &Tf2);
	float Tc_room = 25;
	
	if (Tc_room > 5.0/9.0*(Tf1-32) && Tc_room < 5.0/9.0*(Tf2-32))
		printf("YES\n");
	else
		printf("NO\n");
	
	printf("Temepratures in Celsius %f %f\n", 5.0/9.0*(Tf1-32), 5.0/9.0*(Tf2-32));
}
\end{lstlisting}
Упросите программу, добавив функцию \texttt{float ftoc(float tf)}, которая будет принимать на вход температуру в фаренгейтах и возвращать её в значение в цельсиях.





\item \textbf{Функция, которая принимает, возвращает, считывает и печатает:} Напишите функцию \texttt{float combiner(float a)}, которая принимает вещественное число a, считывает вещественное число b с помощью scanf, печатает их среднее арифметическое($(a+b)/2$) и возвращает их среднее геометрическое ($\sqrt{a b}$). Используйте эту функцию в функции main, чтобы найти среднее арифметическое и среднее геометрическое чисел 42 и 256.


\subsection*{Задачи на функции. Функции и массивы:}
На вход подаётся натуральное число n и n целых чисел. считайте этот массив в функции main:
\begin{lstlisting}
int main()
{
	int n;
	int arr[1000];
	scanf("%d", &n);
	for (int i = 0; i < n; ++i)
		scanf("%d", &arr[i]);
}
\end{lstlisting}
\item \textbf{Array print:} Напишите функцию \texttt{void print\_array(int n, int arr[])}, которая принимает на вход массив и печатает все его числа через пробел. В конце строки напечатайте перенос на новую строку.
\item \textbf{Array sum:} Напишите функцию \texttt{int sum\_array(int n, int arr[])}, которая принимает на вход массив и возвращает сумму массива. Вызовите эту функцию в main(), чтобы найти сумму массива. Сама функция не должна ничего печатать.
\item \textbf{Array max:} Напишите функцию \texttt{int max\_array(int n, int arr[])}, которая принимает на вход массив и возвращает максимальный элемент массива. Вызовите эту функцию в main(), чтобы найти сумму массива. Сама функция не должна ничего печатать.
\item \textbf{Double array:} Напишите функцию \texttt{void double\_array(int n, int arr[])}, которая увеличивает каждый элемент массива в 2 раза. Вызовите эту функцию в main(). Напечатайте массив до и после увеличения в 2 раза, используя функцию print\_array.
\item \textbf{Sort array:} Напишите функцию \texttt{void sort\_array(int n, int arr[])}, которая сортирует массив методом выбора. Вызовите эту функцию в main(). Напечатайте массив до и после сортировки, используя print\_array.

\fi