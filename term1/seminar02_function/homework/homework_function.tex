\documentclass{article}
\usepackage[utf8x]{inputenc}
\usepackage{ucs}
\usepackage{amsmath} 
\usepackage{amsfonts}
\usepackage{marvosym}
\usepackage{wasysym}
\usepackage{upgreek}
\usepackage[english,russian]{babel}
\usepackage{graphicx}
\usepackage{float}
\usepackage{textcomp}
\usepackage{hyperref}
\usepackage{geometry}
  \geometry{left=2cm}
  \geometry{right=1.5cm}
  \geometry{top=1cm}
  \geometry{bottom=2cm}
\usepackage{tikz}
\usepackage{ccaption}
\usepackage{multicol}
\usepackage{hyperref}

\makeatletter
\def\@bignumber#1#2{%
  \ifx#2\end
    #1\let\next\@gobble
  \else
    #1\hspace{0pt plus 1pt}\let\next\@bignumber
  \fi
  \next#2}
\newcommand{\bignumber}[1]{\@bignumber#1\end}
\makeatother

\usepackage{listings}
%\setlength{\columnsep}{1.5cm}
%\setlength{\columnseprule}{0.2pt}

\usepackage[absolute]{textpos}

\begin{document}
\pagenumbering{gobble}

\lstset{
  language=C,                % choose the language of the code
  basicstyle=\linespread{1.1}\ttfamily,
  columns=fixed,
  fontadjust=true,
  basewidth=0.5em,
  keywordstyle=\color{blue}\bfseries,
  commentstyle=\color{gray},
  stringstyle=\ttfamily\color{orange!50!black},
  showstringspaces=false,
  numbersep=5pt,
  numberstyle=\tiny\color{black},
  numberfirstline=true,
  stepnumber=1,                   % the step between two line-numbers.        
  numbersep=10pt,                  % how far the line-numbers are from the code
  backgroundcolor=\color{white},  % choose the background color. You must add \usepackage{color}
  showstringspaces=false,         % underline spaces within strings
  captionpos=b,                   % sets the caption-position to bottom
  breaklines=true,                % sets automatic line breaking
  breakatwhitespace=true,         % sets if automatic breaks should only happen at whitespace
  xleftmargin=.2in,
  extendedchars=\true,
  keepspaces = true,
}
\lstset{literate=%
   *{0}{{{\color{red!20!violet}0}}}1
    {1}{{{\color{red!20!violet}1}}}1
    {2}{{{\color{red!20!violet}2}}}1
    {3}{{{\color{red!20!violet}3}}}1
    {4}{{{\color{red!20!violet}4}}}1
    {5}{{{\color{red!20!violet}5}}}1
    {6}{{{\color{red!20!violet}6}}}1
    {7}{{{\color{red!20!violet}7}}}1
    {8}{{{\color{red!20!violet}8}}}1
    {9}{{{\color{red!20!violet}9}}}1
}

\title{Семинар \#2: Функции. Домашнее задание.\vspace{-5ex}}\date{}\maketitle

\section*{Функции без возвращаемого значения:}
\begin{itemize}
\item \textbf{Задача 1 -- Двоичный код наоборот:} Как известно, чтобы напечатать на экран число в шестнадцатиричном или восьмеричном виде, можно использовать спецификаторы \texttt{\%x} или \texttt{\%o}. Однако, для двоичного представления числа такого спецификатора нет. Напишите функцию \texttt{void print\_binary\_backwards(int number)}, которая будет печатать число в двоичном виде наоборот (так проще).
\begin{center}
\begin{tabular}{ c c }
 вход & выход \\ \hline
 2 & 01  \\ 
 10 & 0101  \\
 97 & 1000011 \\
 1234567 & 111000010110101101001\\
\end{tabular}
\end{center}
\end{itemize}

\section*{Функции с возвращаемым значением:}
Пример функции, которая принимает на вход 3 числа и возвращает максимальное из них:
\begin{lstlisting}
#include <stdio.h>

int max3(int a, int b)
{
	int result = a;
	if (b > result)
		result = b;
	if (c > result)
		result = c;
	return result;
}
int main()
{
	printf("%d\n", max3(10, 40, -5));  // На место max(10, 40, -5) подставится результат
}
\end{lstlisting}
\begin{itemize}
\item \textbf{Задача 2 -- Среднее гармоническое:}  Написать функцию, \texttt{float harmonicmean(float a, float b)}, которая будет вычислять среднее гармоническое. Среднее гармоническое $h$ вычисляется из формулы 
$$\frac{1}{h} = \frac{1}{2}\Big(\frac{1}{a} + \frac{1}{b}\Big)$$

\item \textbf{Задача 3 -- Гамма-функция:} Гамма-функция -- это обобщение понятия факториала на вещественные числа. Определяется следующим образом:
$$
\Gamma \left( x \right) = \int\limits_0^\infty {t^{x - 1} e^{ - t} dt}
$$
Легко вывести, что $\Gamma(n) = (n - 1)!$ для натуральных $n$. Написать функцию, \texttt{double gamma(double x)}, которая будет вычислять значение гамма-функции в точке $x$, при $x > 1$. Для вычисления интеграла использовать метод трапеций с шагом \texttt{step}. Суммирование продолжать до тех пор пока площадь трапеции превышает \texttt{eps = 1e-10} (то есть $10 ^{-10}$). \texttt{step} и \texttt{eps} задать как константы. Понадобятся функции \texttt{pow}(возведение в степень) и \texttt{exp}(экспонента) из библиотеки \texttt{math.h}. Тесты для проверки:

\begin{center}
\begin{tabular}{ c c }
 вход & выход \\ \hline
 2 & 1.0  \\ 
 8 & 5040.0  \\
 20 & 1.216451e+17  \\
 1.5 &        0.8862269255 \\
 2.5 &        1.3293403881\\
 4.14159265 & 7.1880826955\\
\end{tabular}
\end{center}

\item \textbf{Задача 4 -- A500:} Напишите программу, которая будет печатать на экран все числа от $0$ до $10^{7}$, которые одновременно в троичном, четвертичном и пятеричном представлении имеют в своей записи лишь нули и единицы. Для решения этой задачи напишите функцию \texttt{int check\_if\_only\_01(int num, int basse)}, которая будет проверять, что в записи числа \texttt{num} по основанию \texttt{base} есть только нули и единицы и будет возвращать \texttt{1}, если это так и \texttt{0} иначе.
\end{itemize}

\section*{Рекурсия:}
\begin{itemize}
\item \textbf{Задача 5 -- Двоичный код:} Напишите функцию \texttt{void print\_binary(int number)}, которая будет печатать число в двоичном виде, используя рекурсию.
\begin{center}
\begin{tabular}{ c c }
 вход & выход \\ \hline
 2 & 10  \\ 
 10 & 1010  \\
 97 & 1100001 \\
 1234567 & 100101101011010000111\\
\end{tabular}
\end{center}

\item \textbf{Задача 6 -- Число Фибоначчи:} Для вычисления чисел Фибоначчи написана следующая программа.
\begin{lstlisting}
#include <stdio.h>
unsigned long long fib(unsigned int n)
{
    if (n < 2)
        return n;
    else
        return fib(n - 1) + fib(n - 2);
}
int main()
{    
    printf("%llu\n", fib(60));
}
\end{lstlisting}
Однако, в этой программе есть ошибка. При запуске программа зависает и ничего не печатает.  Известно, что 60-е число Фибоначчи помещается в тип \texttt{unsigned long long} (вообще, в этот тип помещаются все числа Фибоначчи до номера \texttt{93} включительно), так что проблема не в переполнении. Найдите в чём причина ошибки. Опишите словесно эту причину и напишите работающую версию функции, вычисляющей числа Фибоначчи. Тесты для проверки:

\begin{center}
\begin{tabular}{ c c }
 вход & выход \\ \hline
 5 & 5  \\ 
 20 & 6765 \\ 
 50 & 12586269025 \\ 
 60 & 1548008755920 \\ 
 75 & 2111485077978050 \\
 93 & 12200160415121876738 \\   
\end{tabular}
\end{center}
\end{itemize}

\newpage
\section*{Передача по указателю:}
\begin{lstlisting}
#include <stdio.h>
int main() 
{ 
    // Предположим у нас есть переменная:
    int x = 42; 
    // Положение этой переменной в памяти характеризуется двумя числами - адресом и 
    // размером переменной. Узнать их можно, используя операторы & и sizeof:
    printf("Size and address of x = %d and %llu\n",  sizeof(x), &x);
    // Обратите внимание, что для отображения адреса использовался модификатор %llu, так 
    // как в 64 битных системах адрес это 64 битное число. Также можно было бы 
    // использовать модификатор %p, для отображения адреса в 16 - ричном виде.
    
    // Для работы с адресами в языке C вводится специальный тип, который называется 
    // указатель. Введём переменную для хранения адреса переменной x:
    int* address_of_x = &x;
    // Теперь в переменной address_of_x типа int* будет храниться число - адрес 
    // переменной x. Если бы x был бы не int, а float, то для хранения адреса x нужно было 
    // бы использовать тип float* .
    
    // Ну хорошо, у нас есть переменная, которая хранит адрес x. Как её дальше 
    // использовать? Очень просто - поставьте звёздочку перед адресом, чтобы получить 
    // переменную x.
    // *address_of_x это то же самое, что и  x
    *address_of_x += 10;
    printf("%d\n", x);
    // Запомните: &  -  по переменной получить адрес
    //            *  -  по адресу получить переменную
} 
\end{lstlisting}

Как использовать указатели для передачи в функции адреса переменной: \\
\begin{lstlisting}
#include <stdio.h>
// Эта функция не удвоит значение, так как функция работает с копией передаваемого параметра
void doubler_naive(int x)
{
	x *= 2;
}

// Эта функция сработает, но таким образом можно изменить только одну переменную за раз.
// К тому же, тут происходит 2 лишних копирования переменной x. Что может быть плохо, если 
// переменная x будет не типа int, а, например, структурой большого размера.
int doubler(int x) 
{
	return 2*x;
}

// Лучший способ - передача по адресу
void doubler_by_address(int* address_of_x) 
{
	*address_of_x *= 2;
}

int main() 
{
	int x = 79;
	printf("%d\n", x);
	doubler_naive(x);
	printf("%d\n", x);
	x = doubler(x);
	printf("%d\n", x);
	doubler_by_address(&x);
	printf("%d\n", x);
}
\end{lstlisting}
\begin{itemize}
\item \textbf{Задача 7 -- Меняем переменную по адресу:} Пусть в функции main() определена переменная \texttt{float\ x = 4.53}. Вам нужно ввести создать переменную типа \texttt{float*} и сохранить в ней адрес x. А затем увеличить x в 2 раза, используя только указатель.
\item \textbf{Задача 8 -- Куб:} Напишите функции \texttt{float cube1(float x)} и \texttt{void cube2(float* address\_of\_x)}, которые будут возводить значение переменной в куб двумя разными методами. Используйте обе функции в функции main() и проверьте их работу. 
\item \textbf{Задача 9 -- Перестановка:} Напишите функцию \texttt{void swap(int* address\_of\_a, int* address\_of\_b)}, которая меняет значения 2-х переменных типа int местами. Используйте эту функцию в функции \texttt{main()} и проверьте её работу. 
\item \textbf{Задача 10 -- Квадратное уравнение:} Написать функцию\\
 \texttt{int solve\_quadratic(double a, double b, double c, double* px1, double* px2)},\\
которая будет решать квадратное уравнение. Функция должна возвращать целое число - количество корней данного уравнения. Результат решения функция должна записывать по адресам \texttt{px1} и \texttt{px2}. Если число корней равно нулю, то по этим адресам не нужно ничего записывать. Если число корней равно одному, то этот корень нужно записать по адресу \texttt{px1}. Используйте эту функцию в программе. Пользователь вводит 3 числа, программа должна напечатать:
\begin{itemize}
\item \texttt{No roots}, если корней нет
\item \texttt{One root: 1.4}, если есть 1 корень, равный 1.4 (к примеру)
\item \texttt{Two roots: 3.1 and 8.7}, если есть 2 корня, равные 3.1 и 8.7 (к примеру)
\end{itemize}
\end{itemize}



\end{document}