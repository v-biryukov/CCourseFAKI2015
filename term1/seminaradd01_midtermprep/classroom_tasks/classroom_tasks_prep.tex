\documentclass{article}
\usepackage[utf8x]{inputenc}
\usepackage{ucs}
\usepackage{amsmath} 
\usepackage{amsfonts}
\usepackage{marvosym}
\usepackage{wasysym}
\usepackage{upgreek}
\usepackage[english,russian]{babel}
\usepackage{graphicx}
\usepackage{float}
\usepackage{textcomp}
\usepackage{hyperref}
\usepackage{geometry}
  \geometry{left=2cm}
  \geometry{right=1.5cm}
  \geometry{top=1cm}
  \geometry{bottom=2cm}
\usepackage{tikz}
\usepackage{ccaption}
\usepackage{multicol}

\hypersetup{
   colorlinks=true,
   citecolor=blue,
   linkcolor=black,
   urlcolor=blue
}

\usepackage{listings}
%\setlength{\columnsep}{1.5cm}
%\setlength{\columnseprule}{0.2pt}

\usepackage[absolute]{textpos}

\usepackage{colortbl,graphicx,tikz}
\definecolor{X}{rgb}{.5,.5,.5}

\renewcommand{\thesubsection}{\arabic{subsection}}

\begin{document}
\pagenumbering{gobble}
\lstset{
  language=C,                % choose the language of the code
  basicstyle=\linespread{1.1}\ttfamily,
  columns=fixed,
  fontadjust=true,
  basewidth=0.5em,
  keywordstyle=\color{blue}\bfseries,
  commentstyle=\color{gray},
  stringstyle=\ttfamily\color{orange!50!black},
  showstringspaces=false,
  numbersep=5pt,
  numberstyle=\tiny\color{black},
  numberfirstline=true,
  stepnumber=1,                   % the step between two line-numbers.        
  numbersep=10pt,                  % how far the line-numbers are from the code
  backgroundcolor=\color{white},  % choose the background color. You must add \usepackage{color}
  showstringspaces=false,         % underline spaces within strings
  captionpos=b,                   % sets the caption-position to bottom
  breaklines=true,                % sets automatic line breaking
  breakatwhitespace=true,         % sets if automatic breaks should only happen at whitespace
  xleftmargin=.2in,
  extendedchars=\true,
  keepspaces = true,
}
\lstset{literate=%
   *{0}{{{\color{red!20!violet}0}}}1
    {1}{{{\color{red!20!violet}1}}}1
    {2}{{{\color{red!20!violet}2}}}1
    {3}{{{\color{red!20!violet}3}}}1
    {4}{{{\color{red!20!violet}4}}}1
    {5}{{{\color{red!20!violet}5}}}1
    {6}{{{\color{red!20!violet}6}}}1
    {7}{{{\color{red!20!violet}7}}}1
    {8}{{{\color{red!20!violet}8}}}1
    {9}{{{\color{red!20!violet}9}}}1
}


\title{Семинар \#7: Повторение. Классные задачи.\vspace{-5ex}}\date{}\maketitle


\subsection*{Основы}
\begin{itemize}
\item На вход подаются 2 целых числа \texttt{a} и \texttt{b}. Напечатайте сначала \texttt{b}, а потом \texttt{a} через пробел.
\item На вход подаются 2 целых числа \texttt{a} и \texttt{b}. Напечатайте остаток деления первого числа на второе.
\item На вход подаются 3 целых числа. Напечатайте \texttt{Yes}, если третье число является суммой двух первых.
\item На вход подаются 2 целых числа \texttt{a} и \texttt{b}. Напечатайте наибольшее из этих чисел.
\item На вход подаются 2 целых числа \texttt{a} и \texttt{b}. Напечатайте все числа от наибольшего из этих чисел до наименьшего.
\begin{center}
\begin{tabular}{ l | l }
 вход & выход \\ \hline
 \texttt{2 8} & \texttt{8 7 6 5 4 3 2}  \\ \hline
 \texttt{9 6} & \texttt{9 8 7 6}  \\ 
\end{tabular}
\end{center}
\item На вход поступает число \texttt{n} и, затем, \texttt{n} целых чисел. Напечатайте сумму этих \texttt{n} чисел.
\begin{center}
\begin{tabular}{ l | l }
 вход & выход \\ \hline
 \texttt{3} & \texttt{11}  \\ 
 \texttt{7 3 1} &  \\ 
\end{tabular}
\end{center}

\item На вход поступает число \texttt{n} и, затем, \texttt{n} целых чисел. Напечатайте наибольшее из этих \texttt{n} чисел.
\begin{center}
\begin{tabular}{ l | l }
 вход & выход \\ \hline
 \texttt{4} & \texttt{8}  \\ 
 \texttt{7 3 8 2} &  \\ 
\end{tabular}
\end{center}

\item На вход поступает число \texttt{n} и, затем, \texttt{n} целых чисел. Напечатайте сумму первого и последнего элемента последовательности.
\begin{center}
\begin{tabular}{ l | l }
 вход & выход \\ \hline
 \texttt{3} & \texttt{12}  \\ 
 \texttt{7 3 5} &  \\ \hline
 \texttt{4} & \texttt{9}  \\ 
 \texttt{5 8 2 4} &  \\
\end{tabular}
\end{center}
\end{itemize}

\subsection*{Переполнение}
\begin{itemize}
\item На вход подаётся 1 целое число \texttt{a} из диапазона от $0$ до $2^{64} - 2$. Напечатайте число, которое на 1 больше.
\begin{center}
\begin{tabular}{ l | l }
 вход & выход \\ \hline
 \texttt{5} & \texttt{6}  \\ \hline
 \texttt{123456789123} & \texttt{123456789124}  \\ 
\end{tabular}
\end{center}
\item На вход подаются 2 целых числа из диапазона от $0$ до $2^{32} - 1$. Напечатайте их произведение.
\begin{center}
\begin{tabular}{ l | l }
 вход & выход \\ \hline
 \texttt{2 2} & \texttt{4}  \\ \hline
  \texttt{123456789 1000000} & \texttt{123456789000000}  \\ \hline
 \texttt{123456789 123456789} & \texttt{15241578750190521}  \\ 
\end{tabular}
\end{center}
\end{itemize}

\subsection*{Вещественные числа}
\begin{itemize}
\item На вход подаются 2 вещественных числа. Напечатайте их сумму.
\item На вход подаются 2 вещественных числа \texttt{x} и \texttt{y}. Напечатайте \texttt{Yes} если точка \texttt{(x, y)} попадает внутрь единичной окружности и \texttt{No} иначе.
\begin{center}
\begin{tabular}{ l | l }
 вход & выход \\ \hline
 \texttt{0.5 -0.5} & \texttt{Yes}  \\ \hline
 \texttt{0.7 0.7} & \texttt{Yes}  \\ \hline
 \texttt{0.7 0.8} & \texttt{No}  \\ 
\end{tabular}
\end{center}
\item На вход подаётся 1 вещественное число \texttt{a} -- значение угла в градусах. Напечатайте значение выражения $sin(a) \cdot tan(a)$.
\begin{center}
\begin{tabular}{ l | l }
 вход & выход \\ \hline
 \texttt{45} & \texttt{0.707}  \\ \hline
 \texttt{10} & \texttt{0.031}  \\ \hline
 \texttt{80} & \texttt{5.585}  \\ 
\end{tabular}
\end{center}
\end{itemize}

\subsection*{Массивы}
\begin{itemize}
\item На вход поступает число \texttt{n} и, затем, \texttt{n} целых чисел. Напечатайте эту последовательность 2 раза.
\begin{center}
\begin{tabular}{ l | l }
 вход & выход \\ \hline
 \texttt{3} & \texttt{7 3 1 7 3 1}  \\ 
 \texttt{7 3 1} &  \\ 
\end{tabular}
\end{center}

\item На вход поступает число \texttt{n} и, затем, \texttt{n} целых чисел. Напечатайте эту последовательность 2 раза. Первый раз в нормальном порядке, второй раз -- в обратном.
\begin{center}
\begin{tabular}{ l | l }
 вход & выход \\ \hline
 \texttt{3} & \texttt{7 3 1 1 3 7}  \\ 
 \texttt{7 3 1} &  \\ 
\end{tabular}
\end{center}

\item На вход поступает число \texttt{n} и, затем, \texttt{n} целых чисел. Напечатайте эту последовательность в обратном порядке, повторив каждое число дважды.
\begin{center}
\begin{tabular}{ l | l }
 вход & выход \\ \hline
 \texttt{3} & \texttt{1 1 3 3 7 7}  \\ 
 \texttt{7 3 1} &  \\ 
\end{tabular}
\end{center}


\item На вход поступает число \texttt{n} и, затем, \texttt{n} целых чисел. Ещё на вход приходит целое число \texttt{k}. Напечатайте эту последовательность в обратном порядке, повторив каждое число \texttt{k} раз.
\begin{center}
\begin{tabular}{ l | l }
 вход & выход \\ \hline
 \texttt{3} & \texttt{1 1 1 1 3 3 3 3 7 7 7 7}  \\ 
 \texttt{7 3 1} &  \\ 
 \texttt{4} &  \\
\end{tabular}
\end{center}

\item На вход поступает число \texttt{n} и, затем, две последовательности по \texttt{n} целых чисел каждая. Сложите эти две последовательности поэлементно и напечатайте её.
\begin{center}
\begin{tabular}{ l | l }
 вход & выход \\ \hline
 \texttt{4} & \texttt{12 4 10 4}  \\ 
 \texttt{7 3 1 2} &  \\ 
 \texttt{5 1 9 2} &  \\ 
\end{tabular}
\end{center}
\end{itemize}


\subsection*{Двумерные массивы}
\begin{itemize}
\item На вход поступают числа \texttt{n} и \texttt{m} и, затем, матрица целых чисел размера \texttt{n} строк на \texttt{m} столбцов. Напечатайте все суммы строк.
\begin{center}
\begin{tabular}{ l | l }
 вход & выход \\ \hline
 \texttt{3 4} & \texttt{13 17 18}  \\ 
 \texttt{7 3 1 2} &  \\ 
 \texttt{5 1 9 2} &  \\ 
 \texttt{7 2 5 4} &  \\ 
\end{tabular}
\end{center}

\item На вход поступают числа \texttt{n} и \texttt{m} и, затем, матрица целых чисел размера \texttt{n} строк на \texttt{m} столбцов. Напечатайте все суммы столбцов.
\begin{center}
\begin{tabular}{ l | l }
 вход & выход \\ \hline
 \texttt{3 4} & \texttt{19 6 15 8}  \\ 
 \texttt{7 3 1 2} &  \\ 
 \texttt{5 1 9 2} &  \\ 
 \texttt{7 2 5 4} &  \\ 
\end{tabular}
\end{center}

\item На вход поступают числа \texttt{n} и \texttt{m} и, затем, матрица целых чисел размера \texttt{n} строк на \texttt{m} столбцов. Напечатайте наибольшиий элемент этой матрицы.
\begin{center}
\begin{tabular}{ l | l }
 вход & выход \\ \hline
 \texttt{3 4} & \texttt{9}  \\ 
 \texttt{7 3 1 2} &  \\ 
 \texttt{5 1 9 2} &  \\ 
 \texttt{7 2 5 4} &  \\ 
\end{tabular}
\end{center}

\item На вход поступают числа \texttt{n} и \texttt{m} и, затем, матрица целых чисел размера \texttt{n} строк на \texttt{m} столбцов. Напечатайте индексы наибольшего элемента матриц. Нумерация строк и столбцов начинается с \texttt{0}.
\begin{center}
\begin{tabular}{ l | l }
 вход & выход \\ \hline
 \texttt{3 4} & \texttt{1 2}  \\ 
 \texttt{7 3 1 2} &  \\ 
 \texttt{5 1 9 2} &  \\ 
 \texttt{7 2 5 4} &  \\ 
\end{tabular}
\end{center}

\item На вход поступают числа \texttt{n} и \texttt{m} и, затем, матрица целых чисел размера \texttt{n} строк на \texttt{m} столбцов. Поменяйте последние 2 столбца местами и напечатайте.
\begin{center}
\begin{tabular}{ l | l }
 вход & выход \\ \hline
 \texttt{3 4} &        \texttt{7 3 2 1}  \\ 
 \texttt{7 3 1 2} &    \texttt{5 1 2 9}\\ 
 \texttt{5 1 9 2} &    \texttt{7 2 4 5}\\ 
 \texttt{7 2 5 4} &  \\ 
\end{tabular}
\end{center}
\end{itemize}


\subsection*{Функции}
\begin{itemize}
\item Напишите функцию, которая принимает 2 целых числа и печатает их сумму.
\item Напишите функцию, которая принимает 2 целых числа и возвращает их сумму. Протестируйте эту функцию в \texttt{main}.

\item Напишите функцию, которая принимает 2 вещественных числа $a$ и  $b$ и возвращает их среднее геометрическое $c$.
$$
c = \sqrt{a \cdot b}
$$
Протестируйте эту функцию в \texttt{main}.
\end{itemize}

\subsection*{Функции и массивы}
\begin{itemize}
\item Напишите функцию, которая принимает на вход массив целых чисел и печатает сумму этих чисел.
\item Напишите функцию, которая принимает на вход массив целых чисел и возвращает сумму этих чисел. Протестируйте эту функцию в \texttt{main}.
\item Напишите функцию, которая принимает на вход массив вещественных чисел и возвращает среднее значение этих чисел. Протестируйте эту функцию в \texttt{main}.
\item Напишите функцию, которая принимает на вход массив целых чисел и возвращает \texttt{1} если все эти числа делятся на \texttt{7}. Если хотя бы одно из чисел не делится на \texttt{7}, то функция должна вернуть \texttt{0}.
\item Напишите функцию, которая принимает на вход массив вещественных чисел и возвращает \texttt{1} если все эти числа по модулю меньше \texttt{1}. Если хотя бы одно из чисел по модулю больше \texttt{1}, то функция должна вернуть \texttt{0}. Протестируйте эту функцию в \texttt{main}.
\end{itemize}


\subsection*{Функции. Рекурсия}
\begin{itemize}
\item Напишите рекурсивную функцию, которая будет принимать целое положительное число и возвращать сумму цифр в этом числе.
\end{itemize}

\subsection*{Символы}
\begin{itemize}
\item Напишите программу, которая принимает на вход число \texttt{n} и 1 символ и печатает этот символ \texttt{n} раз.
\end{itemize}

\subsection*{Простые алгоритмы сортировки ($O(N^2)$)}
\begin{itemize}
\item На вход поступает число \texttt{n} и, затем, \texttt{n} целых чисел. Отсортируйте эти числа по возрастанию и напечатайте.
\item На вход поступает число \texttt{n} и, затем, \texttt{n} целых чисел. Напишите функцию, которая будет сортировать эти числа. Примените эту функцию в \texttt{main} и напечатайте эти числа.
\end{itemize}

\subsection*{Структуры и функции}
\begin{itemize}
\item Структура \texttt{struct point} задаётся следующим образом.
\begin{lstlisting}
struct point {
    float x, y;
};
\end{lstlisting}
\begin{itemize}
\item Напишите функцию, которая не будет ничего принимать, а будет возвращать точку с координатами \texttt{(7, 5)}.
\item Напишите функцию, которая будет принимать точку, и печатать её.
\item Напишите функцию, которая будет принимать точку, менять местами координаты и возвращать её.
\item Напишите функцию, которая будет принимать 2 точки, и возвращать точку, которая лежит посередине между ними.
\item Напишите функцию, которая не будет ничего возвращать, а будет принимать указатель на точку, и менять местами координаты.
\item Протестируйте все эти функции в \texttt{main}.
\end{itemize}
\item Создайте массив из 10-ти элементов типа \texttt{struct point} в функции main. Значения задайте сами.
\item Отсортируйте все эти точки по первой координате и напечатайте.
\item Отсортируйте все эти точки по удалению от начала координат и напечатайте.
\item Напишите функцию, которая принимает на вход массив точек и возвращает точку -- центр масс этих точек (при условии, что все точки имеют одинаковую массу).
\end{itemize}
\end{document}