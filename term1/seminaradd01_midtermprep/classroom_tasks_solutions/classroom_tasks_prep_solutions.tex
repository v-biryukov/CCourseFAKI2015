\documentclass{article}
\usepackage[utf8x]{inputenc}
\usepackage{ucs}
\usepackage{amsmath} 
\usepackage{amsfonts}
\usepackage{marvosym}
\usepackage{wasysym}
\usepackage{upgreek}
\usepackage[english,russian]{babel}
\usepackage{graphicx}
\usepackage{float}
\usepackage{textcomp}
\usepackage{hyperref}
\usepackage{geometry}
  \geometry{left=2cm}
  \geometry{right=1.5cm}
  \geometry{top=1cm}
  \geometry{bottom=2cm}
\usepackage{tikz}
\usepackage{ccaption}
\usepackage{multicol}

\hypersetup{
   colorlinks=true,
   citecolor=blue,
   linkcolor=black,
   urlcolor=blue
}

\usepackage{listings}
%\setlength{\columnsep}{1.5cm}
%\setlength{\columnseprule}{0.2pt}

\usepackage[absolute]{textpos}

\usepackage{colortbl,graphicx,tikz}
\definecolor{X}{rgb}{.5,.5,.5}

\renewcommand{\thesubsection}{\arabic{subsection}}

\begin{document}
\pagenumbering{gobble}
\lstset{
  language=C,                % choose the language of the code
  basicstyle=\linespread{1.1}\ttfamily,
  columns=fixed,
  fontadjust=true,
  basewidth=0.5em,
  keywordstyle=\color{blue}\bfseries,
  commentstyle=\color{gray},
  stringstyle=\ttfamily\color{orange!50!black},
  showstringspaces=false,
  numbersep=5pt,
  numberstyle=\tiny\color{black},
  numberfirstline=true,
  stepnumber=1,                   % the step between two line-numbers.        
  numbersep=10pt,                  % how far the line-numbers are from the code
  backgroundcolor=\color{white},  % choose the background color. You must add \usepackage{color}
  showstringspaces=false,         % underline spaces within strings
  captionpos=b,                   % sets the caption-position to bottom
  breaklines=true,                % sets automatic line breaking
  breakatwhitespace=true,         % sets if automatic breaks should only happen at whitespace
  xleftmargin=.2in,
  extendedchars=\true,
  keepspaces = true,
}
\lstset{literate=%
   *{0}{{{\color{red!20!violet}0}}}1
    {1}{{{\color{red!20!violet}1}}}1
    {2}{{{\color{red!20!violet}2}}}1
    {3}{{{\color{red!20!violet}3}}}1
    {4}{{{\color{red!20!violet}4}}}1
    {5}{{{\color{red!20!violet}5}}}1
    {6}{{{\color{red!20!violet}6}}}1
    {7}{{{\color{red!20!violet}7}}}1
    {8}{{{\color{red!20!violet}8}}}1
    {9}{{{\color{red!20!violet}9}}}1
}

\definecolor{solcolor}{RGB}{226,240,245}

\title{Семинар \#7: Повторение. Классные задачи.\vspace{-5ex}}\date{}\maketitle


\subsection*{Основы}
\begin{itemize}
\item На вход подаются 2 целых числа \texttt{a} и \texttt{b}. Напечатайте сначала \texttt{b}, а потом \texttt{a} через пробел.
\begin{lstlisting}[backgroundcolor = \color{solcolor}]
#include <stdio.h>
int main() {
    int a, b; 
    scanf("%i%i", &a, &b);
    printf("%i %i\n", b, a);
}
\end{lstlisting}
\item На вход подаются 2 целых числа \texttt{a} и \texttt{b}. Напечатайте остаток деления первого числа на второе.
\begin{lstlisting}[backgroundcolor = \color{solcolor}]
#include <stdio.h>
int main() {
    int a, b; 
    scanf("%i%i", &a, &b);
    printf("%i\n", a % b);
}
\end{lstlisting}
\item На вход подаются 3 целых числа. Напечатайте \texttt{Yes}, если третье число является суммой двух первых.
\begin{lstlisting}[backgroundcolor = \color{solcolor}]
#include <stdio.h>
int main() {
    int a, b, c; 
    scanf("%i%i%i", &a, &b, &c);
    if (a + b == c) {
        printf("Yes\n");
    }
    else {
        printf("No\n");
    }
}
\end{lstlisting}
\item На вход подаются 2 целых числа \texttt{a} и \texttt{b}. Напечатайте наибольшее из этих чисел.
\begin{lstlisting}[backgroundcolor = \color{solcolor}]
#include <stdio.h>
int main() {
    int a, b; 
    scanf("%i%i", &a, &b);
    if (a > b) {
        printf("%i\n", a);
    }
    else {
        printf("%i\n", b);
    }
}
\end{lstlisting}
\item На вход подаются 2 целых числа \texttt{a} и \texttt{b}. Напечатайте все числа от наибольшего из этих чисел до наименьшего.
\begin{center}
\begin{tabular}{ l | l }
 вход & выход \\ \hline
 \texttt{2 8} & \texttt{8 7 6 5 4 3 2}  \\ \hline
 \texttt{9 6} & \texttt{9 8 7 6}  \\ 
\end{tabular}
\end{center}
\begin{lstlisting}[backgroundcolor = \color{solcolor}]
#include <stdio.h>
int main() {
    int a, b; 
    scanf("%i%i", &a, &b);
    int max, min;
    if (a > b) {
        max = a;
        min = b;
    }
    else {
        max = b;
        min = a;
    }
    
    for (int i = max; i >= min; --i) {
        printf("%i ", i);
    }
}
\end{lstlisting}
\item На вход поступает число \texttt{n} и, затем, \texttt{n} целых чисел. Напечатайте сумму этих \texttt{n} чисел.
\begin{center}
\begin{tabular}{ l | l }
 вход & выход \\ \hline
 \texttt{3} & \texttt{11}  \\ 
 \texttt{7 3 1} &  \\ 
\end{tabular}
\end{center}

\begin{lstlisting}[backgroundcolor = \color{solcolor}]
#include <stdio.h>
int main() {
    int n; 
    scanf("%i", &n);
    
    int sum = 0;
    for (int i = 0; i < n; ++i) {
        int num;
        scanf("%i", &num);
        sum += num;
    }
    
    printf("%i\n", sum);
}
\end{lstlisting}

\newpage
\item На вход поступает число \texttt{n} и, затем, \texttt{n} целых чисел. Напечатайте наибольшее из этих \texttt{n} чисел.
\begin{center}
\begin{tabular}{ l | l }
 вход & выход \\ \hline
 \texttt{4} & \texttt{8}  \\ 
 \texttt{7 3 8 2} &  \\ 
\end{tabular}
\end{center}

\begin{lstlisting}[backgroundcolor = \color{solcolor}]
#include <stdio.h>
#include <limits.h>
int main() {
    int n; 
    scanf("%i", &n);
    
    int max = INT_MIN;
    for (int i = 0; i < n; ++i) {
        int num;
        scanf("%i", &num);
        if (num > max) {
            max = num;
        }
    }
    
    printf("%i\n", max);
}
\end{lstlisting}

\item На вход поступает число \texttt{n} и, затем, \texttt{n} целых чисел. Напечатайте сумму первого и последнего элемента последовательности.
\begin{center}
\begin{tabular}{ l | l }
 вход & выход \\ \hline
 \texttt{3} & \texttt{12}  \\ 
 \texttt{7 3 5} &  \\ \hline
 \texttt{4} & \texttt{9}  \\ 
 \texttt{5 8 2 4} &  \\
\end{tabular}
\end{center}

\begin{lstlisting}[backgroundcolor = \color{solcolor}]
#include <stdio.h>

int main() {
    int n; 
    scanf("%i", &n);
    int sum = 0;
    
    for (int i = 0; i < n; ++i) {
        int num;
        scanf("%i", &num);
        if (i == 0) {
            sum += num;
        }
        if (i == n - 1) {
            sum += num;
        }
    }
    
    printf("%i\n", sum);
}
\end{lstlisting}
\end{itemize}

\subsection*{Переполнение}
\begin{itemize}
\item На вход подаётся 1 целое число \texttt{a} из диапазона от $0$ до $2^{64} - 2$. Напечатайте число, которое на 1 больше.
\begin{center}
\begin{tabular}{ l | l }
 вход & выход \\ \hline
 \texttt{5} & \texttt{6}  \\ \hline
 \texttt{123456789123} & \texttt{123456789124}  \\ 
\end{tabular}
\end{center}
\begin{lstlisting}[backgroundcolor = \color{solcolor}]
#include <stdio.h>

int main() {
    unsigned long long a;
    scanf("%llu", &a);
    printf("%llu\n", a + 1);
}
\end{lstlisting}
\item На вход подаются 2 целых числа из диапазона от $0$ до $2^{32} - 1$. Напечатайте их произведение.
\begin{center}
\begin{tabular}{ l | l }
 вход & выход \\ \hline
 \texttt{2 2} & \texttt{4}  \\ \hline
  \texttt{123456789 1000000} & \texttt{123456789000000}  \\ \hline
 \texttt{123456789 123456789} & \texttt{15241578750190521}  \\ 
\end{tabular}
\end{center}
\begin{lstlisting}[backgroundcolor = \color{solcolor}]
#include <stdio.h>

int main() {
    unsigned long long a, b;
    scanf("%llu%llu", &a, &b);
    printf("%llu\n", a * b);
}
\end{lstlisting}
\end{itemize}

\subsection*{Вещественные числа}
\begin{itemize}
\item На вход подаются 2 вещественных числа. Напечатайте их сумму. \\
Используя тип \texttt{float}:
\begin{lstlisting}[backgroundcolor = \color{solcolor}]
#include <stdio.h>

int main() {
    float a, b;
    scanf("%f%f", &a, &b);
    printf("%f\n", a + b);
}
\end{lstlisting}
Или используя более точный тип \texttt{double}:
\begin{lstlisting}[backgroundcolor = \color{solcolor}]
#include <stdio.h>

int main() {
    double a, b;
    scanf("%lf%lf", &a, &b);
    printf("%lf\n", a + b);
}
\end{lstlisting}
\item На вход подаются 2 вещественных числа \texttt{x} и \texttt{y}. Напечатайте \texttt{Yes} если точка \texttt{(x, y)} попадает внутрь единичной окружности и \texttt{No} иначе.
\begin{center}
\begin{tabular}{ l | l }
 вход & выход \\ \hline
 \texttt{0.5 -0.5} & \texttt{Yes}  \\ \hline
 \texttt{0.7 0.7} & \texttt{Yes}  \\ \hline
 \texttt{0.7 0.8} & \texttt{No}  \\ 
\end{tabular}
\end{center}
\begin{lstlisting}[backgroundcolor = \color{solcolor}]
#include <stdio.h>

int main() {
    double x, y;
    scanf("%lf%lf", &x, &y);
    if (x * x + y * y < 1) {
        printf("Yes\n");
    }
    else {
        printf("No\n");
    }
}
\end{lstlisting}
\item На вход подаётся 1 вещественное число \texttt{a} -- значение угла в градусах. Напечатайте значение выражения $sin(a) \cdot tan(a)$.
\begin{center}
\begin{tabular}{ l | l }
 вход & выход \\ \hline
 \texttt{45} & \texttt{0.707}  \\ \hline
 \texttt{10} & \texttt{0.031}  \\ \hline
 \texttt{80} & \texttt{5.585}  \\ 
\end{tabular}
\end{center}

\begin{lstlisting}[backgroundcolor = \color{solcolor}]
#include <stdio.h>
#include <math.h>

int main() {
    double a;
    scanf("%lf", &a);
    const double pi = 3.14159265;
    printf("%lf\n", sin(a * pi / 180) * tan(a * pi / 180));
}
\end{lstlisting}
\end{itemize}

\newpage
\subsection*{Массивы}
\begin{itemize}
\item На вход поступает число \texttt{n} и, затем, \texttt{n} целых чисел. Напечатайте эту последовательность 2 раза.
\begin{center}
\begin{tabular}{ l | l }
 вход & выход \\ \hline
 \texttt{3} & \texttt{7 3 1 7 3 1}  \\ 
 \texttt{7 3 1} &  \\ 
\end{tabular}
\end{center}

\begin{lstlisting}[backgroundcolor = \color{solcolor}]
#include <stdio.h>

int main() {
    int n;
    scanf("%i", &n);
    int array[1000];
    for (int i = 0; i < n; ++i) {
        scanf("%i", &array[i]);
    }
    
    for (int i = 0; i < n; ++i) {
        printf("%i ", array[i]);
    }
    for (int i = 0; i < n; ++i) {
        printf("%i ", array[i]);
    }
    printf("\n");
}
\end{lstlisting}

\item На вход поступает число \texttt{n} и, затем, \texttt{n} целых чисел. Напечатайте эту последовательность 2 раза. Первый раз в нормальном порядке, второй раз -- в обратном.
\begin{center}
\begin{tabular}{ l | l }
 вход & выход \\ \hline
 \texttt{3} & \texttt{7 3 1 1 3 7}  \\ 
 \texttt{7 3 1} &  \\ 
\end{tabular}
\end{center}

\begin{lstlisting}[backgroundcolor = \color{solcolor}]
#include <stdio.h>

int main() {
    int n;
    scanf("%i", &n);
    int array[1000];
    for (int i = 0; i < n; ++i) {
        scanf("%i", &array[i]);
    }
    
    for (int i = 0; i < n; ++i) {
        printf("%i ", array[i]);
    }
    for (int i = 0; i < n; ++i) {
        printf("%i ", array[n - 1 - i]);
    }
    printf("\n");
}
\end{lstlisting}

\item На вход поступает число \texttt{n} и, затем, \texttt{n} целых чисел. Напечатайте эту последовательность в обратном порядке, повторив каждое число дважды.
\begin{center}
\begin{tabular}{ l | l }
 вход & выход \\ \hline
 \texttt{3} & \texttt{1 1 3 3 7 7}  \\ 
 \texttt{7 3 1} &  \\ 
\end{tabular}
\end{center}

\begin{lstlisting}[backgroundcolor = \color{solcolor}]
#include <stdio.h>

int main() {
    int n;
    scanf("%i", &n);
    int array[1000];
    for (int i = 0; i < n; ++i) {
        scanf("%i", &array[i]);
    }
    
    for (int i = 0; i < n; ++i) {
        printf("%i ", array[n - 1 - i]);
        printf("%i ", array[n - 1 - i]);
    }
    printf("\n");
}
\end{lstlisting}

\item На вход поступает число \texttt{n} и, затем, \texttt{n} целых чисел. Ещё на вход приходит целое число \texttt{k}. Напечатайте эту последовательность в обратном порядке, повторив каждое число \texttt{k} раз.
\begin{center}
\begin{tabular}{ l | l }
 вход & выход \\ \hline
 \texttt{3} & \texttt{1 1 1 1 3 3 3 3 7 7 7 7}  \\ 
 \texttt{7 3 1} &  \\ 
 \texttt{4} &  \\
\end{tabular}
\end{center}

\begin{lstlisting}[backgroundcolor = \color{solcolor}]
#include <stdio.h>

int main() {
    int n;
    scanf("%i", &n);
    int array[1000];
    for (int i = 0; i < n; ++i) {
        scanf("%i", &array[i]);
    }
    int k;
    scanf("%i", &k);
    
    for (int i = 0; i < n; ++i) {
        for (int j = 0; j < k; ++j) {
       	    printf("%i ", array[n - 1 - i]);
       	}
    }
    printf("\n");
}
\end{lstlisting}

\item На вход поступает число \texttt{n} и, затем, две последовательности по \texttt{n} целых чисел каждая. Сложите эти две последовательности поэлементно и напечатайте её.
\begin{center}
\begin{tabular}{ l | l }
 вход & выход \\ \hline
 \texttt{4} & \texttt{12 4 10 4}  \\ 
 \texttt{7 3 1 2} &  \\ 
 \texttt{5 1 9 2} &  \\ 
\end{tabular}
\end{center}

\begin{lstlisting}[backgroundcolor = \color{solcolor}]
#include <stdio.h>

int main() {
    int n;
    scanf("%i", &n);
    int array1[1000];
    for (int i = 0; i < n; ++i) {
        scanf("%i", &array1[i]);
    }
    int array2[1000];
    for (int i = 0; i < n; ++i) {
        scanf("%i", &array2[i]);
    }
    
    for (int i = 0; i < n; ++i) {
        array1[i] += array2[i];
    }
    for (int i = 0; i < n; ++i) {
        printf("%i ", array1[i]);
    }
    printf("\n");
}
\end{lstlisting}

\end{itemize}

\newpage
\subsection*{Двумерные массивы}
\begin{itemize}
\item На вход поступают числа \texttt{n} и \texttt{m} и, затем, матрица целых чисел размера \texttt{n} строк на \texttt{m} столбцов. Напечатайте все суммы строк.
\begin{center}
\begin{tabular}{ l | l }
 вход & выход \\ \hline
 \texttt{3 4} & \texttt{13 17 18}  \\ 
 \texttt{7 3 1 2} &  \\ 
 \texttt{5 1 9 2} &  \\ 
 \texttt{7 2 5 4} &  \\ 
\end{tabular}
\end{center}

\begin{lstlisting}[backgroundcolor = \color{solcolor}]
#include <stdio.h>

int main() {
    int n, m;
    scanf("%i%i", &n, &m);
    int matrix[100][100];
    for (int i = 0; i < n; ++i) {
        for (int j = 0; j < m; ++j) {
            scanf("%i", &matrix[i][j]);
        }
    }
    
    for (int i = 0; i < n; ++i) {
        int row_sum = 0;
        for (int j = 0; j < m; ++j) {
            row_sum += matrix[i][j];
        }
        printf("%i ", row_sum);
    }
    printf("\n");
}
\end{lstlisting}
Эту задачу можно решить гораздо более эффективно -- без использования двумерного массива:
\begin{lstlisting}[backgroundcolor = \color{solcolor}]
#include <stdio.h>

int main() {
    int n, m;
    scanf("%i%i", &n, &m);
    for (int i = 0; i < n; ++i) {
        int row_sum = 0;
        for (int j = 0; j < m; ++j) {
            int num;
            scanf("%i", &num);
            row_sum += num;
        }
        printf("%i ", row_sum);
    }
    printf("\n");
}
\end{lstlisting}
Если вы запустите эту программу в терминале, то сумма строки будет печататься после ввода каждой строки. Это происходит потому что и ввод и вывод производится в/из одного места. Если же бы мы считывали, к примеру, из одного файла, а записывали в другой, то программа работала бы как надо.

\item На вход поступают числа \texttt{n} и \texttt{m} и, затем, матрица целых чисел размера \texttt{n} строк на \texttt{m} столбцов. Напечатайте все суммы столбцов.
\begin{center}
\begin{tabular}{ l | l }
 вход & выход \\ \hline
 \texttt{3 4} & \texttt{19 6 15 8}  \\ 
 \texttt{7 3 1 2} &  \\ 
 \texttt{5 1 9 2} &  \\ 
 \texttt{7 2 5 4} &  \\ 
\end{tabular}
\end{center}

\begin{lstlisting}[backgroundcolor = \color{solcolor}]
#include <stdio.h>
int main() {
    int n, m;
    scanf("%i%i", &n, &m);
    int matrix[100][100];
    for (int i = 0; i < n; ++i) {
        for (int j = 0; j < m; ++j) {
            scanf("%i", &matrix[i][j]);
        }
    }
    
    int column_sums[100] = {};
    for (int i = 0; i < n; ++i) {
        for (int j = 0; j < m; ++j) {
            column_sums[j] += matrix[i][j];
        }
    }
    
    for (int j = 0; j < m; ++j) {
        printf("%i ", column_sums[j]);
    }
    printf("\n");
}
\end{lstlisting}
Эту задачу можно решить гораздо более эффективно -- без использования двумерного массива:
\begin{lstlisting}[backgroundcolor = \color{solcolor}]
#include <stdio.h>
int main() {
    int n, m;
    scanf("%i%i", &n, &m);
    
    int column_sums[100] = {};
    for (int i = 0; i < n; ++i) {
        for (int j = 0; j < m; ++j) {
            int num;
            scanf("%i", &num);
            column_sums[j] += num;
        }
    }
 
    for (int j = 0; j < m; ++j) {
        printf("%i ", column_sums[j]);
    }
    printf("\n");
}
\end{lstlisting}


\item На вход поступают числа \texttt{n} и \texttt{m} и, затем, матрица целых чисел размера \texttt{n} строк на \texttt{m} столбцов. Напечатайте индексы наибольшего элемента матриц. Нумерация строк и столбцов начинается с \texttt{0}.
\begin{center}
\begin{tabular}{ l | l }
 вход & выход \\ \hline
 \texttt{3 4} & \texttt{1 2}  \\ 
 \texttt{7 3 1 2} &  \\ 
 \texttt{5 1 9 2} &  \\ 
 \texttt{7 2 5 4} &  \\ 
\end{tabular}
\end{center}

\begin{lstlisting}[backgroundcolor = \color{solcolor}]
#include <stdio.h>
#include <limits.h>
int main() {
    int n, m;
    scanf("%i%i", &n, &m);
    
    int max = INT_MIN;
    int imax = 0;
    int jmax = 0;
    for (int i = 0; i < n; ++i) {
        for (int j = 0; j < m; ++j) {
            int num;
            scanf("%i", &num);
            if (num > max) {
                max = num;
                imax = i;
                jmax = j;
            }
        }
    }
    
    printf("%i %i %i\n", max, imax, jmax);
}
\end{lstlisting}

\newpage
\item На вход поступают числа \texttt{n} и \texttt{m} и, затем, матрица целых чисел размера \texttt{n} строк на \texttt{m} столбцов. Поменяйте последние 2 столбца местами и напечатайте.
\begin{center}
\begin{tabular}{ l | l }
 вход & выход \\ \hline
 \texttt{3 4} &        \texttt{7 3 2 1}  \\ 
 \texttt{7 3 1 2} &    \texttt{5 1 2 9}\\ 
 \texttt{5 1 9 2} &    \texttt{7 2 4 5}\\ 
 \texttt{7 2 5 4} &  \\ 
\end{tabular}
\end{center}

\begin{lstlisting}[backgroundcolor = \color{solcolor}]
#include <stdio.h>

int main() {
    int n, m;
    scanf("%i%i", &n, &m);
    int array[100][100];
    for (int i = 0; i < n; ++i) {
       for (int j = 0; j < m; ++j) {
           scanf("%i", &array[i][j]);
       }
    }
    for (int i = 0; i < n; ++i) {
        int temp = array[i][m - 1];
        array[i][m - 1] = array[i][m - 2];
        array[i][m - 2] = temp;
    }
    
    for (int i = 0; i < n; ++i) {
       for (int j = 0; j < m; ++j) {
           printf("%i ", array[i][j]);
       }
       printf("\n");
    }
}
\end{lstlisting}
\end{itemize}


\subsection*{Функции}
\begin{itemize}
\item Напишите функцию, которая принимает 2 целых числа и печатает их сумму.

\begin{lstlisting}[backgroundcolor = \color{solcolor}]
#include <stdio.h>

int func(int a, int b) {
    printf("%i\n", a + b);
}

int main() {
    func(7, 3);   
}
\end{lstlisting}


\newpage
\item Напишите функцию, которая принимает 2 целых числа и возвращает их сумму. Протестируйте эту функцию в \texttt{main}.
\begin{lstlisting}[backgroundcolor = \color{solcolor}]
#include <stdio.h>

int func(int a, int b) {
    return a + b;
}

int main() {
    printf("%i\n", func(7, 3));   
}
\end{lstlisting}
\item Напишите функцию, которая принимает 2 вещественных числа $a$ и  $b$ и возвращает их среднее геометрическое $c$.
$$
c = \sqrt{a \cdot b}
$$
Протестируйте эту функцию в \texttt{main}.

\begin{lstlisting}[backgroundcolor = \color{solcolor}]
#include <stdio.h>
#include <math.h>

double func(double a, double b) {
    return sqrt(a * b);
}

int main() {
    printf("%lf\n", func(7, 3));   
}
\end{lstlisting}
\end{itemize}

\subsection*{Функции и массивы}
\begin{itemize}
\item Напишите функцию, которая принимает на вход массив целых чисел и печатает сумму этих чисел.
\begin{lstlisting}[backgroundcolor = \color{solcolor}]
#include <stdio.h>

void func(int array[], int n) {
    int sum = 0;
    for (int i = 0; i < n; ++i) {
        sum += array[i];
    }
    printf("%i\n", sum);
}

int main() {
    int a[5] = {6, 2, 1, 5, 2};
    func(a, 5);
}
\end{lstlisting}

\newpage
\item Напишите функцию, которая принимает на вход массив целых чисел и возвращает сумму этих чисел. Протестируйте эту функцию в \texttt{main}.
\begin{lstlisting}[backgroundcolor = \color{solcolor}]
#include <stdio.h>

int func(int array[], int n) {
    int sum = 0;
    for (int i = 0; i < n; ++i) {
        sum += array[i];
    }
    return sum;
}

int main() {
    int a[5] = {6, 2, 1, 5, 2};
    printf("%i\n", func(a, 5));
}
\end{lstlisting}
\item Напишите функцию, которая принимает на вход массив вещественных чисел и возвращает среднее значение этих чисел. Протестируйте эту функцию в \texttt{main}.
\begin{lstlisting}[backgroundcolor = \color{solcolor}]
#include <stdio.h>

double average(double array[], int n) {
    double sum = 0;
    for (int i = 0; i < n; ++i) {
        sum += array[i];
    }
    return sum / n;
}

int main() {
    double a[5] = {6, 2, 1, 5, 2};
    printf("%lf\n", average(a, 5));
}
\end{lstlisting}
\item Напишите функцию, которая принимает на вход массив целых чисел и возвращает \texttt{1} если все эти числа делятся на \texttt{7}. Если хотя бы одно из чисел не делится на \texttt{7}, то функция должна вернуть \texttt{0}.
\begin{lstlisting}[backgroundcolor = \color{solcolor}]
#include <stdio.h>

int is_all_div7(int array[], int n) {
    for (int i = 0; i < n; ++i) {
        if (array[i] % 7 != 0) {
            return 0;
        }
    }
    return 1;
}

int main() {
    int a[5] = {7, 147, 7, 14, 21};
    printf("%i\n", is_all_div7(a, 5));
}
\end{lstlisting}
\end{itemize}


\subsection*{Функции. Рекурсия}
\begin{itemize}
\item Напишите рекурсивную функцию, которая будет принимать целое положительное число и возвращать сумму цифр в этом числе.
\begin{lstlisting}[backgroundcolor = \color{solcolor}]
#include <stdio.h>

int digit_sum(int a) {
    if (a < 10) {
        return a;
    }
    return a % 10 + digit_sum(a / 10);
}

int main() {
    printf("%i\n", digit_sum(54316));
}
\end{lstlisting}
\end{itemize}

\subsection*{Символы}
\begin{itemize}
\item Напишите программу, которая принимает на вход число \texttt{n} и 1 символ и печатает этот символ \texttt{n} раз.
\begin{lstlisting}[backgroundcolor = \color{solcolor}]
#include <stdio.h>

int main() {
    int n;
    char x;
    scanf("%i %c", &n, &x);
    for (int i = 0; i < n; ++i) {
        printf("%c", x);
    }
    printf("\n");
}

\end{lstlisting}
\end{itemize}

\newpage
\subsection*{Простые алгоритмы сортировки ($O(N^2)$)}
\begin{itemize}
\item На вход поступает число \texttt{n} и, затем, \texttt{n} целых чисел. Отсортируйте эти числа по возрастанию и напечатайте.
\begin{lstlisting}[backgroundcolor = \color{solcolor}]
#include <stdio.h>

int main() {
    int n;
    scanf("%i", &n);
    int array[1000];
    for (int i = 0; i < n; ++i) {
        scanf("%i", &array[i]);
    }
    
    for (int i = 0; i < n; ++i) {
        int min_index = i;
        for (int j = i; j < n; ++j) {
            if (array[j] < array[min_index]) {
                min_index = j;
            }
        }
        int temp = array[i];
        array[i] = array[min_index];
        array[min_index] = temp;
    }
    
    for (int i = 0; i < n; ++i) {
        printf("%i ", array[i]);
    }
    printf("\n");
}
\end{lstlisting}
\newpage
\item На вход поступает число \texttt{n} и, затем, \texttt{n} целых чисел. Напишите функцию, которая будет сортировать эти числа. Примените эту функцию в \texttt{main} и напечатайте эти числа.

\begin{lstlisting}[backgroundcolor = \color{solcolor}]
#include <stdio.h>

void sort(int array[], int n) {
    for (int i = 0; i < n; ++i) {
        int min_index = i;
        for (int j = i; j < n; ++j) {
            if (array[j] < array[min_index]) {
                min_index = j;
            }
        }
        int temp = array[i];
        array[i] = array[min_index];
        array[min_index] = temp;
    }
}

int main() {
    int n;
    scanf("%i", &n);
    int array[1000];
    for (int i = 0; i < n; ++i) {
        scanf("%i", &array[i]);
    }
    
    sort(array, n);
    
    for (int i = 0; i < n; ++i) {
        printf("%i ", array[i]);
    }
    printf("\n");
}
\end{lstlisting}
\end{itemize}


\newpage
\subsection*{Структуры и функции}
\begin{itemize}
\item Структура \texttt{struct point} задаётся следующим образом.
\begin{lstlisting}
struct point {
    float x, y;
};
\end{lstlisting}
\begin{itemize}
\item Напишите функцию, которая не будет ничего принимать, а будет возвращать точку с координатами \texttt{(7, 5)}.
\begin{lstlisting}[backgroundcolor = \color{solcolor}]
struct point func1() {
    struct point a = {7, 5};
    return a;
}
\end{lstlisting}
\item Напишите функцию, которая будет принимать точку, и печатать её.
\begin{lstlisting}[backgroundcolor = \color{solcolor}]
void func2(struct point a) {
    printf("%i %i\n", a.x, a.y)
}
\end{lstlisting}
\item Напишите функцию, которая будет принимать точку, менять местами координаты и возвращать её.
\begin{lstlisting}[backgroundcolor = \color{solcolor}]
struct point func3(struct point a) {
    float temp = a.x;
    a.x = a.y;
    a.y = temp;
    return a;
}
\end{lstlisting}
\item Напишите функцию, которая будет принимать 2 точки, и возвращать точку, которая лежит посередине между ними.
\begin{lstlisting}[backgroundcolor = \color{solcolor}]
struct point func4(struct point a, struct point b) {
    struct point c = {(a.x + b.x) / 2, (a.y + b.y) / 2};
    return c;
}
\end{lstlisting}
\item Напишите функцию, которая не будет ничего возвращать, а будет принимать указатель на точку, и менять местами координаты.
\begin{lstlisting}[backgroundcolor = \color{solcolor}]
void func5(struct point* p) {
    float temp = p->x;
    p->x = p->y;
    p->y = temp;
}
\end{lstlisting}
\item Протестируйте все эти функции в \texttt{main}.

\end{itemize}
\item Создайте массив из 10-ти элементов типа \texttt{struct point} в функции main. Значения задайте сами.

\newpage
\item Отсортируйте все эти точки по первой координате и напечатайте.
\begin{lstlisting}[backgroundcolor = \color{solcolor}]
#include <stdio.h>

struct point {
    float x, y;
};
typedef struct point Point;

void sort(Point array[], int n) {
    for (int i = 0; i < n; ++i) {
        int min_index = i;
        for (int j = i; j < n; ++j) {
            if (array[j].x < array[min_index].x) {
                min_index = j;
            }
        }
        Point temp = array[i];
        array[i] = array[min_index];
        array[min_index] = temp;
    }
}

int main() {
    Point array[]  = {{5, 1}, {6, 3}, {1, 2}, {4, 1}, {7, -1}};
    
    sort(array, 5);
    
    for (int i = 0; i < 5; ++i) {
        printf("(%g, %g) ", array[i].x, array[i].y);
    }
    printf("\n");
}
\end{lstlisting}

\newpage
\item Отсортируйте все эти точки по удалению от начала координат и напечатайте.
\begin{lstlisting}[backgroundcolor = \color{solcolor}]
#include <stdio.h>

struct point {
    float x, y;
};
typedef struct point Point;

float sqdist(Point a) {
    return a.x * a.x + a.y * a.y;
}

void sort(Point array[], int n) {
    for (int i = 0; i < n; ++i) {
        int min_index = i;
        for (int j = i; j < n; ++j) {
            if (sqdist(array[j]) < sqdist(array[min_index])) {
                min_index = j;
            }
        }
        Point temp = array[i];
        array[i] = array[min_index];
        array[min_index] = temp;
    }
}

int main() {
    Point array[]  = {{5, 1}, {6, 3}, {1, 2}, {4, 1}, {7, -1}};
    
    sort(array, 5);
    
    for (int i = 0; i < 5; ++i) {
        printf("(%g, %g) ", array[i].x, array[i].y);
    }
    printf("\n");
}
\end{lstlisting}

\newpage
\item Напишите функцию, которая принимает на вход массив точек и возвращает точку -- центр масс этих точек (при условии, что все точки имеют одинаковую массу).

\begin{lstlisting}[backgroundcolor = \color{solcolor}]
#include <stdio.h>

struct point {
    float x, y;
};
typedef struct point Point;


Point center(Point array[], int n) {
    Point sum = {0, 0};
    for (int i = 0; i < n; ++i) {
        sum.x += array[i].x;
        sum.y += array[i].y;
    }
    sum.x /= n;
    sum.y /= n;
    return sum;
}

int main() {
    Point array[]  = {{5, 1}, {6, 3}, {1, 2}, {4, 1}, {7, -1}};
    
    Point c = center(array, 5);
    
    printf("(%g, %g)\n", c.x, c.y);
}
\end{lstlisting}
\end{itemize}
\end{document}