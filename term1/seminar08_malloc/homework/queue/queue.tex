\documentclass{article}
\usepackage[utf8x]{inputenc}
\usepackage{ucs}
\usepackage{amsmath} 
\usepackage{amsfonts}
\usepackage{marvosym}
\usepackage{wasysym}
\usepackage{upgreek}
\usepackage[english,russian]{babel}
\usepackage{graphicx}
\usepackage{float}
\usepackage{textcomp}
\usepackage{hyperref}
\usepackage{geometry}
  \geometry{left=2cm}
  \geometry{right=1.5cm}
  \geometry{top=1cm}
  \geometry{bottom=2cm}
\usepackage{tikz}
\usepackage{ccaption}
\usepackage{multicol}

\hypersetup{
   colorlinks=true,
   citecolor=blue,
   linkcolor=black,
   urlcolor=blue
}

\usepackage{listings}
%\setlength{\columnsep}{1.5cm}
%\setlength{\columnseprule}{0.2pt}

\usepackage[absolute]{textpos}

\usepackage{colortbl,graphicx,tikz}
\definecolor{X}{rgb}{.5,.5,.5}


\begin{document}
\pagenumbering{gobble}
\lstset{
  language=C,                % choose the language of the code
  basicstyle=\linespread{1.1}\ttfamily,
  columns=fixed,
  fontadjust=true,
  basewidth=0.5em,
  keywordstyle=\color{blue}\bfseries,
  commentstyle=\color{gray},
  stringstyle=\ttfamily\color{orange!50!black},
  showstringspaces=false,
  numbersep=5pt,
  numberstyle=\tiny\color{black},
  numberfirstline=true,
  stepnumber=1,                   % the step between two line-numbers.        
  numbersep=10pt,                  % how far the line-numbers are from the code
  backgroundcolor=\color{white},  % choose the background color. You must add \usepackage{color}
  showstringspaces=false,         % underline spaces within strings
  captionpos=b,                   % sets the caption-position to bottom
  breaklines=true,                % sets automatic line breaking
  breakatwhitespace=true,         % sets if automatic breaks should only happen at whitespace
  xleftmargin=.2in,
  extendedchars=\true,
  keepspaces = true,
}
\lstset{literate=%
   *{0}{{{\color{red!20!violet}0}}}1
    {1}{{{\color{red!20!violet}1}}}1
    {2}{{{\color{red!20!violet}2}}}1
    {3}{{{\color{red!20!violet}3}}}1
    {4}{{{\color{red!20!violet}4}}}1
    {5}{{{\color{red!20!violet}5}}}1
    {6}{{{\color{red!20!violet}6}}}1
    {7}{{{\color{red!20!violet}7}}}1
    {8}{{{\color{red!20!violet}8}}}1
    {9}{{{\color{red!20!violet}9}}}1
}

\section*{Домашнее задание: Очередь}
\begin{multicols}{2}
\begin{lstlisting}
#define CAPACITY 7
typedef int Data;

struct queue
{
    int front;
    int back;
    Data values[CAPACITY];
};
typedef struct queue Queue;

// ......

int main()
{
    Queue a;
    queue_init(&a);
    enqueue(&a, 100);
    for (int i = 0; i < 20; ++i)
    {
        enqueue(&a, i);
        dequeue(&a);
    }
    enqueue(&a, 200);
    queue_print(&a);
}
\end{lstlisting}
\vfill\null
Очередь — абстрактный тип данных с дисциплиной доступа к элементам «первый пришёл — первый вышел». \\
Реализация с помощью массива:
\begin{center}
\includegraphics[width=1.05\linewidth]{../../images/queue.png}
\end{center}
\end{multicols}

\subsubsection*{Очередь со статическим массивом. Задачи:}
\begin{enumerate}
\item Написать функцию \texttt{void enqueue(Queue* q, Data x)}.
\item Написать функцию \texttt{Data dequeue(Queue* q)}.
\item Написать функцию \texttt{void queue\_init(Queue* q)}.
 Протестируйте очередь: проверьте, что выведет программа, написанная выше.
\item Написать функцию \texttt{int queue\_is\_empty(Queue* q)}, которая возвращает 1 если очередь пуста и 0 иначе.
\item Написать функцию \texttt{int queue\_get\_size(Queue* q)}, которая возвращает число элементов в очереди (не \texttt{CAPACITY}!).
\item Написать функцию \texttt{int queue\_is\_full(Queue* q)}, которая возвращает 1 если очередь заполнена и 0 иначе.
\item Написать функции \texttt{Data queue\_get\_front(Queue* s)} и \texttt{Data queue\_get\_back(Queue* s)}, которые возвращают элементы, находящиеся в начале и в конце очереди соответственно, но не изменяют очередь.
\item Написать функцию \texttt{void queue\_print(Queue* s)}, которая распечатывает все элементы очереди.
\item Что произойдёт, если вызвать enqueue() при полной очереди или dequeue() при пустой? Обработайте эти ситуации. Программа должна печатать сообщение об ошибке и завершаться с аварийным кодом завершения. Чтобы завершить программу таким образом можно использовать функцию \texttt{exit()} из библиотеки \texttt{stdlib.h}. Пример вызова: \texttt{exit(1);}
\newpage
\item Протестируйте очередь на следующих тестах:
\begin{enumerate}
\item В очередь добавляется 4 элемента, затем удаляется 2. Вывести содержимое очереди с помощью \texttt{queue\_print()}
\item В очередь добавляется очень много элементов (больше чем \texttt{CAPACITY}). Программа должна напечатать сообщение об ошибке.
\item В очередь добавляется 3 элемента, затем удаляется 2, затем добавляется очень много элементов (больше чем \texttt{CAPACITY}). Программа должна напечатать сообщение об ошибке.
\item В очередь добавляется 3 элемента, затем удаляется 4. Программа должна напечатать сообщение об ошибке.
\item В очередь добавляется 2 элемента, затем выполняется следующий цикл:
\begin{verbatim}
for (int i = 0; i < 10000; ++i)
{
    enqueue(&a, i);
    dequeue(&a);
}
\end{verbatim}
Вывести содержимое очереди с помощью \texttt{queue\_print()}
\end{enumerate}

\subsubsection*{Очередь с динамическим массивом. Задачи:}

Описание такой очереди выглядит следующим образом:
\begin{lstlisting}
struct queue
{
    int capacity;
    int front;
    int back;
    Data* values;
};
typedef struct queue Queue;
\end{lstlisting}

\item Скопируйте код очереди со статическим массивом в новый файл и измените описание структуры как показано выше. Макрос \texttt{CAPACITY} больше не нужен, его можно удалить.

\item Измените функцию \texttt{void queue\_init(Queue* q)} на \texttt{void queue\_init(Queue* q, int initial\_capacity)}. Теперь она должна присваивать capacity начальное значение initial\_capacity и выделять необходимую память под массив values.

\item Измените функцию \texttt{void enqueue(Queue* q)}. Теперь, при заполнении очереди должно происходить перевыделение памяти с помощью функции \texttt{realloc()}. После перевыделения нужно переместить элементы массива на новые места и изменить \texttt{front} и \texttt{back}. Если \texttt{front != 0}, то нужно переместить элементы массива от \texttt{front} до конца старого массива \texttt{values} в конец нового массива \texttt{values}.

\item Добавьте функцию \texttt{void queue\_destroy(Queue* q)}, которая будет освобождать память, выделенную под массив \texttt{values}.

\item Протестируйте очередь: в очередь добавляется много элементов ($\gg 10^3 >$ \texttt{initial\_capacity}). Программа \textbf{не} должна напечатать сообщение об ошибке (если только совокупный размер элементов не превышает размер доступной оперетивной памяти). 

\item В случае, если \texttt{malloc()} или \texttt{realloc()} не смогли выделить запрашиваемый объём памяти (например, по причине того, что этот объём  больше, чем вся доступная оперативная память или по какой-нибудь иной причине), то они возвращают значение \texttt{NULL}. Программа должна это учитывать и завершаться с ошибкой, если нельзя выделить нужный объём памяти.
Как правильно использовать \texttt{realloc()} можно посмотреть по следующей ссылке:\\
\href{https://stackoverflow.com/questions/21006707/proper-usage-of-realloc}{https://stackoverflow.com/questions/21006707/proper-usage-of-realloc}
\end{enumerate}

\end{document}