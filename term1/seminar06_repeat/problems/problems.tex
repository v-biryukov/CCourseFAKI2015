\documentclass{article}
\usepackage[utf8x]{inputenc}
\usepackage{ucs}
\usepackage{amsmath} 
\usepackage{mathtext}
\usepackage{amsfonts}
\usepackage{upgreek}
\usepackage[english,russian]{babel}
\usepackage{graphicx}
\usepackage{float}
\usepackage{textcomp}
\usepackage{hyperref}
\usepackage{geometry}
  \geometry{left=2cm}
  \geometry{right=1.5cm}
  \geometry{top=1cm}
  \geometry{bottom=2cm}
\usepackage{tikz}
\usepackage{ccaption}



\begin{document}
\pagenumbering{gobble}

\section*{Задачи:}
\begin{enumerate}
\item \textbf{Произведение чисел:} Написать программу, которая считывает 2 числа $a$ и $b$ и печатает их произведение. $0 \le a, b \le 2^{32}-1$
\item \textbf{mod 7:} Написать программу, которая печатает все числа делящиеся на 7 в интервале от 700 до 1000.
\item \textbf{Часть года:} Написать функцию на вход которой подаётся целое число -- число дней прошедших с начала года. Она должна возвращать вещественное число -- доля прошедшего года(от 0 до 1). В году 365 дней.
\item \textbf{Математическая функция:} Написать функцию, которая вычисляет выражение $\sin(\sqrt{x})$ от положитльного числа $x$. Вызвать её в функции main().
\item \textbf{Нормализация:} На вход программе подаётся целое число $n$ и $n$ вещественных чисел. Нужно эти числа нормировать (то есть разделить на их сумму) и напечатать.

\end{enumerate}


\end{document}