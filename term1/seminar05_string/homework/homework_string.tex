\documentclass{article}
\usepackage[utf8x]{inputenc}
\usepackage{ucs}
\usepackage{amsmath} 
\usepackage{amsfonts}
\usepackage{marvosym}
\usepackage{wasysym}
\usepackage{upgreek}
\usepackage[english,russian]{babel}
\usepackage{graphicx}
\usepackage{float}
\usepackage{textcomp}
\usepackage{hyperref}
\usepackage{geometry}
  \geometry{left=2cm}
  \geometry{right=1.5cm}
  \geometry{top=1cm}
  \geometry{bottom=2cm}
\usepackage{tikz}
\usepackage{ccaption}
\usepackage{multicol}

\hypersetup{
   colorlinks=true,
   citecolor=blue,
   linkcolor=black,
   urlcolor=blue
}

\usepackage{listings}
%\setlength{\columnsep}{1.5cm}
%\setlength{\columnseprule}{0.2pt}

\usepackage[absolute]{textpos}


\usepackage{colortbl,graphicx,tikz}
\definecolor{X}{rgb}{.5,.5,.5}

\renewcommand{\thesubsection}{\arabic{subsection}}

\begin{document}
\pagenumbering{gobble}
\lstset{
  language=C,                % choose the language of the code
  basicstyle=\linespread{1.1}\ttfamily,
  columns=fixed,
  fontadjust=true,
  basewidth=0.5em,
  keywordstyle=\color{blue}\bfseries,
  commentstyle=\color{gray},
  stringstyle=\ttfamily\color{orange!50!black},
  showstringspaces=false,
  numbersep=5pt,
  numberstyle=\tiny\color{black},
  numberfirstline=true,
  stepnumber=1,                   % the step between two line-numbers.        
  numbersep=10pt,                  % how far the line-numbers are from the code
  backgroundcolor=\color{white},  % choose the background color. You must add \usepackage{color}
  showstringspaces=false,         % underline spaces within strings
  captionpos=b,                   % sets the caption-position to bottom
  breaklines=true,                % sets automatic line breaking
  breakatwhitespace=true,         % sets if automatic breaks should only happen at whitespace
  xleftmargin=.2in,
  extendedchars=\true,
  keepspaces = true,
}
\lstset{literate=%
   *{0}{{{\color{red!20!violet}0}}}1
    {1}{{{\color{red!20!violet}1}}}1
    {2}{{{\color{red!20!violet}2}}}1
    {3}{{{\color{red!20!violet}3}}}1
    {4}{{{\color{red!20!violet}4}}}1
    {5}{{{\color{red!20!violet}5}}}1
    {6}{{{\color{red!20!violet}6}}}1
    {7}{{{\color{red!20!violet}7}}}1
    {8}{{{\color{red!20!violet}8}}}1
    {9}{{{\color{red!20!violet}9}}}1
}


\makeatletter
\newcount\my@repeat@count
\newcommand{\myrepeat}[2]{%
  \begingroup
  \my@repeat@count=\z@
  \@whilenum\my@repeat@count<#1\do{#2\advance\my@repeat@count\@ne}%
  \endgroup
}
\makeatother

\title{Семинар \#5: Строки. Домашнее задание.\vspace{-5ex}}\date{}\maketitle

\subsection*{Задача 1: Номер буквы}
Считайте символ буквы латинского алфавита и напечатайте его номер в алфавите. Если на вход подаётся не буква, то нужно напечатать \texttt{Not a letter}.
\begin{center}
\begin{tabular}{ l | l }
 вход & выход \\ \hline
 \texttt{P} & \texttt{16} \\
 \texttt{b} & \texttt{2} \\
 \texttt{B} & \texttt{2} \\
 \texttt{\#} & \texttt{Not a letter}\\ 
\end{tabular}
\end{center}

\subsection*{Задача 2: Лесенка}
Считать слово и напечатать лесенку из этого числа. Например, для слова \texttt{Hello} нужно напечатать лесенку: 
\begin{center}
\begin{center}
\begin{tabular}{ c | l }
 вход & выход \\ \hline
 \texttt{Hello} & \texttt{H}  \\ 
 & \texttt{He} \\
 & \texttt{Hel} \\
 & \texttt{Hell} \\
 & \texttt{Hello}\\ 
\end{tabular}
\end{center}
\end{center}

\subsection*{Задача 3: Чередование}
Считать 2 слова и печатать их чередуя по одному символу. То есть сначала напечатать первый символ первого числа, потом первый символ второго числа, потом второй символ первого числа, второй символ второго и т. д. Если какое-то из чисел закончится, то нужно допечатать оставшееся число
\begin{center}
\begin{tabular}{ l | l }
 вход & выход \\ \hline
 \texttt{cat dog} & \texttt{cdaotg}  \\ 
 \texttt{cat elephant} & \texttt{cealtephant} \\
 \texttt{elephant dog} & \texttt{edloegphant} \\
 \texttt{aaaa bbbb} & \texttt{abababab} \\
 \texttt{aaaa b} & \texttt{abaaa}\\ 
 \texttt{a b} & \texttt{ab}\\ 
\end{tabular}
\end{center}



\subsection*{Задача 4: Палиндром}
На вход подаётся строка. Проверить является ли эта строка палидромом.

\begin{multicols}{2}
\begin{center}
\begin{tabular}{ l | l }
 вход & выход \\ \hline
 \texttt{cat} & \texttt{No} \\
 \texttt{abba} & \texttt{Yes} \\
 \texttt{aba} & \texttt{Yes} \\
 \texttt{a} & \texttt{Yes} \\
 \texttt{aa} & \texttt{Yes} \\
 \texttt{ab} & \texttt{No} 
\end{tabular}
\end{center}


\begin{center}
\begin{tabular}{ l | l }
 вход & выход \\ \hline
 \texttt{abcdedcba} & \texttt{Yes} \\
 \texttt{abcdedcb} & \texttt{No} \\
 \texttt{abcdedcbab} & \texttt{No} \\
 \texttt{abcdedcbb} & \texttt{No} \\
 \texttt{abcxedcba} & \texttt{No} \\
 \texttt{abcxexcba} & \texttt{Yes} 
\end{tabular}
\end{center}
\end{multicols}


\subsection*{Задача 5: Восклицание}
На вход подаётся строка. Напечатать эту же строку, но ставя восклицательный знак после каждого слова.
\begin{center}
\begin{tabular}{ l | l }
 вход & выход \\ \hline
 \texttt{Better late than never} & \texttt{Better! late! than! never!} \\
 \texttt{cat \quad dog elephant} & \texttt{cat! \quad dog! elephant!}  \\ 
 \texttt{cat} & \texttt{cat!} \\
 \texttt{a} & \texttt{a!} \\
\end{tabular}
\end{center}


\subsection*{Задача 6: Правильная скобочная последовательность}
На вход подаётся скобочная последовательность(строка, состоящая из символов \texttt{'('} и \texttt{')'}). Нужно выяснить является ли эта скобочная последовательность допустимой или нет.
\begin{multicols}{2}
\begin{center}
\begin{tabular}{ l | l }
 вход & выход \\ \hline
 \texttt{(()())} & \texttt{Yes} \\
 \texttt{(()(()()))} & \texttt{Yes} \\
 \texttt{(()))()())} & \texttt{No} \\
 \texttt{(((())))} & \texttt{Yes} \\
 \texttt{(((()))))} & \texttt{No} \\
 \texttt{(((()))} & \texttt{No} \\
\end{tabular}
\end{center}

\begin{center}
\begin{tabular}{ l | l }
 вход & выход \\ \hline
 \texttt{()} & \texttt{Yes} \\
 \texttt{)(} & \texttt{No} \\
 \texttt{))} & \texttt{No} \\
 \texttt{((} & \texttt{No} \\
 \texttt{()()} & \texttt{Yes} \\
 \texttt{(} & \texttt{No} \\
 \texttt{)} & \texttt{No} \\
\end{tabular}
\end{center}
\end{multicols}

\subsection*{Задача 7: Наиди ошибку в скобках}
На вход подаётся строка, которая может содержать скобки. Нужно выяснить является ли выражение допустимым с точки зрения положения скобок и, если нет, напечатать строку указывающую на положение ошибки. Положение об ошибке отметить с помощью символа \string^ (циркумфлекс).  Ошибка -- это закрывающая скобка, если до этого нет открывающей или первая незакрытая открывающая скобка. Если ошибки нет, то нужно напечатать \texttt{OK}.
\begin{center}
\begin{tabular}{ l | l }
 вход & выход \\ \hline
 \texttt{(x + y) * (z + w)} & \texttt{OK} \\ \hline
 \texttt{(x + y) * )z + w)} & \texttt{(x + y) * )z + w)} \\
                            & \texttt{\myrepeat{10}{\space}\string^} \\ \hline
 \texttt{((x + y) * (z + w} & \texttt{((x + y) * (z + w} \\
                            & \texttt{\string^} \\ \hline
 \texttt{f(g((x + y) * z), g(x))} & \texttt{OK} \\ \hline
 \texttt{f(g((x + y)) * z), g(x))} & \texttt{f(g((x + y)) * z), g(x))} \\
                            & \texttt{\myrepeat{23}{\space}\string^} \\ \hline
\end{tabular}
\end{center}


\subsection*{Задача 8: Удаление символа}
Напишите функцию \texttt{void delete\_chars(char str[], char c)}, которая будет удалять все символы, равные \texttt{c} из строки \texttt{str}. Постарайтесь сделать эту функцию как можно более эффективной.

\begin{center}
\begin{tabular}{ l | l }
 \texttt{str, c} & \texttt{str} после вызова \texttt{delete\_chars(str, c)} \\ \hline
 \texttt{cat a} & \texttt{ct} \\
 \texttt{elephant e} & \texttt{lphant} \\
 \texttt{aaaa a} & \texttt{} \\
 \texttt{a a} & \texttt{} \\
 \texttt{ababababa a} & \texttt{bbbb} \\
\end{tabular}
\end{center}
\subsection*{Задача 9: Сокровище}
Вы находитесь на плоскости в начале координат (точке с координатами \texttt{(0, 0)}) и вам нужно найти закопанное сокровище. Путь до него задаётся последовательностью команд, состоящих из направления и расстояния, которое нужно пройти в этом направлении. Вам нужно найти координаты сокровища. Используйте функцию \texttt{strcmp}.
\begin{center} 
\begin{tabular}{ l | l }
 вход & выход \\ \hline
 \texttt{6} & \texttt{-30 20}\\
 \texttt{North 10} & \\
 \texttt{East 20} &\\
 \texttt{South 50} &\\
 \texttt{West 60} &\\
 \texttt{East 10} &\\
 \texttt{North 60} &\\
\end{tabular}
\end{center}

\subsection*{Задача 10: Переворот слов в файле}
Счмтайте файл \texttt{input.txt}  и переверните все слова из этого файла и запишите результат в файл \texttt{output.txt}.
\begin{center} 
\begin{tabular}{ l | l }
 файл \texttt{input.txt} & файл \texttt{output.txt} \\ \hline
 \texttt{Cat and Dog} & \texttt{taC dna goD}\\
\end{tabular}
\end{center}
Протестируйте программу на файле \texttt{invisible\_man.txt}.


\end{document}