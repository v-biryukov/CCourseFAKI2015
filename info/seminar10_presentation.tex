\documentclass[14pt,pdf,hyperref={unicode}]{beamer}

% \documentclass[aspectratio=43]{beamer}
% \documentclass[aspectratio=1610]{beamer}
% \documentclass[aspectratio=169]{beamer}

\usepackage{lmodern}

% подключаем кириллицу 
\usepackage[T2A]{fontenc}
\usepackage[utf8]{inputenc}
\usepackage{listings}
\usepackage{graphicx}
\usepackage{hyperref}

% отключить клавиши навигации
\setbeamertemplate{navigation symbols}{}

% тема оформления
\usetheme{CambridgeUS}

% цветовая схема
\usecolortheme{seahorse}

\definecolor{light-gray}{gray}{0.90}

\lstset{basicstyle=\ttfamily,breaklines=true}

\title{info}   
\subtitle{ФАКИ 2016}
\author{Бирюков В. А.} 
\date{\today} 
% \logo{\includegraphics[height=5mm]{images/logo.png}\vspace{-7pt}}

\begin{document}

\lstset{language=C}

% титульный слайд
\begin{frame}
\titlepage
\end{frame} 

\defverbatim[colored]\makeset{
\begin{lstlisting}[language=C++,basicstyle=\ttfamily,keywordstyle=\color{blue}]
void make_set(int X) {
  parent[X] = X;
}
\end{lstlisting}
}

\lstset{
  language=C,                % choose the language of the code
  keywordstyle=\color{blue},
  numbers=none,                   % where to put the line-numbers
  stepnumber=1,                   % the step between two line-numbers.        
  numbersep=5pt,                  % how far the line-numbers are from the code
  backgroundcolor=\color{light-gray},  % choose the background color. You must add \usepackage{color}
  showspaces=false,               % show spaces adding particular underscores
  showstringspaces=false,         % underline spaces within strings
  showtabs=false,                 % show tabs within strings adding particular underscores
  tabsize=2,                      % sets default tabsize to 2 spaces
  captionpos=b,                   % sets the caption-position to bottom
  breaklines=true,                % sets automatic line breaking
  breakatwhitespace=true,         % sets if automatic breaks should only happen at whitespace
}





\begin{frame}[fragile]
\frametitle{Операционные системы} 
\begin{center}
\includegraphics[width=0.2\linewidth]{images/linux.png}
\end{center}
\begin{itemize}
\item Linux
\item Работа с командной строкой:
\href{https://stepic.org/course/%D0%92%D0%B2%D0%B5%D0%B4%D0%B5%D0%BD%D0%B8%D0%B5-%D0%B2-Linux-73/syllabus}{[Курс на stepic]}. 
\item Программы ssh, vim, tmux и др.
\end{itemize}
\end{frame}


\begin{frame}[fragile]
\frametitle{Компилируемые языки программирования} 
\begin{center}
\includegraphics[width=0.2\linewidth]{images/cplusplus.jpg}
\end{center}
\begin{itemize}
\item Язык C
\href{http://ocw.mit.edu/courses/electrical-engineering-and-computer-science/6-087-practical-programming-in-c-january-iap-2010/}{[Курс на mit.edu]}. 
\item Язык C++
\href{http://ocw.mit.edu/courses/electrical-engineering-and-computer-science/6-087-practical-programming-in-c-january-iap-2010/}{[Курс на coursera]}. 
\item Java, $C\#$ -- если будет нужно
\end{itemize}
\end{frame}

\begin{frame}[fragile]
\frametitle{Интерпретируемые языки программирования: Python} 
\begin{center}
\includegraphics[width=0.2\linewidth]{images/python.png}
\end{center}
\begin{itemize}
\item Язык Python
\href{https://www.udacity.com/course/intro-to-computer-science--cs101}{[Курс на udacity]}.
\item Библиотеки Python: numpy, matplotlib, scipy
\end{itemize}
\end{frame}


\begin{frame}[fragile]
\frametitle{Системы контроля версий: Git} 
\begin{center}
\includegraphics[width=0.4\linewidth]{images/git.jpg}
\end{center}
\begin{itemize}
\item git
\href{https://www.udacity.com/course/how-to-use-git-and-github--ud775}{[Курс на udacity]}.
\item github
\end{itemize}
\end{frame}


\begin{frame}[fragile]
\frametitle{Практика программирования} 
\begin{center}
\includegraphics[width=0.4\linewidth]{images/gsoc.png}
\end{center}
\begin{itemize}
\item Задачи на Spoj, codeforces
\item Google summer of code
\end{itemize}
\end{frame}


\end{document}
