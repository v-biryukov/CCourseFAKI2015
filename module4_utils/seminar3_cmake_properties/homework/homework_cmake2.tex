\documentclass{article}
\usepackage[utf8x]{inputenc}
\usepackage{ucs}
\usepackage{amsmath} 
\usepackage{amsfonts}
\usepackage{marvosym}
\usepackage{wasysym}
\usepackage{upgreek}
\usepackage[english,russian]{babel}
\usepackage{graphicx}
\usepackage{float}
\usepackage{textcomp}
\usepackage{hyperref}
\usepackage{geometry}
  \geometry{left=2cm}
  \geometry{right=1.5cm}
  \geometry{top=1cm}
  \geometry{bottom=2cm}
\usepackage{tikz}
\usepackage{ccaption}
\usepackage{multicol}
\usepackage{fancyvrb}
\usepackage{xcolor}

\hypersetup{
   colorlinks=true,
   citecolor=blue,
   linkcolor=black,
   urlcolor=blue
}

\usepackage{listings}
%\setlength{\columnsep}{1.5cm}
%\setlength{\columnseprule}{0.2pt}

\usepackage[absolute]{textpos}


\usepackage{colortbl,graphicx,tikz}
\definecolor{X}{rgb}{.5,.5,.5}

\renewcommand{\thesubsection}{\arabic{subsection}}

\begin{document}
\pagenumbering{gobble}
\lstset{
  language=C++,                % choose the language of the code
  basicstyle=\linespread{1.1}\ttfamily,
  columns=fixed,
  fontadjust=true,
  basewidth=0.5em,
  keywordstyle=\color{blue}\bfseries,
  commentstyle=\color{gray},
  stringstyle=\ttfamily\color{orange!50!black},
  showstringspaces=false,
  numbersep=5pt,
  numberstyle=\tiny\color{black},
  numberfirstline=true,
  stepnumber=1,                   % the step between two line-numbers.        
  numbersep=10pt,                  % how far the line-numbers are from the code
  backgroundcolor=\color{white},  % choose the background color. You must add \usepackage{color}
  showstringspaces=false,         % underline spaces within strings
  captionpos=b,                   % sets the caption-position to bottom
  breaklines=true,                % sets automatic line breaking
  breakatwhitespace=true,         % sets if automatic breaks should only happen at whitespace
  xleftmargin=.2in,
  extendedchars=\true,
  keepspaces = true,
}
\lstset{literate=%
   *{0}{{{\color{red!20!violet}0}}}1
    {1}{{{\color{red!20!violet}1}}}1
    {2}{{{\color{red!20!violet}2}}}1
    {3}{{{\color{red!20!violet}3}}}1
    {4}{{{\color{red!20!violet}4}}}1
    {5}{{{\color{red!20!violet}5}}}1
    {6}{{{\color{red!20!violet}6}}}1
    {7}{{{\color{red!20!violet}7}}}1
    {8}{{{\color{red!20!violet}8}}}1
    {9}{{{\color{red!20!violet}9}}}1
}
\newcommand\upquote[1]{\textquotesingle#1\textquotesingle}

\renewcommand{\thesubsection}{\arabic{subsection}}
\makeatletter
\def\@seccntformat#1{\@ifundefined{#1@cntformat}%
   {\csname the#1\endcsname\quad}%    default
   {\csname #1@cntformat\endcsname}}% enable individual control
\newcommand\section@cntformat{}     % section level 
\newcommand\subsection@cntformat{Задача \thesubsection.\space} % subsection level
\newcommand\subsubsection@cntformat{\thesubsubsection.\space} % subsubsection level
\makeatother


\makeatletter
\newcount\my@repeat@count
\newcommand{\myrepeat}[2]{%
  \begingroup
  \my@repeat@count=\z@
  \@whilenum\my@repeat@count<#1\do{#2\advance\my@repeat@count\@ne}%
  \endgroup
}
\makeatother

\title{Семинар: CMake 2. Домашнее задание.\vspace{-5ex}}\date{}\maketitle
\subsection{Свойства таргета}
На языке Cmake напишите функцию \texttt{print\_target\_info}, которая будет принимать на вход название таргета и печатать следующие свойства этого тарегета:
\begin{itemize}
\item \texttt{NAME} 
\item \texttt{TYPE} 
\item \texttt{SOURCES} 
\item \texttt{SOURCE\_DIR} 
\item \texttt{BINARY\_DIR} 
\item \texttt{INCLUDE\_DIRECTORIES} 
\item \texttt{COMPILE\_DEFINITIONS} 
\item \texttt{COMPILE\_OPTIONS} 
\item \texttt{LINK\_LIBRARIES} 
\item \texttt{LINK\_OPTIONS} 
\end{itemize}




\subsection{Передача в функцию}
\begin{Verbatim}[commandchars=\\\{\}]
\textcolor{teal}{function}(func)
    \textcolor{teal}{message}(\textcolor{olive}{"Number of arguments = "} ${ARGC})
    \textcolor{teal}{message}(\textcolor{olive}{"Arguments:"})
    \textcolor{teal}{foreach}(arg IN LISTS ARGV)
        \textcolor{teal}{message}(${arg})
    \textcolor{teal}{endforeach}()
    \textcolor{teal}{message}(\textcolor{olive}{"----------------------------"})
\textcolor{teal}{endfunction}()
\end{Verbatim}



Для ответа на этот вопрос запишите результаты по каждому пункт в один текстовый файл.



\end{document}
